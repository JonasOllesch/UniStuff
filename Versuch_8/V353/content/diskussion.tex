\section{Diskussion}
\label{sec:Diskussion}


Beim Vergleichen der Zeitkonstanten $ \dfrac{1}{RC}$ der ersten zwei Messungen ergibt sich ein Verhältnis von  $8,57 \pm 0,29$. 
Die einzelnen Werte liegen außerhalb der Standardabweichungen der einzelnen Werte, aber sie befinden sich in der gleichen Größenordnung.
Die Ergebnisse der ersten beiden Messungen entsprechen den aus der Theorie zu erwartenden Graphen.
Die berechneten Zeitkonstanten unterliegen jedoch einigen Fehlern, da die Innenwiderstände sämtlicher Bauteile nicht beachtet wurden, dass hat einen systematischen Fehler zur Folge, welcher jedoch die Form der Ausgleichskurven nicht verändert.
Weitere Fehler sind dadurch entstanden, dass das Bild am Oszilloskop nicht dauerhaft stabil war. Es war teilweise ein Laufen zu beobachten, welches die Messung erschwierig hat. 

Die Messung der Amplitudenspannung und der Phasenverschiebung sind vollständig unbrauchbar, da bei diesen Messungen das Oszilloskop falsch eingestellt wurde.
Das ist daran zu erkennen, dass der Wert von $RC = (2,570 \pm 1,380) \cdot 10^{-5} \, \unit{\second} \,$ mehrere Größenordnungen von den andere Werten entfernt ist.

Eine altenierende Triggerung hat dazu geführt, dass eine zufriedenstellende Messung der gewünschten Größen nicht möglich war.
Die falsche Einstellung des Oszilloskop ist erst nach dem Ende der Messreihe aufgefallen.
% nach Absprache mit der Praktikumsbetreuung sollte der Versuch jedoch nicht erneut durchgeführt werden.