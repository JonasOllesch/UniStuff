\section{Diskussion}
\label{sec:Diskussion}


Beim Vergleichen der Zeitkonstanten $ \dfrac{1}{RC}$ der Messungen aus \autoref{fig:ausgleichsa)} und der \autoref{fig:ausgleichsb)} ergibt sich ein Verhältnis von  $8.57 \pm 0.29$. 
Die einzelnen Werte liegen außerhalb der Standardabweichungen der einzelnen Werte, aber sie befinden sich in der gleichen Größenordnung.
Die Messungen aus \autoref{fig:ausgleichsa)} und \autoref{fig:ausgleichsb)} lassen sich gut durch die theoretischen Funktionen beschreiben.
Die berechneten Zeitkonstanten unterliegen noch weiteren Fehlern, da die Innenwiderstände sämtlicher Bauteile nicht beachtet wurden. 
Das hat einen systematischen Fehler zur Folge, welcher jedoch die Form der Ausgleichskurven nicht verändert.
Weitere Fehler sind dadurch entstanden, dass das Bild am Oszilloskop nicht dauerhaft stabil war. Es war teilweise ein Laufen zu beobachten, welches die Messung erschwirig hat. 

Die Messung der Amplitudenspannung und der Phasenverschiebung sind vollständig unbrauchbar, da bei diesen Messungen das Oszilloskop falsch eingestellt wurde. 
Eine altenierende Triggerung hat dazu geführt, dass eine zufriedenstellende Messung der gewünschten Größen nicht möglich war.
Die falsche Einstellung des Oszilloskop ist erst nach dem Ende der Messreihe aufgefallen, nach Absprache mit der Praktikumsbetreuung sollte der Versuch jedoch nicht erneut durchgeführt werden.