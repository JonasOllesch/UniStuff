\section{Energieverlust und Bethe-Bloch-Gleichung (10P)}

Die Bethe-Bloch-Gleichung beschreibt den Energieverlust schwerer Teilchen auf ihrem Weg durch Materie. Vernachlässigt man die Dichte- und Schalenkorrektur, ist sie gegeben durch
  \begin{equation*}
    -\frac{\text{d}E}{\text{d}x}=\underbrace{\frac{4\pi\,N_\text{A}\,r_e^2\,m_e c^2}{M_u}}_{\approx\SI{0.3}{\mega\electronvolt\,\centi\meter^2\per\gram}}\frac{Z\rho}{A}\,\frac{z^2}{\beta^2}\left(\ln\left[\frac{2\gamma^2\beta^2m_ec^2}{I}\right]-\beta^2\right)
  \end{equation*}
  dabei ist $N_\text{A}$ die Avogadro-Konstante, $r_e$ der klassische Elektronenradius, $M_u$ die molare Massenkonstante,
  $Z$ die Ordnungszahl des Target-Mediums, $A$ seine Massenzahl, $I$ sein mittleres Ionisationspotential, $\rho$ seine Dichte
  und $z$ die Ladungszahl des Projektilteilchens.

  Rechnen Sie nichtrelativistisch.
  Zudem können Sie den nur schwach geschwindigkeitsabhängigen Term durch
  \begin{equation*}
	  \ln\left[\frac{2\gamma^2\beta^2m_ec^2}{I}\right]-\beta^2\approx 5
  \end{equation*}
  nähern.
  Die Dichte von Silizium beträgt ungefähr $\SI{2}{\gram\per\centi\meter^{3}}$, die Ordnungszahl $\num{14}$ und die Massenzahl des am häufigsten auftretenden Isotops ist $\num{28}$.

  \begin{enumerate}
  \item Welcher Prozess trägt hauptsächlich zum Energieverlust nach der Bethe-Bloch-Gleichung bei? (1P)
  \solution{(1P)
  Schnell geladene Teilchen, die sich durch Materie bewegen, wechselwirken mit den Elektronen der Atome im Material (inelastischen Stößen mit Hüllenelektronen).
  Durch die Wechselwirkung werden die Atome angeregt oder ionisiert, was zu einem Energieverlust des sich bewegenden Teilchens führt.
  }

  \item Lässt sich die Bethe-Bloch-Gleichung auch für den Durchgang von Elektronen durch Materie verwenden? Begründen Sie Ihre Antwort. (1P)
  \solution{(1P)
  Für Elektronen ist der Energieverlust aus zwei Gründen anders:
  Die Ununterscheidbarkeit von den Hüllenelektronen und der starke Energieverlust durch Bremsstrahlung.
  }

  \item Bestimmen Sie den Energieverlust von Kaonen (\mbox{$m_{K^+}\approx\SI{500}{MeV/c^2}$}) mit einer kinetischen Energie von $\SI{25}{MeV}$ beim Durchdringen einer dünne Siliziumschicht von $\SI{1}{mm}$ Dicke.
  Der Energieverlust ist hierbei klein gegenüber der kinetischen Energie des Teilchens. (3P)
  \solution{(3P)
  $\Delta E<<E \Rightarrow \beta , \gamma \approx const:$
  \begin{align}
    \Delta E &\approx -\frac{\mathrm{d}E}{\mathrm{d}x} \Delta x = \SI{0.3}{\mega\electronvolt\,\centi\meter^2\per\gram} \cdot \frac{14}{28} \cdot \SI{2}{\gram\per\centi\meter\cubed} \cdot 5 \cdot \frac{1}{\beta^2} \cdot \SI{1}{mm}
    \intertext{Aus $E_{\text{kin}}=\frac{1}{2}mv^2=\frac{1}{2}m\beta^2\mathrm{c}^2$ folgt $\beta^2=\frac{2\,E_{\text{kin}}}{m\mathrm{c}^2}$ und somit}
    \Delta E &\approx \frac{3}{2} \cdot \frac{500}{2 \cdot 25} \cdot 0.1 \si{MeV} = \frac{3}{2} \si{MeV} = \SI{1.5}{MeV}
  \end{align}
  }

  \item Wie dick müsste die Siliziumschicht sein, damit die Kaonen komplett absorbiert werden? Beachten Sie, dass sich hier die kinetische Energie des Teilchens und damit seine Geschwindigkeit ändert. \textbf{Tipp:} Integrieren Sie den Kehrwert der Bethe-Bloch-Gleichung über die Energie. (3P) 
  \solution{(3P)
  Um die maximale Eindringtiefe $R$ und damit die Dicke zu bestimmen, muss über $\frac{dx}{dE}$ integriert werden.
  \begin{align}
    R&=-\int_{E_{\text{max}}}^0\frac{dx}{dE}dE=\frac{1}{\num{0.3}}\cdot\frac{\si{\gram}}{\si{\mega\electronvolt\centi\meter\squared}}\cdot\frac{28}{14}\cdot\frac{1}{\num{2}}\frac{\si{\centi\meter\cubed}}{\si{\gram}}\cdot\frac{1}{5}\int_{0}^{\SI{25}{MeV}}\beta^2dE
    \intertext{Aus $E_{\text{kin}}=\frac{1}{2}mv^2=\frac{1}{2}m\beta^2\mathrm{c}^2$ folgt $\beta^2=\frac{2\,E_{\text{kin}}}{m\mathrm{c}^2}$ und somit}
    R&=\frac{2}{3}\cdot\frac{2}{mc^2}\int_0^{\SI{25}{MeV}}E\,dE\cdot\frac{\si{\centi\meter}}{\si{\mega\electronvolt}}=\frac{2}{3}\cdot\frac{2}{500}\cdot\frac{1}{2}\cdot25^2\,\si{\centi\meter}=\frac{5}{6} \si{\milli\meter} \approx \SI{8.33}{\milli\meter}
  \end{align}
  }

  \item Skizzieren Sie die Bethe-Bloch-Gleichung in Abhängigkeit von $\beta\gamma$ und erklären Sie in diesem Zusammenhang den Begriff des minimal ionisierenden Teilchens (MIP). (2P)
  \solution{(2P)
  Ein minimal ionisierendes Teilchen (MIP) ist ein Teilchen, dessen Energieverlustrate durch Materie nahe dem Minimum der (Bethe-Bloch-Formel) liegt.
  }
  \end{enumerate}