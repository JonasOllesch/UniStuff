\section{Radioaktive Zerfälle (10P)}

In einem ursprünglich kein Blei enthaltendem Uran-Mineral entstehen die Blei-Isotope \ce{^{207}Pb} und \ce{^{206}Pb} durch den radioaktiven Zerfall der folgenden Uran-Isotope:
\begin{itemize}
          \item \ce{^{235}U} \qquad Häufigkeit heute: $0,72\%$, Halbwertszeit: $7,038\cdot10^8$ \unit{a}
          \item \ce{^{238}U} \qquad Häufigkeit heute: $99,28\%$, Halbwertszeit: $4,0468\cdot10^9$ \unit{a} 
\end{itemize}
Das Mineral sei $600$ Mio. Jahre alt.

\begin{enumerate}
\item Stellen Sie die Zerfallsgesetze der Uran-Isotope auf. (2P)

\item \ce{^{235}U} zerfällt über die natürliche Uran-Actinium-Reihe zu \ce{^{207}Pb}, \ce{^{238}U} über die natürliche Uran-Radium-Reihe zu \ce{^{206}Pb}. Erklären Sie kurz, was natürliche Zerfallsreihen sind. Stellen Sie weitergehend die Zerfallsgesetze der Blei-Isotope auf. (2P)

\item Welches Masseverhältnis
Blei zu Uran enthält das Mineral heute? (2P)

\item Wie groß ist das
Häufigkeitsverhältnis \ce{^{207}Pb} zu \ce{^{206}Pb}? (2P)
  \solution{
      Test solution
  }

\item Bestimmen Sie nun das Alter der Erde unter den Annahmen, dass an
ihrem Anfang \ce{^{235}U} und \ce{^{238}U} gleich häufig vorkamen und zum heutigen Zeitpunkt die für Uran angegebenen Häufigkeiten gelten. (2P)
\end{enumerate}