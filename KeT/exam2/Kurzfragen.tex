\section{Kurzfragen - (20P)}
\begin{enumerate}
\item Was sind die magischen Zahlen und wie verhält sich die Bindungsenergie in der Nähe der magischen Zahlen? (2P)

\vspace{3cm}\item Nennen Sie zwei Kenngrößen eines Beschleunigers. (2P)

\vspace{3cm}\item Welche Strahlungsart wird bei einem Übergang von $\ce{^{224}Ra} \rightarrow \ce{^{220}Rn}$ frei? Welcher Kernprozess liegt diesem Übergang zugrunde? (2P)

\vspace{3cm}\item Wodurch ist die maximale Energie, auf die ein Teilchen beschleunigt werden kann, bei einem Linearbeschleuniger limitiert? (2P)

\vspace{4cm}\item Welche zwei Eigenschaften muss ein Kandidat für ein Dunkle-Materie-Teilchen haben? (2P)

\vspace{3cm}\item Sind Flavour-Changing-Neutral-Currents (FCNC) im Standardmodell auf Tree-Niveau erlaubt? Begründen Sie Ihre Antwort. (2P)

\vspace{3cm}\item Vergleichen Sie die CKM- mit der PMNS-Matrix. Nennen Sie mindestens eine Gemeinsamkeit und einen Unterschied. (2P)

\vspace{3cm}\item Was ist eine additive Quantenzahl? Nennen Sie zwei Beispiele. (2P)


\vspace{3cm}\item Welche der folgenden Reaktionen sind im Standardmodell verboten, welche sind erlaubt?
Begründen Sie Ihre Entscheidungen. Geben Sie mögliche Wechselwirkungen
für die erlaubten Prozesse an. Nehmen Sie an, dass Neutrinos masselos sind.  
Zeichnen Sie die dominanten Feynman-Diagramme für die erlaubten Prozesse.
Geben Sie dabei alle intermediären Teilchen an. (4P)

\begin{itemize}
    \item $\mu^- \to e^- \gamma $
    \item $B_s \to \mu^+ \mu^- \gamma $
    \item $B_s \to \phi \phi$
    \item $n \to p  e^- \bar{\nu_e}$
\end{itemize}

\end{enumerate}

\null\newpage\clearpage