\section{Schwache Wechselwirkung (10P)}

In dieser Aufgabe untersuchen Sie Prozesse der schwachen Wechselwirkung.

\begin{enumerate}
    \item Erklären Sie was die V-A Struktur der schwachen Wechselwirkung beschreibt.

    \item Der Übergang $\Gamma_{i \to f}$ von einem Anfangszustand zu einem Endzustand kann durch Fermis Goldene Regel beschrieben werden.
    Geben Sie diese an und beschreiben Sie ihre Komponenten.
    \item Zeichen Sie das Feynman-Diagramm, welches den Zerfall des $\mu^-$ auf Tree-Level beschreibt.

    \item Wenden Sie die Feynman-Regel an, um das Matrixelement $M$ zu bestimmen.
    
    \textit{Hinweis: Eine explizite Berechnung von $M$ ist nicht notwendig.} 

    Die Zerfallsrate in Abhängigkeit der Elektronenenergie ist gegeben durch 
    \begin{equation}
        \dfrac{d \Gamma}{d E_e} = \left(\dfrac{g_\text{W}}{M_\text{W}c}\right)^4 \cdot \left(1 - \dfrac{4 E_e}{3 m_\mu c²}\right) \,.
    \end{equation}
    \item Bestimmen Sie die totale Zerfallsrate und geben sie eine Gleichung für die Lebenszeit eines $\mu^-$.
    \item Vergleichen Sie das Ergebnis mit dem Literaturwert. Erklären Sie mögliche Abweichungen.

    \item Welcher der beiden Zerfälle $\pi^- \to e^- \bar{\nu}_{e}$ und $\pi^- \to \mu^- \bar{\nu}_{\mu}$ tritt häufiger auf? Begründen Sie. (2P)

    \item Betrachten Sie den Prozess $e^+ e^- \to \mu^+ \mu^-$ auf Tree-Level. Bei welchen Energien tritt dieser Prozess besonders häufig auf? 

\end{enumerate}