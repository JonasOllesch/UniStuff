\section{Schwache Wechselwirkung (10P)}

\begin{enumerate}
    \item Erklären Sie was die V-A Struktur der schwachen Wechselwirkung beschreibt.
    \solution{Vektor - Aktialvektor bedeuted, dass die schwache WW nur an linkshändigen Teilchen koppelt. Mathematische ist $1 - \gamma^5$ linkshändige Prjektor, der dazu führt. dass Parität maximal gebrochen wird.}

    \item Der Übergang $\Gamma_{i \to f}$ von einem Anfangszustand zu einem Endzustand kann durch Fermis Goldene Regel beschrieben werden.
    Geben Sie diese an und beschreiben Sie ihre Komponenten.
    \solution{
        $\Gamma_{i \to f} = \dfrac{2\pi}{\hbar} \left| M_{f,i}\right|^2 \mathrm{d} \Pi_\text{LIPS}$ Das Matrixelement $M_{f,i}$ beschreibt die Amplitude des Übergang und  $d \Pi_\text{LIPS}$ beschreibt die Anzahl der möglichen Endzustände.
    }
    \item Zeichen Sie das Feynman-Diagramm, welches den Zerfall des $\mu^-$ auf Tree-Level beschreibt.
    \begin{solution}
        {\begin{tikzpicture}
            \begin{feynman}
              \vertex (i1) {\(\mu^-\)};
              \vertex[right=of i1] (w) [dot] {};
              \vertex[below right=of w] (f1) {\(e^-\)};
              \vertex[above right=of w] (f2) {\(\overline{\nu}_e\)};
              \vertex[above left=of w] (f3) {\(\nu_\mu\)};
              
              \diagram* {
                (i1) -- [fermion] (w),
                (w) -- [boson, edge label=\(W^-\)] (f2),
                (w) -- [fermion] (f1),
                (i1) -- [fermion] (f3),
              };
            \end{feynman}
          \end{tikzpicture}}
    \end{solution}
    \item Wenden Sie die Feynman-Regel an, um das Matrixelement $M$ zu bestimmen.
    
    \textit{Hinweis: Eine explizite Berechnung von $M$ ist nicht notwendig.} 

    Die Zerfallsrate in Abhängigkeit der Elektronenenergie ist gegeben durch 
    \begin{equation}
        \dfrac{d \Gamma}{\mathrm{d} E_\text{e}} = \left(\dfrac{g_\text{W}}{M_\text{W}c}\right)^4 \cdot \left(1 - \dfrac{4 E_e}{3 m_\mu c²}\right) \,.
    \end{equation}
    \item Bestimmen Sie die totale Zerfallsrate und geben sie eine Gleichung für die Lebenszeit eines $\mu^-$.
    \item Vergleichen Sie das Ergebnis mit dem Literaturwert. Erklären Sie mögliche Abweichungen.

    \item Welcher der beiden Zerfälle $\pi^- \to e^- \bar{\nu}_{\text{e}}$ und $\pi^- \to \mu^- \bar{\nu}_{\mu}$ tritt häufiger auf? Begründen Sie.

    \item Betrachten Sie den Prozess $e^+ e^- \to \mu^+ \mu^-$ auf Tree-Level. Bei welchen Energien tritt dieser Prozess besonders häufig auf? 

\end{enumerate}