\section{Durchführung}
\label{sec:Durchführung}
Mithilfe einer Rechteckschwingung wird der linke Schwingkreis angeregt, die Anregungsfrequenz wird dabei so lange variiert, bis der auf dem Oszilloskop dargestellte Spannungsverlauf maximal wird. Die dazu eingestellte
Frequenz ist die Resonanzfrequenz. Nun wird der rechte Schwingkreis angeschlossen, der mit einem variablen Widerstand ausgestattet ist. Er wird nun mit der Resonanzfrequenz des anderen Schwingkreises angeregt, der
Widerstand wird modifiziert, bis der dargestellte Spannungsverlauf erneut maximal wird. \\

Anschließend wird der linke Schwingkreis mit einer Frequenz von $f = 400 \, \unit{\hertz}$ angeregt, am Kopplungkondensator werden verschiedene Widerstände mit $2 \leq C_k \leq 12 \, \unit{\nano\farad}$ eingestellt und
es wird auf dem Bildschirm das Schwebungsverhalten beobachtet. Dazu wird die Anzahl der Schwebungsmaxima in einer Schwingungsperiode abgezählt und das Verhältnis ermittelt. \\

Die Rechteckspannung wird durch eine Sinusspannung ausgetauscht, mithilfe der Lissjous-Figuren wird nach den Fundamentalfrequenzen $ν^+$ und $ν^-$ gesucht. \\

Abschließend wird am Generator ein Frequenzfenster eingestellt, am Oszilloskop wird in Abhängigkeit von $C_k$ der zeitliche Abstand der Maxima abgelesen.