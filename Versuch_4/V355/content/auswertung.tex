\section{Auswertung}
\label{sec:Auswertung}


Zunächst wurde die Resonanzfrequenz bestimmt. Der auf dem Oszilloskop dargestellte Spannungsverlauf wurde dabei bei
einer Frequenz von 
\begin{equation*}
    ν = 35,78 \, \unit{\mega\hertz}
\end{equation*} maximal. Bei dieser Frequenz handelt es sich also um die Resonanzfrequenz.

Bei den folgenden Messungen wurden, mit einem Fehler von $3\%$ die Kopplungskapazitäten $12 \,\unit{\nano\farad};\, 9.99 \,\unit{\nano\farad};\, 
8.18 \,\unit{\nano\farad};\, 6.86 \,\unit{\nano\farad};\, 4.74 \,\unit{\nano\farad};\, 2.86 \,\unit{\nano\farad};\, 2.19 \,\unit{\nano\farad};\,$
verwendet. 

\subsection{a) Bestimmung des Verhältnisses von Schwingungs und Schwebungsfrequenz}
\label{subsec:a}

In \autoref{tab:SchwebSchwingVerh} ist die Anzahl $n$ der Schwingungsmaxima pro Schwebungsperiode bei 
verschiedenen Kopplungskapazitäten sowie der dazugehörige Theoriewert dargestellt.
Die Theoriewerte der Frequenzen berechnen sich dabei mit $ν^+ = \frac{1}{2π}ω$ und $ν^-=\frac{1}{2π}ω$ aus
\eqref{eq:w_+} und \eqref{eq:w_-}, die Frequenzverhältnisse ergeben sich dann aus
\begin{equation*}
    n_t= \frac{ν^+_t + ν^-_t}{2(ν^-_t - ν^+_t)} \,.
\end{equation*}

\begin{table}[H]
    \centering
    \sisetup{table-format=2.0}
    \begin{tabular}{S[table-format=1.3] S S[table-format= 2.2]}
        \toprule
        {$C_k \mathbin{/} \unit{\nano\farad}$} & {$n$} & {$n_t$}\\
        \midrule
        12         &         16  & {$16.71 \pm 0.81$} \\
        9.99       &         14  & {$14.09 \pm 0.68$} \\
        8.18       &         12  & {$11.72 \pm 0.56$} \\
        6.86       &         10  & {$9.996 \pm 0.47$} \\  
        4.74       &          8  & {$7.220 \pm 0.33$} \\
        2.86       &          4  & {$4.749 \pm 0.20$} \\
        2.19       &          3  & {$3.863 \pm 0.15$} \\
        0.997      &          1  & {$2.261 \pm 0.08$} \\
        \bottomrule
    \end{tabular}
    \caption{Verhältnis der Schwingung- und Schwebungsmaxima bei unterschiedlichen Kopplungswiderständen.}
    \label{tab:SchwebSchwingVerh}
\end{table}

Dabei sei ein Messfehler von $\pm 1$ anzunehmen. 

\subsection{c) Frequenzabhängigkeit des Stromverlaufes mit Wobbelgenerator}

Um den Stromverlauf in Frequenzabhängigkeit zu bestimmen, wurde der zeitliche Abstand der Spannungsmaxima auf dem Oszilloskop
gemessen. Die aufgenommenen Messdaten sind in \autoref{tab:Maximaabstand} aufgeführt. Dabei ist $t_1$ der Abstand des kleinen
zum ersten großen und $t_2$ der Abstand des kleinen zum zweiten großen Peak. Mit einer Zeitskalierung von $1 \unit{\milli\second}$ entspricht also ein Kästchen einer Millisekunde.
Aus den Kehrwerten dieser über die Skalierung des Oszilloskops gemessenen Zeiten lässt sich die Frequenz $f_1$ bzw. $f_2$ bestimmen.

\begin{table}[H]
    \centering
    \sisetup{table-format=1.0}
    \begin{tabular}{S[table-format=1.3] S S[table-format=3.2] S S[table-format=3.2]}
        \toprule
        {$C_k \mathbin{/} \unit{\nano\farad}$} & {$t_1 \mathbin{/} \unit{\milli\second}$} & {$f_1 \mathbin{/} \unit{\hertz}$} 
        & {$t_2 \mathbin{/} \unit{\milli\second}$} & {$f_2 \mathbin{/} \unit{\hertz}$}\\
        \midrule
        12      &    3.4     & 294.12 & 4.2 & 238.10 \\
        9.99    &    3.5     & 285.71 & 4.2 & 238.10 \\
        8.18    &    3.6     & 277.78 & 4.2 & 238.10 \\
        6.86    &    3.5     & 285.71 & 4.2 & 238.10 \\  
        4.74    &    3.6     & 277.78 & 4.6 & 217.39 \\
        2.86    &    3.5     & 285.71 & 5.2 & 192.31 \\
        2.19    &    3.5     & 285.71 & 5.6 & 178.57 \\
        0.997   &    3.5     & 285.71 & 5.8 & 172.41 \\
        \bottomrule
    \end{tabular}
    \caption{Zeitliche Abstände des kleinen Peak zu den beiden höheren Peaks.}
    \label{tab:Maximaabstand}
\end{table}



\subsection{b) Bestimmung der Eigenfrequenz}

Die theoretischen Werte für $\nu_+ $ und $\nu_-$ wurden mit den Geleichungen \eqref{eq:w_+} und \eqref{eq:w_-}
berechneten. Die Kreisfrquenzen werden in die Fundamentalfrequenzen umgerechnet.
Für $\nu_-$ gilt 
\begin{equation*}
    \nu_-     = \dfrac{1}{2\pi \sqrt{L \left(\dfrac{1}{C}+\dfrac{2}{C_k}\right)^{-1}}}\,
\end{equation*}

und für $\nu_+$ 
\begin{equation*}
    \nu_+     =\dfrac{1}{2\pi \sqrt{LC}}\,.
\end{equation*}
Für unsere Messwerte ergibt sich die folgende Tabelle.
{$\dfrac{\text{d}T_1}{\text{d}t}$}
\begin{table}[H]
    \centering
    \begin{tabular}{S S S S S S S}
      \toprule
        {$C_k \mathbin{/} \unit{\nano\farad}$} & {$\nu^-_t \mathbin{/} \unit{\kilo\hertz}$} & {$\nu^-_e \mathbin{/} \unit{\kilo\hertz}$} & {a in \%} & {$\nu^+_t \mathbin{/} \unit{\kilo\hertz}$} & {$\nu^+_e \mathbin{/} \unit{\kilo\hertz}$} & {a in \%}\\
      \midrule
         12       & 38.1 &     38.1    &  0.0 & {}      &{}   &{}   \\ 
         9,99     & 38.59&     38.5    &  0.02& {}      &{}   &{}   \\
         8,18     & 36.63&     39.1    &  0.06& {}      &{}   &{}   \\ 
         6,86     & 36.79&     39.7    &  0.07& {}      &{}   &{}   \\ 
         4,74     & 37.34&     41.2    &  0.09& {}      &{}   &{}   \\ 
         2,86     & 38.12&     44.3    &  0.14& {}      &{}   &{}   \\ 
         2,19     & 38.71&     46.5    &  0.17& {}      &{}   &{}   \\ 
         0,997    & 41.41&     56.3    &  0.26& {}      &{}   &{}   \\ 


      \bottomrule
    \end{tabular}
    \caption{Die theoretischen und experimentellen Eigenfrequenzen des Schwingkreises.}
  \end{table}