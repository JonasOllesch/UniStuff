\section{Auswertung}
\label{sec:Auswertung}
Zunächst wurde die Resonanzfrequenz bestimmt. Der auf dem Oszilloskop dargestellte Spannungsverlauf wurde dabei bei
einer Frequenz von 
\begin{equation*}
    f = 35,78 \unit{\mega\hertz}
\end{equation*} maximal.

\subsection{a) Bestimmung des Verhältnisses von Schwingungs und Schwebungsfrequenz}
\label{subsec:c}

In \autoref{tab:SchwebSchwingVerh} ist die Anzahl $n$ der Schwingungsmaxima pro Schwebungsperiode dargestellt.

\begin{table}
    \centering
    \sisetup{table-format=2.0}
    \begin{tabular}{S[table-format=1.3] S}
        \toprule
        {$C_k \mathbin{/} \unit{\nano\farad}$} & {$n$} \\
        \midrule
        12         &         16 \\
        9.99       &         14 \\
        8.18       &         12 \\
        6.86       &         10 \\  
        4.74       &         8  \\
        2.86       &         4  \\
        2.19       &         3  \\
        0.997      &         1  \\
        \bottomrule
    \end{tabular}
    \caption{Verhältnis der Schwingung- und Schwebungsmaxima bei unterschiedlichen Kopllungswiderständen.}
    \label{tab:SchwebSchwingVerh}
\end{table}