\section{Auswertung}
\label{sec:Auswertung}

\subsection{Bestimmung der Eigenfrequenz}

Die theoretischen Werte für $\nu_+ $ und $\nu_-$ wurden mit den Geleichungen \eqref{eq:w_+} und \eqref{eq:w_-}
berechneten. Die Kreisfrquenzen werden in die Fundamentalfrequenzen umgerechnet.
Für $\nu_-$ gilt 
\begin{equation*}
    \nu_-     = \dfrac{1}{2\pi \sqrt{L \left(\dfrac{1}{C}+\dfrac{2}{C_k}\right)^{-1}}}\,
\end{equation*}

und für $\nu_+$ 
\begin{equation*}
    \nu_+     =\dfrac{1}{2\pi \sqrt{LC}}\,.
\end{equation*}
Für unsere Messwerte ergibt sich die folgende Tabelle.
{$\dfrac{\text{d}T_1}{\text{d}t}$}
\begin{table}[H]
    \centering
    \label{tab:Diffquo}
    \begin{tabular}{S S S S S S S}
      \toprule
      & {$\dfrac{\C_k}{\nano\farad}$} & {$\dfrac{\nu^⁻_t}{\kilo\hertz$} & {$\dfrac{\nu^-_e}{\kilo\hertz$}\\
      \midrule
      


      \bottomrule
    \end{tabular}
    \caption{Die theoretischen und experimentellen Eigenfrequenz des Schwingkreises}
  \end{table}