\section{Diskussion}
\label{sec:Diskussion}

Die gemessenen Werte unterlagen trotz teilweiser geringer Abweichung zu den aus der Theorie erwarteten Werten einigen Fehlern.
So war es bei der ersten Messung teilweise nicht eindeutig wo eine Schwebungsperiode endete, da sich die Schwingungen und die Schwebungen überlagern.
Auch ist es möglich, dass der variable Kondensator verstellt wurde, weil dieser nach der ersten Einstellung nicht nachjustiert wurde.
Während allen Messungen flackerte der Schirm der Oszilloskos, sodass das feine Ablesen nicht möglich war.

% Machst du  noch die für a) und b) '^^ ?? natürlich :)

% Zu c)

Das größte Problem bei den Ergebnissen aus c) lag in der Beschädigung des Stellrades der Zeitachsenskalierung. Dieses konnte auch über die eigentlich eingetragenen Skalierungen hinaus gedreht werden, sodass es nicht
möglich war, definitiv festzustellen, welche Achsenskalierung tatsächlich aktiv war. Auch die weiß nachgezeichneten Pfeile waren soweit verblasst, dass nicht abschätzbar war, wie stark die Achse skaliert wurde. Zwar
wurde versucht, durch Abzählen der Klicks, die beim Drehen des Stellrades entstehen und der Einteilung der Skala festzustellen, welche Skalierung genutzt wurde, dabei wurde aber wahrscheinlich ein Fehler gemacht.
Des Weiteren konnte das Bild auf dem Oszilloskop nicht zum Stehen gebracht werden, was das ohnehin schon ungenaue Abzählen der Kästchen zur Frequenzermittlung weiter erschwerte.
Wird nämlich $5 \, \unit{\micro\second}$ anstatt $1 \unit{\milli\second}$ gewählt, nähern sich die Werte, zumindest für niedrige Widerstände, stark der Resonanzfrequenz an.
Diese "korrigierten" Werte seien hier in \autoref{tab:korrekiwerti} dargestellt.

\begin{table}[H]
    \centering
    \sisetup{table-format=1.0}
    \begin{tabular}{S[table-format=1.3] S S[table-format=5.2] S S[table-format=5.2]}
        \toprule
        {$C_k \mathbin{/} \unit{\nano\farad}$} & {$t_1 \mathbin{/} \unit{\second}$} & {$f_1 \mathbin{/} \unit{\hertz}$} 
        & {$t_2 \mathbin{/} \unit{\second}$} & {$f_2 \mathbin{/} \unit{\hertz}$}\\
        \midrule
        12      &    {$1,7 * 10^{-5}$}     & 58823.53 & {$2,1 * 10^{-5}$} & 47619.05 \\
        9.99    &    {$1,75 * 10^{-5}$}    & 57142.86 & {$2,1 * 10^{-5}$} & 47619.05 \\
        8.18    &    {$1,8 * 10^{-5}$}     & 55555.56 & {$2,1 * 10^{-5}$} & 47619.05 \\
        6.86    &    {$1,75 * 10^{-5}$}    & 57142.86 & {$2,1 * 10^{-5}$} & 47619.05 \\  
        4.74    &    {$1,8 * 10^{-5}$}     & 55555.56 & {$2,3 * 10^{-5}$} & 43478.26 \\
        2.86    &    {$1,75 * 10^{-5}$}    & 57142.86 & {$2,6 * 10^{-5}$} & 38461.54 \\
        2.19    &    {$1,75 * 10^{-5}$}    & 57142.86 & {$2,8 * 10^{-5}$} & 35714.29 \\
        0.997   &    {$1,75 * 10^{-5}$}    & 57142.86 & {$2,9 * 10^{-5}$} & 34482.76 \\
        \bottomrule
    \end{tabular}
    \caption{Korrigierte zeitliche Abstände des kleinen Peak zu den beiden höheren Peaks sowie die dazugehörigen Frequenzen.}
    \label{tab:korrekiwerti}
\end{table}