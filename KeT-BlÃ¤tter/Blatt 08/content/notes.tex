\section{Notes}
\label{sec:notes}

\subsection{Total Antisymmetry of the VVV-Coupling in the Z-Penguin Lagrangian}

We want to see why the Z-Penguin paper writes the VVV-term from the 3-point Lagrangian $\mathcal{L}_3$ as
\begin{align}
    \mathcal{L} \supset \frac{i}{6} \sum{v_1 v_2 v_3} 
    g_{v_1v_2v_3}^{abc} \left( V_{v_1,\mu}^a V_{v_2,\nu}^b
    \, \partial_{\vphantom{v_3}}^{[\mu} V_{v_3}^{c,\nu]} + V_{v_3,\mu}^c V_{v_1,\nu}^a  
    \, \partial_{\vphantom{v_2}}^{[\mu} V_{v_2}^{b,\nu]} + V_{v_2,\mu}^b
    V_{v_3,\nu}^c \, \partial_{\vphantom{v_1}}^{[\mu} V_{v_1}^{a,\nu]}
    \right) 
\end{align}
and whether the coupling $c_{v_1v_2v_3}$ in the most-general VVV-vertex (summation implied)
\begin{equation}
    \mathcal{L}_3 = c_{v_i v_j v_k} (\partial^{\mu} V_{v_i}^{\nu}) V_{v_j, \mu} V_{v_k, \nu}
\end{equation}
is the same as the coupling used in the paper.

For that consider that we can always separate our derivative term into a fully symmetric and fully antisymmetric part so that
\begin{align}
    \partial^{\mu} V_{v_i}^{\nu} = \frac{1}{2} \left(\partial_{\vphantom{v_i}}^{[\mu} V_{v_i}^{c,\nu]} + \partial_{\vphantom{v_i}}^{\{ \mu} V_{v_i}^{c,\nu \}} \right)
\end{align}
with the curly brackets denoting the symmetrisation 
\begin{equation}
    \partial_{\vphantom{v_i}}^{\{ \mu} V_{v_i}^{c,\nu \}} = \partial^\mu V_{v_i}^{c,\nu} + \partial^\nu V_{v_i}^{c,\mu}
\end{equation}
and the square brackets the antisymmetrisation
\begin{equation}
    \partial_{\vphantom{v_i}}^{[\mu} V_{v_i}^{c,\nu]} = \partial^\mu V_{v_i}^{c,\nu} - \partial^\nu V_{v_i}^{c,\mu} \,.
\end{equation}

We also know that for unitarity reasons, \cite{unitarity}, the coupling $c_{v_i v_j v_k}$ must be totally antisymmetric in the indices $v_i, v_j, v_k$.
This however means that in the product with the coupling, the totally symmetric part cannot contribute, which leads to
\begin{align}
    \mathcal{L}_3 = \frac{1}{2}c_{v_i v_j v_k} \partial_{\vphantom{v_i}}^{[\mu} V_{v_i}^{c,\nu]} V_{v_j, \mu} V_{v_k, \nu} \,.
\end{align}

Nothing is stopping us from just writing the right hand term thrice and compensating it by dividing by 3.
Then, by renaming and swapping indices and using the total antisymmetry of $c_{v_i v_j v_k}$, we get the form used in \cite{penguin}.
It should look something like this
\begin{align*}
    \mathcal{L}_3 &= \frac{1}{2}c_{v_i v_j v_k} \partial_{\vphantom{v_i}}^{[\mu} V_{v_i}^{c,\nu]} V_{v_j, \mu} V_{v_k, \nu} \\
                  &= \frac{1}{6}c_{v_i v_j v_k} \left( \partial_{\vphantom{v_i}}^{[\mu} V_{v_i}^{c,\nu]} V_{v_j, \mu} V_{v_k, \nu} 
                    + \partial_{\vphantom{v_i}}^{[\mu} V_{v_i}^{c,\nu]} V_{v_j, \mu} V_{v_k, \nu} 
                    + \partial_{\vphantom{v_i}}^{[\mu} V_{v_i}^{c,\nu]} V_{v_j, \mu} V_{v_k, \nu} \right) \\
                  &= \frac{1}{6} \left( c_{v_i v_j v_k} \partial_{\vphantom{v_i}}^{[\mu} V_{v_i}^{c,\nu]} V_{v_j, \mu} V_{v_k, \nu} 
                    + c_{v_j v_i v_k} \partial_{\vphantom{v_j}}^{[\mu} V_{v_j}^{c,\nu]} V_{v_i, \mu} V_{v_k, \nu} 
                    + c_{v_k v_j v_i} \partial_{\vphantom{v_k}}^{[\mu} V_{v_k}^{c,\nu]} V_{v_j, \mu} V_{v_i, \nu} \right) \\
                  &= \frac{1}{6} \left( c_{v_i v_j v_k} \partial_{\vphantom{v_i}}^{[\mu} V_{v_i}^{c,\nu]} V_{v_k, \mu} V_{v_j, \nu} 
                    + c_{v_i v_j v_k} \partial_{\vphantom{v_j}}^{[\mu} V_{v_j}^{c,\nu]} V_{v_k, \mu} V_{v_i, \nu} 
                    + c_{v_i v_j v_k} \partial_{\vphantom{v_k}}^{[\mu} V_{v_k}^{c,\nu]} V_{v_i, \mu} V_{v_j, \nu} \right) \,, \\
\end{align*}
where in the last step we not only rearranged the $i,j,k$ inside the $c_{v_i v_j v_k}$, but also renamed the indices in the last two terms 
and swapped an additional time. This is exactly the form used in the paper, albeit in reversed order.
Our coupling $c_{v_i v_j v_k}$ is thus the same up to a factor of $i$, which we very well just might have forgotten.


\subsection{Complex Vector Bosons}

While it is stated in \cite{penguin} that they deal with only real vector bosons in their Lagrangian, the later used vector boson masses suggest otherwise,
instead implying that we are in the mass basis, in which it is well known that at least our charged standard model vector bosons are complex with
\begin{equation}
    W^{\pm} = \frac{1}{\sqrt{2}}(W_1 \mp i W_2) \,.
\end{equation}
This confused us.
We decided to have another closer look, this time taking the standard model Lagrangian, more explicitly all terms that include an interaction of the form
\begin{equation}
    \partial_{\vphantom{v_i}}^{[\mu} V_{v_i}^{c,\nu]} V_{v_j, \mu} V_{v_k, \nu} \,, \qquad v_i, v_j, v_k \in {W^{\pm}, Z\,, A} \,.
\end{equation}
The desired terms of the (symmetry-broken) standard model Lagrangian are as follows:
\begin{equation}\begin{aligned}
    \mathcal{L}_\text{SM} &\supset -\frac14\left|\partial_\mu A_\nu-\partial_\nu A_\mu-\mathrm{i}e(W_\mu^-W_\nu^+-W_\nu^-W_\mu^+)\right|^2 \\
    &-\frac14\left|\partial_\mu Z_\nu-\partial_\nu Z_\mu+\mathrm{i}e\frac{c_\mathrm{w}}{s_\mathrm{w}}(W_\mu^-W_\nu^+-W_\nu^-W_\mu^+)\right|^2 \\
    &-\frac12\left|\partial_\mu W_\nu^+-\partial_\nu W_\mu^+-\mathrm{i}e(W_\mu^+A_\nu-W_\nu^+A_\mu) +\mathrm{i}e\frac{c_\mathrm{w}}{s_\mathrm{w}}(W_\mu^+Z_\nu-W_\nu^+Z_\mu)\right|^2
\end{aligned}
\label{eq:SMVVV}
\end{equation}
To see now how the standard model couplings correspond to the coupling $g_{VVV}$ in \cite{penguin}, we simply need to expand the terms in \eqref{eq:SMVVV}
and collect the terms.
This would have been a lot easier to do in Mathematica, but someone decided to do it by hand, yielding the following result:
\begin{align*}
    \mathcal{L}_\text{SM}^{VVV} = &- i \, \text{e} \, \partial^{[ \mu} \, A^{\nu ]} \, W_\mu^- \, W_\nu^+ + i \, \text{e} \frac{c_\text{W}}{s_\text{W}} \, \partial^{[ \mu} \, Z^{\nu ]} \, W_\mu^- \, W_\nu^+ \\
                                  &+ i \, \text{e} \, \partial^{[ \mu} \, W^{+ \nu]} \, A_\mu \, W_\nu^-  - i \, \text{e} \frac{c_\text{W}}{s_\text{W}} \, \partial^{[ \mu} \, W^{+ \nu]} \, Z_\mu \, W_\nu^-  \\
                                  &- i \, \text{e} \, \partial^{[ \mu} \, W^{- \nu]} \, A_\mu \, W_\nu^+  + i \, \text{e} \frac{c_\text{W}}{s_\text{W}} \, \partial^{[ \mu} \, W^{- \nu]} \, Z_\mu \, W_\nu^+  \\
\end{align*}
We already see that, if we decide to not absorb the $i$ into it, the couplings are real.
Now, our hope is to compare this to the terms in the Z-Penguin paper and find the exact same structure.
So, as before, I senselessly expanded the sum in the paper for our standard model case of vector bosons and collected the terms.
This allows us to write the exact same terms as in our standard model Lagrangian, but in terms of the coupling $g_{VVV}$.
To get a grasp on what exactly these terms look like, here is one of them:
\begin{align*}
    \mathcal{L}^3_\text{Z-Peng} \supset &\frac{i}{6} \left(g_{A W^+ W^-} - g_{W^+ A W^-} + g_{W^+ W^- A} - g_{W^- W^+ A} + g_{W^- A W^+} - g_{A W^- W^+} \right) \\
                                        &A_\mu W_\nu^+ \partial^{[ \mu} W^{- \nu ]}
\end{align*}

Every single of the resulting terms shares the same factor in the form of the alternating sum of the permutations of the couplings. pümmün :)
What we can also see that, for a totally antisymmetric coupling
\begin{equation}
    g_{V_i V_j V_k} = - g_{P \{V_i V_j V_k \}(1)} \,,
\end{equation} 
the terms in the Z-Penguin paper are exactly the same as in the standard model Lagrangian.
And while this does not prove that the coupling NEEDS to be totally antisymmetric, it is nice to see that for the case of a totally antisymmetric coupling, everything works out perfectly,
with
\begin{align}
    g_{A W^+ W^-} &= - \text{e} \\
    g_{Z W^+ W^-} &= - \text{e} \frac{c_\text{W}}{s_\text{W}}
\end{align}