\section{Exercise 29 - \textit{Sample Variance}}
\label{sec:ex29}

Assume $x_1, ..., x_\text{n}$ as square-integrable, pairwise uncorrelated, real-valued random variables $X_1, ..., X_\text{n}$ with variance $\sigma^2$ and mean $\mu$.

\subsection*{(a)}

First, let's have a look at the arithmetic mean
\begin{equation*}
    \bar{X} = \dfrac{1}{n} \sum_{i=1}^n X_i \,.
\end{equation*}
and find out whether it is an unbiased estimator of the mean $\mu$. \\

To do that, we will have to rewrite the expression for $\bar{X}$. \\
Since $\frac{1}{n}$ is just a constant, it can just be pulled past the expected value as seen below:

\begin{align*}
    \langle \bar{X} \rangle &=  E \left(\dfrac{1}{n} \sum_{i=1}^n X_i \right) \\
                            &= \dfrac{1}{n} E \left(\sum_{i=1}^n X_i \right) \,.
\end{align*}

Since it doesn't matter whether the expression is summed first and the mean taken afterward, or the other way around, the arithmetic mean becomes

\begin{align*}
    \langle \bar{X} \rangle &= \dfrac{1}{n} \sum_{i=1}^n E(X_i) \\
                            &= \dfrac{1}{n} \sum_{i=1}^n \mu  \\
                            &= \dfrac{1}{n} \, n \, \mu \\
                            &= \mu = \langle \mu \rangle \,.
\end{align*}

So, the arithmetic mean is indeed unbiased.

\subsection*{(b)}

Here, the most import rule for calculating with variances is
\begin{equation*}
    \text{Var}(aX) = a^2 \text{Var}(X) \,.
\end{equation*}

With that, the expression for the variance can be simplified as follows:
\begin{equation*}
    \text{Var}(\bar{X}) = \text{Var} \left(\dfrac{1}{n} \sum_{i=1}^n X_i \right) = \dfrac{1}{n^2} \text{Var} \left(\sum_{i=1}^n X_i \right)
\end{equation*}
Similarly to task (a),
\begin{align*}
    \text{Var}(\bar{X}) &= \dfrac{1}{n^2} \text{Var} \left(\sum_{i=1}^n X_i \right) \\
                        &= \dfrac{1}{n^2} \sum_{i=1}^n \text{Var}(X_i) \\
                        &= \dfrac{1}{n^2} n \sigma^2 \\
                        &= \dfrac{1}{n} \sigma^2 \,,
\end{align*}
which is what was to be shown.

\subsection*{(c)}

With the help of (b), task (c) is solved below:

\begin{align*}
    \langle S_0^2 \rangle &= E \left( \dfrac{1}{n} \sum_{i=1}^n (X_i - \mu)^2 \right) \\
                          &= \dfrac{1}{n} \sum_{i=1}^n E((X_i - \mu)^2) && \text{||} \, \text{by definition:} \quad E((X_i - \mu)^2) = \text{Var}(X_i) \\
                          &= \dfrac{1}{n} \sum_{i=1}^n \text{Var}(X_i) \\
                          &= \dfrac{1}{n}  \, \sigma^2 \,.
\end{align*}

So, the estimator $S_0^2$ is biased, but can be corrected by adding a factor $n$ so that
\begin{equation*}
    \langle \tilde{S}_0^2 \rangle = n \langle S_0^2 \rangle = \sigma^2 \,.
\end{equation*}

\subsection*{(d)}

\begin{align*}
    \langle \sigma^2 \rangle    &= \left\langle \dfrac{1}{n} \sum_{i=1}^n (X_i - \bar{X})^2 \right\rangle \\
                                &= \dfrac{1}{n} \langle  \sum_{i=1}^n (X_i - \mu + \mu \, - \bar{X})^2 \rangle \\ 
                                &= \dfrac{1}{n} \langle  \sum_{i=1}^n (X_i - \mu)^2 -2(X_i -\mu)(\bar{X} - \mu) + (\bar{X} - \mu)^2 \rangle \\
                                &= \dfrac{1}{n} \langle  \sum_{i=1}^n ( X_i - \mu)^2 -2 \sum_{i=1}^n (X_i -\mu)(\bar{X} - \mu) + \sum_{i=1}^n(\bar{X} - \mu)^2 \rangle  \\
                                &= \dfrac{1}{n} \langle  \sum_{i=1}^n ( X_i - \mu)^2 -2n(\bar{X} -\mu)(\bar{X} - \mu) + n(\bar{X} - \mu)^2 \rangle  \\
                                &= \dfrac{1}{n} \langle  \sum_{i=1}^n ( X_i - \mu)^2 -n(\bar{X} - \mu)^2 \rangle  \\
                                &= \dfrac{1}{n} \left( \sum_{i=1}^n \langle X_i - \mu \rangle^2 -n\langle\bar{X} -\mu \rangle^2 \right) \\
                                &= \dfrac{1}{n} \left( \text{Var}(X_i) -n \text{Var}(\bar{X})\right)  \\
                                &= \sigma^2 - \dfrac{\sigma^2}{n}   \\
                                &= \dfrac{n-1}{n} \sigma^2                               
\end{align*}

As we can clearly see, the estimator is indeed biased. To correct it, we just need to add a factor of $\frac{1}{n-1}$ to the definition of the standard deviation.

    