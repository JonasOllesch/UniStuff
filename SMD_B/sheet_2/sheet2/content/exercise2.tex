\section{Task 2 - \textit{Lab Experiment}}

\subsection*{(a)}

The design matrix is a $"\text{simple}"$ $12$ by $2$ matrix of the form
\begin{equation*}
    \textbf{A} = \left(
        \begin{array}{c c}
            f_1(\psi_1) & f_2(\psi_1) \\
            \vdots & \vdots \\
            f_1(\psi_{12}) & f_2(\psi_{12}) \\
        \end{array}
    \right) \,.
\end{equation*}

Using the table and ansatz 
\begin{equation*}
    f(\psi) = a_1 f_1(\psi) + a_2 f_2(\psi)
\end{equation*}
from the exercise, where
\begin{align*}
    f_1(\psi) &= \cos (\psi) \\
    f_2(\psi) &= \sin (\psi) \,,
\end{align*}
we can just calculate it as shown in \autoref{list:designmat} and get
\begin{equation*}
    \textbf{A} = \left(
        \begin{array}{c c}
             1                    &  0  \\
             0.866                &  0.5 \\
             0.5                  &  0.866 \\
             6.123 \cdot 10^{-17} &  1 \\
            -0.5                  &  0.866 \\
            -0.866                &  0.5 \\
            -1                    &  1.225 \cdot 10^{-16} \\
            -0.866                & -0.5 \\
            -0.5                  & -0.866 \\
            -1.837 \cdot 10^{-16} & -1 \\
            0.5                   & -0.866 \\
            0.866                 & -0.5 \\
        \end{array}
    \right)
\end{equation*}

\begin{lstlisting}[language = Python, caption={Calculation of design matrix \textbf{A}. The file \texttt{Daten.txt} holds the table data from the exercise sheet.}, label = {list:designmat}]
    Data = np.array(np.genfromtxt('Daten.txt'))
    A = np.zeros(shape=(12,2))
    A[:,0]= np.cos(Data[:,0]*(2*np.pi/360))
    A[:,1]= np.sin(Data[:,0]*(2*np.pi/360))
\end{lstlisting}

It should be noted that the three values with the lowest exponents are in reality just zero and will be treated as such. \\

To calculate the solution vector $\textbf{â}$, we use the definition from the lecture, namely
\begin{equation}
    \textbf{â} = (\textbf{A}^T\textbf{A})^{-1}\textbf{A}^T \textbf{y} \,,
    \label{eq:solveca}
\end{equation}
where $\textbf{y}$ is just the vector of the asymmetry values. \\

\eqref{eq:solveca} can be implemented into code as seen in \autoref{list:solveca}


\begin{lstlisting}[language = Python, caption={Calculation of solution vector \textbf{â}.}, label = {list:solveca}]
    A_T= A.transpose()
    tmp1 = np.matmul(A_T,A)
    tmp2 = np.linalg.inv(tmp1)
    tmp3 = np.matmul(tmp2,A_T)
    a = np.matmul(tmp3,Data[:,1])
\end{lstlisting}
and yields
\begin{equation*}
    \hat{\textbf{a}} = \left(\begin{array}{c}
        a_1 \\  
        a_2
    \end{array} \right) 
    = \left(\begin{array}{c}
        -0.0375 \\  
        0.0773
    \end{array} \right)
\end{equation*}

If one were to plot those values for $a_1$ and $a_2$ (see Max's, Jonas M.'s and Stefan's solution for that :D), one would see that they fit the data quite nicely, but since it is not part of the task, we will not
do it here.

\subsection*{(b)}

When trying to calculate the covariance matrix $\textbf{V}[\hat{\textbf{a}}]$ using
\begin{equation}
    \textbf{V}[\hat{\textbf{a}}] = (\textbf{A}^T\textbf{A})^{-1}\textbf{V}[\textbf{y}] \textbf{A} (\textbf{A}^T\textbf{A})^{-1} \,,
    \label{eq:Vara}
\end{equation}
one should first have a look at $\textbf{V}[\textbf{y}]$. \\
In case that all different values of $\textbf{y}$ are uncorrelated, as they are here, \eqref{eq:Vara} simplifies a lot and becomes
\begin{equation*}
    \textbf{V}[\hat{\textbf{a}}] = \sigma^2 (\textbf{A}^T \textbf{A})^{-1} \,.
\end{equation*}
Using the standard deviation of $\sigma^2 = 0.011$ for the values of $\textbf{y}$, the covariance matrix is
\begin{equation*}
    \left(
        \begin{array}{c c}
        2.01666667 \cdot 10^{-5}  & -1.67921232 \cdot 10^{-21} \\
        -1.67921232 \cdot 10^{-21} &  2.01666667 \cdot 10^{-5} \\
        \end{array}
    \right)
\end{equation*}
Similarly to the design matrix, the elements on the off-diagonal are negligibly small. \\
Now, the errors of $a_1$ and $a_2$ are just the square root of the main diagonal entries, giving

\begin{equation*}
    \hat{\textbf{a}}
    = \left(\begin{array}{c}
        -0.0375 \pm 0.0045 \\
         0.0774 \pm 0.0045
    \end{array} \right)
\end{equation*}
with a correlation coefficient of zero, since $a_1$ and $a_2$ are uncorrelated.

\subsection*{(d)}

To calculate $A_0$ and $\delta$, the two ansätze from (a) are set equal.
By using
\begin{equation*}
    \cos(\alpha + \beta) = \cos\alpha \cos\beta - \sin\alpha \sin\beta
\end{equation*}
and comparing coefficients of the resulting equation, which will be omitted here, we get
\begin{equation*}
    \delta = -\arctan\left(\dfrac{a_2}{a_1}\right) = 64.1462 \pm 2.9926
\end{equation*}
and
\begin{equation*}
    A_0 = \dfrac{a_1}{\cos(\delta)} =\dfrac{a_1}{\cos \left(\arctan \left(\dfrac{a_2}{a_1}\right)\right)} = -0.0860 \pm 0.0045 \,.
\end{equation*}

If done by hand, the errors would be calculated using gaussian error propagation, here, numpy's uncertainties package does the work (see \autoref{list:errors} for comparison).

\begin{lstlisting}[language = Python, caption={Calculation of errors of $a_1, a_2, A_0$ and $\delta$.}, label = {list:errors}]
    def calc_A_0 (a1,a2):
    return  a1/(unumpy.cos(unumpy.arctan(a2/a1)))

    def calc_delta(a1,a2):
    return (-1)*unumpy.arctan(a2/a1)
    
    Err_a1 = np.sqrt(Var_a[0][0])
    Err_a2 = np.sqrt(Var_a[1][1])
    a1 = ufloat(a[0],Err_a1)
    a2 = ufloat(a[1],Err_a2)
    print("a1" , repr(a1))
    print("a2" , repr(a2))
    A_0 = calc_A_0(a1, a2)
    delta = calc_delta(a1, a2)
\end{lstlisting}