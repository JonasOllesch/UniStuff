\section{Task 2 - \textit{Lab Experiment}}

\subsection*{(a)}

The design matrix is a $"\text{simple}"$ $12$ by $2$ matrix of the form
\begin{equation*}
    \textbf{A} = \left(
        \begin{array}{c c}
            f_1(\psi_1) & f_2(\psi_1) \\
            \vdots & \vdots \\
            f_1(\psi_{12}) & f_2(\psi_{12}) \\
        \end{array}
    \right) \,.
\end{equation*}

Using the table and ansatz from the exercise, we can just calculate it as shown in \autoref{list:designmat} and get
\begin{equation*}
    \textbf{A} = \left(
        \begin{array}{c c}
             1                    &  0  \\
             0.866                &  0.5 \\
             0.5                  &  0.866 \\
             6.123 \cdot 10^{-17}  &  1 \\
            -0.5                  &  0.866 \\
            -0.866                &  0.5 \\
            -1                    &  1.225 \cdot 10^{-16} \\
            -0.866                & -0.5 \\
            -0.5                  & -0.866 \\
            -1.837 \cdot 10^{-16} & -1 \\
            0.5                   & -0.866 \\
            0.866                 & -0.5 \\
        \end{array}
    \right)
\end{equation*}

\begin{lstlisting}[language = Python, caption={Calculation of design matrix \textbf{A}.}, label = {list:designmat}]
    Data = np.array(np.genfromtxt('Daten.txt'))
    A = np.zeros(shape=(12,2))
    A[:,0]= np.cos(Data[:,0]*(2*np.pi/360))
    A[:,1]= np.sin(Data[:,0]*(2*np.pi/360))
\end{lstlisting}


It should be noted that the three values with the lowest exponents are in reality just zero and will be treated as such.

To calculate the solution vector $\textbf{â}$, we use the definition from the lecture, namely
\begin{equation}
    \textbf{â} = (\textbf{A}^T\textbf{A})^{-1}\textbf{A}^T \textbf{y} \,,
    \label{eq:solveca}
\end{equation}
where $\textbf{y}$ is just the vector of the asymmetry values. \\

\eqref{eq:solveca} can be implemented into code as seen in \autoref{list:solveca}


\begin{lstlisting}[language = Python, caption={Calculation of solution vector \textbf{â}.}, label = {list:solveca}]
    A_T= A.transpose()
    tmp1 = np.matmul(A_T,A)
    tmp2 = np.linalg.inv(tmp1)
    tmp3 = np.matmul(tmp2,A_T)
    a = np.matmul(tmp3,Data[:,1])
\end{lstlisting}

