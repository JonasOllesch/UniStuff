\section*{Exercise 14 - \textit{Balloon Experiment}}

\subsection*{(a)}

First, we have to calculate the likelihood with the seven
measurements from the exercise.

With
\begin{equation*}
    x = [4135, 4202, 4203, 4218, 4227, 4231, 4310]
\end{equation*}
and assuming a poissonian distribution for the measurements, we get
the negative log-likelihood
\begin{equation*}
    -\ln(\mathscr{L}) = 7\lambda - \sum_{i=1}^7 x_i \ln{\lambda}
    + \sum_{i=1}^7 \ln(x_i!) \,.
\end{equation*}

To find the maximum, we just take the derivative and set it equal to zero.

Here, we get

\begin{equation*}
    \frac{\partial \mathscr{L}}{\partial \lambda} = 7 
    - \frac{1}{\lambda} \sum_{i=1}^7x_i = 0 \,,
\end{equation*}

which, if rearranged for $\lambda$ just yields the mean
\begin{equation*}
    \hat{\lambda} = 4128 \,.
\end{equation*}

\subsection*{(b)}

Now, we just replace $\lambda$ with a linear polynomial of the from
$y = mx + n$.