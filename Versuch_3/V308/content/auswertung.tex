\section{Auswertung}
\label{sec:Auswertung}

\subsection{Die lange Spule}
\label{sec:langeSpule}
%lange Spule Plot
In \autoref{fig:langeSpule} ist die magnetische Flussdichte in Abhängigkeit der Tiefe $x$ dargestellt. Die Tiefe
wird dabei vom linken Rand der Spule gemessen. Der mittlere Spulendurchmesser beträgt dabei $d=41 \, \unit{\milli\meter}$, 
die Länge $l=0,18 \, \unit{\meter}$ mit einer Windungszahl von $n=300$.
\begin{figure}[H]
    \centering
    \includegraphics{build/langeSpule.pdf}
    \caption{Magnetische Flussdichte in Abhängigkeit der Tiefe x bei einem Strom von $I= 1 \, \, \unit{\ampere}$.}
    \label{fig:langeSpule}
  \end{figure}

Der Messwert in der Mitte der Spule wird nun mit dem Theoriewert verglichen.
Dabei ergibt der Theoriewert nach \eqref{LangB} mit $l=0,18 \, \unit{\meter}$, $N=300$ und $I= 1\, \unit{\ampere}$
\begin{equation*}
  B_{theo}= 0.00209 \, \unit{\tesla},
\end{equation*}
der nächste Messwert liegt, aufgrund der nur begrenzt langen Hallsonde, bei $x=0,06\, \unit{\meter}$ und lautet
\begin{equation*}
  B_{mess} = 0.00277 \, \unit{\tesla}\,.
\end{equation*}
Da das Feld im Inneren der langen Spule in der Theorie homogen ist, sollte der Messunterschied zur tatsächlichen Spulenmitte
gering sein. 

\newpage

\subsection{Das Spulenpaar}
\label{sec:Spulenpaar}
%Spulenpaar Plot
Die Messdaten innerhalb und außerhalb des Spulenpaares wurden zu drei unterschiedlichen Längen gemessen. 
Bei den aufgetragenen Kurven handelt es sich um die nach \eqref{eq:BFeld} zum Vergleich ermittelten Theoriewerte
der Magnetfeldstärke.

In \autoref{fig:SpulenPaar10} ist der Verlauf der magnetischen Flussdichte in Abhängigkeit des fließenden Stroms
für einen Spulenabstand von $l_1= 10 \, \unit{\centi\meter}$ mitsamt den Theoriewerten dargestellt.
  \begin{figure}[H]
    \centering
    \includegraphics{build/SpulenPaar10.pdf}
    \caption{Verlauf der magnetischen Flussdichte eines Spulenpaares bei einem Strom von $I= 4 \, \, \unit{\ampere}$.}
    \label{fig:SpulenPaar10}
  \end{figure}

  Analog zu \autoref{fig:SpulenPaar10} veranschaulicht \autoref{fig:SpulenPaar15} den Magnetisierungsverlauf und die Theoriewerte
  für einen Spulenabstand von $l_2=15\, \unit{\centi\meter}$.
  \begin{figure}[H]
    \centering
    \includegraphics{build/SpulenPaar15.pdf}
    \caption{Verlauf der magnetischen Flussdichte eines Spulenpaares bei einem Strom von $I= 4 \, \, \unit{\ampere}$.}
    \label{fig:SpulenPaar15}
  \end{figure}

  Die Messdaten und Theoriewerte für den Spulenabstand von $l_3=20\, \unit{\centi\meter}$ sind in 
  \autoref{fig:SpulenPaar20} dargestellt.
  \begin{figure}[H]
    \centering
    \includegraphics{build/SpulenPaar20.pdf}
    \caption{Verlauf der magnetischen Flussdichte eines Spulenpaares bei einem Strom von $I= 4 \, \, \unit{\ampere}$.}
    \label{fig:SpulenPaar20}
  \end{figure}

\newpage

\subsection{Hysteresekurve der Ringspule}
%Hystereseplot

In \autoref{fig:Hystereseplot} ist die Hysteresekurve der magnetisierten Ringspule dargestellt. Dabei ist die magnetische
Flussdichte $B$ gegen den fließenden Strom $I$ aufgetragen. Die grüne Kurve bezeichnet dabei die Neukurve, also die Magnetisierungskurve
ohne Startmagnetisierung des ferromagntischen Materials, die blaue Kurve stellt die erste, die rote Kurve die zweite Ent- und Remagnetisierung dar.

\begin{figure}[H]
    \centering
    \includegraphics{build/Hystereseplot.pdf}
    \caption{Hystereseplot der magnetisierten Ringspule in Abhängigkeit des fließenden Stroms.}
    \label{fig:Hystereseplot}
  \end{figure}

  Die Koerzitivkraft, hier durch die Nullstelle der blauen Kurve gegeben, lässt sich ablesen zu

  \begin{equation*}
    I_c \approx -0,5 \, \unit{\ampere} \,.
  \end{equation*}

  Die Remanenz, also Restmagnetisierung des Stoffes lässt sich als y-Achsenabschnitt der blauen Kurve ermitteln, also
  als zweiter Messwert für $I=0 \, \unit{\ampere}$.
  Es ergibt sich
  \begin{equation*}
    B_r = 130,5 \unit{\milli\tesla} = 0,1305 \unit{\tesla}. 
  \end{equation*}
