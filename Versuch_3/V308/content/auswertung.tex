\section{Auswertung}
\label{sec:Auswertung}

%lange Spule Plot
In \autoref{fig:langeSpule} ist die magnetische Flussdichte in Abhängigkeit der Tiefe $x$ dargestellt. Die Tiefe
wird dabei vom linken Rand der Spule gemessen. Der mittlere Spulendurchmesser beträgt dabei $d=41 \unit{\milli\meter}$, 
die Länge $l=0,18 \unit{\unit}$ mit einer Windungszahl von $n=300$.
\begin{figure}[H]
    \centering
    \includegraphics{build/langeSpule.pdf}
    \caption{Magnetische Flussdichte in Abhängigkeit der Tiefe x bei einem Strom von $I= 1 \, \unit{\ampere}$.}
    \label{fig:langeSpule}
  \end{figure}

Der Messwert in der Mitte der Spule wird nun mit dem Theoriewert verglichen.
Dabei ergibt der Theoriewert nach \eqref{LangB} mit $l=0,18 \unit{\meter}$, $N=300$ und $I= 1\unit{\ampere}$
\begin{equation*}
  B_{theo}= 0.00209 \unit{\tesla},
\end{equation*}
der nächste Messwert liegt, aufgrund der nur begrenzt langen Hallsonde, bei $x=0,06\unit{\meter}$ und lautet
\begin{equation*}
  B_{mess} = 0.0277 \unit{\tesla}\,.
\end{equation*}
Da das Feld im Inneren der langen Spule in der Theorie homogen ist, sollte der Messunterschied zur tatsächlichen Spulenmitte
gering sein. 



%Helmholtzspulenpaar Plot
Die Messdaten des Spulenpaares wurden zu drei unterschiedlichen Längen gemessen.
  \begin{figure}[H]
    \centering
    \includegraphics{build/SpulenPaar10.pdf}
    \caption{Verlauf der magnetischen Flussdichte eines Helmholtzspulenpaares bei einem Strom von $I= 4 \, \unit{\ampere}$.}
    \label{fig:SpulenPaar10}
  \end{figure}

%Hystereseplot

\begin{figure}[H]
    \centering
    \includegraphics{build/Hystereseplot.pdf}
    \caption{Hystereseplot der magnetisierten Ringspule in Abhängigkeit des fließenden Stroms.}
    \label{fig:Hystereseplot}
  \end{figure}
