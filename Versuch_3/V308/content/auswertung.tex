\section{Auswertung}
\label{sec:Auswertung}

%lange Spule Plot
In \autoref{fig:langeSpule} ist die magnetische Flussdichte in Abhängigkeit der Tiefe $x$ dargestellt. Die Tiefe
wird dabei vom linken Rand der Spule gemessen. Der mittlere Spulendurchmesser beträgt dabei $d=41 \unit{\milli\meter}$
mit einer Windungszahl von $n=300$.
\begin{figure}[H]
    \centering
    \includegraphics{build/langeSpule.pdf}
    \caption{Magnetische Flussdichte in Abhängigkeit der Tiefe x bei einem Strom von $I= 1,4 \, \unit{\ampere}$.}
    \label{fig:langeSpule}
  \end{figure}

%Theoriewerte

Es folgt ein Vergleich der Messwerte mit den Theoriewerten, die über \eqref{eq:BFeld} bestimmt werden.
Die Mess- und Theoriewerte seien dabei in der folgenden Tabelle dargestellt.

%Helmholtzspulenpaar Plot

  \begin{figure}[H]
    \centering
    \includegraphics{build/SpulenPaar10.pdf}
    \caption{Verlauf der magnetischen Flussdichte eines Helmholtzspulenpaares bei einem Strom von $I= 4 \, \unit{\ampere}$.}
    \label{fig:SpulenPaar10}
  \end{figure}

%Hystereseplot

\begin{figure}[H]
    \centering
    \includegraphics{build/Hystereseplot.pdf}
    \caption{Hystereseplot der magnetisierten Ringspule in Abhängigkeit des fließenden Stroms.}
    \label{fig:Hystereseplot}
  \end{figure}
