\section{Diskussion}
\label{sec:Diskussion}Es kann gesagt werden, dass die Messungen für die Hysteresekurve den Erwartungen nahe zu entspricht, jedoch treffen die Kurven nach der Remagnetisierung nicht aufeinander. Dieser Unterschied kann dadurch erklärt werden, dass der Sprom, der druch die Spule fließ nicht genau bestimmt werden kann, da die Qualität des Gleichstromrichters der in der Spannungsquelle verbaut ist kann keine Aussage gemacht werden.
Bei dem Experiment mit der langen Spule ist es nicht möglich gewesen die Spule zu befestigen. Sie musste während der Messungen von Hand festgehalten werden. Es könnte zu Verschiebungen zwischen Hallsonde und Spule gekommen sein, sodass sich die Sonde nicht mehr zentral in der Spule befand.
Bei dem Spulenpaar gibt es keine einheitliche Abstandsskalar. Der Abstand zwischen der Spulen wurde auf einem anderen Maßband gemessen, als die Position der Hallsonde, deswegen sind die Theoriekurven trotz Korrektur gegen die Messwerte verschoben. Mit einer weiteren Korrektur erhält man die folgenden Plots.

\begin{figure}[H]
    \centering
    \includegraphics{build/SpulenPaar10korrektur.pdf}
    \caption{Verlauf der magnetischen Flussdichte eines Spulenpaares $l = 15 \unit{\centi\meter}$ und der Korrektur von 1.55 $ \unit{\centi\meter}$  bei einem Strom von $I= 4 \, \, \unit{\ampere}$.}
    \label{fig:SpulenPaar10korrektur}
  \end{figure}

  \begin{figure}[H]
    \centering
    \includegraphics{build/SpulenPaar15korrektur.pdf}
    \caption{Verlauf der magnetischen Flussdichte eines Spulenpaares $l = 20 \unit{\centi\meter}$ und der Korrektur von 2.173 $ \unit{\centi\meter}$bei einem Strom von $I= 4 \, \, \unit{\ampere}$.}
    \label{fig:SpulenPaar15korrektur}
  \end{figure}

  \begin{figure}[H]
    \centering
    \includegraphics{build/SpulenPaar20korrektur.pdf}
    \caption{Verlauf der magnetischen Flussdichte mit eines Spulenpaares $l = 10 \unit{\centi\meter}$ und der Korrektur von 1.825 $ \unit{\centi\meter}$bei einem Strom von $I= 4 \, \, \unit{\ampere}$.}
    \label{fig:SpulenPaar20korrektur}
  \end{figure}

  Eine weitere Fehlerquelle war die Ausrichtung der Hallsonde. Die verwendeten Hallsonden können nur Magnetfelder messen die senkrecht zur Hallsonde verlaufen. Bei einer Verdrehung der Hallsonde zu dem tatsächlichen Verlauf der Feldlinien kann die Magnetfeldstärke nicht genau bestimmt werden. Kleinere Abweichungen können durch nicht regelmäßige Wicklung der Spulen oder eine erhöhte Luftfeuchtigkeit, welche die Suszeptibilität des Raums verändert.