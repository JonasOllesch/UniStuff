\section{Durchführung}
\label{sec:Durchführung}
Mithilfe einer Hallsonde wird die Magnetfeldstärke einer langen Spule in Äbhänigigkeit der Tiefe $x$ in Abständen von $0,5\, \unit{\centi\meter}$ gemessen.
Ebenso wird die Magnetfeldstärke eines Helmholtzspulenpaares innerhalb und außerhalb der Spulen mit drei unterschiedlichen Spulenabständen, gewählt werden hier $d_1=10 \, \unit{\centi\meter}$, $d_2= 15\, \unit{\centi\meter}$ und
$d_3 = 20\, \unit{\centi\meter}$, gemessen.
Abschließend wird die Magnetisierung in einer Ringspule gemessen. Dazu wird der fließende Strom zunächst  auf $I = 10 \, \unit{\ampere}$ hochgeregelt, es wird in $1 \, \unit{\ampere}$-Schritten gemessen. Dann wird der Strom wieder 
heruntergeregelt, bei Erreichen von $I= 0 \, \unit{\ampere}$ wird die Polung der Spule umkehrt und der Strom wird erneut auf $I = 10 \, \unit{\ampere}$ erhöht. Schlussendlich wird er wieder auf null gebracht und die Pole werden ein weiteres Mal getauscht,
beim erneuten Erreichen von $I = 10 \, \unit{\ampere}$ lassen sich die aufgenommenen Messdaten in einer Hysteresekurve darstellen.
