\section{Auswertung}
\label{sec:Auswertung}
Für eine verknüpfte Messgröße $f(x_1,...,x_N)$ gilt die Gaußsche Fehlerfortpflanzung mit
\begin{equation}
Δf(x_1,...,x_N)=\sqrt{\sum_{i=1}^N (\frac{\partial f}{\partial x_i}Δx_i)^2}
\label{eq:gaussF}
\end{equation}
%T_1 und T_2

%Einzelfrequenzen

Die unterschiedlichen Pendellängen sollen dabei in unterschiedlichen Abschnitten verhandelt werden.

\subsection{Pendellänge 70 cm}
\subsubsection{Messung der Einzelfrequenzen} \label{sec:Einzfrel1}

In \autoref{tab:freiePendel1} sind die unabhängigen Schwingungsdauern der Pendel bei einer Pendellänge von $l=70 \unit{\centi\meter}$ dargestellt.

\begin{table}[H]
  \centering
  \caption{Periodendauern der einzelnen Pendel bei $l=70 \unit{\centi\meter}$}
  \label{tab:freiePendel1}
  \sisetup{table-format=1.3}
  \begin{tabular}{S[table-format=2.0] S S}
    \toprule
    {Messung} & {$T_1 \mathbin{/} \unit{\second}$} & {$T_2 \mathbin{/} \unit{\second}$} \\
    \midrule
    1 & 1.642 & 1.692 \\
    2 & 1.626 & 1.708 \\
    3 & 1.660 & 1.744 \\
    4 & 1.646 & 1.658 \\
    5 & 1.664 & 1.732 \\
    6 & 1.652 & 1.668 \\
    7 & 1.650 & 1.694 \\
    8 & 1.648 & 1.658 \\
    9 & 1.680 & 1.724 \\
   10 & 1.660 & 1.612 \\
    \bottomrule
  \end{tabular}
\end{table}

Aus den Mittelwerten und der Abweichung ergibt sich 

\begin{align*}
  T_1 & = (1.653 \pm 0.014) \, \unit{\second} \\
  T_2 & = (1.689 \pm 0.038) \, \unit{\second}.
\end{align*}

\subsubsection{Gleichphasige und gegenphasige Schwingung}

\autoref{tab:GleigphaSchwingung1} stellt die gleich- und gegenphasigen Schwingungsdauern bei einer Pendellänge von $l=70 \unit{\centi\meter}$ dar.

\begin{table}[H]
  \centering
  \caption{Periodendauern bei der gleich- und gegenphasigen Schwingung mit einer Pendellänge von 70 cm.}
  \label{tab:GleigphaSchwingung1}
  \sisetup{table-format=1.2}
  \begin{tabular}{S[table-format=2.0] S S[table-format=1.3] }
    \toprule
    {Messung} & {$T_+ \mathbin{/} \unit{\second}$} & {$T_- \mathbin{/} \unit{\second}$} \\
    \midrule
    1 & 1.784 & 1.420 \\
    2 & 1.664 & 1.432 \\
    3 & 1.652 & 1.570 \\
    4 & 1.736 & 1.614 \\
    5 & 1.708 & 1.520 \\
    6 & 1.826 & 1.548 \\
    7 & 1.608 & 1.510 \\
    8 & 1.656 & 1.540 \\
    9 & 1.698 & 1.566 \\
   10 & 1.694 & 1.544 \\
    \bottomrule
  \end{tabular}
\end{table}

Auch hier folgen aus Mittelwerte und Unsicherheit
\begin{align*}
  T_+ & = (1.7025 \pm 0.0619) \, \unit{\second} \\
  T_- & = (1.5260 \pm 0.0571) \, \unit{\second}.
\end{align*}

%Gekoppelte Schwingung

\subsubsection{Gekoppelte Schwingungen}

In \autoref{tab:gekoppelteSchwingung1} sind die Schwebungsdauer und die Schwingfrequenz für die gekoppelte Schwingung aufgetragen.

\begin{table}[H]
  \centering
  \caption{Periodendauern und Schwebungsdauern bei der gekoppelten Schwingung.}
  \label{tab:gekoppelteSchwingung1}
  \sisetup{table-format=1.2}
  \begin{tabular}{S[table-format=2.0] S S}
    \toprule
    {Messung} & {$T \mathbin{/} \unit{\second}$} & {$T_S \mathbin{/} \unit{\second}$} \\
    \midrule
    1 & 1.586 & 18.12 \\
    2 & 1.440 & 17.95 \\
    3 & 1.588 & 18.51 \\
    4 & 1.486 & 18.33 \\
    5 & 1.632 & 18.36 \\
    6 & 1.446 & 17.40 \\
    7 & 1.542 & 18.29 \\
    8 & 1.604 & 16.81 \\
    9 & 1.454 & 16.99 \\
   10 & 1.614 & 17.47 \\
   \bottomrule
  \end{tabular}
\end{table}

Aus den Mittelwerten und Abweichungen ergeben sich erneut

\begin{align*}
 T   &= (1.539  \pm 0.072) \, \unit{\second} \\ 
 T_S &= (17.823 \pm 0.580) \, \unit{\second}\, . 
\end{align*}


\subsection{Pendellänge 100 cm}
\subsubsection{Messung der Einzelfrequenzen}

Analog zu \autoref{sec:Einzfrel1} sind in \autoref{tab:freiePendel2} die unabhängigen Periodendauern aufgetragen.
\begin{table}[H]
  \centering
  \caption{Periodendauern der einzelnen Pendel mit einer Pendellänge von 100 cm.}
  \label{tab:freiePendel2}
  \sisetup{table-format=1.3}
  \begin{tabular}{S[table-format=2.0] S S}
    \toprule
    {Messung} & {$T_1 \mathbin{/} \unit{\second}$} & {$T_2 \mathbin{/} \unit{\second}$} \\
    \midrule
    1 & 1.992 & 1.980 \\
    2 & 1.988 & 2.012 \\
    3 & 1.946 & 1.980 \\
    4 & 1.986 & 1.974 \\
    5 & 2.002 & 1.976 \\
    6 & 2.028 & 2.002 \\
    7 & 1.942 & 1.948 \\
    8 & 2.044 & 1.984 \\
    9 & 1.948 & 2.026 \\
   10 & 1.968 & 1.956 \\
    \bottomrule
  \end{tabular}
\end{table}

Aus den Mittelwerten und der Abweichung ergibt sich 

\begin{align*}
  T_1 = (1.984 \pm 0.033) \, \unit{\second} \\
  T_2 = (1.984 \pm 0.023) \, \unit{\second} \,.
\end{align*}

\subsubsection{Gleichphasige und gegenphasige Schwingung}

\autoref{tab:GleigphaSchwingung2} stellt die gleich- und gegenphasigen Schwingungsdauern bei einer Pendellänge von $l=70 \unit{\centi\meter}$ dar.

\begin{table}[H]
  \centering
  \caption{Periodendauern bei der gleich- und gegenphasigen Schwingung mit einer Pendellänge von 70 cm.}
  \label{tab:GleigphaSchwingung2}
  \sisetup{table-format=1.2}
  \begin{tabular}{S[table-format=2.0] S S[table-format=1.3] }
    \toprule
    {Messung} & {$T_+ \mathbin{/} \unit{\second}$} & {$T_- \mathbin{/} \unit{\second}$} \\
    \midrule
    1 & 1.938 & 1.766 \\
    2 & 1.982 & 1.848 \\
    3 & 2.034 & 1.874 \\
    4 & 1.958 & 1.822 \\
    5 & 1.960 & 1.888 \\
    6 & 1.962 & 1.836 \\
    7 & 2.030 & 1.880 \\
    8 & 1.952 & 1.826 \\
    9 & 1.982 & 1.894 \\
   10 & 1.984 & 1.834 \\
    \bottomrule
  \end{tabular}
\end{table}

Es folgen wieder
\begin{align*}
  T_+ &= (1.9781 \pm 0.0303) \, \unit{\second} \\
  T_- &= (1.8470 \pm 0.0369) \, \unit{\second} \,.
\end{align*}

\subsubsection{Gekoppelte Schwingungen}

In \autoref{tab:gekoppelteSchwingung2} sind die Schwebungsdauer und die Schwingfrequenz für die gekoppelte Schwingung aufgetragen.

\begin{table}[H]
  \centering
  \caption{Periodendauern bei der gegenphasigen Schwingungen}
  \label{tab:gekoppelteSchwingung2}
  \sisetup{table-format=1.3}
  \begin{tabular}{S[table-format=2.0] S S[table-format=2.2]}
    \toprule
    {Messung} & {$T \mathbin{/} \unit{\second}$} & {$T_S \mathbin{/} \unit{\second}$} \\
    \midrule
    1 & 1.936 & 27.64 \\
    2 & 1.912 & 28.10 \\
    3 & 1.830 & 27.42 \\
    4 & 1.858 & 27.42 \\
    5 & 1.850 & 27.74 \\
    6 & 1.888 & 27.89 \\
    7 & 1.860 & 27.03 \\
    8 & 1.828 & 27.73 \\
    9 & 1.938 & 28.60 \\
   10 & 1.970 & 28.37 \\
   \bottomrule
  \end{tabular}
\end{table}
Analog ergeben sich auch hier
\begin{align*}
  T  &= (1.887  \pm 0.047) \, \unit{\second}\\
 T_S &= (27.794 \pm 0.440) \, \unit{\second} \, .
\end{align*}

%%%%%%%%%%%%%%%%%%%%%%%%%%%


\subsection{Berechnung der Schwingfrequenzen}

\subsubsection{Pendellänge 70 cm} \label{sec:Schwingf1}

Über die Mittelwerte der Periodendauern ergeben sich die Messfrequenzen zu
\begin{align*}
\omega_1 & = \frac{2π}{T_1}   = 3.802 \pm 0.032     \,\unit{\hertz}\\
\omega_2 & = \frac{2π}{T_2}  = 3.720 \pm 0.085      \,\unit{\hertz}\\
\omega_+ & = \frac{2π}{T_+}  = 3.69 \pm 0.13        \,\unit{\hertz}\\
\omega_- & = \frac{2π}{T_-}  = 4.12 \pm 0.15        \,\unit{\hertz}\\
\omega_S & = \frac{2π}{T_S} = 0.353 \pm 0.011       \,\unit{\hertz}\,.
\end{align*} 

Bevor die Schwebungsfrequenz berechnet werden kann, ist es notwendig, den Kopplungsgrad der Federn zu ermitteln.
Mit \eqref{eq:Kopplungsgrad} ergibt sich
\begin{equation*}
  K = 0.1088 \pm 0.0515
\end{equation*}

Mit $l_1=70 \, \unit{\centi\meter}$ folgt für die Theoriewerte der Frequenzen aus \eqref{eq:omega+}, \eqref{eq:omega-} und \eqref{eq:omegaS}

\begin{align*}
\omega_1 = \omega_+ & = 3.744 \, \unit{\hertz} \\ 
\omega_{-} & = 3.785            \, \unit{\hertz} \\ 
\omega_{S} & = 0.041           \, \unit{\hertz} \,.
\end{align*}

\subsubsection{Pendellänge 100 cm}

Analog zu \autoref{sec:Schwingf1} ergeben sich hier die Schwingfrequenzen


\begin{align*}
  \omega_1 & =  \frac{2π}{T_1}         = 3.166   \pm 0.052 \,\unit{\hertz}  \\
  \omega_2 & =  \frac{2π}{T_2}         = 3.167   \pm 0.036 \,\unit{\hertz}  \\
  \omega_+ & =  \frac{2π}{T_+}         = 3.176   \pm 0.049 \,\unit{\hertz}  \\
  \omega_- & =  \frac{2π}{T_-}         = 3.402   \pm 0.068 \,\unit{\hertz}  \\
  \omega_S & =  \frac{2π}{T_S}        = 0.2261  \pm 0.0036 \,\unit{\hertz} \,,
\end{align*}
der Kopplungsgrad
\begin{equation*}
  K = 0.0686 \pm 0.0251
\end{equation*}
und damit die Theoriewerte
\begin{align*}
\omega    = \omega_+ & = 3.132 \, \unit{\hertz} \nonumber \\
\omega_- & = 3.154          \, \unit{\hertz} \nonumber \\
\omega_S & = 0.0218         \, \unit{\hertz} \nonumber \,.
\end{align*}