\section{Vorbereitungsaufgaben}
\subsection{Was ist eine harmonische Schwingung?}
Eine harmonische Schwingung bezeichnet einen Schwingprozess, dessen Projektion sich als Kreisbewegung darstellen lässt. Die Schwingung selbst entspricht einer Sinus- oder Cosinusfunktion.
Es gilt also $\phi(t)=A\cos(\omega(t))$ oder $\phi(t)= A\sin(\omega(t))$, dabei stellt $A$ die Maximalamplitude der Schwingung dar.

\subsection{Wie weit kann man ein Fadenpendel mit einer Pendellänge von l = 70cm auslenken, damit die Kleinwinkelnäherung noch gilt?}
Die Kleinwinkelnäherung gilt unabhängig von der Pendellänge für Winkel $φ\leq10°$.
\label{sec:Vorbereitungsaufgaben}

% Fertig und korrigiert.