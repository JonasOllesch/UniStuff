\section{Diskussion}
\label{sec:Diskussion}

\subsection{Abweichung von Theorie und Messung}

Die relative Abweichung des gemessenen und theoretischen Wertes der Kugel beträgt mit 
\begin{equation*}
    I_{K,m} = (0,00138 \pm 0,00017)\, \unit{\newton\meter}  \, .
\end{equation*}

und

\begin{equation*}
    I_{K,t} = 0,00131 \, \unit{\newton\meter}. % die theorischen Werten haben keine Unsicherheit aber wenn die noch eine bekommen sollen, dann kann man da bestimmt noch was machen xD
\end{equation*}

zu

\begin{equation*}
    \left|\frac{I_{K,t} - I_{K,m}}{I_{K,t}} \right| \cdot 100 = 5,344 \,\% \,. % \left|\frac{I_{K,m}}{I_{K,t}} \cdot 100 - 100 \right| = 5,344 \% \,.
\end{equation*} \\

Analog ergibt sich dann für den Zylinder mit

\begin{equation*}
    I_{K,m} = (0,00034 \pm 0,00004) \, \unit{\newton\meter}
\end{equation*} 

und

\begin{equation*}
    I_{Z,t} = 0,00043 \,  \unit{\newton\meter}.
\end{equation*}

eine Abweichung von
\begin{equation*}
    \left|\frac{I_{Z,t} - I_{Z,m}}{I_{Z,t}} \right| \cdot 100 = 20,930 \% \,.
\end{equation*} \\

Die Abweichungen von Theorie und Messung der einzelnen Holzpuppenpositionen belaufen sich auf

\begin{equation*}
  100 \cdot \left|\frac{I_{P_1,t} - I_{P_1,m}}{I_{P_1,t}} \right| = 21,429 \%
\end{equation*}
und
\begin{equation*}
  100 \cdot \left|\frac{I_{P_2,t} - I_{P_2,m}}{I_{P_2,t}} \right| = 31,818 \% \,.
\end{equation*} \\

Zum Vergleich der experimentellen und theoretischen Werte der Holzpuppe werden im Folgenden insbesondere die Verhältnisse der Trägheitsmomente $\frac{I_{P_1}}{I_{P_2}}$ der einzelnen Positionen betrachtet.
Es ergibt sich mit
\begin{equation*}
  d_{m}= \frac{I_{P_1,m}}{I_{P_2,m}} = 0,489
\end{equation*}
und
\begin{equation*}
  d_{t} = \frac{I_{P_1,t}}{I_{P_2,t}} = 0,424 
\end{equation*}
eine relative Abweichung von
\begin{equation*}
    \left|\frac{d_{t} - d_m}{d_t} \right| \cdot 100 = 15,252 \,\% \,.
\end{equation*}

\subsection{Fehlerdiskussion}

Die Abweichung des Verhältnisses zwischen Theorie und Messung ist mit $d_{ges} = 15,252 \,\%$ recht gering, 
auch zwischen den einzelnen Theorie- und Messwerten der unterschiedlichen Positionen bestehen mit maximal rund $32 \%$ in Anbetracht des Aufbaus nur geringfügige Abweichungen. \\

Die wohl größte Fehlerquelle bei der Bestimmung der Trägheitsmomente der Puppe liegt darin, dass die Gliedmaßen nur schwer perfekt auszurichten sind. 
Einerseits konnte aufgrund der Gelenke nicht sichergestellt werden, dass Hände und Füße der Puppe voll ausgestreckt waren, was die Annäherung als Zylinder ungenauer macht, 
in der Näherung wurden Hände und Füße gar nicht berücksichtigt.
Andererseits war vor allem der Torso, aber auch alle anderen Körperteile nicht vollständig symmetrisch, was das Schwingungsverhalten beeinflusste und die Näherung weiter verschlechterte.
Zusätzlich dazu erfuhr das System aufgrund einer mangelnden Befestigtung während der Schwingung Störungen, die die Schwingungsdauer beeinflusst haben könnten. \\

Für alle drei Körper, also auch Kugel und Zylinder spielen auch Reibung und Luftwiderstand eine Rolle, zusammen mit der hohen Ungenauigkeit der Schwingungsdauermessung 
lassen sich die Abweichungen von Theorie und Messung erklären.
Auch die Winkelscheibe selbst war zum Messen ungeeignet, da sie ständig, von der Drehachse rutschte oder sich verschob.