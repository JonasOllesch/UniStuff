\section{Diskussion}
\label{sec:Diskussion}

Am Aufbau der Wärmepumpe war zu erkennen, dass es sich bei den Reservoiren nicht zum abgeschlossene System handelt, da die Behälter oben eine Öffnung haben durch, in die die Thermometer eingeführt wurden.
Aufgrund dieser Öffnungen sind neben den zu erwartenden Verlusten weitere enstanden. Auch haben die Barometer nur grobe Skalen, es ist lediglich möglich, die Drücke auf etwa {$0.1$} bar abzuschätzen.
Zwar wurde das Wasser in den Reservoires in Bewegung gehalten, um eine bessere Durchmischung zu gewährleisten, jedoch ist gegen Ende der Messung im kalten Reservoir Wasser am der Kupferschlange  festgefroren, was den Wärmeaustausch behinderte.
Auch die Rohre zwischen den Reservoiren, dem Drosselventil und dem Kompressor sind nicht darauf ausgelegt, eine hohe Güteziffer zu erreichen, weil diese länger als nötig sind.