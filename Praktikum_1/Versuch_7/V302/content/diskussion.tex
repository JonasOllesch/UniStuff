\section{Diskussion}
\label{sec:Diskussion}
Vergleicht man die Abweichung zwischen der Induktivitätsmessbrücke und der Maxwellbrücke, so fällt auf, dass $\dfrac{R_{ind}}{R_{max}} = 0,984 \pm 0,004 \thickapprox 1 $ ist.
Beim Vergleich von  $\dfrac{L_{x,max}}{L_{x,ind}} = 10,4 \pm 0,4 $ ist zu erkennen, dass die Abweichung deutlich größer ist, dabei ist die Messung der Maxwellbrücke wahrscheinlich besser, da der Wert von $(0,131 \pm 0,006) \,\unit{\henry}$ besser zu einer Spule der verwendeten Größe passt.
Diese Abweichung lässt sich durch die Annahmen, dass die Spule $L_2$ bei der Induktivitätsmessbrücke keinen Innenwiderstand besitzt, erklären.
Zwar wurde auch angenommen, dass der Kondensator $C_4$ der Maxwellbrücke ebenfalls verlustfrei ist, allerdings arbeiten Kondensatoren prinzipiell verlustärmer. \\

Zum Vergleich der Theoriekurve und den Messdaten der Wien-Robinson-Brücke lässt sich sagen, dass sie, bis auf das Datentupel $(ν, U_{Br}(ν)) = (800 \,\unit{\hertz}, 720 \,\unit{\milli\volt})$ recht gut zueinander passen. \\

Der Klirrfaktor von $1,234 \cdot 10^{-3}$ ist ebenfalls sehr gering, die Genauigkeit um $ν_0$ ist mit einem Minimum von $1 \,\unit{\milli\volt}$ sehr hoch. 