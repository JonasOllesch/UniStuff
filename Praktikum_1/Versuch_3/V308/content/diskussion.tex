\section{Diskussion}
\label{sec:Diskussion}
Es kann gesagt werden, dass die Messungen für die Hysteresekurve den Erwartungen nahe zu entspricht, jedoch treffen die Kurven nach der Remagnetisierung nicht aufeinander. 
Dieser Unterschied kann dadurch erklärt werden, dass der Strom, der druch die Spule fließt, nicht genau bestimmt werden kann, da über die Qualität des Gleichstromrichters,
 der in der Spannungsquelle verbaut ist, keine Aussage gemacht werden kann.\\

Bei dem Experiment mit der langen Spule ist es nicht möglich gewesen, die Spule zu befestigen. 
Sie musste während der Messungen von Hand festgehalten werden. 
Außerdem ist das Feld im Inneren der Spule, wie bei der Messung zu beobachten war, nicht vollständig homogen, was auch die hohe relative Abweichung von der theoretischen Flussdichte im Inneren der langen Spule erklärt.
Aufgrund der begrenzten Länge der Hallsonde war es ebenfalls nicht möglich, das Zentrum der Spule zu erreichen, eine "wahre" Messung des inneren Feldes im Spulenmittelpunkt war also nicht durchführbar.
Des Weitern könnte es zu Verschiebungen zwischen Hallsonde und Spule gekommen sein, sodass sich die Sonde nicht mehr zentral in der Spule befand, was ebenfalls die Messgenauigkeit verringert.\\

Bei dem Spulenpaar gibt es keine einheitliche Abstandsskala.
Der Abstand zwischen der Spulen wurde auf einer anderen Skala gemessen, als die Position der Hallsonde, deswegen sind die Theoriekurven trotz Korrektur gegen die Messwerte verschoben.
Eine Korrektur von  $1.5 \, \unit{\centi\meter}$ liefert jedoch relativ genau Ergebnisse, die in \autoref{fig:SpulenPaar10korrektur}, \autoref{fig:SpulenPaar15korrektur} und \autoref{fig:SpulenPaar20korrektur}
dargestellt sind.

\begin{figure}[H]
    \centering
    \includegraphics[scale=0.7]{build/SpulenPaar10korrektur.pdf}
    \caption{Korrektur des Verlaufes der magnetischen Flussdichte des Spulenpaares im Abstand von $l = 15 \unit{\centi\meter}$ .}
    \label{fig:SpulenPaar10korrektur}
  \end{figure}

  \begin{figure}[H]
    \centering
    \includegraphics[scale=0.7]{build/SpulenPaar15korrektur.pdf}
    \caption{Korrektur des Verlaufes der magnetischen Flussdichte des Spulenpaares im Abstand von $l = 20 \unit{\centi\meter}$.}
    \label{fig:SpulenPaar15korrektur}
  \end{figure}

  \begin{figure}[H]
    \centering
    \includegraphics[scale=0.7]{build/SpulenPaar20korrektur.pdf}
    \caption{Korrektur des Verlaufes der magnetischen Flussdichte des Spulenpaares im Abstand von $l = 10 \unit{\centi\meter}$.}
    \label{fig:SpulenPaar20korrektur}
  \end{figure}

  Eine weitere Fehlerquelle war die Ausrichtung der Hallsonde. Die verwendeten Hallsonden können nur Magnetfelder messen die senkrecht zur Hallsonde verlaufen. Bei einer Verdrehung der Hallsonde zu dem tatsächlichen Verlauf der Feldlinien kann die Magnetfeldstärke nicht genau bestimmt werden. Kleinere Abweichungen können durch nicht regelmäßige Wicklung der Spulen oder eine erhöhte Luftfeuchtigkeit, welche die Suszeptibilität des Raums verändert.