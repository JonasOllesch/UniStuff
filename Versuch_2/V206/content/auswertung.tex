\section{Auswertung}
\label{sec:Auswertung}

%Tabelle mit Messdaten
Es folgt eine Tabelle mit den aufgenommenen Messdaten.
\begin{table}[H]
  \centering
  \caption{Veränderung der Reservoirtemperaturen und -drücke sowie der Leistung im Messverlauf.}
  \label{tab:Messdaten}
  \sisetup{table-format=2.1}
  \begin{tabular}{S[table-format=2.0] S S S S S[table-format=3.0]}
    \toprule
    {$t \mathbin{/} \unit{\minute}$} 
    & {$T_1 \mathbin{/} \unit{\celsius}$} & {$T_2 \mathbin{/} \unit{\celsius}$} 
    & {$p_b \mathbin{/} \unit{\bar}$} & {$p_a \mathbin{/} \unit{\bar}$} 
    & {$N \mathbin{/} \unit{\watt}$} \\
    \midrule
     0 & {21.4} & {21.4} &   { 3.2} & {4.7} & {  0} \\
     1 & {22.0} & {21.3} &   { 5.0} & {3.0} & {115} \\
     2 & {23.3} & {23.3} &   { 5.2} & {3.2} & {115} \\
     3 & {24.6} & {18.8} &   { 5.5} & {3.4} & {120} \\
     4 & {25.8} & {17.7} &   { 5.8} & {3.4} & {122} \\
     5 & {27.0} & {16.4} &   { 6.0} & {3.5} & {125} \\
     6 & {28.5} & {15.2} &   { 6.4} & {3.5} & {125} \\
     7 & {29.9} & {14.2} &   { 6.8} & {3.2} & {125} \\
     8 & {31.2} & {13.2} &   { 6.9} & {3.0} & {123} \\
     9 & {32.6} & {12.2} &   { 7.2} & {3.0} & {123} \\
    10 & {33.8} & {11.3} &   { 7.5} & {2.8} & {123} \\
    11 & {35.1} & {10.3} &   { 7.9} & {2.8} & {124} \\
    12 & {36.2} & { 9.3} &   { 8.0} & {2.6} & {125} \\
    13 & {37.3} & { 8.5} &   { 8.3} & {2.4} & {125} \\
    14 & {38.3} & { 7.6} &   { 8.5} & {2.4} & {125} \\
    15 & {39.3} & { 6.8} &   { 8.9} & {2.4} & {126} \\
    16 & {40.2} & { 5.9} &   { 9.0} & {2.3} & {127} \\
    17 & {41.1} & { 5.1} &   { 9.1} & {2.2} & {115} \\
    18 & {41.9} & { 4.4} &   { 9.4} & {2.1} & {115} \\
    19 & {42.8} & { 3.7} &   { 9.8} & {2.1} & {115} \\
    20 & {43.7} & { 2.9} &   { 9.5} & {2.0} & {115} \\
    21 & {44.3} & { 2.4} &   {10.0} & {2.0} & {115} \\
    22 & {45.0} & { 1.7} &   {10.2} & {1.9} & {115} \\
    23 & {45.7} & { 1.1} &   {10.4} & {1.8} & {114} \\
    24 & {46.4} & { 0.5} &   {10.6} & {1.8} & {114} \\
    25 & {47.0} & {$-0.1$} & {10.9} & {1.8} & {114} \\
    26 & {47.7} & {$-0.6$} & {11.0} & {1.7} & {113} \\
    27 & {48.2} & {$-1.1$} & {11.1} & {1.7} & {113} \\
    28 & {48.8} & {$-1.6$} & {11.2} & {1.6} & {113} \\
    29 & {49.4} & {$-2.1$} & {11.4} & {1.6} & {112} \\
    30 & {49.8} & {$-2.6$} & {11.6} & {1.6} & {111} \\
    31 & {50.4} & {$-3.0$} & {11.9} & {1.6} & {111} \\
    \bottomrule
  \end{tabular}
\end{table}

\subsection{Temperaturverläufe und nicht lineare Approximation}
%erster Plot

\begin{figure}[H]
  \centering
  \includegraphics{plot_1.pdf}
  \caption{Temperaturverlauf in den Reservoiren.}
  \label{fig:plot1}
\end{figure}

Als Ansatz zur Approximation der Temperaturverläufe wurde eine quadratische Näherung der Form 
\begin{equation}
  T(t)=At^2+Bt+C
\end{equation}
verwendet.

\noindent In der folgenden Tabelle seien die Parameter $A, B$ und $C$ der Näherungen für $T_1(t)$ und $T_2(t)$ dargestellt.

\begin{table}[H]
  \centering
  \label{tab:ApproxTemp}
  \sisetup{table-format=1.5}
  \begin{tabular}{S S S S S S[table-format=2.5] S[table-format=3.5]}
    \toprule
    & {$A \mathbin{/} \unit{\dfrac{\celsius}{\minute^2}}$} & {$A \mathbin{/} \unit{\dfrac{\kelvin}{\second^2}}$} 
    & {$B \mathbin{/} \unit{\dfrac{\celsius}{\minute}}$} & {$B \mathbin{/} \unit{\dfrac{\kelvin}{\second}}$} 
    & {$C \mathbin{/} \unit{\celsius}$} &  {$C \mathbin{/} \unit{\kelvin}$} \\
    \midrule
    {$T_1(t)$} & {$-0.01756$} &  {$-4.87778*10^{-6}$} & {1.50604} & {0.02510} & {20.45657} & {293.60657} \\
    {$T_2(t)$} & {0.01510} & {$4.19443*10^{-6}$} & {$-1.29451$} & {$-0.02158$} & {22.77799} & {295.92799} \\
    \bottomrule
  \end{tabular}
\end{table}
Bei Betrachtung der Kurvenparameter ist, abgesehen von den Vorzeichen, wie erwartet eine große Ähnlichkeit zwischen den
beiden Temperaturkurven zu beobachten.

\newpage

\subsection{Bestimmung der Differenzenquotienten}

Mithilfe der Näherungen für die Temperaturverläufe lassen sich die Differenzenquotienten $\dfrac{ΔT_i}{Δt}$ als Diefferenzialquotienten
$\dfrac{\text{d}T_i}{\text{d}t}$ schreiben.
Zur Bestimmung der Diefferenzialquotienten wurden vier Zeitpunkte mit einem Abstand von je 380 Sekunden gewählt.
Als Zeitpunkte wurden dienen dabei $t_i = 380i \, \unit{\second}$ mit $i=1,2,3$.

\begin{table}[H]
  \centering
  \label{tab:Diffquo}
  \sisetup{table-format=1.5}
  \begin{tabular}{S S S}
    \toprule
    & {$\dfrac{\text{d}T_1}{\text{d}t}$} & {$\dfrac{\text{d}T_2}{\text{d}t}$} \\
    \midrule
    {$t_1 = 380  \, \unit{\second}$} & {$0.02139 \pm 0.00035$} & {$-0.018387 \pm 0.00085$} \\
    {$t_2 = 760  \, \unit{\second}$} & {$0.01769 \pm 0.00042$} & {$-0.015110 \pm 0.00101$} \\
    {$t_3 = 1140 \, \unit{\second}$} & {$0.01398 \pm 0.00050$} & {$-0.012012 \pm 0.00123$} \\
    {$t_4 = 1520 \, \unit{\second}$} & {$0.01027 \pm 0.00061$} & {$-0.008824 \pm 0.00148$} \\
    \bottomrule
  \end{tabular}
\end{table}


\subsection{Bestimmung des Massendurchsatzes}

\subsection{Bestimmung der mechanischen Kompressorleistung}


\subsection{Sonstige Plots}

In diesem Abschnitt befinden sich lediglich sonstige Plots, die der Vollständigkeit halber zwar enthalten sind, zur 
Bestimmung vorangegangener Größen aber nicht von großer Bedeutung sind.

%3. Plot

\begin{figure}
  \centering
  \includegraphics{plot_2.pdf}
  \caption{Zeitlicher Druckverlauf in den Reservoiren.}
  \label{fig:plot3}
\end{figure}

%5. Plot

\begin{figure}
  \centering
  \includegraphics{plot_5.pdf}
  \caption{Zeitlicher Verlauf der Leistung.}
  \label{fig:plot5}
\end{figure}
