\section{Diskussion}
\label{sec:Diskussion}

Am Aufbau der Wärmepumpe war zu erkennen, dass es sich bei den Reservoiren nicht zum abgeschlossene System handelt, da die Behälter oben eine Öffnung haben durch, welche die Thermometer in eingeführt wurden.
Aufgrund dieser Öffnungen sind neben den zu erwartenden Verlusten weitere enstanden. Auch haben die Barometer grobe Skalen und es ist nur möglich die Drücke auf {$0.1$} bar abzuschätzen.
Zwar wurde das Wasser in den Reservoires in Bewegung gehalten um eine bessere Durchmischung zu gewährleisten, jedoch ist gegen Ende der Messung Wasser am der Kupferschlange in kalten Reservoir festgefroren, was den Wärmeaustausch behinderte.
Auch die Rohre zwischen den Reservoiren, dem Drosselventil und dem Kompressor sind nicht darauf ausgelet eine hohe Güteziffer zu erreichen, weil diese länger als nötig sind.