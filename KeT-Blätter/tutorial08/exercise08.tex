\documentclass{exercise}
\usepackage{global-settings}
\usepackage{multicol}
\usepackage{subcaption}
\usepackage[version=4]{mhchem}
\def\modelsolution{1}

\setcounter{tutorial}{8}
\setcounter{exercise}{1}
\release{Mittwoch, 20.11.2024}
\submission{Mittwoch, 27.11.2024, 14 Uhr}

\begin{document}

\clearpage
\makeheader

\exercise{Fragen (2P)}
Stellen Sie \emph{pro Person} zwei relevante Fragen zu den Inhalten der Vorlesung \enquote{Einf\"uhrung in die Kern- und Elementarteilchenphysik}.

\exercise{Lorentztransformationen ()}

Die Lorentzgruppe ist die Gruppe an Transformationen auf Vierervektoren, unter denen die Minkowskimetrik (die Metrik des flachen Raumes) invariant bleibt:
\begin{equation}
    L := O(3,1) =  \left\{ \Lambda \in \mathbb{R}^{4 \times 4} \, | \, \Lambda_{\phantom{\mu} \mu}^\alpha \, g_{\alpha \beta} \, \Lambda_{\phantom{\nu} \nu}^\beta   = g_{\mu \nu} \right\} \,.
\end{equation} 

Eine wichtige Untergruppe der Lorentztransformationen (LTs) sind die eigentlich-orthochronen Lorentztransformationen
\begin{equation}
    L_+^\uparrow := \left\{ \Lambda \in L \, | \, \det \Lambda = 1 \, \wedge \, \text{sgn} \Lambda^0_{\phantom{0} 0} = 1 \right\} \,,
\end{equation} 
die ausschließlich Boosts und Rotationen umfasst.
Neben den eigentlich-orthochronen LTs existieren drei weitere Komponenten, die zusammen die Lorentzgruppe bilden.
Diese sind über Parit\"at und/oder Zeitumkehr mit der Untergruppe der eigentlich-orthochronen LTs verbunden.

\begin{enumerate}
    \item Geben Sie die Matrixformen der Paritäts- und Zeitumkehrtransformationen $P$ und $T$ an.
    
    \solution{
    \begin{align*}
        P &= \begin{pmatrix}
            1 & 0 & 0 & 0 \\
            0 & -1 & 0 & 0 \\
            0 & 0 & -1 & 0 \\
            0 & 0 & 0 & -1
        \end{pmatrix}
    \end{align*}
    \begin{align*}
        T &= \begin{pmatrix}
            -1 & 0 & 0 & 0 \\
            0 & 1 & 0 & 0 \\
            0 & 0 & 1 & 0 \\
            0 & 0 & 0 & 1
        \end{pmatrix}
    \end{align*}
    \textit{1 Punkt}
    }

    \item Zeigen Sie, dass $\det \Lambda = \pm 1$ gilt. 
    \solution{
        Wir starten mit
        \begin{equation}
            \Lambda^T g \Lambda = g \,.
        \end{equation}
        Dann folgt
        \begin{align*}
            \det(\Lambda^T) \det(g) \det(\Lambda) &= \det(g) \\
            \det(\Lambda^T) \det(\Lambda) &= 1 \\
            \det^2(\Lambda) &= 1 \\
            \det(\Lambda) &= \pm 1 \,.
        \end{align*}
        \textit{1 Punkt}
    }

    \item Betrachten Sie einen Lorentzboost in $x$-Richtung mit
    \begin{equation}
        \Lambda = \begin{pmatrix}
            \gamma & -\beta \gamma & 0 & 0 \\
            -\beta \gamma & \gamma & 0 & 0 \\
            0 & 0 & 1 & 0 \\
            0 & 0 & 0 & 1
        \end{pmatrix}
    \end{equation}
    und dem bekannten $\gamma = 1/\sqrt{1-\beta^2}$.
    Zeigen Sie, dass $\Lambda \in L_+^\uparrow$ gilt.

    \solution{
        Wir teilen den Beweis in drei Schritte auf:
        \begin{enumerate}
            \item  $\Lambda \in L$
            
            Offensichtlich ist $\Lambda$ eine reelle $4 \times 4$ da $\gamma \,, \beta \in \mathbb{R}$.
            Des Weiteren gilt:
            \begin{align*}
                \Lambda^T g \Lambda &= \begin{pmatrix}
                    \gamma & -\beta \gamma & 0 & 0 \\
                    -\beta \gamma & \gamma & 0 & 0 \\
                    0 & 0 & 1 & 0 \\
                    0 & 0 & 0 & 1
                    \end{pmatrix}^T
                    \begin{pmatrix}
                    1 & 0 & 0 & 0 \\
                    0 & -1 & 0 & 0 \\
                    0 & 0 & -1 & 0 \\
                    0 & 0 & 0 & -1
                    \end{pmatrix}
                    \begin{pmatrix}
                    \gamma & -\beta \gamma & 0 & 0 \\
                    -\beta \gamma & \gamma & 0 & 0 \\
                    0 & 0 & 1 & 0 \\
                    0 & 0 & 0 & 1
                    \end{pmatrix} \\
                    &= \begin{pmatrix}
                    \gamma & +\beta \gamma & 0 & 0 \\
                    -\beta \gamma & -\gamma & 0 & 0 \\
                    0 & 0 & -1 & 0 \\
                    0 & 0 & 0 & -1
                    \end{pmatrix}
                    \begin{pmatrix}
                    \gamma & -\beta \gamma & 0 & 0 \\
                    -\beta \gamma & \gamma & 0 & 0 \\
                    0 & 0 & 1 & 0 \\
                    0 & 0 & 0 & 1  
                    \end{pmatrix} \\
                    &= \begin{pmatrix}
                    1 & 0 & 0 & 0 \\
                    0 & -1 & 0 & 0 \\
                    0 & 0 & -1 & 0 \\
                    0 & 0 & 0 & -1
                    \end{pmatrix} = g \, 
            \end{align*}
            wobei wir $\gamma^2 = 1 - \beta^2$ nutzen.

            \item  $\text{sgn} \Lambda^0_{\phantom{0} 0} = 1$

            Das stimmt offensichtlich, weil $\Lambda^0_{\phantom{0} 0} = \gamma > 0$ immer gilt.

            \item  $\det \Lambda = 1$ 
            \begin{align*}
                \det \Lambda &= \det \begin{pmatrix}
                    \gamma & -\beta \gamma & 0 & 0 \\
                    -\beta \gamma & \gamma & 0 & 0 \\
                    0 & 0 & 1 & 0 \\
                    0 & 0 & 0 & 1 \\
                \end{pmatrix} \\
                &= 1 \cdot \begin{vmatrix}
                    \gamma & -\beta \gamma & 0 \\
                    -\beta \gamma & \gamma & 0 \\
                    0 & 0 & 1 \\
                    \end{vmatrix} \\
                &= \gamma^2 - \beta^2 \gamma^2 = 1 \,.
            \end{align*}
        \end{enumerate}
        \textit{2 Punkte}
    }
\end{enumerate}
\end{document}