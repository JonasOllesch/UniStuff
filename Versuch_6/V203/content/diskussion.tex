\section{Diskussion}
\label{sec:Diskussion}
Bei der Auswertung der Messung bis 15 $\unit{\bar}$ gibt es zwei mathematische Lösungen. Physikalisch gibt es nur eine sinnvolle Lösung und zwar die positive Wurzel \autoref{fig:L+}, da die Verdampfungswärme bei zunehmender Temperatur geringer werden muss. 

Bei einem Literaturwert von ${4,0626} \cdot 10^4 \dfrac{\unit{\joule}}{\unit{\mol}}$ \cite{chemiede} hat die Messung eine relative Abweichung von $15\%$ ergeben.
Diese Abweichung kann zu Teilen dadurch erklärt werden, dass die Verdampfungswärme als Konstante angenommen wurde.
Bei Wasser ist die Verdampfungswärme jedoch abhängig von der Temperatur.
Weiter kann angenommen werden, dass der Glaskolben nicht vollständig luftdicht war, somit war ein Druckausgleich zwischen Kolben und Umwelt möglich.
Eine weitere Fehlerquelle ist das Druckmessgerät. Bei der Evakuierung war ein instantaner Druckabfall von ungefähr $400 \unit{\milli\bar}$ zu beobachten gewesen.
Der Druckabfall muss die Messung nicht verfälscht haben, aber er ist bemerkenswert.
Nach eine Temperatur von $48 \unit{\celsius}$ wurde die Heizung auf die maximale Stufe gestellt. Dieser Eingrifft könnte zu einem Knick in der Dampfdruckkurve geführt haben.
Eine weitere Fehlerquelle ist der Umgebungsdruck $p_0$. Er wurde während der gesammten Messung als konstant angesehen, aber eine Änderung kann nicht ausgeschlossen werden.