\section{Auswertung}
\label{sec:auswertung}

\subsection{Messung bis zu 1 bar}
\label{subsec:Auswertung_a}
Um die Verdampfungswärme $L$ zu bestimmen wird \eqref{eq:lnp} verwendet. 
Der Umgebungsdruck $p_0$ wurde zu Begin des Experimentes gemessen und beträgt 

\begin{equation*}
    p_0 = 1016 \cdot 10² \unit{\pascal}    \,.
\end{equation*} \\

Die aufgenommenen Messwerte sind in \autoref{tab:Messung1} aufgetragen.

\begin{table}[H]
    \centering
    \caption{Druckmessung für $p \leq 1 \,\unit{\bar}$.}
    \label{tab:Messung1}
    \begin{tabular}{S[table-format=2.0] S[table-format=3.0] S[table-format=3.0] S[table-format=3.0]}
      \toprule
        {$T \mathbin{/} \unit{\celsius}$} & {$p \mathbin{/} \unit{\milli\bar}$} & {$T \mathbin{/} \unit{\celsius}$} & {$p \mathbin{/} \unit{\milli\bar}$}\\
      \midrule
            18 & 23  & 60  & 227  \\
            19 & 37  & 61  & 233  \\
            20 & 42  & 62  & 249  \\
            21 & 46  & 63  & 254  \\
            22 & 49  & 64  & 262  \\
            23 & 53  & 65  & 270  \\
            24 & 56  & 66  & 280  \\
            25 & 60  & 67  & 296  \\
            26 & 63  & 68  & 310  \\
            27 & 66  & 69  & 322  \\
            28 & 69  & 70  & 336  \\
            29 & 74  & 71  & 348  \\
            30 & 77  & 72  & 363  \\
            31 & 81  & 73  & 377  \\
            32 & 84  & 74  & 396  \\
            33 & 89  & 75  & 409  \\
            34 & 91  & 76  & 427  \\
            35 & 95  & 77  & 448  \\
            36 & 99  & 78  & 449  \\
            37 & 103 & 79  & 480  \\
            38 & 107 & 80  & 503  \\
            39 & 112 & 81  & 521  \\
            40 & 116 & 82  & 540  \\
            41 & 120 & 83  & 562  \\
            42 & 124 & 84  & 583  \\
            43 & 129 & 85  & 606  \\
            44 & 133 & 86  & 630  \\
            45 & 138 & 87  & 651  \\
            46 & 144 & 88  & 679  \\
            47 & 149 & 89  & 707  \\
            48 & 156 & 90  & 733  \\
            49 & 161 & 91  & 760  \\
            50 & 166 & 92  & 791  \\
            51 & 174 & 93  & 817  \\
            52 & 178 & 94  & 850  \\
            53 & 185 & 95  & 882  \\
            54 & 190 & 96  & 916  \\
            55 & 196 & 97  & 945  \\
            56 & 202 & 98  & 983  \\
            57 & 207 & 99  & 1014 \\
            58 & 214 & 100 & 1048 \\
            59 & 220 & {-} & {-}  \\
      \bottomrule
    \end{tabular}
  \end{table}

Wenn nun $\ln{\left(\dfrac{p}{p_0} \right)}$ gegen $\dfrac{1}{T}$ aufgetragen wird, kann die Verdampfungswärme $L$ durch die in \autoref{fig:linreg1} dargestellte lineare Regression ausgerechnet werden, die durch

\begin{equation}
  y = a x + b
  \label{eq:linreg}
\end{equation} gegeben ist.

\begin{figure}
  \centering
  \includegraphics{build/Messung_a.pdf}
  \caption{Lineare Regression für $p \leq 1 \,\unit{\bar}$.}
  \label{fig:linreg1}
\end{figure}

Dabei sind die Parameter gegeben durch 

\begin{align*}
a & = (-4.24 \pm 0.04) \cdot 10^3 \, \unit{\kelvin}\\
b & = 11.31 \pm 0.13 \,.
\end{align*}

Aus dem Vergleich von \eqref{eq:lnp} und \eqref{eq:linreg} wird der Zusammenhang

\begin{equation*}
  a = - \frac{L}{R} \Leftrightarrow L = -a R
\end{equation*} deutlich.

Daraus bestimmt sich die Verdampfungswärme $L$ zu

\begin{equation*}
  L = (3.524 \pm 0.035) \cdot 10^4 \,\unit{\frac{\joule}{\mol}} \,.
\end{equation*} \\

Die benötigte Energie, um das Volumen eines Mol des Wassers auf ein Mol Wasserdampf zu vergrößern, wird
als äußere Wärmemenge bezeichnet und ist durch

\begin{equation}
  L_a = W = pV = RT = 3101.295 \,\unit{\frac{\joule}{\mol}}
  \label{eq:äußereVW}
\end{equation} gegeben.

Aus der Differenz der Verdampfungswärme $L$ sowie der äußeren Verdampfungswärme $L_a$ ergibt sich die
innere Verdampfungswärme $L_i$, die zur Überwindung der molekularen Bindungskräfte nötig ist, zu 
\begin{equation}
  L_i = L - L_a = (3.214 \pm 0.035) \cdot 10^4 \,\unit{\frac{\joule}{\mol}} \,.
  \label{eq:innereVW}
\end{equation} \\

Durch Division mit der Avogadrokonstante $N_A = 6.022 \cdot 10^{23} \,\unit{\mol}^{-1}$ ergibt sich
mit $1 \,\unit{\electronvolt} = 1,602 \cdot 10^{-19} \,\unit{\joule}$

\begin{equation}
  L_i = 0.333 \pm 0.004 \,\unit{\frac{\joule}{\mol}} \,.
\end{equation}

\subsection{Messung bis 15 bar}
Um die Temperaturabhängigkeit der Verdampfungswärme $L$ festzustellen, wird \eqref{eq:clauclapL} verwendet.
Außerdem darf $V_F$ gegenüber $V_D$ vernachlässigt werden.
Mit \eqref{eq:approxV_D} und 
\begin{equation*}
  A = 0.9 \,\unit{\frac{\joule\cubic\meter}{\square\mol}}
\end{equation*} folgt

\begin{equation}
  V_D = \frac{RT}{2p} \pm \sqrt{\frac{R^2 T^2}{4 p^2} - \frac{A}{p}}\,.
\end{equation} \\

Einsetzen in \eqref{eq:clauclapL} liefert nun

\begin{align}
  L &= T \left(\frac{RT}{2p} \pm \sqrt{\frac{R^2 T^2}{4 p^2} - \frac{A}{p}} \right) 
  \frac{\mathrm{d}p}{\mathrm{d}T} \\
    &= \frac{T}{p} \left(\frac{RT}{2} \pm \sqrt{\left(\frac{R T}{2} \right)^2 - A p}\right) 
    \frac{\mathrm{d}p}{\mathrm{d}T} \,. 
    \label{eq:VWMess2}
\end{align}

\newpage

Die von $p = 1 \,\unit{\bar}$ bis $p = 15 \,\unit{\bar}$ aufgenommenen Messwerte sind in 
\autoref{tab:Messung2} aufgetragen.

\begin{table}[H]
  \centering
  \caption{Messreihe für  $1 \,\unit{\bar} \leq p \leq 15 \,\unit{\bar} $.}
  \label{tab:Messung2}
  \begin{tabular}{S S}
    \toprule
    {$p \mathbin{/} \unit{\bar}$} & {$T \mathbin{/} \unit{\celsius}$} \\
    \midrule
     1 & 121 \\
     2 & 134 \\
     3 & 143 \\
     4 & 152 \\
     5 & 157 \\
     6 & 162 \\
     7 & 166 \\
     8 & 171 \\
     9 & 176 \\
    10 & 179 \\
    11 & 183 \\
    12 & 187 \\
    13 & 189 \\
    14 & 191 \\
    15 & 192 \\
    \bottomrule
  \end{tabular}
\end{table}

Zur Bestimmung von $\frac{\mathrm{d}p}{\mathrm{d}T}$ in \eqref{eq:VWMess2} wird ein Ausgleichspolynom
3. Grades der Form
\begin{equation*}
  p(T) = a T^3 + b T^2 + c T + d
\end{equation*} verwendet.

Die Regression ist in \autoref{fig:linreg2} dargestellt.

\begin{figure}
  \centering
  \includegraphics{build/Messung_b.pdf}
  \caption{Lineare Regression für $1 \,\unit{\bar} \leq p \leq 15 \,\unit{\bar}$.}
  \label{fig:linreg2}
\end{figure}

Werden $p(T)$ und 
\begin{equation*}
  \frac{\mathrm{d}p}{\mathrm{d}T} = 3 a T^2 + 2 b T + c
\end{equation*} nun in \eqref{eq:VWMess2} eingesetzt, ergeben sich zwei mögliche Verläufe der 
Verdampfungswärme.
Wird die Wurzel addiert, ergibt sich der in \autoref{fig:L+}, bei Subtraktion der
in \autoref{fig:L-} dargestellte Kurvenverlauf.

\begin{figure}
  \centering
  \includegraphics{build/L_positiv.pdf}
  \caption{Verlauf der Verdampfungswärme bei Addition der Wurzel.}
  \label{fig:L+}
\end{figure}

\begin{figure}
  \centering
  \includegraphics{build/L_negativ.pdf}
  \caption{Verlauf der Verdampfungswärme bei Subtraktion der Wurzel.}
  \label{fig:L-}
\end{figure}





