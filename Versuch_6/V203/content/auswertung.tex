\section{Auswertung}
\label{sec:auswertung}

\subsection{Messung bis zu 1 bar}
\label{subsec:Auswertung_a}
Um die Verdampfungswärme $L$ zu bestimmen wird 
%\eqref{eq:..} 
verwendet. Wenn nun $\log{(\dfrac{\unit{p}}{\unit{p_0}})}$ gegen $\dfrac{1}{T}$ aufgetragen wird, kann die Verdampfungswärme $L$ durch eine einfache lineare Regression ausgerechnet werden. 
Der Umgebungsdruck $p_0$ wurde zu Begin des Experimentes gemessen und beträgt 

\begin{equation*}
    p_0 = 1016 \cdot 10² \unit{\pascal}    \,.
\end{equation*}

\begin{table}[H]
    \centering
    \caption{Druck bei weniger als ein Bar.}
    \begin{tabular}{S S S S}
      \toprule
        {$T \mathbin{/} \unit{\celsius}$} & {$p \mathbin{/} \unit{\milli\bar}$} & {$T \mathbin{/} \unit{\celsius}$} & {$p \mathbin{/} \unit{\milli\bar}$}\\
      \midrule
            18          &23             & 60          &227     \\
            19          &37             & 61          &233     \\
            20          &42             & 62          &249     \\
            21          &46             & 63          &254     \\
            22          &49             & 64          &262     \\
            23          &53             & 65          &270     \\
            24          &56             & 66          &280     \\
            25          &60             & 67          &296     \\
            26          &63             & 68          &310     \\
            27          &66             & 69          &322     \\
            28          &69             & 70          &336     \\
            29          &74             & 71          &348     \\
            30          &77             & 72          &363     \\
            31          &81             & 73          &377     \\
            32          &84             & 74          &396     \\
            33          &89             & 75          &409     \\
            34          &91             & 76          &427     \\
            35          &95             & 77          &448     \\
            36          &99             & 78          &449     \\
            37          &103            & 79          &480     \\
            38          &107            & 80          &503     \\
            39          &112            & 81          &521     \\
            40          &116            & 82          &540     \\
            41          &120            & 83          &562     \\
            42          &124            & 84          &583     \\
            43          &129            & 85          &606     \\
            44          &133            & 86          &630     \\
            45          &138            & 87          &651     \\
            46          &144            & 88          &679     \\
            47          &149            & 89          &707     \\
            48          &156            & 90          &733     \\
            49          &161            & 91          &760     \\
            50          &166            & 92          &791     \\
            51          &174            & 93          &817     \\
            52          &178            & 94          &850     \\
            53          &185            & 95          &882     \\
            54          &190            & 96          &916     \\
            55          &196            & 97          &945     \\
            56          &202            & 98          &983     \\
            57          &207            & 99          &1014    \\
            58          &214            & 100         &1048    \\
            59          &220            & {-}           &{-}       \\
      \bottomrule
    \end{tabular}
  \end{table}