\section{Diskussion}
\label{sec:Diskussion}


Verglichen mit einem Theoriewert von $\tau_{\mu,\text{theo}} \approx 2,197 \,\unit{\micro\second}$ \cite{pdg} weicht die experimentell bestimmte Lebensdauer von
$\tau_{\mu,\text{exp}} =  \left(2,090  \pm 0,090 \right) \,\unit{\micro\second}$ um $4,869 \,\%$  ab.

Es wird eine geringere experimentelle Lebensdauer erwartet, da $\mu^{-}$ von Atomen eingefangen werden können. 
Dort können sie durch die Überlagerung ihrer Wellenfunktion mit der vom Kern durch diesen absorbiert werden und durch den Kanal
%\[
%\ce{$\mu^{-}$ + p -> n + $\bar{{\nu}_\mu$}}
%\]
\begin{equation*}
    \mu^{-} +  p \rightarrow  n + \bar{{\nu}_\mu}
\end{equation*}
zerfallen, der nicht durch die PET registriert werden kann.

\subsection{Untergrundrate}

Anhand \eqref{eq:theo_unter} für die theoretische Bestimmung des Untergrundes ergibt sich ein Wert von
\begin{equation*}
    B_\text{theo} = 1,226 \,.
\end{equation*}

Verglichen mit dem experimentell bestimmten Wert von $B_\text{exp} = 0,340 \pm \, 0,267$ besteht eine relative Abweichung von
$72,268 \,\%$. Aus diesem Wert lässt sich ableiten, dass zusätzliche Effekte berücksichtigt werden müssen, um den theoretischen und den experimentellen Wert in Einklang zu bringen.

Die ersten drei Kanäle des MCA werden bei der Ausgleichsfunktion nicht berücksichtigt, da diese aufgrund der aufgebauten Schaltung keine Einträge beinhalten können.
Auch sind die Zählungen in den folgenden Kanälen geringer als erwartet, was die Vermutung zulässt, dass der MCA für geringe Zeiten eine schlechte Auflösung hat.
Auffällig ist auch, dass die Kanäle $31$ und $33$ überbesetzt sind. Ein eindeutiger Grund kann dafür nicht angegeben werden.
Hinzu kommt, dass am TAC $N_\text{stop} = 5046$ Stoppsignale, am MCA aber nur $N_{\mu} = 3235$ Myonen registriert wurden.
Dieser Unterschied entsteht durch die Diskrepanz der Suchzeit und des am TAC eingestellten Zeitfensters.



\subsection{Halbwertsbreite}

Die Pulse, die die Verzögerungsleitungen und die Diskriminatoren durchlaufen haben die Form einer Gaußglocke.
Diese Pulse haben eine eingestellte Breite von $\Delta t = \SI{10}{\nano\second}, \,\SI{15}{\nano\second}, \, \SI{20}{\nano\second}\,.$
und können durch verschieden lange Verzögerungsleitungen gegeneinader verschoben werden.
Wenn beide Pulse überlappen und eine Schwellspannung überschreiten, gibt die Koinzidenz ein Signal aus.
Es wird ein Plateau erwartet, wo sich die beiden Pulse überlappen.
Wenn es sich bei den einzelnen Pulsen um Rechteckspannungen handelt, wird ein Plateau mit der doppelten Breite einzelnen Pulse erwartet.
Es wird davon ausgegangen, dass gaußförmige Pulse das Plateau verschmieren.
Die theoretischen und experimentellen Halbwertsbreiten sind in \autoref{tab:ETH} zu sehen.

\begin{table}[H]
    \centering
    \caption{Experimentell und theoretische Halbwertsbreiten und ihre Abweichung für verschiedene Koinzidenzen.}
    \label{tab:ETH}
    \begin{tabular}{c c c c}
    \toprule
     $t_\text{Koin} \mathbin{/} \unit{\nano\second}$&$\Delta t_\text{Halb}^{\text{FWHM, theo}} \mathbin{/} \unit{\nano\second}$ &$\Delta t_\text{Halb}^{\text{FWHM, exp}} \mathbin{/} \unit{\nano\second}$  & $\left|\Delta t_\text{Halb}^{\text{FWHM, theo}} -  \Delta t_\text{Halb}^{\text{FWHM, exp}} \right| \mathbin{/} \unit{\nano\second} $ \\
    \midrule
        $10$&$ 20$ &$15,32 \pm 2,88$ & $ 4,68$\\ 
        $15$&$ 30$ &$21,64 \pm 1,68$ & $ 8,36$\\ 
        $20$&$ 40$ &$40,39 \pm 0,15$ & $ 0,39$\\ 
    \bottomrule
    \end{tabular}
    \end{table}

Es ist zu erkennen, dass sich die erwartete Halbwertsbreite und die theoretische teilweise stark unterscheiden. 
Dabei wird die Breite bei kleinen Koinzidenzen unterschätzt. 

Bei einer Koinzidenz von $\SI{40}{\nano\second}$ liegen beide Werte mit einer relativen Abweichung von $0,98 \, \%$ nah beieinander.

