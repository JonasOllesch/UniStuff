\section{Diskussion}
\label{sec:Diskussion}

\subsection{Verzögerungszeit}

Als Verzögerungszeit der Verzögerungsleitungen wurde $\Delta t = 3 \,\unit{\nano\second}$ gewählt.
Wie aus dem Fit der gemessenen Ereignisse pro Verzögerungen abzuleiten ist, liegt die optimale Verzögerungszeit bei
$\Delta t = Hier könnte Ihr Wert stehen!! \,\unit{\nano\second}$.


\subsection{Vergleich der Untergrundraten}

Anhand \textbf{Formel referenzieren!} für die theoretische Bestimmung des Untergrundes ergibt sich ein Wert von
\begin{equation}
    U_\text{theo} = irgendwas \,\frac{\text{counts}}{\unit{\second}} \,.
\end{equation}

Verglichen mit dem experimentell bestimmten Wert von $U_\text{exp} = irgendwas \,\frac{\text{counts}}{\unit{\second}}$ besteht eine Abweichung von
$irgendwas \,\%$.

\subsection{Vergleich der Lebensdauern}

Verglichen mit einem Theoriewert von $\tau_{\mu,\text{theo}} \approx 2,196 981 1 \,\unit{\micro\second}$ \cite{pdg} weicht die experimentell bestimmte Lebensdauer von
$\tau_{\mu,\text{exp}} =  \left(2,090  \pm 0,090 \right) \,\unit{\micro\second}$ um $irgendwas \,\%$ ab.


\subsection{Sonstige Effekte}

Wie in \autoref{sec:theorie} erläutert verringert sich durch die Bildung myonischer Atome die Lebensdauer
negativ geladener Myonen. Dadurch fällt der bestimmte Wert etwas niedriger aus. \\

Hinzu kommt, dass am TAC $N_\text{stop} = 5046$ Stoppsignale, am MCA aber nur $N_{\mu} = 3235$ Myonen registriert wurden. Das spricht dafür,
dass die Funktionsweise des MCAs für geringe Zeiten eingeschränkt ist.
