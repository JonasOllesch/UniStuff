\section{Diskussion}
\label{sec:Diskussion}

\subsection{Verzögerungszeit}

Als Verzögerungszeit der Verzögerungsleitung wurde $\Delta t = 3 \,\unit{\nano\second}$ gewählt.


\subsection{Vergleich der Untergrundraten}

Anhand \eqref{eq:theo_unter} für die theoretische Bestimmung des Untergrundes ergibt sich ein Wert von
\begin{equation}
    B_\text{theo} = 1,226 \,.
\end{equation}

Verglichen mit dem experimentell bestimmten Wert von $B_\text{exp} = 0,340 \pm \, 0,267$ besteht eine relative Abweichung von
$72,268 \,\%$. Aus diesem Wert lässt sich ableiten, dass zusätzliche Effekte berücksichtigt werden müssen, um den theoretischen und den experimentellen Wert in Einklang zu bringen.

\subsection{Vergleich der Lebensdauern}

Verglichen mit einem Theoriewert von $\tau_{\mu,\text{theo}} \approx 2,197 \,\unit{\micro\second}$ \cite{pdg} weicht die experimentell bestimmte Lebensdauer von
$\tau_{\mu,\text{exp}} =  \left(2,090  \pm 0,090 \right) \,\unit{\micro\second}$ um $4,869 \,\%$  ab.

Eine weitere Fehlerquelle ist der MCA. Die ersten drei Kanäle wurden bei der Ausgleichsfunktion nicht berücksichtigt, da diese aufgrund der aufgebauten Schaltung keine Einträge beinhalten können.
 Nach einem exponentiellen Abfall, wären hier die meisten Einträge zu erwarten. 
Auch sind die Zählungen in den folgenden Kanälen geringer als erwartet, was die Vermutung zulässt, dass der MCA für geringe Zeiten eine schlechte Auflösung hat.
Auffällig ist auch, dass die Kanäle $31$ und $33$ überbesetzt sind. Ein eindeutiger Grund kann dafür nicht angegeben werden.
Hinzu kommt, dass am TAC $N_\text{stop} = 5046$ Stoppsignale, am MCA aber nur $N_{\mu} = 3235$ Myonen registriert wurden.
Dieser Unterschied entsteht durch die Diskrepanz der Suchzeit und des am TAC eingestellten Zeitfensters.

Wie in \autoref{sec:theorie} erläutert verringert sich durch die Bildung myonischer Atome die Lebensdauer
negativ geladener Myonen. Dadurch fällt der bestimmte Wert etwas niedriger aus. \\

%%%%% Auswertung der Breite?? Was möchte er da haben??
% Ich habe keine Ahnung :(
