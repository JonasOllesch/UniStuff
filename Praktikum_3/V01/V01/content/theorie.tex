\section{Theoretische Grundlagen}
\label{sec:theorie}

\subsection{Entstehung und Zerfall kosmischer Myonen}

Treffen Protonen kosmischer Höhenstrahlung auf die Moleküle der Luft, kann ein hadronischer Teilchenschauer ausgelöst werden, bei dem unter anderem Pionen und Kaonen entstehen. \\
Die hier betrachteten, in einer Höhe von etwa $10 - 15 \,\unit{\kilo\meter}$ \cite{grup} entstehenden, kosmischen Myonen sind ein Teil dieses Luftschauers, der direkt aus den Pion- und Kaonzerfällen entsteht. \\

Unter Erzeugung eines Myons und Myonneutrinos zerfallen diese wie folgt:

\begin{align*}
    \pi^- &\rightarrow \mu^- + \bar{\nu}_\mu \\
    \pi^+ &\rightarrow \mu^+ + \nu_\mu \\
    K^+ &\rightarrow \mu^+ + \nu_\mu \,.
\end{align*}

Myonen besitzen ähnliche Eigenschaften wie Elektronen, mit dem Unterschied, dass Myonen etwa die 200-fache Masse \cite{pdg} besitzen.
Mit einer Lebensdauer von $2,187 \cdot 10^{-6} \,\unit{\second}$ \cite{pdg} sollten die kosmischen Myonen nicht in der Lage sein, die Erdoberfläche zu erreichen.
Die hier entstehenden Myonen bewegen sich hochrelativistisch, erfahren also Zeitdilatation.
Es erreicht $1 \,\text{Myon} \, (\unit{\centi\meter}^2 \, \unit{\minute})^{-1}$ die Erdoberfläche \cite{grup}. \\

Myonen können es unter Austausch eines W-Bosons über den folgenden Kanal zerfallen:

\begin{align*}
    \mu^+ &\rightarrow e^+ + \nu_e + \bar{\nu}_\mu \\
    \mu^- &\rightarrow e^- + \bar{\nu}_e + \nu_\mu\,, 
\end{align*}

wobei immer ein Neutrino und ein Antineutrino entsteht, sodass die Leptonzahl erhalten bleibt. \\
Für positive Myonen ist dieser Zerfall der einzig mögliche, negative Myonen in Ruhe können dagegen auch durch Myoneinfang zerfallen.
Das eingefangene Myon löst aufgrund seiner gegenüber Elektronen größeren Masse eine Elektronenkaskade aus,
bis es auf das niedrigste Energieniveau fällt und vom Kern unter Entstehung eines Neutrons und eines Neutrinos eingefangen wird \cite{myoeifa}:

\begin{equation*}
    \mu^- + p \rightarrow n + \nu_\mu \,.
\end{equation*}

Dadurch erhält das negative Myon genau genommen eine geringere Lebensdauer.


\subsection{Berechnung der Lebensdauer}

Zur Berechnung der Lebensdauer wird im Folgenden von einem exponentiellen Zusammenhang der Form
\begin{equation}
    N(t) = N_0 \, e^{-\lambda \, t}
    \label{eq:expdec}
\end{equation}
zwischen der Anzahl der Myonen zum Zeitpunkt $t_0 = 0$ und der Zeit $t$ ausgegangen. 
Dabei bezeichnet $N_0$ die Anzahl der Myonen zu Beginn, $N(t)$ die Anzahl nach einer Zeit $t$ und $\lambda$ die charakteristische Zerfallskonstante des Prozesses. \\

Die Lebensdauer $\tau$ lässt sich mit dem zeitlichen Erwartungswert $<t>$ identifizieren.
Es gilt
\begin{equation}
    \tau = <t> = \int_0^\infty \text{d} t \, \lambda \, t \, e^{-\lambda \, t} = \frac{1}{\lambda} \,.
\end{equation} 

Für diesen Versuch muss \eqref{eq:expdec} um einen Untergrund $B$ ergänzt werden, um alle gezählten Ereignisse zu beschreiben, sodass
\begin{equation}
    N(t) = N_0 \, e^{-\lambda \, t} + B
    \label{eq:exponential_decay}
\end{equation}
gilt. \\

Zur Abschätzung dieses Hintergrundes treffe innerhalb einer Suchzeit $T_\text{s}$ nach Eintreffen des ersten Myons ein weiteres als Untergrund ein.
Für die durchschnittliche Anzahl gemessener Myonen pro Sekunde während der gesamten Messzeit $T_\text{mess}$ gilt
\begin{equation*}
    n = \frac{N_\text{start}}{T_\text{mess}} \,,
\end{equation*}
wobei $N_\text{start}$ die Anzahl der gegebenen Startsignale, also die Anzahl der eintreffenden ersten Myonen beschreibt. \\

Das Produkt $T_\text{s} \cdot n$ beschreibt den Erwartungswert der Messungen in der Suchzeit. \\

Damit lässt sich die poissonverteilte Wahrscheinlichkeit für das Eintreffen $k$ weiterer Myonen durch
\begin{equation*}
    P(k) = \frac{(T_\text{s} \cdot n)^k}{k!} \, e^{T_\text{s} \cdot n}
\end{equation*}
beschreiben. \\

Über
\begin{equation*}
    N_\text{fehl} (k) = N_\text{start} \cdot P(k)
\end{equation*}
lässt sich so die Anzahl der Fehlereignisse berechnen, wobei hier insbesondere $P(1)$ wichtig ist. \\
Mit $P(1)$ ergibt sich
\begin{equation}
    N_\text{fehl} (1) =  {N_\text{start}}²  \cdot \frac{T_\text{s}}{T_\text{mess}} \, e^{{ N_\text{start}} \cdot \frac{T_\text{s}}{T_\text{mess}}}
    \label{eq:theo_unter}
\end{equation}


\newpage

\subsection{Erklärung wichtiger Bauelemente}

Die Erklärung der im Versuch verwendeten Bauelemente erfolgt nach der in \autoref{fig:schaltung} erkennbaren Reihenfolge.

\begin{figure}[H]
    \centering
    \includegraphics[width=.6\textwidth]{figures/V01.pdf}
    \caption{Schaltbild der verwendeten Messelektronik \cite{ap03}.}
    \label{fig:schaltung}
\end{figure}

\subsubsection{Szintillator}

Bei dem ersten verwendeten Bauteil handelt es sich um einen flüssigen organischen Szintillator. Der Unterschied zu anorganischen Szintillatoren besteht darin, dass hier nicht das Szintillatormaterial
ionisiert wird und sich ein Elektron zum Aktivierungszentrum bewegen muss, um ein Photon auszulösen, sondern dass das Szintillatormaterial auf ein höheres Energieniveau angeregt wird.
Diese Anregungsenergie wird innerhalb einiger Nanosekunden wieder freigesetzt \cite{kolawerm}. \\
Die Zeitauflösung organischer Szintillatoren ist deutlich höher als die anorganischer Szintillatoren.
Aufgrund der Vielfalt unterschiedlicher Anregungsniveaus organischer Szintillatoren ist die Energieauflösung verglichen mit den diskreten Ionisationsenergien der anorganischen Szintillatoren geringer.
Für die Lebensdauermessung ist besonders die Zeitauflösung wichtig, also eignen sich organische Szintillatoren am besten. \\


\subsubsection{Photomultiplier}

Der Photomultiplier (PM) wandelt mithilfe des Photoeffekts ein eintreffendes Photon in ein elektrisches Signal um.
Über eine anliegende Beschleunigungsspannung wird das vom Photon ausgelöste Elektron zur Anode beschleunigt.
Im Sekundärelektronenvervielfacher, der aus mehreren aufeinanderfolgen Elektroden in zunehmend positivem Potential löst es weitere Elektronen aus \cite{myoeifa}.
An der Anode lassen sich diese Elektronen als Strom messen.
Insgesamt wird das vom Szintillator eintreffende Signal um einen Faktor von bis zu $10^9$, für gewöhnlich um $10^5$ - $10^7$-fach, verstärkt \cite{kolawerm}.


\subsubsection{Verzögerungsleitung}

In den Verzögerungsleitungen liegen Leitungen unterschiedlicher Länge, sodass das eintreffende Signal mit Schaltern über die Leitungen umgeleitet werden kann.
Das Signal legt eine längere Strecke zurück und ist damit verzögert. Die hier verwendeten Verzögerungsleitungen verfügen über sieben unterschiedliche Schalter, mit denen Einzelverzögerung
zwischen $0,5 \,\unit{\nano\second}$ und $32 \,\unit{\nano\second}$ eingestellt werden können. Diese Schalter funktionieren additiv.


\subsubsection{Diskriminator}

Die verbauten Diskriminatoren verfügen über eine regelbare Schwellamplitude. Alle eintreffende Impulse unter dieser Amplitude werden nicht weitergegeben, sodass Störsignale,
die beispielsweise durch zufällig ausgelöste Elektronen im PM entstehen, unterdrückt.


\subsubsection{Koinzidenz}

Die Koinzidenz liefert nur dann einen Output, wenn an allen Eingängen gleichzeitig ein Eingangssignal eintrifft.
Es operiert wie ein Und-Gatter.

\subsubsection{Monostabile Kippstufe}

Die monostabile Kippstufe, auch Monoflop genannt, besitzt zwei unterschiedliche Zustände. Trifft ein Impuls am Monoflop ein, wird dieser von seinem Grundzustand in den angeregten Zustand geschaltet,
in dem er für eine definierbare Zeit verbleibt, bevor er sich wieder umschaltet. Dabei gibt er in beiden Positionen ein Signal aus.


\subsubsection{Time-Amplitude-Converter}

Der Time-Amplitude-Converter (TAC) verfügt über zwei Eingänge. Treffen zwei Signale im Abstand $\Delta t$ ein, gibt er ein Ausgangssignal, dessen Amplitude proportional zu $\Delta t$  ist.


\subsubsection{Vielkanalanalysator}

Der Vielkanalanalysator (engl. Multichannel Analyser [kurz MCA]) sortiert Eingangssignale nach ihrer Amplitude in unterschiedliche Kanäle und erstellt so ein Histogramm der Eingangssignale.





