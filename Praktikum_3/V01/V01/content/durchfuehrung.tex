\section{Durchführung}
\label{sec:Durchführung}

Die Messapparatur wird \autoref{fig:schaltung} entsprechend aufgebaut.
Der flüssige Szintillator befindet sich dabei in einem Tank von etwa $50 \,\unit{\liter}$ Volumen.
Um zu gewährleisten, dass die Photomultiplier ordnungsgemäß funktionieren, werden diese direkt an das Oszilloskop angeschlossen und es wird überprüft, ob von beiden PMT ein Signal eintrifft. \\

Anschließend werden die Schwellspannung der Diskriminatoren zur Rauschunterdrückung so eingestellt, dass etwa $30$ Impulse pro Sekunde eintreffen.
Um Fehler gering zu halten, wird diese Messung über ein hinreichend großes Zeitfenster durchgeführt. \\

Zur Justierung der Koinzidenz wird an den Diskriminatoren zunächst eine Pulsdauer von $\Delta t = 20 \,\unit{\nano\second}$, dann eine Pulsdauer von $\Delta t = 15 \,\unit{\nano\second}$ und letztlich eine
Pulsdauer von $\Delta t = 10 \,\unit{\nano\second}$ eingestellt.
An den Verzögerungsleitungen wird schrittweise die Verzögerung erhöht und dabei die Pulsrate gemessen, erneut über ein hinreichend großes Zeitintervall.
Dabei wird zunächst die linke, dann die rechte Verzögerungsleitung verwendet, aber zu keinem Zeitpunkt beide Leitungen gleichzeitig.
Es sollte sich eine Messreihe ergeben, die ein Plateau besitzt. \\

Zur Durchführung des restlichen Versuches wird eine Verzögerung gewählt, die bestenfalls im Bereich von circa $20$ Impulsen pro Sekunde im Zentrum des gemessenen Plateaus liegt.
Der verbleibende Teil der Schaltung wird aufgebaut und es wird eine Suchzeit von $T_\text{s} = 15 \,\unit{\micro\second}$ eingestellt, sodass auch Myonenzerfälle, die weit über der erwarteten Lebendauer
liegen, vom TAC erfasst werden, dessen Messbereich entsprechend angepasst wird. \\

Um den MCA zu kalibrieren, werden die angeschlossenen Diskriminatoren durch einen Doppelimpulsgenerator ersetzt.
Durch Messung der Kanalzuordnungen je nach Impulslänge können zur Lebensdauermessung unterschiedliche Kanäle unterschiedlichen Lebensdauern zugeteilt werden. \\

Zur eigentlichen Messung werden die Diskriminatoren erneut angeschlossen und die Messung am MCA sowie am Impulszähler zeitgleich gestartet.




