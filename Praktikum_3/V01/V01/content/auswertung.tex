\section{Auswertung}
\label{sec:auswertung}


Die Messung lief insgesamt $t = 160955 \, \unit{\second}$. Das entspricht $44,71 \, \unit{\hour}$. In dieser Zeit wurden $N_{start} = 2595364$ Startsignale und $N_{stop} = 5046$ Stoppsignale gemessen.
Während der Messzeit wurden insgesamt $N_{\mu} = 3235 $ Myonen registriert.

\subsection{Kalibrierung des MCA}
\label{sec:KaliMCA}

Der MCA wird mithilfe eines Doppelimpulsgenerator justiert. Doppelimpulsgenerator erzeugt zwei Impulse mit einem festen Abstand.
Diese Impulse werden von MCA je nach zeitlichen Abstand einem Kanal zugeordnet. In \autoref{tab:KaliMCA_tab} sieht man die Kanalnummer und den dazugehörenden Abstand der Impulse.

\begin{table}
    \centering
    \caption{Einordnung der Impulsabstände in die Kanäle des MCA}
    \label{tab:KaliMCA_tab}
    \begin{tabular}{c c}
    \toprule
     Kanal &  $\text{Zeit} \mathbin{/} \unit{\micro\second}$ \\
    \midrule
     430,0 &                             9,5 \\
     384,0 &                             8,5 \\
     338,0 &                             7,5 \\
     292,0 &                             6,5 \\
     246,0 &                             5,5 \\
     200,0 &                             4,5 \\
     154,0 &                             3,5 \\
     108,0 &                             2,5 \\
      62,0 &                             1,5 \\
      16,0 &                             0,5 \\
    \bottomrule
    \end{tabular}
    \end{table}

Die Zeit für die Kanäle, die zwischen den bestimmten Kanälen liegt, werden durch eine lineare Regression bestimmt \autoref{fig:lin_reg_Channel_zeit}.
Die Parameter des Fits sind durch 
\begin{equation*}
    \text{Impulsabstände} = \left(0,021 \pm 0,000 \right) \unit{\micro\second} \cdot  \text{Kanalnummer} + \left(0,152 \pm 0,000 \right) \unit{\micro\second}
    \label{eq:Kanal_Zeit_Rechnung}
\end{equation*}
gegeben.

\begin{figure}[H]
    \centering
    \includegraphics{build/Channel_Kalibrierung.pdf}
    \caption{Lineare Regression durch die Kanäle des MCA und die dazugehörenden Zeiten.}
    \label{fig:lin_reg_Channel_zeit}
\end{figure}

\subsection{Bestimmung der Lebensdauer}
\label{sec:Bes_Leb}

Zunächst werden die Kanalnummer mit der Gleichung \autoref{eq:Kanal_Zeit_Rechnung} in Zeiten umgerechnet.
In dem Plot \autoref{fig:expon} sind die Anzahl der Myonen, die in einen Kanal gemessen wurden, gegen die Lebensdauer dieser Myonen aufgetragen.
Es wird eine Ausgleichsfunktion der Form von \autoref{eq:exponential_decay} gefittet.

\begin{figure}[H]
    \centering
    \includegraphics{build/Myonenzerfall_Zeit.pdf}
    \caption{Exponentieller Zerfall der Myonen.}
    \label{fig:expon}
\end{figure}

Die Parameter des Fits sind gegeben durch:

\begin{align*}
    &N_0         &=& 34,251 \pm \, 0,758                                 \\
    &\lambda     &=& (0,478 \pm \, 0,0206)  \, \unit{\dfrac{1}{\micro\second}} \\
    &B           &=& 0,340  \pm \, 0,267
\end{align*}

Aus der Inversen der Zerfallskonstanten wird die Lebensdauer berechnet
\begin{equation*}
    \tau = \dfrac{1}{\lambda} = \left(2,090  \pm 0,090 \right) \unit{\micro\second} \,.
\end{equation*}