\section{Auswertung}
\label{sec:auswertung}


Die Messung lief insgesamt $t = 160955 \, \unit{\second}$. Das entspricht $44,71 \, \unit{\hour}$. In dieser Zeit wurden $N_\text{start} = 2595364$ Startsignale und $N_\text{stop} = 5046$ Stoppsignale gemessen.
Während der Messzeit wurden insgesamt $N_{\mu} = 3235 $ Myonen registriert.

\subsection{Auswahl der Verzögerungszeit}
\label{sec:Verzögerungszeit}
Um eine angebrachte Verzögerungszeit zu finden, wird die Verzögerungszeit variiert und die Zählrate hinter der Koinzidenz gemessen.
Die Messwerte sind in \autoref{fig:koin_10}, \autoref{fig:koin_15} und \autoref{fig:koin_20} aufgetragen. Die Unsicherheit ist durch $ \sqrt{N}$ gegeben, wobei $N$ die Menge der gezählten Ereignisse ist.
Das negative Vorzeichen vor der Verzögerungszeit bedeutet, dass die linke Verzögerungsleitung verwendet wurde. Bei der rechten Verzögerungsleitung ist das Vorzeichen positiv.
Die Halbwertsbreite wurde mithilfe eines Polynoms des 8. Grades bestimmt. Die Regression wurde mit dem Package $scipy.optimize.curve\_ fit $  \cite{scipyop} durchgeführt. %%%%%%%%%%%%%%%%%%%%%%%%%%%%%%%%%%%%%% Warum 8. Grad??
Über die Gleichung \eqref{eq:Auflosung} kann aus der Halbwertsbreite die Auflösung berechnet werden. Die Ereignisse sind in \autoref{tab:Halbwertsbreiten} zu sehen.

\begin{equation}
    t_\text{Auf} = \left|2 \cdot t_\text{Koin} - t_\text{Halb} \right|
    \label{eq:Auflosung}
\end{equation}

\begin{table}[H]
    \centering
    \caption{Die durchschnittliche Höhe des Plateaus, die Auflösungszeit und die Halbwertsbreite für die verschiedenen Koinzidenz.}
    \label{tab:Halbwertsbreiten}
    \begin{tabular}{c c c c}
    \toprule
     $t_\text{Koin} \mathbin{/} \unit{\nano\second} $ &$\bar{N}$ & $\Delta t_\text{Halb} \mathbin{/} \unit{\nano\second}$ & $t_\text{Auf} \mathbin{/} \unit{\nano\second}$ \\
    \midrule
        $10$&  $20.15 \pm 0,67 $ &  $17,891$ & $2,109$\\ 
        $15$&  $24,87 \pm 1,78 $ &  $22,855$ & $7,145$\\ 
        $20$&  $21,63 \pm 1,26 $ &  $40,831$ & $0,831$\\ 
    \bottomrule
    \end{tabular}
    \end{table}

\begin{figure}[H] 
    \centering
    \includegraphics[width=.8\textwidth]{build/Kalibrierung_Koinzidenz_10.pdf}
    \caption{Aufgenommene Zählrate bei Änderung der Verzögerungsleitung für eine Koinzidenz von $t_\text{Koin} = 10 \, \unit{\nano\second}$ sowie Bestimmung der Halbwertsbreite.}
    \label{fig:koin_10}
\end{figure}

\begin{figure}[H] 
    \centering
    \includegraphics[width=.8\textwidth]{build/Kalibrierung_Koinzidenz_15.pdf}
    \caption{Aufgenommene Zählrate bei Änderung der Verzögerungsleitung für eine Koinzidenz von $t_\text{Koin} = 15 \, \unit{\nano\second}$ sowie Bestimmung der Halbwertsbreite.}
    \label{fig:koin_15}
\end{figure}

\begin{figure}[H]
    \centering
    \includegraphics[width=.8\textwidth]{build/Kalibrierung_Koinzidenz_20.pdf}
    \caption{Aufgenommene Zählrate bei Änderung der Verzögerungsleitung für eine Koinzidenz von $t_\text{Koin} = 20 \, \unit{\nano\second}$ sowie Bestimmung der Halbwertsbreite.}
    \label{fig:koin_20}
\end{figure}

Als Verzögerungszeit der Verzögerungsleitung wurde $\Delta t = 3 \,\unit{\nano\second}$ bei $t_\text{Koin} = 15 \, \unit{\nano\second}$ gewählt.

%\begin{table}[H]
%    \centering
%    \caption{Die Zählrate nach der Koinzidenz $ t = 10 \, \unit{\nano\second}$, bei systematischer Variation der Verzögerungszeit, auf eine Messzeit von  $1 \unit{\second}$ gerechnet.}
%    \label{tab:koin_10}
%    \begin{tabular}{c c c c}
%    \toprule
%     $\Delta t \mathbin{/} \unit{\nano\second}$ &  $\text{Zählung}$ & $\Delta t \mathbin{/} \unit{\nano\second}$ &  $\text{Zählung}$ \\
%    \midrule
%        $-32    $   &   $0,125 \pm 0,032$   &  $32$            &   $0,050 \pm 0,020$\\
%        $-24    $   &   $0,200 \pm 0,058$   &  $24$            &   $0,350 \pm 0,076$\\
%        $-20    $   &   $0,450 \pm 0,087$   &  $20$            &   $0,766 \pm 0,113$\\
%        $-16    $   &   $1,783 \pm 0,172$   &  $16$            &   $2,416 \pm 0,201$\\
%        $-12    $   &   $2,383 \pm 0,199$   &  $12$            &   $8,100 \pm 0,636$\\
%        $-8     $   &   $9,633 \pm 0,567$   &  $8 $            &   $15,950\pm 0,893$\\
%        $-4     $   &   $11,767\pm 0,626$   &  $4 $            &   $18,950\pm 0,973$\\
%        $-3     $   &   $12,800\pm 0,653$   &  $3 $            &   $20,450\pm 1,011$\\
%        $-2     $   &   $21,100\pm 1,027$   &  $2 $            &   $19,850\pm 0,996$\\
%        $-1     $   &   $20,050\pm 1,001$   &  $1 $            &   $20,500\pm 1,012$\\
%    \bottomrule
%    \end{tabular}
%    \end{table}
%
%    \begin{table}[H]
%        \centering
%        \caption{Die Zählrate nach der Koinzidenz $ t = 15 \, \unit{\nano\second}$, bei systematischer Variation der Verzögerungszeit, auf eine Messzeit von $1 \unit{\second}$ gerechnet.}
%        \label{tab:koin_15}
%        \begin{tabular}{c c c c}
%        \toprule
%         $\Delta t \mathbin{/} \unit{\nano\second}$ &  $\text{Zählung}$ & $\Delta t \mathbin{/} \unit{\nano\second}$ &  $\text{Zählung}$ \\
%        \midrule
%            $-32    $   &   $0,091    \pm 0,027$  & $32$         &   $1,250    \pm 0,102$\\
%            $-24    $   &   $0,150    \pm 0,050$  & $24$         &   $0,517    \pm 0,093$\\
%            $-20    $   &   $0,217    \pm 0,060$  & $20$         &   $1,950    \pm 0,180$\\
%            $-16    $   &   $1,767    \pm 0,173$  & $16$         &   $7,250    \pm 0,348$\\
%            $-12    $   &   $8,000    \pm 0,365$  & $12$         &   $18,80    \pm 0,970$\\
%            $-8     $   &   $14,067   \pm 0,685$  & $8 $         &   $22,550   \pm 1,062$\\
%            $-4     $   &   $20,567   \pm 0,828$  & $4 $         &   $25,200   \pm 1,123$\\
%            $-3     $   &   $15,900   \pm 0,728$  & $3 $         &   $25,150   \pm 1,121$\\
%            $-2     $   &   $23,600   \pm 1,086$  & $2 $         &   $24,850   \pm 1,115$\\
%            $-1     $   &   $24,100   \pm 1,098$  & $1 $         &   $28,650   \pm 1,197$\\
%        \bottomrule
%        \end{tabular}
%        \end{table}    
%
%
%
%        \begin{table}[H]
%            \centering
%            \caption{Die Zählrate nach der Koinzidenz $ t = 20 \, \unit{\nano\second}$, bei systematischer Variation der Verzögerungszeit, auf eine Messzeit von $1 \unit{\second}$ gerechnet.}
%            \label{tab:koin_20}
%            \begin{tabular}{c c c c}
%            \toprule
%             $\Delta t \mathbin{/} \unit{\nano\second}$ &  $\text{Zählung}$ & $\Delta t \mathbin{/} \unit{\nano\second}$ &  $\text{Zählung}$ \\
%            \midrule
%                $-32    $   &   $0,275   \pm 0,048$  &$32$          &   $0,183   \pm 0,039$\\
%                $-24    $   &   $6,400   \pm 0,327$  &$24$          &   $1,950   \pm 0,180$\\
%                $-20    $   &   $15,317  \pm 0,505$  &$20$          &   $7,600   \pm 0,356$\\
%                $-16    $   &   $21,933  \pm 0,855$  &$16$          &   $15,867  \pm 0,727$\\
%                $-12    $   &   $20,667  \pm 0,830$  &$12$          &   $20,100  \pm 1,002$\\
%                $-8     $   &   $21,067  \pm 0,838$  &$8 $          &   $21,350  \pm 1,033$\\
%                $-4     $   &   $21,867  \pm 0,854$  &$4 $          &   $24,100  \pm 1,098$\\
%                $-3     $   &   $21,300  \pm 0,843$  &$3 $          &   $23,400  \pm 1,082$\\
%                $-2     $   &   $20,633  \pm 0,829$  &$2 $          &   $22,600  \pm 1,063$\\
%                $-1     $   &   $19,500  \pm 0,806$  &$1 $          &   $22,650  \pm 1,064$\\
%            \bottomrule
%            \end{tabular}
%            \end{table}  
%



\subsection{Kalibrierung des MCA}
\label{sec:KaliMCA}

Der MCA wird mithilfe eines Doppelimpulsgenerator justiert. Doppelimpulsgenerator erzeugt zwei Impulse mit einem festen Abstand.
Diese Impulse werden von MCA je nach zeitlichen Abstand einem Kanal zugeordnet. In \autoref{tab:KaliMCA_tab} sieht man die Kanalnummer und den dazugehörenden Abstand der Impulse.

\begin{table}[H]
    \centering
    \caption{Einordnung der Impulsabstände in die Kanäle des MCA}
    \label{tab:KaliMCA_tab}
    \begin{tabular}{c c}
    \toprule
     Kanal &  $\text{Zeit} \mathbin{/} \unit{\micro\second}$ \\
    \midrule
     430,0 &                             9,5 \\
     384,0 &                             8,5 \\
     338,0 &                             7,5 \\
     292,0 &                             6,5 \\
     246,0 &                             5,5 \\
     200,0 &                             4,5 \\
     154,0 &                             3,5 \\
     108,0 &                             2,5 \\
      62,0 &                             1,5 \\
      16,0 &                             0,5 \\
    \bottomrule
    \end{tabular}
    \end{table}

Die Zeit für die Kanäle, die zwischen den bestimmten Kanälen liegt, werden, wie in \autoref{fig:lin_reg_Channel_zeit} dargestellt, durch eine lineare Regression bestimmt.
Die Parameter des Fits sind durch 
\begin{equation}
    \text{Impulsabstände} = \left(0,021 \pm 0,000 \right) \,\unit{\micro\second} \cdot  \text{Kanalnummer} + \left(0,152 \pm 0,000 \right) \,\unit{\micro\second}
    \label{eq:Kanal_Zeit_Rechnung}
\end{equation}
gegeben.

\begin{figure}[H]
    \centering
    \includegraphics{build/Channel_Kalibrierung.pdf}
    \caption{Lineare Regression durch die Kanäle des MCA und die dazugehörenden Zeiten.}
    \label{fig:lin_reg_Channel_zeit}
\end{figure}

\subsection{Bestimmung der Untergrundrate}
\label{sec:Bes_Unt}
Es wird zunächst eine Abschätzung der Untergrundrate durchgeführt. Aus \eqref{eq:theo_unter} ergibt sich ein Wert von $N_\text{start} = 627,893$.
Der Untergrund verteilt sich auf $512$ Kanäle. Dadurch folgt eine Untergrundrate von $B_\text{theo} = 1,226$. %%% Warum haben wir hier die 511 statt 512? aus irgendeinem Grund steht bei uns in der Datei 511. Jetzt haben wir halt 512 ist mir auch egal und ändert im Prinzip überhaupt nix


\subsection{Bestimmung der Lebensdauer}
\label{sec:Bes_Leb}

Zunächst werden die Kanalnummer mit der Gleichung \eqref{eq:Kanal_Zeit_Rechnung} in Zeiten umgerechnet.
In \autoref{fig:expon} sind die Anzahl der Myonen, die in einen Kanal gemessen wurden, gegen die Lebensdauer dieser Myonen aufgetragen.
Es wird eine Ausgleichsfunktion der Form \eqref{eq:exponential_decay} gefittet.

\begin{figure}[H]
    \centering
    \includegraphics{build/Myonenzerfall_Zeit.pdf}
    \caption{Vom MCA aufgenommene Messdaten der unterschiedlichen Myonenlebensdauern samt exponentiellem Fit.}
    \label{fig:expon}
\end{figure}

Die Parameter des Fits sind gegeben durch:

\begin{align*}
    &N_0                      =& 34,251 \pm& \, 0,758                                 \\
    &\lambda            	  =& (0,478 \pm& \, 0,0206)  \, \unit{\dfrac{1}{\micro\second}} \\
    &B_{\text{exp}}           =& 0,340  \pm& \, 0,267
\end{align*}

Aus der Inversen der Zerfallskonstanten wird die Lebensdauer zu 
\begin{equation*}
    \tau = \dfrac{1}{\lambda} = \left(2,090  \pm 0,090 \right) \,\unit{\micro\second}
\end{equation*}
berechnet.
