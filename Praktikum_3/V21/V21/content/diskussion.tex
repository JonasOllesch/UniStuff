\section{Diskussion}
\label{sec:Diskussion}

\subsection{Erdmagnetfeld}

Die gemessenen Werte werden mit dem Literaturwert von $1,938 \cdot 10^{-5} \, \unit{\tesla}$ \cite{Pfeiler2017} verglichen.
Daraus ergeben sich eine relative Abweichungen von $30,08 \, \%$ für $b_1 = \left( 2,521 \pm 0,057 \right) \cdot 10^{-5}\, \unit{\tesla}$ und ein 
Abweichung von $24,56 \, \% $ für $b_2 = \left( 2,412  \pm 0,016 \right) \cdot 10^{-5}\, \unit{\tesla}$.
Eine Erklärung für diese große Abweichungen könnten Störfelder sein, die das Erdmagnetfeld überlagern.\\


\subsection{Landé-Faktor}


Für Rb-85 wird ein Landé-Faktor von $\frac{1}{2} $ erwartet, was eine relative Abweichung von $1,54 \, \%$ entspricht.
Bei Rb-87 ist der Literaturwert $\frac{1}{3}$ und die Abweichung beträgt $1,81 \, \%$\\


\subsection{Kernspin}


Das in der Natur vorkommende Rubidium besteht zu $72,17 \, \%$ aus Rb-85 und zu $27,83 \, \%$ aus Rb-87 \cite{Pfeiler2017} .
Die Differenz zu den gemessenen Daten kann dadurch erklärt werden, dass die Probe mit Rb-85 angereichert wurde.\\


\subsection{Isotopenverhältnis}

Für den Nulldurchgang werden 128 Pixel gezählt, was deutlich weniger ist als die 73 Pixel aus dem ersten und zweiten Peak zusammen.
Eine Erklärung könnte sein, dass die Auflösung der Apparatur nicht ausreichend ist und die Peaks möglicherweise breiter und nicht so groß ausfallen wie erwartet.\\


\subsection{Quadratischer Zeeman-Effekt}

Der quadratischen Zeeman-Effekt wird auf eine Größenordnung von $10^{-5} \, \unit{\electronvolt}$ geschätzt.
Der hier beobachteten Energien ist sehr gering und es kann davon ausgegangen werden, dass
die verwendeten Magnetfelder klein genug sind, um den quadratischen Zeeman-Effekt nicht berücksichtigen zu müssen. 