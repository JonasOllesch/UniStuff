\section{Discussion}
\label{sec:Diskussion}


The values of $C_\text{V}$ and $C_\text{p} $ follow the expected course of the function. 
With higher temperatures the values approach the classical value of $3R$ and surpass it, 
but all values lie within the uncertainty. 
More data point for temperatures below $100 \unit{\kelvin}$ are desirable, because large increase in heat capacity is expected for a small increase in temperature.
The observational error for the Debye temperature is $12.47 \%$. An explanatory approach would be the continuous loss of heat to the surroundings, that can not be prevented. 
The temperature of the recipient was in general $5 \, \unit{\kelvin}$ lower than the temperature of the copper sample.
A precise alignment of both temperatures is too difficult to achieve, without extensive knowledge about the used equipment.
For better results at low temperatures, another coolant could be used. Such as liquid helium.
In addition unknown systematic errors of the measured devices are not taken into account.