\section{Evaluation}
\label{sec:Auswertung}



The error propagation is performed using $uncertainties$ \cite{unp}. The linear interpolation is performed using $numpy$ \cite{numpy}.
Figures are compiled using $matplotlib$\cite{Hunter:2007}. Due to fluctuations of the measuring device, we assume an uncertainty of
$\pm 0.5 \, \unit{\ohm}$ for the resistances and $\pm 0.5 \, \unit{\second} $ for the time measurements. 
The assumed error of the voltage and the current is $1 \, \%$.



The Debye temperature can be determent with the help of \autoref{tab:Debye}
The value for the Debye temperature is obtained by finding the closed in the table for a given $C_V$.
The row of the table corresponse  to the first before the decimal point and the colum corresponse
with the first digit after the decimal point. 
The Debye temperature is only calculated for temperature below $T = 170 \, unit{\kelvin}$.

\begin{table}[H]
\centering
\caption{The Debye temperature for a given $C_V$  \cite{v47}.}
\label{tab:Debye}
\begin{tabular}{c | c c c c c c c c c c}
\toprule
 $\dfrac{\Theta_D}{T} $ &0 & 1 & 2 &3 & 4 & 5&6 & 7 & 8 &9\\
\midrule
    0 & 0 & 1 & 2 & 3 & 4 & 5 & 6 & 7 & 8 & 9 \\
    1 & 0 & 1 & 2 & 3 & 4 & 5 & 6 & 7 & 8 & 9 \\
    2 & 0 & 1 & 2 & 3 & 4 & 5 & 6 & 7 & 8 & 9 \\
    3 & 0 & 1 & 2 & 3 & 4 & 5 & 6 & 7 & 8 & 9 \\
    4 & 0 & 1 & 2 & 3 & 4 & 5 & 6 & 7 & 8 & 9 \\
    5 & 0 & 1 & 2 & 3 & 4 & 5 & 6 & 7 & 8 & 9 \\
    6 & 0 & 1 & 2 & 3 & 4 & 5 & 6 & 7 & 8 & 9 \\
    7 & 0 & 1 & 2 & 3 & 4 & 5 & 6 & 7 & 8 & 9 \\
    8 & 0 & 1 & 2 & 3 & 4 & 5 & 6 & 7 & 8 & 9 \\
    9 & 0 & 1 & 2 & 3 & 4 & 5 & 6 & 7 & 8 & 9 \\
\bottomrule
\end{tabular}
\end{table}