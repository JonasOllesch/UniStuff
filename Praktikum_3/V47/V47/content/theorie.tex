\section{Theory}
\label{sec:theorie}

If a substance is exposed to an external heat or energy source,
it will heat up trying to reach thermodynamic equilibrium.
The capacity at which it does so is called the heat capacity
\begin{equation*}
    C = \frac{\Delta Q}{\Delta T} \,,
\end{equation*}
defined by the amount of heat needed to heat said substance by a certain amount of
$\si{\kelvin}$. \\

More specifically, two different kinds of heat capacity exist. \\
The heat capacity 
\begin{equation*}
    C_{V} = \left( \frac{\partial U}{\partial T} \right) \big \vert _V
\end{equation*}
as the derivative of the inner energy by temperature under constant volume $V$ and respectively
\begin{equation*}
    C_{p} = \left( \frac{\partial U}{\partial T} \right) \big \vert _p
\end{equation*}
for constant pressure $p$. \\

In practice, holding the pressure constant is significantly simpler than keeping a constant volume, so $C_p$ is the heat capacity most often measured directly.
The correlation between $C_V$ and $C_p$, where the former is always greater, is given by
\begin{equation}
    C_V = C_p - 9 \, \alpha^2 \, \kappa \, V_0 \, T \,,
\end{equation}
where $\alpha$ is the bulk modulus describing the substance's resistance to compression, $\kappa$ is modulus of elasticity (Young's modulus), referring to the tensile/compressive stiffness of said solid substance
under lengthwise stress, $V_0$ is the substance's molar volume and $T$ the temperature. \\

The following subchapters deal with three different approaches regarding the heat capacity's behavior for changing temperatures, namely the classical, the Einstein and the Debye approach.

\subsection{The classical approach}

In the classical approach, assuming a uniform distribution of energy inside the substance, each of the atom's degrees of freedom is assigned a fixed energy of
\begin{equation*}
    E = \frac{1}{2} k_\text{B} T \,,
\end{equation*}
where $T$, again, is the temperature and $k_\text{B}$ is the Boltzmann constant. \\

Here, the substance in question is copper in its solid state.
Inside the copper lattice, each atom is fixes in place by interactions with its neighbors, meaning they do not possess any translative or rotational, but vibrational degrees of freedom along each of the three axes. \\
As vibrational degrees of freedom contain both kinetic and potential energy, the total amount of energy per atom amounts to
\begin{equation*}
    E = 3 k_\text{B} T \,
\end{equation*}
or
\begin{equation*}
    E = 3 k_\text{B} T \cdot N_\text{L} = 3 R T
\end{equation*}
when expanded to a single mol of copper, with $N_\text{L}$ describing the Loschmidt constant and the universal gas constant $R$.\\
From that, $C_V$ calculates as
\begin{equation*}
    C_V = \left( \frac{\partial U}{\partial T} \right) \big \vert _V = 3 R \,.
\end{equation*} \\

As can be seen, the heat capacity assumed by the classical approach is independent of both pressure and temperature.
This contradicts practical measurements, where a distinct relation between heat capacity and temperature can be observed. 

\subsection{Einstein approach}