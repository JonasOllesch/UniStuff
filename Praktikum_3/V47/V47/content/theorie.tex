\section{Theory}
\label{sec:theorie}

If a substance is exposed to an external heat or energy source,
it will heat up trying to reach thermodynamic equilibrium.
The capacity at which it does is the heat capacity
\begin{equation}
    C = \frac{\Delta Q}{\Delta T} \,,
    \label{eq:heat_cap}
\end{equation}
defined by the amount of heat needed to heat said substance by a certain amount of
$\si{\kelvin}$. \\

More specifically, two different kinds of heat capacity exist. \\
The heat capacity 
\begin{equation*}
    C_{V} = \left( \frac{\partial U}{\partial T} \right) \big \vert _V
\end{equation*}
as the derivative of the inner energy by temperature under constant volume $V$ and respectively
\begin{equation*}
    C_{p} = \left( \frac{\partial H}{\partial T} \right) \big \vert _p
\end{equation*}
for constant pressure $p$ with the entalpy $H$. \\

In practice, holding the pressure constant is significantly simpler than keeping a constant volume, so $C_p$ is the heat capacity most often measured directly.
The correlation between $C_V$ and $C_p$, where the former is always greater, is given by
\begin{equation}
    C_V = C_p - 9 \, \alpha^2 \, \kappa \, V_0 \, T \,,
    \label{eq:CVtoCp}
\end{equation}
where $ \kappa $ is the bulk modulus describing the substance's resistance to compression, $\alpha$ is the linear coefficient of expansion, $V_0$ is the substance's molar volume and $T$ its temperature. \\

The following subchapters deal with three different approaches regarding the heat capacity's behavior for changing temperatures, namely the classical, the Einstein and the Debye approach.

\subsection{The classical approach}

In the classical approach, assuming a uniform distribution of energy inside the substance, each of the atom's degrees of freedom is assigned a fixed energy of
\begin{equation*}
    E = \frac{1}{2} k_\text{B} T \,,
\end{equation*}
where $T$, again, is the temperature and $k_\text{B}$ is the Boltzmann constant. \\

Here, the substance in question is copper in its solid state.
Inside the copper lattice, each atom is fixes in place by interactions with its neighbors, meaning they do not possess any translative or rotational, but only vibrational degrees of freedom along each of the three axes. \\
As vibrational degrees of freedom contain both kinetic and potential energy, the total amount of energy per atom amounts to
\begin{equation*}
    E = 3 k_\text{B} T \,
\end{equation*}
or
\begin{equation*}
    E = 3 k_\text{B} T \cdot N_\text{L} = 3 R T
\end{equation*}
when expanded to a single mol of copper, with $N_\text{L}$ describing the Loschmidt constant and the universal gas constant $R$.\\
From that, $C_V$ calculates as
\begin{equation*}
    C_V = \left( \frac{\partial U}{\partial T} \right) \big \vert _V = 3 R \,.
\end{equation*} \\

As can be seen, the heat capacity assumed by the classical approach is independent of both pressure and temperature.
This contradicts experimental observations, where a distinct relation between heat capacity and temperature can be noticed. 

\subsection{Einstein approach}

The Einstein approach assumes that all $3N$ eigen oscillations of the lattice oscillate at the same frequency $\omega_\text{E}$.
The average energy follows the Bose-Einstein distribution, more specifically
\begin{equation*}
   \langle U \rangle = 3 N \hbar \omega_\text{E} \left(\frac{1}{2} + \frac{1}{\text{e}^{\beta \hbar \omega_\text{E}} - 1} \right) \,,
\end{equation*}
where $\hbar$ is the Planck constant and $\beta = \frac{1}{k_\text{B} T}$. \\

Using the characteristic Einstein temperature $\theta_\text{E} = \frac{\hbar \omega_\text{E}}{k_\text{B}}$, the heat capacity can be expressed as
\begin{equation}
    C^\text{E}_V = 3 N k_\text{B} \left(\frac{\theta_\text{E}}{T} \right)^2 \frac{\text{e}^{\frac{\theta_\text{E}}{T}}}{\left(\text{e}^{\frac{\theta_\text{E}}{T}} - 1 \right)^2} \,.
    \label{eq:einsteinheatcap}
\end{equation}

Approximately, the heat capacity follows
\begin{equation*}
    C^\text{E}_V = \begin{cases}
        3 N k_\text{B} \left(\frac{\theta_\text{E}}{T} \right)^2 \text{e}^{-\frac{\theta_\text{E}}{T}} &\qquad \text{for} \quad T << \theta_\text{E} \\ 
        3 N k_\text{B} &\qquad \text{for} \quad T >> \theta_\text{E} \,.
    \end{cases} 
\end{equation*}

As can be seen above, for high temperatures, the Einstein approximation once again predicts a constant heat capacity coinciding with the classical result.
This approximation work quite well for higher temperatures, where optical phonons dictate the processes inside the copper. \cite{}

\subsection{Debye approach}

The Debye approach on the other hand better describes lower temperatures, where acoustic phonons dominate the processes inside the substance.
Here, all phonon branches are approximated through $3$ branches of linear dispersion $\omega_i = v_i q$,  from which the heat capacity follows as
The singular modes of the Einstein model are replaced with a more complex spectrum of possible eigen frequency $Z(\omega)$ in a crystal. 
The heat capacity is given by 
\begin{equation}
    C_\text{V} = \dfrac{\dif}{\dif T} \int^{\omega_\text{max}}_0 Z(\omega) \dfrac{\hbar \omega}{\exp{\left( \hbar \omega \mathbin{/} k_\text{B} T \right)} -1} \dif\omega.
    \label{eq:CV_debye}
\end{equation}
In the first approximation $Z(\omega)$ can be determent by counting all possible eigen frequency for a cube with edge length $L$ in a frequency interval form $\omega$ to $\omega + \dif \omega$.
The result is 
\begin{equation}
    Z(\omega)\dif\omega = \dfrac{L³ \omega}{2\pi}\dfrac{1}{\dfrac{1}{v^{3}_\text{l}} + \dfrac{2}{v^{3}_\text{t} }}  \dif\omega \, 
    \label{eq:densityofstates}
\end{equation}
where $v_\text{l}$ is the longitudinal and $v_\text{t}$ the transversal phase velocity.
A real crystal with, $ N_\text{L}$ atoms, has only a limited number of possible frequency. The upper boundary is called the Debye frequency $\omega_\text{D}$, which can be calculated throw the integral 

\begin{equation*}
    \int^{\omega_\text{D}}_0 Z(\omega) \dif \omega = 3 N_\text{L}. 
\end{equation*}
In combination with \eqref{eq:densityofstates} resulting

\begin{equation}
    w^{3}_\text{D} = \dfrac{18 \pi N_\text{L}}{L^3}\dfrac{1}{\dfrac{1}{v^{3}_\text{l}} + \dfrac{2}{v^{3}_\text{t} }} .
    \label{eq:Debye_frequency}
\end{equation}


Now \eqref{eq:densityofstates} is substituted into the \eqref{eq:CV_debye}

\begin{equation*}
    C^\text{D}_V = 9 N k_\text{B} \left( \frac{T}{\theta_\text{D}} \right)^3 \int^{\frac{\theta_\text{D}}{T}}_0 \dif x \, \frac{x^4 \text{e}^x}{(\text{e}^x -1)^2} \,
\end{equation*}
follows, with the Debye temperature

\begin{equation*}
    \theta_\text{D} = \frac{\hbar v_\text{s}}{k_\text{B}} \left(6 \pi^2 \frac{N}{V}\right)^{\frac{1}{3}} \,,
\end{equation*}
the speed of sound $v_\text{s}$ in medium and $N$ wave vectors inside the first Brillouin zone. \\

For high and low temperatures, the heat capacity can be expressed as
\begin{equation*}
    C^\text{D}_V= \begin{cases}
        \frac{12 \pi^4}{5} N k_\text{B} \left(\frac{T} {\theta_\text{D}}\right)^3 &\qquad \text{for} \quad T << \theta_\text{D} \\ 
        3 N k_\text{B} &\qquad \text{for} \quad T >> \theta_\text{D} \,.
    \end{cases} 
\end{equation*}

\newpage