\section{Theory}
\label{sec:theorie}

If a substance is exposed to an external heat or energy source,
it will heat up trying to reach thermodynamic equilibrium.
The capacity at which it does so is called the heat capacity
$C$, defined by the amount of heat needed to heat said substance by a single
$\si{\kelvin}$. \\

More specifically, two different kinds of heat capacity exist.
The heat capacity 
\begin{equation}
    C_\text{V} = \left( \frac{\partial U}{\partial T} \right) \big \vert _V
\end{equation}