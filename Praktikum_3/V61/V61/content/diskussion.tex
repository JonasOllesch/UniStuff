\section{Diskussion}
\label{sec:Diskussion}



\subsection{Untersuchung der Polarisation}
\label{sec:Polarisation}

Die Messwerte der Polarisationsmessung folgen in erster Ordnung dem Gesetzt von Malus, was die lineare Polarisation des Lasers bestätigt.
Das Verhältnis aus dem größten und dem kleinsten Messwerte beträgt $0.21 \% $, was durch Unreinheiten im Polfilter erklärt werden kann.

\subsection{Bestimmung der Wellenlänge}
\label{sec:Wellenlänge}

Die theoretische Wellenlänge des Lasers liegt bei $633 \unit{\nano\meter}$\cite{eicheich}. Der theoretische Wert liegt innerhalb der Unsicherheit von zwei der Messwerte aus \autoref{tab:Wellenlänge_ubersicht}.
Als Mittelwert ergibt sich $ \left( 633,89 \pm 6,82 \right) \unit{\nano\meter}$. Dieser Mittelwert hat eine relative Abweichung von $ 0,14 \,\% $ vom Literaturwert.
Eine mögliche Fehlerquelle ist das ungenaue Abmessen der Abstände zwischen dem Schirm und dem Gitter und zwischen den verschiedenen Maxima. 


\subsection{Messung der TEM Moden}
\label{sec:TEM_moden}
Sowohl an die $\text{TEM}_{00}$ und $\text{TEM}_{01}$ lassen sich Theoriekurven fitten. Dabei ist festzustellen, dass die Messreihe der $\text{TEM}_{00}$ deutlich besser zur Theorie passt.
Zur Stabilisierung der Moden wurde ein Wolframdraht verwendet. Dieser Draht ist nicht senkrecht zur Polarisationsrichtung des Lasers in seiner Halterung eingespannt.
Auf dem Schirm war ein gedrehtes Bild zu sehen. Die Messung könnte verbessert werden, indem zunächst die genau Polarisationsrichtung des Lasers überprüft wird und danach die Ausrichtung des Drahtes an diese angepasst wird.    
Auch könnten die Moden in zwei Dimensionen vermessen werden.

\subsection{Untersuchung der Stabilitätsbedingung}
\label{sec:Stab_be}
Im Experiment konnte keine eindeutige Resonatorlänge festgestellt werden bei der die Stabilitätsbedingung verletzt wird.
Es konnte gezeigt werden, dass sich die kritische Resonatorlänge im Bereich von $ \left( 135 \pm 5 \right) \, \unit{\centi\meter}$ für den ersten Aufbau und im Bereich von
$ \left( 140 \pm 5 \right) \, \unit{\centi\meter}$ für den zweiten Aufbau befinden muss. Eine genauere Messung war nicht möglich, da der Strahl zu instabil wurde.
Es konnten keine Messwerte für eine Resonatorlänge von über $2,1 \, \unit{\meter}$ gemessen werden, da die optische Schiene nicht lang genug ist.

\subsection{Messung der longitudinal Moden}
\label{sec:Stab_be}

Die Frequenz der longitudinal Moden wird nun mit der Verschiebung der Wellenlänge des Lasers verglichen, die durch den Dopplereffekt entsteht.
Die Dopplerbreite eines HeNe-Lasers beträgt $ \Delta f = 1500 \, \unit{\mega\hertz}$ \cite{eicheich}. Für jede Resonatorlänge liegt die gemessene Modenlänge weit unter der Dopplerbreite.
Der Laser befindet sich in diesem Aufbau im Multimodenbetrieb und die longitudinalen Moden treten als Schwebung auf.
