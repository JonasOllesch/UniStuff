\section{Auswertung}
\label{sec:auswertung}

Die folgenden Ausgleichsfunktionen werden mit Funktion $curve\_fit$ aus der Python\cite{py}-Bibliothek $scipy$\cite{2020SciPy-NMeth} bestimmt.
Die Fehlerrechnung wird mithilfe der Bibliothek $uncertainties$ \cite{unp} durchgeführt.
Die Grafiken werden durch die Bibliothek $matplotlib$\cite{Hunter:2007} erstellt. Der gemessene Strom an der Photodiode wird in Ampere ausgelesen.  
Aufgrund von Schwankungen des Photostroms wird eine Unsicherheit von $10 \%$ angenommen.

\subsection{Untersuchung der Polarisation}
\label{sec:Polarisation}

Die Intentsitätsverteilung in Abhängigkeit vom Raumwinkel $\theta$ ist durch die Funktion
\begin{equation*}
    I(\theta) = I_0 \cdot \cos(\theta + \theta_0)² %am besten in der Theorie erklären wo diese Formel überhaupt herkommt
\end{equation*}
gegeben, wobei $\theta_0 $ ein konstanter Offset gegenüber der optischen Achse ist. 
Die Parameter des Fit werden als
\begin{align*}
    I_0      =& (69,16 \pm    1,27) \, \unit{\micro\ampere}   \text{und}  \\
    \theta_0 =& (92,18 \pm    0,12) \, °                        \\
\end{align*}
bestimmt. Die grafische Darstellung ist in  \autoref{fig:pol_pol} und \autoref{fig:pol}

\begin{figure}[H]
    \centering
    \includegraphics[width=\textwidth]{build/polarisation.pdf}
    \caption{Der Photostrom hinter einem Polfilter in Abhängigkeit des Raumwinkels.}
    \label{fig:pol}
\end{figure}

\begin{figure}[H]
    \centering
    \includegraphics[width=\textwidth]{build/polarisation_pol.pdf}
    \caption{Der Photostrom hinter einem Polfilter in Abhängigkeit des Raumwinkels in einem polaren Koordinatensystem.}
    \label{fig:pol_pol}
\end{figure}

\subsection{Bestimmung der Wellenlänge}
\label{sec:Wellenlänge}

Um die Wellenlänge des Lasers zu bestimmen, wird der Strahl an einem Gitter gebeugt. Auf einem Schirm, der sich in einem Abstand $d$ hinter dem Gitter befindet, kann ein Interferenzmuster beobachtet werden. Es wird der Abstand $\Delta x_\text{n}$ zum zentralen Maxima $n = 0$ gemessen und mithilfe der Bragg-Bedingung
\begin{equation}
    n \lambda = 2 g \sin{\alpha}
    \label{eq:bragg}
\end{equation}
kann die Wellenlänge $\lambda$ berechnet werden, dabei ist $n$ das $n$-te Maxima von $n=0$, $g$ ist die Gitterkonstante des verwendeten Gitters und $\alpha$ der Winkel zwischen der optischen Achse des dem betrachteten Maximas.
$\alpha$ wird über die trigonometrische Beziehung  
\begin{equation*}
    \alpha = \arctan(\frac{\Delta x_\text{n}}{d})
    \label{eq:trig}
\end{equation*} 
bestimmt. Die aufgenommenen Messwerte sind in \autoref{tab:Well_80}, \autoref{tab:Well_100}, \autoref{tab:Well_600} und \autoref{tab:Well_1600} zu sehen.

\begin{table}[H]
    \centering
    \caption{Der Abstand des $n$-ten Maximas zu $n = 0$. Negative $n$ bezeichnen Maxima die links von $n = 0$ liegen und positive $n$, die die rechts von $n = 0$ liegen. Die Messwerte wurden für $d = 77,3 \, \unit{\centi\meter}$ und $g = \dfrac{1 \, \unit{\milli\meter}}{80}$ aufgenommen.}
    \label{tab:Well_80}
    \begin{tabular}{c c c c}
    \toprule
     $n$ & $\Delta x_\text{n}$ &   $n$ & $\Delta x_\text{n}$\\
    \midrule
        $-8$  &$ 34,5$ &        $1$ &  $ 3,9$ \\
        $-7$  &$ 29,5$ &        $2$ &  $ 7,9$ \\
        $-6$  &$ 24,8$ &        $3$ &  $12,2$ \\
        $-5$  &$ 20,3$ &        $4$ &  $16,1$ \\
        $-4$  &$ 16,0$ &        $5$ &  $20,2$ \\
        $-3$  &$ 12,1$ &        $6$ &  $24,5$ \\
        $-2$  &$ 8,0 $ &        $7$ &  $29,0$ \\
        $-1$  &$ 4,1 $ &        $8$ &  $33,7$ \\
       $ 0 $  &$ 0,0 $ &        $-$ &  $   -$ \\
    \bottomrule
    \end{tabular}
    \end{table}

\begin{table}[H]
    \centering
    \caption{Der Abstand des $n$-ten Maximas zu $n = 0$. Negative $n$ bezeichnen Maxima die links von $n = 0$ liegen und positive $n$, die die rechts von $n = 0$ liegen. Die Messwerte wurden für $d = 77,3 \, \unit{\centi\meter}$ und $g = \dfrac{1 \, \unit{\milli\meter}}{100}$ aufgenommen.}
    \label{tab:Well_100}
    \begin{tabular}{c c c c}
    \toprule
     $n$ & $\Delta x_\text{n}$ &   $n$ & $\Delta x_\text{n}$\\
    \midrule
    $-6$ &  $31,7 $&      $1$   &$4,9$ \\
    $-5$ &  $25,7 $&      $2$   &$9,8$ \\
    $-4$ &  $20,2 $&      $3$   &$14,9$ \\
    $-3$ &  $14,9 $&      $4$   &$20,1$ \\
    $-2$ &  $9,8  $&      $5$   &$25,5$ \\
    $-1$ &  $4,9  $&      $6$   &$31,3$ \\   
    $ 0$ &  $0,0  $&      $-$   &$-$    \\
    \bottomrule
    \end{tabular}
    \end{table}

\begin{table}[H]
    \centering
    \caption{Der Abstand des $n$-ten Maximas zu $n = 0$. Negative $n$ bezeichnen Maxima die links von $n = 0$ liegen und positive $n$, die die rechts von $n = 0$ liegen. Die Messwerte wurden für $d = 29,4 \, \unit{\centi\meter}$ und $g = \dfrac{1 \, \unit{\milli\meter}}{600}$ aufgenommen.}
    \label{tab:Well_600}
    \begin{tabular}{c c c c}
    \toprule
     $n$ & $\Delta x_\text{n}$ &   $n$ & $\Delta x_\text{n}$\\
    \midrule
    $-2$ & $36,0$ & $1$ & $ 11,9$ \\
    $-1$ & $12,1$ & $2$ & $ 33,4$ \\
    $0$ & $ 0.0$  & $- $ & $ - $\\
   \bottomrule
    \end{tabular}
    \end{table}
\begin{table}[H]
    \centering
    \caption{Der Abstand des $n$-ten Maximas zu $n = 0$. Negative $n$ bezeichnen Maxima die links von $n = 0$ liegen und positive $n$, die die rechts von $n = 0$ liegen. Die Messwerte wurden für $d = 29,4 \, \unit{\centi\meter}$ und $g = \dfrac{1 \, \unit{\milli\meter}}{600}$ aufgenommen.}
    \label{tab:Well_600}
    \begin{tabular}{c c c c}
    \toprule
     $n$ & $\Delta x_\text{n}$ &   $n$ & $\Delta x_\text{n}$\\
    \midrule
    $-2$ & $36,0$ & $1$ & $ 11,9$ \\
    $-1$ & $12,1$ & $2$ & $ 33,4$ \\
    $ 0$ & $ 0,0$ & $-$ & $ -   $\\
   \bottomrule
    \end{tabular}
    \end{table}
\begin{table}[H]
    \centering
    \caption{Der Abstand des $n$-ten Maximas zu $n = 0$. Negative $n$ bezeichnen Maxima die links von $n = 0$ liegen und positive $n$, die die rechts von $n = 0$ liegen. Die Messwerte wurden für $d = 29,4 \, \unit{\centi\meter}$ und $g = \dfrac{1 \, \unit{\milli\meter}}{600}$ aufgenommen.}
    \label{tab:Well_1200}
    \begin{tabular}{c c c c}
    \toprule
     $n$ & $\Delta x_\text{n}$ &   $n$ & $\Delta x_\text{n}$\\
    \midrule
    $-1$ & $19,8$   & $1$ & $ 17,9$   \\
    $0$  & $0   $   & $-$ & $ -   $   \\
    \bottomrule
    \end{tabular}
    \end{table}

Um systematische Fehler zu vermeiden wird der Abstand zu den beiden Maxima gleicher Ordnung gemittelt. 
Aus diesen gemittelten Abständen werden über die Gleichungen \eqref{eq:trig} und \eqref{eq:bragg} die Wellenlängen in \autoref{tab:Wellenlänge_ubersicht} berechnet. 
Es wird angenommen, dass die Maxima bis auf $1 \unit{\milli\meter}$ bestimmt werden konnten.

\begin{table}[H]
    \centering
    \caption{Die berechnete Wellenlänge zu den verwendeten Gittern und Abständen.}
    \label{tab:Wellenlänge_ubersicht}
    \begin{tabular}{c c c}
    \toprule
     $g \mathbin{/} \unit{\milli\meter}$ & $d \mathbin{/} \unit{\centi\meter}$ &$\lambda \mathbin{/} \unit{\nano\meter}$ \\
    \midrule
    $1/80  $   & $77,3$   & $637,10 \pm 5,94$\\
    $1/100 $   & $77,3$   & $630,12 \pm 1,41$\\
    $1/600 $   & $29,4$   & $632,81 \pm 2,99$\\
    $1/1200$   & $16,4$   & $635,32 \pm 0,54$\\
    \bottomrule
    \end{tabular}
    \end{table}

\subsection{Messung der TEM Moden}
\label{sec:TEM_moden}

Die Intentsitätsverteilung der $\text{TEM}_{00}$ und $\text{TEM}_{01}$  werden durch Funktionen 

\begin{align*}
    I_{00}(x) =& I_0 \cdot e^{-(x-x_0)²/w ²} \, \text{und} \\
    I_{01}(x) =& I_0 \cdot (x-x_{0})^2 \left( e^{-(x-x_1)²/w ²} + a e^{-(x-x_2)²/w ²} \right) \\
\end{align*}
gefittet.
Die Parameter der $\text{TEM}_{00}$  wurden als 
\begin{align*}
   I_0 =& \left( 162,61 \pm 4,87 \right) \cdot 10^{-7} \, \dfrac{\unit{\ampere}}{\unit{\meter}} \\
   x_0 =& \left( 8,12   \pm 0,08 \right) \cdot 10^{-3} \, \unit{\meter}                         \\
   w   =& \left(11,90   \pm 0,05 \right) \cdot 10^{-3} \, \unit{\dfrac{1}{\meter}}              \\
\end{align*}
und die der $\text{TEM}_{01}$ als
\begin{align*}
    I_0 =& \left( 12,94  \pm 1,87 \right) \cdot 10^{-9} \, \unit{\ampere}           \\
    x_0 =& \left( -3,17  \pm 0,44 \right) \cdot 10^{-3} \, \unit{\meter}            \\
    x_1 =& \left( -5,02  \pm 1,84 \right) \cdot 10^{-3} \, \unit{\meter}            \\
    x_2 =& \left( -0,97  \pm 0,43 \right) \cdot 10^{-3} \, \unit{\meter}            \\
    w   =& \left(  7,93  \pm 1,48 \right) \cdot 10^{-3} \, \unit{\dfrac{1}{\meter}} \\
    w_2 =& \left(-11,17  \pm 0,18 \right) \cdot 10^{-3} \, \unit{\dfrac{1}{\meter}} \\
 \end{align*}
 bestimmt.

\subsection{Untersuchung der Stabilitätsbedingung}
\label{sec:Stab_be}
Der theoretische Verlauf der Stabilitätsfaktoren ist in \autoref{fig:stab_be_theo} zu sehen.
Im Experiment wurde für die Anordnung $r_1 = inf$ und $r_2 = ???$ bei einer Resonatorlänge von $135 \, \unit{\centi\meter}$ ein stabiler Strahl erzeugt.
In der zweiten Anordnung $r_1 = ???$ und $r_2 = ???$ wurde der Strahl bei einer Länge von $140 \, \unit{\centi\meter}$ instabil.
 \begin{figure}[H]
    \centering
    \includegraphics[width=\textwidth]{build/Stabilität_theo.pdf}
    \caption{Theoretische Sta für zwei Anordnungen.}
    \label{fig:stab_be_theo}
\end{figure}

\subsection{Messung der longitudinal Moden}
\label{sec:Stab_be}
Die Peaks der gemessenen longitudinal Moden in Abhängigkeit der Resonatorlänge sind in \autoref{tab:longi} zu sehen.
Zusätzlich ist der mittlere Abstand zwischen zwei Moden in der Tabelle zu sehen.
%Die Frequenz der longitudinal Moden wird nun mit der Verschiebung der Wellenlänge des Lasers verglichen, die durch den Dopplereffekt entsteht.
%Die Dopplerbreite eines HeNe-Lasers beträgt $ \Delta f = 1500 \, \unit{meta\hertz}$\cite{eicheich}.



\begin{table}[H]
    \centering
    \caption{Die Peakpositionen in den longitudinal Moden. Die Postion und der durchschnittliche Abstand sind in $\unit{\mega\hertz}$.}
    \label{tab:longi}
    \begin{tabular}{c c c c c c c c c c | c}
    \toprule
      $L \mathbin{/} \unit{\centi\meter}$ & & & & & & & & & $\Delta \mathbin{/} \unit{\mega\hertz}$ &  \\
    \midrule
        $44$ &  $345$ & $690$ &$ 1031$ & $-   $ & $-    $ & $-   $ & $-   $ & $-   $ &  $-   $ &  $ 343,00 \pm 2,0 $ \\
        $46$ &  $330$ & $630$ &$ 990$  & $-   $ & $-    $ & $-   $ & $-   $ & $-   $ &  $-   $ &  $ 330,00 \pm 30,00 $ \\
        $50$ &  $315$ & $630$ &$ 945$  & $-   $ & $-    $ & $-   $ & $-   $ & $-   $ &  $-   $ &  $ 315,00 \pm 0,00 $ \\
        $52$ &  $308$ & $611$ &$ 915$  & $1219$ & $-    $ & $-   $ & $-   $ & $-   $ &  $-   $ &  $ 303,67 \pm 0,47 $ \\
        $54$ &  $293$ & $585$ &$ 881$  & $1170$ & $-    $ & $-   $ & $-   $ & $-   $ &  $-   $ &  $ 292,33 \pm 2,87 $ \\
        $56$ &  $281$ & $563$ &$ 844$  & $-   $ & $-    $ & $-   $ & $-   $ & $-   $ &  $-   $ &  $ 281,50 \pm 0,50 $ \\
        $58$ &  $270$ & $536$ &$ 806$  & $1073$ & $-    $ & $-   $ & $-   $ & $-   $ &  $-   $ &  $ 267,67 \pm 1,70 $ \\
        $60$ &  $259$ & $518$ &$ 776$  & $1039$ & $-    $ & $-   $ & $-   $ & $-   $ &  $-   $ &  $ 260,00 \pm 2,16 $ \\
        $62$ &  $251$ & $503$ &$ 754$  & $1001$ & $-    $ & $-   $ & $-   $ & $-   $ &  $-   $ &  $ 250,00 \pm 2,16 $ \\
        $64$ &  $244$ & $488$ &$ 728$  & $971 $ & $-    $ & $-   $ & $-   $ & $-   $ &  $-   $ &  $ 242,33 \pm 1,70 $ \\
        $66$ &  $236$ & $473$ &$ 705$  & $941 $ & $-    $ & $-   $ & $-   $ & $-   $ &  $-   $ &  $ 235,00 \pm 2,16 $ \\
        $68$ &  $229$ & $456$ &$ 683$  & $911 $ & $1136 $ & $-   $ & $-   $ & $-   $ &  $-   $ &  $ 226,75 \pm 1,09 $ \\
        $70$ &  $221$ & $446$ &$ 664$  & $885 $ & $1110 $ & $-   $ & $-   $ & $-   $ &  $-   $ &  $ 222,25 \pm 2,95 $ \\
        $72$ &  $218$ & $431$ &$ 645$  & $863 $ & $1076 $ & $-   $ & $-   $ & $-   $ &  $-   $ &  $ 214,50 \pm 2,06 $ \\
        $74$ &  $206$ & $409$ &$ 611$  & $816 $ & $1020 $ & $-   $ & $-   $ & $-   $ &  $-   $ &  $ 203,50 \pm 1,12 $ \\
        $76$ &  $199$ & $398$ &$ 600$  & $799 $ & $998  $ & $1196$ & $-   $ & $-   $ &  $-   $ &  $ 199,40 \pm 1,36 $ \\ 
        $78$ &  $195$ & $390$ &$ 585$  & $776 $ & $971  $ & $1166$ & $-   $ & $-   $ &  $-   $ &  $ 194,20 \pm 1,60 $ \\ 
        $80$ &  $191$ & $379$ &$ 570$  & $758 $ & $949  $ & $1136$ & $-   $ & $-   $ &  $-   $ &  $ 189,00 \pm 1,67 $ \\ 
        $85$ &  $180$ & $356$ &$ 536$  & $713 $ & $893  $ & $1069$ & $1249$ & $-   $ &  $-   $ &  $ 178,17 \pm 1,86 $ \\ 
        $90$ &  $169$ & $338$ &$ 510$  & $679 $ & $848  $ & $1013$ & $1185$ & $-   $ &  $-   $ &  $ 169,33 \pm 2,36 $ \\ 
        $95$ &  $161$ & $319$ &$ 476$  & $638 $ & $795  $ & $956 $ & $1114$ & $1275$ &  $-   $ &  $ 159,14 \pm 1,96 $ \\ 
        $100$ & $150$ & $304$ &$ 454$  & $608 $ & $754  $ & $908 $ & $1061$ & $1211$ &  $-   $ &  $ 151,57 \pm 2,82 $ \\ 
        $110$ & $139$ & $278$ &$ 413$  & $551 $ & $686  $ & $825 $ & $960 $ & $1099$ &  $-   $ &  $ 137,14 \pm 1,88 $ \\ 
        $120$ & $128$ & $251$ &$ 379$  & $506 $ & $630  $ & $758 $ & $881 $ & $1005$ &  $1133$ &  $ 125,63 \pm 2,18 $ \\
        $130$ & $233$ & $465$ &$ 694$  & $926 $ & $1159 $ & $1391$ & $-   $ & $-   $ &  $-   $ &  $ 231,60 \pm 1,36 $ \\ 
    \bottomrule
    \end{tabular}
    \end{table}