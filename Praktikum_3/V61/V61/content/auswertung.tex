\section{Auswertung}
\label{sec:auswertung}

Die folgenden Ausgleichsfunktionen werden mit Funktion $curve\_fit$ aus der Python\cite{py}-Bibliothek $scipy$\cite{2020SciPy-NMeth} bestimmt.
Die Fehlerrechnung wird mithilfe der Bibliothek $uncertainties$ \cite{unp} durchgeführt.
Die Grafiken werden durch die Bibliothek $matplotlib$\cite{Hunter:2007} erstellt. Der gemessene Strom an der Photodiode wird in Ampere ausgelesen.  
Aufgrund von Schwankungen des Photostroms wird eine Unsicherheit von $10 \%$ angenommen.

\subsection{Untersuchung der Polarisation}
\label{sec:Polarisation}

Die Intentsitätsverteilung in Abhängigkeit vom Raumwinkel $\theta$ ist durch die Funktion
\begin{equation*}
    I(\theta) = I_0 \cdot \cos(\theta + \theta_0)² %am besten in der Theorie erklären wo diese Formel überhaupt herkommt
\end{equation*}
gegeben, wobei $\theta_0 $ ein konstanter Offset gegenüber der optischen Achse ist. 
Die Parameter des Fit werden als
\begin{align*}
    I_0      =& (69,16 \pm    1,27) \, \unit{\micro\ampere}   \text{und}  \\
    \theta_0 =& (92,18 \pm    0,12) \, °                        \\
\end{align*}
bestimmt. Die grafische Darstellung ist in  \autoref{fig:pol_pol} und \autoref{fig:pol}

\begin{figure}[H]
    \centering
    \includegraphics[width=\textwidth]{build/polarisation.pdf}
    \caption{Der Photostrom hinter einem Polfilter in Abhängigkeit des Raumwinkels.}
    \label{fig:pol}
\end{figure}

\begin{figure}[H]
    \centering
    \includegraphics[width=\textwidth]{build/polarisation_pol.pdf}
    \caption{Der Photostrom hinter einem Polfilter in Abhängigkeit des Raumwinkels in einem polaren Koordinatensystem.}
    \label{fig:pol_pol}
\end{figure}

\subsection{Bestimmung der Wellenlänge}
\label{sec:Wellenlänge}

Um die Wellenlänge des Lasers zu bestimmen, wird der Strahl an einem Gitter gebeugt. Auf einem Schirm, der sich in einem Abstand $d$ hinter dem Gitter befindet, kann ein Interferenzmuster beobachtet werden. Es wird der Abstand $\Delta x_\text{n}$ zum zentralen Maxima $n = 0$ gemessen und mithilfe der Bragg-Bedingung
\begin{equation}
    n \lambda = 2 g \sin{\alpha}
    \label{eq:bragg}
\end{equation}
kann die Wellenlänge $\lambda$ berechnet werden, dabei ist $n$ das $n$-te Maxima von $n=0$, $g$ ist die Gitterkonstante des verwendeten Gitters und $\alpha$ der Winkel zwischen der optischen Achse des dem betrachteten Maximas.
$\alpha$ wird über die trigonometrische Beziehung  
\begin{equation*}
    \alpha = \arctan(\frac{\Delta x_\text{n}}{d})
    \label{eq:trig}
\end{equation*} 
bestimmt. Die aufgenommenen Messwerte sind in \autoref{tab:Well_80}, \autoref{tab:Well_100}, \autoref{tab:Well_600} und \autoref{tab:Well_1600} zu sehen.

\begin{table}[H]
    \centering
    \caption{Der Abstand des $n$-ten Maximas zu $n = 0$. Negative $n$ bezeichnen Maxima die links von $n = 0$ liegen und positive $n$, die die rechts von $n = 0$ liegen. Die Messwerte wurden für $d = 77,3 \, \unit{\centi\meter}$ und $g = \dfrac{1 \, \unit{\milli\meter}}{80}$ aufgenommen.}
    \label{tab:Well_80}
    \begin{tabular}{c c c c}
    \toprule
     $n$ & $\Delta x_\text{n}$ &   $n$ & $\Delta x_\text{n}$\\
    \midrule
        $-8$  &$ 34,5$ &        $1$ &  $ 3,9$ \\
        $-7$  &$ 29,5$ &        $2$ &  $ 7,9$ \\
        $-6$  &$ 24,8$ &        $3$ &  $12,2$ \\
        $-5$  &$ 20,3$ &        $4$ &  $16,1$ \\
        $-4$  &$ 16,0$ &        $5$ &  $20,2$ \\
        $-3$  &$ 12,1$ &        $6$ &  $24,5$ \\
        $-2$  &$ 8,0 $ &        $7$ &  $29,0$ \\
        $-1$  &$ 4,1 $ &        $8$ &  $33,7$ \\
       $ 0 $  &$ 0,0 $ &        $-$ &  $   -$ \\
    \bottomrule
    \end{tabular}
    \end{table}

\begin{table}[H]
    \centering
    \caption{Der Abstand des $n$-ten Maximas zu $n = 0$. Negative $n$ bezeichnen Maxima die links von $n = 0$ liegen und positive $n$, die die rechts von $n = 0$ liegen. Die Messwerte wurden für $d = 77,3 \, \unit{\centi\meter}$ und $g = \dfrac{1 \, \unit{\milli\meter}}{100}$ aufgenommen.}
    \label{tab:Well_100}
    \begin{tabular}{c c c c}
    \toprule
     $n$ & $\Delta x_\text{n}$ &   $n$ & $\Delta x_\text{n}$\\
    \midrule
    $-6$ &  $31,7 $&      $1$   &$4,9$ \\
    $-5$ &  $25,7 $&      $2$   &$9,8$ \\
    $-4$ &  $20,2 $&      $3$   &$14,9$ \\
    $-3$ &  $14,9 $&      $4$   &$20,1$ \\
    $-2$ &  $9,8  $&      $5$   &$25,5$ \\
    $-1$ &  $4,9  $&      $6$   &$31,3$ \\   
    $ 0$ &  $0,0  $&      $-$   &$-$    \\
    \bottomrule
    \end{tabular}
    \end{table}

\begin{table}[H]
    \centering
    \caption{Der Abstand des $n$-ten Maximas zu $n = 0$. Negative $n$ bezeichnen Maxima die links von $n = 0$ liegen und positive $n$, die die rechts von $n = 0$ liegen. Die Messwerte wurden für $d = 29,4 \, \unit{\centi\meter}$ und $g = \dfrac{1 \, \unit{\milli\meter}}{600}$ aufgenommen.}
    \label{tab:Well_600}
    \begin{tabular}{c c c c}
    \toprule
     $n$ & $\Delta x_\text{n}$ &   $n$ & $\Delta x_\text{n}$\\
    \midrule
    $-2$ & $36,0$ & $1$ & $ 11,9$ \\
    $-1$ & $12,1$ & $2$ & $ 33,4$ \\
    $0$ & $ 0.0$  & $- $ & $ - $\\
   \bottomrule
    \end{tabular}
    \end{table}
\begin{table}[H]
    \centering
    \caption{Der Abstand des $n$-ten Maximas zu $n = 0$. Negative $n$ bezeichnen Maxima die links von $n = 0$ liegen und positive $n$, die die rechts von $n = 0$ liegen. Die Messwerte wurden für $d = 29,4 \, \unit{\centi\meter}$ und $g = \dfrac{1 \, \unit{\milli\meter}}{600}$ aufgenommen.}
    \label{tab:Well_600}
    \begin{tabular}{c c c c}
    \toprule
     $n$ & $\Delta x_\text{n}$ &   $n$ & $\Delta x_\text{n}$\\
    \midrule
    $-2$ & $36,0$ & $1$ & $ 11,9$ \\
    $-1$ & $12,1$ & $2$ & $ 33,4$ \\
    $ 0$ & $ 0,0$ & $-$ & $ -   $\\
   \bottomrule
    \end{tabular}
    \end{table}
\begin{table}[H]
    \centering
    \caption{Der Abstand des $n$-ten Maximas zu $n = 0$. Negative $n$ bezeichnen Maxima die links von $n = 0$ liegen und positive $n$, die die rechts von $n = 0$ liegen. Die Messwerte wurden für $d = 29,4 \, \unit{\centi\meter}$ und $g = \dfrac{1 \, \unit{\milli\meter}}{600}$ aufgenommen.}
    \label{tab:Well_1200}
    \begin{tabular}{c c c c}
    \toprule
     $n$ & $\Delta x_\text{n}$ &   $n$ & $\Delta x_\text{n}$\\
    \midrule
    $-1$ & $19,8$   & $1$ & $ 17,9$   \\
    $0$  & $0   $   & $-$ & $ -   $   \\
    \bottomrule
    \end{tabular}
    \end{table}

Um systematische Fehler zu vermeiden wird der Abstand zu den beiden Maxima gleicher Ordnung gemittelt. 
Aus diesen gemittelten Abständen werden über die Gleichungen \eqref{eq:trig} und \eqref{eq:bragg} die Wellenlängen in \autoref{tab:Wellenlänge_ubersicht} berechnet. 
Es wird angenommen, dass die Maxima bis auf $1 \unit{\milli\meter}$ bestimmt werden konnten.

\begin{table}[H]
    \centering
    \caption{Die berechnete Wellenlänge zu den verwendeten Gittern und Abständen.}
    \label{tab:Wellenlänge_ubersicht}
    \begin{tabular}{c c c}
    \toprule
     $g \mathbin{/} \unit{\milli\meter}$ & $d \mathbin{/} \unit{\centi\meter}$ &$\lambda \mathbin{/} \unit{\nano\meter}$ \\
    \midrule
    $1/80  $   & $77,3$   & $637,10 \pm 5,94$\\
    $1/100 $   & $77,3$   & $630,12 \pm 1,41$\\
    $1/600 $   & $29,4$   & $632,81 \pm 2,99$\\
    $1/1200$   & $16,4$   & $635,32 \pm 0,54$\\
    \bottomrule
    \end{tabular}
    \end{table}

\subsection{Messung der TEM Moden}
\label{sec:TEM_moden}