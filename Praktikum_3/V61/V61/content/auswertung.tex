\section{Auswertung}
\label{sec:auswertung}

Die folgenden Ausgleichsfunktionen werden mit Funktion $curve\_fit$ aus der Python\cite{py}-Bibliothek $scipy$\cite{2020SciPy-NMeth} bestimmt.
Die Fehlerrechnung wird mithilfe der Bibliothek $uncertainties$ \cite{unp} durchgeführt.
Die Grafiken werden durch die Bibliothek $matplotlib$\cite{Hunter:2007} erstellt. Der gemessene Strom an der Photodiode wird in Ampere ausgelesen.  
Aufgrund von Schwankungen des Photostroms wird eine Unsicherheit von $10 \%$ angenommen.

\subsection{Untersuchung der Polarisation}
\label{sec:Polarisation}

Die Intentsitätsverteilung in Abhängigkeit vom Raumwinkel $\theta$ ist durch die Funktion
\begin{equation*}
    I(\theta) = I_0 \cdot \cos(\theta + \theta_0)² %am besten in der Theorie erklären wo diese Formel überhaupt herkommt
\end{equation*}
gegeben, wobei $\theta_0 $ ein konstanter Offset gegenüber der optischen Achse ist. 
Die Parameter des Fit werden als
\begin{align*}
    I_0      =& (69,16 \pm    1,27) \, \unit{\micro\ampere}   \text{und}  \\
    \theta_0 =& (92,18 \pm    0,12) \, °                        \\
\end{align*}
bestimmt. Die grafische Darstellung ist in  \autoref{fig:pol_pol} und \autoref{fig:pol}

\begin{figure}[H]
    \centering
    \includegraphics[width=\textwidth]{build/polarisation.pdf}
    \caption{Der Photostrom hinter einem Polfilter in Abhängigkeit des Raumwinkels.}
    \label{fig:pol}
\end{figure}

\begin{figure}[H]
    \centering
    \includegraphics[width=\textwidth]{build/polarisation_pol.pdf}
    \caption{Der Photostrom hinter einem Polfilter in Abhängigkeit des Raumwinkels in einem polaren Koordinatensystem.}
    \label{fig:pol_pol}
\end{figure}

\subsection{Bestimmung der Wellenlänge}
\label{sec:Wellenlänge}

Um die Wellenlänge des Lasers zu bestimmen, wird der Strahl an einem Gitter gebeugt. Es wird der Abstand zum zentralen Maxima $n = 0$ gemessen und mithilfe der Bragg-Bedingung
\begin{equation}
    n \lambda = 2*g *\sin{\alpha}
    \label{eq:bragg}
\end{equation}
kann die Wellenlänge $\lambda$ berechnet werden, dabei ist $n$ das $n$-te Maxima von $n=0$, $g$ ist die Gitterkonstante des verwendeten Gitters und $\alpha$ der Winkel zwischen der optischen Achse des Maximas.