\section{Auswertung}
\label{sec:auswertung}


Folgende Ausgleichsfunktionen werden mit Funktion $curve\_fit$ aus der Python\cite{py}-Bibliothek $scipy$\cite{2020SciPy-NMeth} bestimmt.
Die Integration wird mithilfe der Funktion $scipy.integrate.simpson$ \cite{2020SciPy-NMeth}verwirklicht.
Die Fehlerrechnung wird mithilfe der Bibliothek $uncertainties$ \cite{unp} durchgeführt.
Grafiken werden durch die Bibliothek $matplotlib$\cite{Hunter:2007} erstellt. 

\subsection{Bestimmung der Heizraten}

In \autoref{tab:Messdaten1} und \autoref{tab:Messdaten2} sind die Messdaten und die Heizraten der verschiedenen Messreihen zu sehen.
Es ergeben sich die durchschnittlichen Heizraten von 
\begin{align}
    \bar{b_1} &=   1,49 \pm  0,23 \label{eq:TheoH1} \\
    \bar{b_2} &=   1,94 \pm  0,35 \label{eq:TheoH2} 
\end{align}

\begin{table}[H]
    \centering
    \caption{Messdaten und Heizrate für eine angestrebte Heizrate von $ b_1 = 1,5 \, \unit{\dfrac{\kelvin}{\minute}}$.}
    \label{tab:Messdaten2}
    \begin{tabular}{c c c c c c c c}
    \toprule
     $t \mathbin{/} \unit{\min}$ & $T \mathbin{/} \unit{\celsius}$ & $I \mathbin{/} \unit{\pico\ampere} $&  $b \mathbin{/} \unit{\dfrac{\kelvin}{\minute}}$ & $t \mathbin{/} \unit{\min}$ & $T \mathbin{/} \unit{\celsius}$ & $I \mathbin{/} \unit{\pico\ampere} $&  $b \mathbin{/} \unit{\dfrac{\kelvin}{\minute}}$\\
    \midrule
    0   &-50,0   &0,195  &-    &  34  &1,2     &1,0    &1,3 \\  
    1   &-47,8   &0,21   &2,2  &  35  &2,8     &1,0    &1,6 \\
    2   &-46,4   &0,225  &1,4  &  36  &4,4     &1,1    &1,6 \\
    3   &-44,8   &0,24   &1,6  &  37  &6,1     &1,2    &1,7 \\
    4   &-43,3   &0,26   &1,5  &  38  &7,8     &1,3    &1,7 \\
    5   &-41,9   &0,28   &1,4  &  39  &9,4     &1,4    &1,6 \\
    6   &-40,4   &0,3    &1,5  &  40  &10,8    &1,5    &1,4 \\
    7   &-39,0   &0,33   &1,4  &  41  &12,3    &1,55   &1,5 \\
    8   &-37,6   &0,37   &1,4  &  42  &13,3    &1,65   &1,0 \\
    9   &-36,3   &0,42   &1,3  &  43  &14,9    &1,7    &1,6 \\
    10  &-34,7   &0,5    &1,6  &  44  &16,2    &1,75   &1,3 \\
    11  &-33,1   &0,59   &1,6  &  45  &17,7    &1,8    &1,5 \\
    12  &-31,5   &0,7    &1,6  &  46  &19,2    &1,85   &1,5 \\
    13  &-30,0   &0,86   &1,5  &  47  &20,7    &1,9    &1,5 \\
    14  &-28,4   &1,05   &1,6  &  48  &22,3    &1,95   &1,6 \\
    15  &-26,8   &1,25   &1,6  &  49  &23,8    &2,0    &1,5 \\    
    16  &-25,3   &1,55   &1,5  &  50  &25,3    &2,05   &1,5 \\
    17  &-23,8   &1,85   &1,5  &  51  &26,8    &2,0    &1,5 \\
    18  &-22,4   &2,3    &1,4  &  52  &28,2    &1,95   &1,4 \\
    19  &-20,9   &2,65   &1,5  &  53  &29,7    &1,9    &1,5 \\    
    20  &-19,4   &3,4    &1,5  &  54  &31,2    &1,8    &1,5 \\
    21  &-17,8   &3,9    &1,6  &  55  &32,8    &1,7    &1,6 \\
    22  &-16,3   &4,5    &1,5  &  56  &34,2    &1,6    &1,4 \\
    23  &-14,8   &4,9    &1,5  &  57  &35,8    &1,5    &1,6 \\
    24  &-13,2   &5,2    &1,6  &  58  &37,5    &1,4    &1,7 \\
    25  &-11,6   &5,3    &1,6  &  59  &39,1    &1,25   &1,6 \\
    26  &-10,1   &5,0    &1,5  &  60  &40,7    &1,5    &1,6 \\
    27  &-8,6    &4,5    &1,5  &  61  &42,3    &1,0    &1,6 \\
    28  &-7,1    &3,8    &1,5  &  62  &44,0    &0,85   &1,7 \\
    29  &-5,7    &3,0    &1,4  &  63  &45,5    &0,75   &1,5 \\
    30  &-4,3    &2,4    &1,4  &  64  &47,0    &0,65   &1,5 \\
    31  &-2,8    &1,65   &1,5  &  65  &48,4    &0,6    &1,4 \\
    32  &-1,4    &1,25   &1,4  &  66  &49,7    &0,5    &1,3 \\
    33  &-0,1    &1,05   &1,3  &  67  &51,2    &0,48   &1,5 \\
    \bottomrule
    \end{tabular}
    \end{table}

    \begin{table}[H]
        \centering
        \caption{Messdaten und Heizrate für eine angestrebte Heizrate von $ b_2 = 2 \, \unit{\dfrac{\kelvin}{\minute}}$.}
        \label{tab:Messdaten1}
        \begin{tabular}{c c c c c c c c}
        \toprule
         $t \mathbin{/} \unit{\min}$ & $T \mathbin{/} \unit{\celsius}$ & $I \mathbin{/} \unit{\pico\ampere} $&  $b \mathbin{/} \unit{\dfrac{\kelvin}{\minute}}$ & $t \mathbin{/} \unit{\min}$ & $T \mathbin{/} \unit{\celsius}$ & $I \mathbin{/} \unit{\pico\ampere} $&  $b \mathbin{/} \unit{\dfrac{\kelvin}{\minute}}$\\
        \midrule
        0   &-50,0   &0,015   & 0,0&26  &0,9    & 0,155 & 1,9 \\
        1   &-48,1   &0,0165  & 1,9&27  &2,9    & 0,135 & 2,0 \\
        2   &-46,3   &0,018   & 1,8&28  &4,8    & 0,13  & 1,9\\
        3   &-44,8   &0,02    & 1,5&29  &7,0    & 0,145 & 2,2 \\
        4   &-43,3   &0,022   & 1,5&30  &9,0    & 0,16  & 2,0\\
        5   &-41,7   &0,026   & 1,6&31  &11,0   & 0,175 & 2,0 \\
        6   &-40,0   &0,03    & 1,7&32  &13,0   & 0,185 & 2,0 \\
        7   &-35,7   &0,036   & 2,3&33  &15,0   & 0,20  & 2,0\\
        8   &-35,1   &0,049   & 2,6&34  &16,8   & 0,215 & 1,8 \\
        9   &-32,6   &0,066   & 2,5&35  &18,7   & 0,225 & 1,9 \\
        10  &-30,5   &0,085   & 2,1&36  &20,5   & 0,235 & 1,8 \\
        11  &-28,7   &0,105   & 1,8&37  &22,7   & 0,250 & 2,2 \\
        12  &-26,9   &0,135   & 1,8&38  &25,2   & 0,26  & 2,5\\
        13  &-25,2   &0,17    & 1,7&39  &27,4   & 0,275 & 2,2 \\
        14  &-23,3   &0,215   & 1,9&40  &29,6   & 0,27  & 2,2\\
        15  &-21,2   &0,28    & 2,1&41  &31,6   & 0,255 & 2,0 \\
        16  &-19,0   &0,36    & 2,2&42  &33,5   & 0,245 & 1,9 \\
        17  &-16,8   &0,46    & 2,2&43  &35,6   & 0,22  & 2,1\\
        18  &-14,8   &0,55    & 2,0&44  &37,6   & 0,20  & 2,0\\
        19  &-12,7   &0,64    & 2,1&45  &39,5   & 0,175 & 1,9 \\
        20  &-10,6   &0,67    & 2,1&46  &41,5   & 0,155 & 2,0 \\
        21  &-8,8    &0,66    & 1,8&47  &43,6   & 0,13  & 2,1\\
        22  &-6,7    &0,57    & 2,1&48  &45,4   & 0,110 & 1,8 \\
        23  &-4,7    &0,45    & 2,0&49  &47,3   & 0,093 & 1,9 \\
        24  &-2,8    &0,33    & 1,9&50  &49,1   & 0,078 & 1,8 \\
        25  &-1,0    &0,21    & 1,8&51  &50,8   & 0,07  & 1,7\\
        \bottomrule
        \end{tabular}
        \end{table}
  
  

\subsection{Signalhintergrundtrennung}
\label{sec:Signalhintergrundtrennung}
        
Die grafische Darstellung der Messdaten ist in \autoref{fig:Messdaten1} und \autoref{fig:Messdaten2} zusehen.
Es wird ein linearer Hintergrund angenommen. Für die Berechnung des Hintergrundes werden die orangen Messwerte verwendet.
Zusätzlich werden die Messwerte, die sich im Plot rechts vom Hintergrund befinden, ausgeschlossen, da diese durch nicht gewünschte Effekte zustande kommen. 
\begin{figure}[H]
    \centering
    \includegraphics[]{build/Messung1.pdf}
    \caption{Plot der Messdaten zusammen mit der linearen Regression des Hintergrundes für die erste Messreihe.}
    \label{fig:Messdaten1}
\end{figure}
\begin{figure}[H]
    \centering
    \includegraphics[]{build/Messung2.pdf}
    \caption{Plot der Messdaten zusammen mit der linearen Regression des Hintergrundes für die zweite Messreihe.}
    \label{fig:Messdaten2}
\end{figure}
Die Parameter der Ausgleichsfunktionen
\begin{equation*}
    I\left(T\right) = \alpha_i T + \beta_i
\end{equation*}
für die erste bzw. zweite Messreihe lassen sich zu 
\begin{align*}
    \alpha_1 &=         (1,65 \pm 0,03 ) \cdot 10^{-14} \,\unit{\dfrac{1}{\kelvin}} \\
    \beta_1  &=         (-3,51 \pm-0,08) \cdot 10^{-12} \,\unit{\ampere} 
\end{align*}
und 
\begin{align*}
    \alpha_2 &=         ( 2,38 \pm 0,06 ) \cdot 10^{-14}  \,\unit{\dfrac{1}{\kelvin}} \\
    \beta_2  &=         (-5,23 \pm 0,02 ) \cdot 10^{-12} \,\unit{\ampere} 
\end{align*}
bestimmen.
Der Hintergrund wird von den Messwerten subtrahiert und die bereinigten Daten sind in 
\autoref{fig:BereinigteDaten1} und \autoref{fig:BereinigteDaten2} aufgetragen.

\begin{figure}[H]
    \centering
    \includegraphics[]{build/Messung1_bereinigt.pdf}
    \caption{Plot der bereinigten Daten der ersten Messreihe.}
    \label{fig:BereinigteDaten1}
\end{figure}
\begin{figure}[H]
    \centering
    \includegraphics[]{build/Messung2_bereinigt.pdf}
    \caption{Plot der bereinigten Daten der zweiten Messreihe.}
    \label{fig:BereinigteDaten2}
\end{figure}
\subsection{Berechnung über die Stromdichte}
\label{sec:BerechnungStromdichte}

Die Relaxationszeit wird über die Hilfsfunktion $f(T) = \ln \left(\int^\infty_{I(T)} \dif t' I(t') \right) - \ln (I(T))$ bestimmt.
Einige der Daten konnten nicht mit in die Berechnung eingebracht werden,
da für $\ln \left(f(T)\right)$ Werte $f(T) < 0$ außerhalb des Definitionsbereichs liegen. 
$f(T)$ wird durch eine Gerade der Form $f(T) = \dfrac{a_i}{k_b T} + b_i$ genähert.
In die Berechnung werden nur die Punkte einbezogen, die annähernd auf einer Geraden liegen.
Die grafische Darstellung der Ausgleichsgeraden sind in \autoref{fig:int1} und \autoref{fig:int2} zu sehen.
Die Parameter werden als 
\begin{align*}
    \alpha_1 &=   (5,76   \pm 0,27) \cdot 10^{-20} \, \unit{\joule} , \\
    \beta_1 &= -15,43   \pm 0,79                                ,\\
    \alpha_2 &=   (5,57   \pm 0,31) \cdot 10^{-20} \,\unit{\joule} \, \, \text{und}\\
    \beta_1 &= -14,80   \pm 0,88\\
\end{align*}
bestimmt.
Über die \eqref{eq:lnrelaxozeitpol} werden die Aktivierungsenergien zu 
\begin{align*}   
    W_1 &=   (0,359 \pm 0,017) \, \unit{\electronvolt} \, \, \text{und}  \\
    W_2 &=   (0,348 \pm 0,019) \, \unit{\electronvolt} \\
\end{align*}
und die Relaxationszeiten zu 
\begin{align*}
    \tau_{0}^{1} &=   1,99 \pm 1,56 \cdot 10^{-7} \, \unit{\second} \, \, \text{und} \\
    \tau_{0}^{2} &=   3,75 \pm 3,31 \cdot 10^{-7} \, \unit{\second}         \\
\end{align*}
bestimmt.
\begin{figure}[H]
    \centering
    \includegraphics[]{build/int1.pdf}
    \caption{Plot von $\ln \left(f(T)\right)$ für die erste Messreihe.}
    \label{fig:int1}
\end{figure}
\begin{figure}[H]
    \centering
    \includegraphics[]{build/int2.pdf}
    \caption{Plot von $\ln \left(f(T)\right)$ für die zweite Messreihe.}
    \label{fig:int2}
\end{figure}



\subsection{Berechnung über die Polarisation}
\label{sec:BerechnungPolarisation}

Es wird der Logarithmus des gemessenen Stroms geplottet und die Daten im kleinen Temperaturbereich durch eine Gerade genähert.
Die Ausgleichsgeraden sind in \autoref{fig:pol1} und \autoref{fig:pol2} zusehen.
Für die Parameter der Geraden werden die Werte 
\begin{align*}
    \alpha_1 &=   -1,65 \pm 0,07  \cdot 10^{-19} \, \unit{\joule} , \\
    \beta_1  &=   20,31 \pm 2,11                              , \\
    \alpha_2 &=   -1,43 \pm 0,08  \cdot 10^{-19} \, \unit{\joule} \, \,\text{und} \\
    \beta_1  &=   14,00 \pm 2,51
\end{align*}
bestimmt.


\begin{figure}[H]
    \centering
    \includegraphics[]{build/pol1.pdf}
    \caption{Plot von $\ln(I\mathbin{/} \unit{\ampere}) $ für die erste Messreihe.}
    \label{fig:pol1}
\end{figure}
\begin{figure}[H]
    \centering
    \includegraphics[]{build/pol2.pdf}
    \caption{Plot von $\ln(I\mathbin{/} \unit{\ampere}) $ für die zweite Messreihe.}
    \label{fig:pol2}
\end{figure}

Daraus ergeben sich die Aktivierungsenergien 
\begin{align*}
    W_1 &= 1,03 \pm 0,05 \, \unit{\electronvolt}  \, \,  \text{und} \\
    W_2 &= 0,89 \pm 0,05 \, \unit{\electronvolt}    \,.
\end{align*}
Über \eqref{eq:tau0pol}, mit $T¹_M = 261,55 \, \unit{\kelvin}$, $T²_M = 262,55\, \unit{\kelvin}$ 
und den Heizraten \eqref{eq:TheoH1} und \eqref{eq:TheoH2}, werden die Relaxationszeiten zu 
\begin{align*}
    \tau¹_0 &= \left( 3,57 \pm 7,31 \cdot 10^{-18} \right) \unit{\second} \, \, \text{und}  \\
    \tau²_0 &= \left( 1,46 \pm 3,60 \cdot 10^{-15} \right) \unit{\second}   
\end{align*}
bestimmt.

\subsection{Relaxationszeit in Abhängigkeit der Temperatur}

Die Relaxationszeiten als Funktion der Temperatur aus dem Polarisationsstrom ist in \autoref{fig:taupol} aufgetragen und aus der Stromdichte in \autoref{fig:tauint} 
\begin{figure}[H]
    \centering
    \includegraphics[]{build/taupol.pdf}
    \caption{Plot von $\ln \left(\frac{\tau(T)}{\si{\second}} \right)$ für den Polarisationsansatz. Dabei wird $\tau(T)$ für den Logarithmus von Einheiten bereinigt.}
    \label{fig:taupol}
\end{figure}
\begin{figure}[H]
    \centering
    \includegraphics[]{build/tauint.pdf}
    \caption{Plot von $\ln \left(\frac{\tau(T)}{\si{\second}} \right)$ für den Ansatz über die Stromdichte. Dabei wird $\tau(T)$ für den Logarithmus von Einheiten bereinigt.}
    \label{fig:tauint}
\end{figure}