\section{Diskussion}
\label{sec:Diskussion}

Es kann gesehen werden, dass die erreichten Heizraten von 
\begin{align*}
    \bar{b_1} &=   1,49 \pm  0,23 \\
    \bar{b_2} &=   1,94 \pm  0,35  
\end{align*}

nah an den angestrebten Heizraten von $ b_1 = 1,5 \, \unit{\dfrac{\kelvin}{\minute}}$ 
und $ b_2 = 2 \, \unit{\dfrac{\kelvin}{\minute}}$ liegen.
Gegen Ende der zweiten Messreihe konnte keine höhere Heizrate erzielt werden, da die Heizspannung auf $110 \,\si{\volt}$. \\
Die Aktivierungsenergien von 

\begin{align*}
W_1 &=   (0,359 \pm 0,017) \, \unit{\electronvolt} \, \, \text{und}  \\
W_2 &=   (0,348 \pm 0,019) \, \unit{\electronvolt}
\end{align*} 

liegen nahe beieinander. \\
Auch die Aktivierungsenergien aus der Stromdichte

\begin{align*}
    W_1 &= 1,03 \pm 0,05 \, \unit{\electronvolt}  \, \,  \text{und} \\
    W_2 &= 0,89 \pm 0,05 \, \unit{\electronvolt}    \,.
\end{align*}

liegen eng zusammen. Jedoch liegen die Aktivierungsenergien aus der Stromdichte deutlich höher.
Die Relaxationszeiten aus dem Polarisationsstrom liegen in der gleichen Größenordnung.
Bei der Stromdichte gibt es jedoch einen Unterschied von drei Größenordnung.
Ein kleiner Unterschied in den Aktivierungsenergie kann zu großen Unterschieden in der Relaxationszeit führen, weil 
die Aktivierungsenergie im Exponenten von \autoref{eq:tau0pol} steht.
Eine bessere Abschirmung der Messgeräte ist notwendig, da Bewegungen im Raum, 
oder elektronische Geräte kleinen Ströme maßgeblich beeinflussen können.
Weitere Fehler treten durch eine Verunreinigung der Probe auf sowie ihr Kontakt mit Wasser.
Dieser Kontakt könnte den zweiten Peak in den Messdaten erklären.
