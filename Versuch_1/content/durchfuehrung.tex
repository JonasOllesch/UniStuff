\section{Durchführung}
Zwei Pendel werden durch das Verschieben der befestigten Massen, $m=1\,\si{kg}$, auf die gleiche Länge gebracht, sodass sie mit näherungsweise derselben Frequenz $ω$ schwingen.
Dabei wird die Pendellänge im ersten Messvorgang zunächst auf $l_1=70\,\si{cm}$ gestellt. \\

Zu Beginn wird die Kopplungsfeder entfernt, um die unabhängigen Periodendauern $T_1$ und $T_2$ der Pendel messen zu können. Dabei werden zehn Mal je fünf Periodendauern gemessen.
Anschließend wird die Kopplungsfeder erneut angebracht, es werden, erneut zehn Mal, die fünffachen Periodendauern von \textbf{gleichsinniger Schwingung ($T_+$)} und \textbf{gegensinniger Schwingung ($T_-$)} 
sowie die Schwingungsdauer ($T$) und Schwebungsdauer ($T_S$) der \textbf{gekoppelten Schwingung} gemessen.
Der Messvorgang wird ein weiteres Mal mit einer Pendellänge von $l_2=100\,\si{cm}$ wiederholt.
\label{sec:Durchführung}

% Fertig und korrigiert
