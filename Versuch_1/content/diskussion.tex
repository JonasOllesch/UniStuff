\section{Diskussion}
\label{sec:Diskussion}

\subsection{Abweichung der Messwerte}
Die Abweichungen der gemessenen Frequenzen von den Theoriewerten berechnen sich durch
\begin{equation}
    ω_{abw} = 100\% \frac{ω_{mess}-ω_{theo}}{ω_{theo}} \text{.}
\end{equation}

Es ergeben sich also
\begin{align}
ω_{+,1,abw} & = 1.422\%   & ω_{+,2,abw} & = 1.409\% \\
ω_{-,1,abw} & = 8.758\%   & ω_{-,2,abw} & = 7.872\%\\
ω_{S,1,abw} & = 760.976\% & ω_{S,2,abw} & = 937.156\%
\end{align}
Bei Betrachtung der Abweichungen der Messfrequenzen von den Theoriewerten fällt auf, dass diese, zumindest
bei der gleich- und gegenphasigen Schwingung, einigermaßen gering sind. Vor allem für die gleichphasige Schwingung
sind diese mit knapp 1,5\% sehr gering. Die der gegenphasigen Schwingung sind mit knapp 8\% um einiges höher, können aber auch noch angenommen werden.
Die Abweichungen der Schwebungsfrequenzen ist dagegen sehr groß, allerdings musste zur Berechnung der experimentell
bestimmte Kopplungsgrad $K$ genutzt werden.

\subsection{Fehlerdiskussion}

Die negativen Werte der theoretischen Schwebungsfrequenzen lassen sich auf einen Fehler bei der Bestimmung des Kopplungsgrades zurückführen.
Solange der Kopplungsgrad positiv ist, gilt stets $\sqrt{\frac{g}{l}}\leq \sqrt{\frac{g+2K}{l}}$.
Dieser Fehler hat unter anderem seinen Ursprung darin, dass nicht über eine Lichtschranke, sondern manuell über eine \\
(Smartphone-)Stoppuhr gemessen wurde. Diese Messung wird durch die menschliche Reaktionszeit von knapp $200 \unit{\milli\second}$ sehr ungenau.
Zwar wurde durch die Messung von je fünf Periodendauern anstelle von einer versucht, diesen Fehler einzugrenzen, dennoch bleibt er bestehen. \\

Eine weitere Fehlerquelle ist die angenommene Kleinwinkelnäherung, die zwar für kleine Winkel sehr genau ist, aber erstens nicht absolut genau und zweitens aufgrund der manuellen
Pendelauslenkung nicht immer zu gewährleisten ist. Auch die Annahme, dass die Pendelmassen und Längen völlig identisch sind, führt durch das unterschiedliche Trägheitsmoment einen geringen Fehler
in das System ein.
