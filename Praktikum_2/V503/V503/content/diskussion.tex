\section{Diskussion}

Als ausschlaggebende Fehlerquelle ist hier die fehlende Messung der Schwebegeschwindigkeit $v_0$ anzugeben.\\
Ohne diese Messreihe ist es nicht möglich, anhand von $v_\text{ab} - v_\text{auf} = 2 v_0$ eine Auswahl tatsächlich brauchbarer Werte zu treffen, sodass hier alle aufgenommenen Werte zur Auswertung verwendet wurden,
auch die der Tröpfchen, dessen Ladung sich unter Umständen im Laufe der Messung änderte. \\
Auch die nur begrenzte menschliche Reaktionszeit führt zu relativ groben Ungenauigkeiten und es ist zweifelhaft, dass die Methode, die den kleinsten gemeinsamen Nenner der Ladungen finden sollte, dies
tatsächlich getan hat.
Hier ergeben für die Elementarladung bei einem Literaturwert von $1,602176 \cdot 10^{-19} \,\si{\coulomb}$ \cite{elchar} und einem
Messwert von $\qty{5.2(0.8)e-19} \,\si{\coulomb}$ eine Abweichung von etwa $230 \,\%$, für die Avogadrokonstante sich bei einem Theoriewert von $6.02214076 \cdot 10^{23} \dfrac{1}{\si{\mol}}$ \cite{na} und einem Messwert von $\qty{1.85(0.29)e23} \,\dfrac{1}{\si{\mol}}$ eine Abweichung von knapp
$69 \,\%$. \\

Im Rahmen der groben Ungenauigkeiten der Messung ist es überraschend, dass die Elementarladung dennoch in derselben Größenordnung liegt. \\
Die Abweichung der Avogadrokonstante ist demnach relativ annehmbar.
