\section{Durchführung}
\label{sec:Durchführung}

\subsection{Versuchsaufbau}
Der Aufbau des Versuchs ist in \autoref{fig:abb2} dargestellt. Das wichtigste Bauteil sind die Kondensatorplatten, die einen Abstand von $d = (7,6250 \pm 0,005) \, \unit{\milli\meter}$ haben.
In der Mitte oberen Platte befindet sich ein Loch. Durch dieses Loch werden die Öltröpfchen mit einem Zerstäuber eingesprüht.
Die Dichte des verwendeten Öls beträgt $\rho_{öl} = 886 \, \unit{\kilo\gram} \mathbin{/} m^3$. Der Raum zwischen den Platten kann mit einer Halogenlampe $8$ beleuchtet werden, um die Öltröpfchen mit dem Mikroskop $5$ besser beobachten zu können.
$4$ ist der Schalter für ein Thorium-232 Prä­pa­rat. Die meisten Öltröpfchen sollten ionisiert in die Kammer gelangen, jedoch kann die Ladung der einzelnen Tröpfchen mit dem $\alpha$-Strahler nachjustiert werden. Dieser Schalter hat drei Positionen.
Bei der " $OFF$ " Position wird das Prä­pa­rat abgeschirmt und mit bei "$ON$" Position wird Strahlung durchgelassen. In der mittleren Position werden Öltröpfchen zwischen die Kondensatorplatten gesprüht.
Das elektrische Feld zwischen den Kondensatorplatten kann durch den Schalter $7$ kontrolliert werden.
An den Buchsen $2$ kann der Widerstand des Thermistors mit einem Multimeter gemessen werden.


\begin{figure}[H]
    \centering
    \includegraphics[width=1.0\textwidth]{figures/Abb2.pdf}
    \caption{Aufbau des Millikan-Öltröpfchenversuchs \cite{ap12}.}
    \label{fig:abb2}
\end{figure}

\subsection{Versuchsdurchführung}
1. Das Feld zwischen den Platten wird abgeschaltet. Danach wird eine hinreichende Anzahl an Öltröpfchen zwischen die Kondensatorplatten gesprüht. \\

2. Öltröpfchen mit einer Geschwindigkeit zwischen $v = 0.1 \, \unit{\milli\meter} \mathbin{/} \unit{\second}$ und $v = 0.01 \, \unit{\milli\meter} \mathbin{/} \unit{\second}$ eignen sich am besten zur Beobachtung.
Danach sollte geprüft werden, ob das beobachtete Tröpfchen geladen ist. Das Thorium-232 kann dazu verwendet werden, um die Ladung des Tröpfchens zu erhöhen. Währenddessen sollte $7$ auf " $OFF$ " stehen. \\

3. Die eigentliche Messung beginnt damit, dass das elektrische Feld mit $7$ eingeschaltet wird und die Zeit gemessen, die das Tröpfchen braucht, um eine bestimmte Strecke zurück zulegen.
Danach wird das Feld umgepolt und die gleiche Strecke durchlaufen. Es wird erneut die Zeit gemessen. Für eine größere Messgenauigkeit kann dieser Prozess mehrfach wiederholt werden.
Im Anschluss wird die Geschwindigkeit $v_0$ gemessen. \\

4. Der Messprozess wird für andere Tröpfchen wiederholt.\\

5. Der Messprozess wird für andere Kondensatorspannungen wiederholt, jedoch sollte die angelegte Spannung $500 \, \unit{\volt}$ nicht überschreiten.\\

Die Halogenlampe $8$ kann angeschaltet werden, damit die Öltröpfchen in der Kammer besser zu erkennen sind.
Ist die Lampe eingeschaltet erwärmt sie die Millikan Kammer $3$, deswegen muss die Temperatur alle $15 \, \unit{\min}$ gemessen werden.
