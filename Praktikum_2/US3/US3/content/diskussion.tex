\section{Diskussion}
\label{sec:diskussion}

Zum Aufbau ist anzumerken, dass das Prisma nicht fest am Schauch fixiert werden konnte. Die Kunststoffplatte, welche die Stabilität des Prisma garantieren sollte, hat nicht in die vorgesehende Halterung gepasst.
Ohne das Festsetzen des Prismas am Schlauch konnte es zu Luftblasen in dem Ultraschallgel kommen und damit zu Fehlern in der Messung. \\

An den ersten drei Plots \autoref{fig:graph1a}, \autoref{fig:graph1b} und \autoref{fig:graph1c} ist eindeutig ein linearer Zusammenhang zwischen der Schrömungsgeschwindigkeit und der Frequenzverschiebung festzustellen.
Die Plots in \autoref{fig:graph2a} und \autoref{fig:graph2b} weisen einen ähnlichen Verlauf auf. \\
Um eine akkurate Ausgeleichsfunktion zu plotten, wären jedoch mehr Messdaten vonnöten. 
Aufgrund der wenigen Messdaten an den Peaks ist ein Maximum bei einer Messtiefe von ungefähr $18  \, \unit{\micro\second}$ anzunehmen. 
Bei einer Prismadicke von $30 \, \unit{\milli\meter} $, einem Außenradius des Rohres von $15 \, \unit{\milli\meter} $ und einem Innenradius von $10 \, \unit{\milli\meter}  $, entspricht der Peak bei einer Messtiefe von  $42.63 \, \unit{\milli\meter} $ in der Streuintensität ungefähr der gegenüberliegenden Innenseite des Rohres.
Ein Fehler in dieser Rechnung könnte sein, dass die Schallgeschwindigkeit in der Strömungsrohre von der bei den Berechnungen verwendeten abweicht.\\

Die Plots der Momentangeschwindigkeiten in \autoref{fig:graph2a} und \autoref{fig:graph2b} bestätigen einen Zusammenhang zwischen Messtiefe und Momentangeschwindigkeit.
Was die Aussage zulässt, dass die Flussgeschwindigkeit in der Mitte des Rohres größer ist als am Rand.
Eine Erklärung ist die Reibung zwischen dem Rohr und der fließenden Flüssigkeit, welche die Flussgeschwindigkeit reduziert.



%Die Plots der Momentangeschwindigkeiten in \autoref{fig:graph2a} und \autoref{fig:graph2b} bestätigen keinen Zusammenhang zwischen Messtiefe und Momentangeschwindigkeit.
%Was die Aussage zulässt, dass die Flussgeschwindigkeit nicht von dem Abstand zum Mittelpunkt des Rohres abhängt. Die Praktikumsaufsicht meinte das diese Aussage wohl nicht so ganz richig ist und im Nachhinein stimme ich dem zu.