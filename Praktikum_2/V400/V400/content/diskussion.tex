\section{Diskussion}
\label{sec:Diskussion}

%\subsection{Das Reflexionsgesetz}
Bei dieser Messreihe konnte ein systematischer Fehler von ein bis zwei Grad beobachtet werden, da das Gesetz " Einfallswinkel gleich Ausfallswinkel " allgemein bekannt ist.
Bis auf diesen systematischen Fehler kann diese Relation bestätigt werden. \\

%\subsection{Brechungsgesetz}
Der Literaturwert für den Brechungsindex für Plexiglas bei Raumtemperatur liegt bei $n =  1,49 $\cite{ap02}, was für den Mittelwert von $n = 1,43 \pm 0,05$ eine Abweichung von $2,05 \,\% $ ergibt.

Die Differenz im Gangunterschied ist maximal $\varDelta  = 1,1 \,\unit{\milli\meter}$ und kann leicht durch Messungenauigkeiten entstehen, deswegen ist nicht eindeutig feststellbar, 
welche der Methoden besser geeignet ist. \\

Bei der Messreihe mit dem Prisma gab es größere Messungenauigkeiten, da der Winkelschirm, von dem der Austrittswinkel gemessen wurde, nicht genau zu der Winkelskala am Boden der Messaperatur passte. \\

Der grüne Laser hat eine nominale Wellenlänge von $\lambda = 532 \,\unit{\nano\meter}$, der rote eine Wellenlänge von $\lambda = 635  \,\unit{\nano\meter}$\cite{ap02}.
Beim grünen Laser und einem 600 Linien$\mathbin{/} \unit{\milli\meter}$-Gitter ergibt sich eine Abweichung von $ 1,88 \, \% $. Für den roten Laser beim gleichen Gitter beträgt die Abweichung $ 2,51 \, \% $
Für das 300 Linien$\mathbin{/} \unit{\milli\meter}$-Gitter haben die Mittelwerte Abweichungen von $ 1,50 \, \% $ und  $ 0,31 \, \% $
Bei 100 Linien$\mathbin{/} \unit{\milli\meter}$-Gitter ergeben sich $ 0,75 \, \% $ und $ 2,83 \, \% $ als Abweichungen. 
Allgemein sind die Abweichungen ziemlich gering, wenn beachtet wird, dass die Laser nur auf ungefähr $3 \,\unit{\milli\meter}$ genau fokussiert werden konnten. 