\section{Diskussion}
\label{sec:Diskussion}

Bei allen Messungen liegen die theoretischen Werte weit außerhalb der Abweichung der gemessenen Werte. Ist der gemessene Absorptionskoeffizient größer als der theoretische, könne weitere Effekte wie der Photoeffekt oder die Paarbildung auftreten.
Diese Effekte reduzieren die Energie der Strahlung zusätzlich. Sie werden aber im Theoriewert nicht beachtet.
Weitere Fehlerquelle sind Verunreinigungen des Materials. Diese Verunreinigungen sorgen dafür, dass der Absorptionskoeffizient verändert wird.
Ein weiterer systematischer Fehler ist eine Änderung des Abstandes zwischen Probe und Geiger-Müller-Zählrohr.
Zusätzliche Fehler treten durch die statistische Natur des radioaktiven Zerfalls auf. Dieser Fehler kann nur durch eine längere Messdauer ausgeglichen werden.

Ein Fehler, der nicht verhindert werden kann, ist die Fluktuation in der Hintergrundstrahlung.
Die Hintergrundstrahlung kann sich zum Beispiel durch das zunehmen oder abnehmen von Sonnenwinden Änderung.