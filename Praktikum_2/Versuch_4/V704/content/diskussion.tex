\section{Diskussion}
\label{sec:Diskussion}

Bei allen Messungen liegen die theoretischen Werte weit außerhalb der Abweichung der gemessenen Werte. Ist der gemessene Absorptionskoeffizient größer als der theoretische, könne weitere Effekte wie der Photoeffekt oder die Paarbildung auftreten.
Diese Effekte reduzieren die Energie der Strahlung zusätzlich. Sie werden aber im Theoriewert nicht beachtet.
Für das Cs-137 und Bleiplatten als Absorber ergibt sich ein experimenteller Wert von $ \mu = \left( 96,958 \pm 5,079 \right) \unit{\dfrac{1}{\meter}} $ und ein theoretischer Wert von $\mu_{\text{theo-Pb}} =  69,39 \, \unit{\dfrac{1}{\meter}} $ \eqref{eq:absorpttheoPb}.
Das resultiert in einer relativen Messabweichung von $ 39,72 \, \%$
Bei der Messreihe mit Kupferplatten als Absorber beträgt der experimentelle Wert $\mu =  \left( 40,481 \pm 0,681 \right) \unit{\dfrac{1}{\meter}}$ und der theoretische Wert $\mu_{\text{theo-Cu}} = 63,2  \, \dfrac{1}{\unit{\meter}}$. Daraus geht eine relativen Messabweichung von $ 35 \, \%$  hervor.
Beide Messreihen wären aussagekräftiger, wenn die Abstände zwischen den gemessenen Dicken regelmäßiger gewählt worden wären. \\


Für die Messreihe mit dem $ \beta $- Strahler Tc-99 beträgt die theoretisch maximale Zerfallsenerige $ 	0,294  \, \unit{\mega\eV} $ \cite{ap05}. Im Experiment konnte jedoch ein Wert von $E_\text{max} = \left(1,22 \pm 0,28 \right) \, \unit{\mega\eV}$ festgestellt werden. 
Die relative Abweichung beträgt damit $ 314 \, \% \,.$ Es ist anzunehmen, dass bei einer längeren Messdauer exaktere Werte hätten aufgenommen werden können. Die große Unsicherheit von $ 22,95 \, \%$ weist darauf hin, dass die Messung zu ungenau war.

Weitere Fehlerquelle sind Verunreinigungen des Materials. Diese Verunreinigungen sorgen dafür, dass der Absorptionskoeffizient verändert wird.
Ein weiterer systematischer Fehler ist eine Änderung des Abstandes zwischen Probe und Geiger-Müller-Zählrohr.
Zusätzliche Fehler treten durch die statistische Natur des radioaktiven Zerfalls auf. Dieser Fehler kann nur durch eine längere Messdauer ausgeglichen werden.
Ein Fehler, der nicht verhindert werden kann, ist die Fluktuation in der Hintergrundstrahlung.
Die Hintergrundstrahlung kann sich zum Beispiel durch das zunehmen oder abnehmen von Sonnenwinden Änderung.