\section{Diskussion}
\label{sec:Diskussion}

Zunächst soll nun auf die Abweichungen der experimentellen und theoretischen Werte eingegangen werden und anschließend mögliche Fehlerquellen diskutiert werden.

\subsection{Abweichungen}

Bei einem Theoriewert von $\lambda_\text{Laser,theo} = 635 \,\si{\nano\meter}$ besteht eine Abweichung von $2,22 \,\%$ zum experimentell bestimmten Wert von
$\lambda_\text{Laser,exp} = 649,07\,\si{\nano\meter}$. \\

Für den Brechungsindex beträgt die Abweichung zwischen dem experimentell ermittelten Wert $n_\text{Luft,exp} = 1,000550$ und dem Theoriewert von $n_\text{Luft,theo} = 1,000272$ \cite{spekrumbrechluft} etwa $0,0258 \,\%$. \\


\subsection{Diskussion möglicher Fehlerquellen}

Die offensichtlichste Fehlerquelle ist die Anfälligkeit der Apparatur für kleinste Stöße. \\
Wird der Tisch auch nur in geringstem Maße berührt, wird die Messung um bis zu $30$ Maxima verfälscht, da durch die Vibrationen das selbe Maximum mehrfach gezählt wird. \\

Auch das versehentliche Abblocken von Licht durch Fehlpositionierung beim Ablesen der Daten oder Tageslichtschwankungen können das Messergebnis verfälschen, da die Photodiode nur generelle Lichtintensitäten,
nicht nur die des vom Laser emittierten Lichts erfasst. \\

Unter Berücksichtigung dieser Fehlerquellen sind die Abweichungen der aufgenommenen Messdaten, insbesondere die des Brechungsindexes, sehr gering.
Es lässt sich also im Allgemeinen auf eine hohe Qualität der Versuchsapparatur schließen.
