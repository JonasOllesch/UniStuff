\section{Theorie}
\label{sec:theorie}

In diesem Abschnitt wird das Prinzip der Interferenz erläutert, welches die Grundlage für das Michelson-Interferometer bildet.

\subsection{Grundlagen}

Licht breitet sich im einfachsten Fall als ebene Welle der Form 
\begin{equation}
    \vec{E}(x,t) = \vec{E_0} \cos{(kx - \omega t + \delta)}
    \label{eq:ebenewellen}
\end{equation}
aus, dabei ist k die Wellenzahl, $\omega$ die Kreisfrequenz und $\delta$ der Phasenwinkel. Für Licht gilt das Prinzip der linearen Superposition. Diese Prinzip sagt aus, dass die Überlagerung mehrerer Lichtwellen durch die Addition der einzelnen Felder berechnen lässt.
Aufgrund der hohen Lichtfrequenz kann dieses Experiment nicht direkt experimentell belegt werden. Aus diesem Grund wird die Intensität betrachtet. Für die Intensität gilt die folgende Beziehung 
\begin{equation*}
    I \propto |\vec{E}| \, .
\end{equation*}
Nach einiger Rechnung ergibt sich für die Intensität überlagerter Wellen die Beziehung \eqref{eq:Iges}. Die einzelnen Intensitäten werden nicht einfach zu $2 const E_{0}^{2}$ addiert, sondern es gibt einen Interferenzterm
\begin{equation*}
    I = 2 \cdot const \, \vec{E}_0 \cos{(\delta_2 - \delta_1)}.
\end{equation*}


%%%%%% Label mal die Formel, nach der wir Lambda berechnen (also Δd = z λ/2) mal bitte 'eq:deltad', dann passt das in der Auswertung alles schön :)
%%%%%% Label mal die Formel b Δn = zλ/2 bitte 'eq:bdeltan' und die Formel n(p_0,T_0) = 1 + Δn T/T_0 p0/(p-p') bitte 'eq:brechungsindex' dange :D
