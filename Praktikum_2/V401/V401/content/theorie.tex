\section{Theorie}
\label{sec:theorie}

In diesem Abschnitt wird das Prinzip der Interferenz erläutert, wechles die Grundlage für das Michalson-Interferometer bildet.

\subsection{Grundlagen}

Licht breitet sich im einfachsten Fall als ebene Welle der Form 
\begin{equation}
    \vec{E}(x,t) = \vec{E_0} \cos{(kx - \omega t + \delta)}
    \label{eq:ebenewellen}
\end{equation}
aus, dabei ist k die Wellenzahl, $\omega$ die Kreisfrquenz und $\delta$ der Phasenwinkel. Für Licht gilt das Prinzip der linearen Superposition. Diese Prinzip sagt aus, dass die Überlagerung mehrer Lichtwellen durch die Addition der einzelnen Felder berechnen lässt.
Aufgrund der hohen Lichfrequenz kann dieses Experiment nicht direkt experimentell belegt werden. Aus diesem Grund wird die Intensität betrachtet. Für die Intensität gilt die folgende Beziehung 
\begin{equation*}
    I \propto |\vec{E}| \, .
\end{equation*}
Nach einiger Rechung ergibt sich für die Intensität überlagerter Wellen die Beziehung \eqref{eq:Iges}. Die einzelnen Intensitäten werden nicht einfach zu $2 const E_{0}^{2}$ addiert, sondern es gibt einen Interferenzterm
\begin{equation*}
    I = 2 \cdot const \, \vec{E}_0 \cos{(\delta_2 - \delta_1)}.
\end{equation*}
