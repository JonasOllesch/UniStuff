\section{Theorie}
\label{sec:theorie}

Werden Atome mit geeigneter Energie beschossen, lässt sich mithilfe
der Energieübergänge der Elektronen die Struktur der Elektronenhüllen aufklären. \\

Auf die gleiche Weise funktioniert auch das Franck-Hertz-Experiment,
bei dem im einem abgeschlossenen Raum monoenergetische Elektronen mit
Quecksilberdampf interagieren.
Die dabei auftretenden (un-)elastischen Stöße zwischen den Elektronen und Hg-Atomen
lassen sich verwenden, um die die vom Quecksilberatom aufgenommene Energie zu ermitteln. \\

Für den unelastischen Stoß, bei dem das Hg-Atom aus seinem Ruhezustand in den
ersten angeregten Zustand übergeht, gilt mit der Ruhemasse $m_0$
des Elektrons sowie $v_\text{vor}$ und $v_\text{nach}$ als Geschwindigkeiten
des Elektrons vor und nach dem Stoß

\begin{equation*}
    E_1 - E_0 = \dfrac{m_0 \, v^2_\text{vor}}{2} -\dfrac{m_0 \, v^2_\text{nach}}{2}
\end{equation*}

Das Experiment selbst ist dabei nach \autoref{fig:abb1} aufgebaut.

\begin{figure}[H]
    \centering
    \includegraphics{figures/Abb_1.pdf}
    \caption{Schematischer Aufbau des Franck-Hertz-Versuches\cite{ap08}.}
    \label{fig:abb1}
\end{figure}

In Folge des glühelektrischen Effekts treten bei Erhitzung des Glühdrahts
ein Großzahl an Elektronen aus, die nach Durchlaufen der Beschleunigungsstrecke
eine kinetische Energie von

\begin{equation*}
    \dfrac{m_0}{2} v^2_\text{vor} = \text{e}_0 U_\text{B} \,,
\end{equation*}
wobei $\text{e}_0$ die Elementarladung der Elektronen und $U_\text{B}$
die an die Elektrode angelegte Gleichspannung darstellen. \\

Die Elektronen treffen auf die Auffängerelektrode, die der Beschleunigerelektrode gegenüber
die Gegenspannung $U_\text{A}$ besitzt. \\

Für die Geschwindigkeitskompnente $v_\text{z}$ in Feldrichtung gilt die Ungleichung
\begin{equation*}
    \dfrac{m_0}{2} v^2_\text{z} \geq \text{e}_0 U_\text{A} \,.
\end{equation*}
Alle Elekronen, die diese Ungleichung nicht erfüllen, kehren zur Beschleunigerelektrode zurück.
Dort können sie nun mit den QUecksilberatomen stoßen, wobei zwei Fälle unterschieden werden müssen. \\

Bei mittlerer Energie treten lediglich elastische Stöße auf, die
Energieabgabe $\Delta E$ beträgt dabei

\begin{equation*}
    \Delta E = \frac{4 m_0 M}{(m_0 + M)^2} E \approx 1,1 \cdot 10^{-5} E \,.
\end{equation*}

Ist die Elektronenenergie $E$ (durch Erhöhen der Beschleunigungsspannung $U_\text{B}$)
größer gleich der Energiedifferenz $E_1 - E_0$ zwischen erstem angeregten und Ruhezustand
des Hg-Atoms, überträgt es exakt diesen Energiebetrag an das Quecksilberatom. \\

Dieses fällt nach einer Relaxationszeit von $~10^{-8} \unit{\second}$
in seinen Ruhezustand zurück und emittiert ein Lichtquant der Energie
\begin{equation*}
    \text{h} \nu = E_1 - E_0
\end{equation*}

Wird nun der Auffängerstrom $I_\text{A}$ an der Auffängerelektrode
beobachtet, lässt sich feststellen, dass, wie in \autoref{fig:abb2} zu erkennen,
jähe Abnahmen des Stroms genau da finden, wo die Elektronen genug Energie besitzen,
um das Quecksilber anzuregen, aber nicht, um die Auffängerelektrode zu erreichen. \\

Dieser Vorgang lässt sich erneut beobachten, wenn die Elektronen erneut
eine Energie von $E_1 - E_0$ erreicht haben und ein weiteres Mal unelastisch mit
dem Hg stoßen können.

\begin{figure}[H]
    \centering
    \includegraphics{figures/Abb_2.pdf}
    \caption{Zusammenhang zwischen Auffängerstrom $I_\text{A}$ und Beschleunigungsspannung $U_\text{B}$\cite{ap08}.}
    \label{fig:abb2}
\end{figure}

Der Abstand $U_1$ zwischen zwei Maxima beträgt dabei

\begin{equation}
    U_1 := \frac{1}{\text{e}_0} (E_1 - E_0) \,.
\end{equation}