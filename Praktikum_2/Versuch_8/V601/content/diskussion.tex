\section{Diskussion}

Bei der Durchführung des Versuches fallen insbesondere zwei Fehlerquellen ins Auge.
Die Skala am XY-Schreiber passend auf den benötigten Bereich einzustellen, gestaltet sich häufig schwieriger als erwartet.
Dadurch kann es passieren, dass die aufgenommenen Kurven nicht vollständig abbildbar sind oder feinere Details verloren gehen.
Hier wurde die Bremsspannung z.B. nur bis $U_\text{B} \approx 9 \,\unit{\volt}$ statt $10 \,\unit{\volt}$ durchlaufen.
Außerdem kommt es bei der Beschriftung der x-Achse leicht zu Fehlern, da die Einstellung am XY-Schreiber einige Zwischenstufen bietet, deren Skalierung nicht eindeutig scheint. \\

Bei der Aufnahme der Franck-Hertz-Kurven stellt vor allem die Schwankungsfreudigkeit der Heizelemente eine Schwierigkeit dar.
Selbst wenn der Temperaturregler nicht verstellt wird, kommt es während der Messung zu Temperaturschwankungen von bis zu $2 \,\unit{\celsius}$ oder mehr. \\

Eine weitere Fehlerquelle ist der Faktor aus \autoref{tab:dampfd_freiweg}. Dieser Faktor soll zwischen 1000 und 4000 liegen, was aber nur bei $155 \, \unit{\celsius}$ und $182 \, \unti{\celsius}$ der Fall ist.
Auch das Ablesen der vom Schreiber aufgenommenen Werte kann leicht zu Fehlern führen. \\

Die Anregungsenergie von Quecksilberbeträgt $4,9 \unit{\electronvolt}$  \cite{ap09} . In diese Messungen hat $\left( 7,525 \pm \, 1,003\right) \, \unit{electronvolt}$ und  $ \left(6,042 \pm 0,172 \right)\, \unit{electronvolt}$ ergeben, was in einer relativen Abweichung von $53,53 \, \%$ und $23,31 \, \%$ resultiert.