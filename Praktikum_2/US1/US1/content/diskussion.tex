\section{Diskussion}
\label{sec:Diskussion}
Als erstes wurde die Schallgeschwindigkeit in Acrylglas gemessen.%Korrektur
Diese wurde zu $c = 2691,79 \,\unit{\frac{\meter}{\second}}$ bestimmt, was verglichern mit einem Theoriewert von
$c_{\text{Acryl},\text{t}} = 2730 \,\unit{\frac{\meter}{\second}} $ einer Abweichung von $1,4 \, \%$ beim Impuls-Echo-Verfahren 
bzw. einer Abweichung von $88,70 \, \%$ beim Durchschallungsverfahren. 
Wie zu erkennen ist, unterliegen insbesondere die Messdaten zum Durchschallungsverfahren einer großen Abweichung.
Nach Bereinigung der Messwerte, also vernachlässigen der Ausreißer, ergeben sich die in \autoref{fig:graph3} und \autoref{fig:graph4} %Korrektur
dargestellten Graphen.

Aus den verbesserten Plots ergibt sich eine Schallgeschwindigkeit von $c = 2724,92 \,\unit{\frac{\meter}{\second}}$ für das
Impuls-Echo-Verfahren, also eine Abweichung von $0,186 \,\%$, beim Durchschallungsverfahren eine Schallgeschwindigkeit von
$2697,99 \,\unit{\frac{\meter}{\second}}$, die eine Abweichung von $1,17 \, \% $ vom Theoriewert abweicht. %Korrektur
Davon ausgehend, dass beim Durchschallungsverfahren einfach nur ein Rechenfehler aufgetreten ist, sind beide Verfahren, nach
Bereigigung der Messdaten, für eine Schallgeschwindigkeitsbestimmung geeignet. \\

\begin{figure}[H]
    \centering
    \includegraphics{build/Graph_a_oA.pdf}
    \caption{Bereinigte Messdaten und Fit zum Impuls-Echo-Verfahren.}
    \label{fig:graph3}
\end{figure}

\begin{figure}[H]
    \centering
    \includegraphics{build/Graph_b_oA.pdf}
    \caption{Bereinigte Messdaten und Fit zum Durchschallungsverfahren.}
    \label{fig:graph4}
\end{figure}

Die Anpassungsschicht der verbesserten Plots ergibt sich aus dem Betrag des y-Achsenabschnittes.
Hier sind $b = -0,006465 \,\unit{\meter}$ für das Impuls-Echo-Verfahren bzw. $b = -0,00229 \,\unit{\meter}$ für das
Durchschallungsverfahren. 
Die Werte sind aufgrund ihrer Größenordnung in der Realität also unbrauchbar, der grobe Bereich der Anpassungsschicht befindet sich im Millimeterbereich.\\ 

In \autoref{fig:graph1a} und \autoref{fig:graph1b} sind Messwerte bei ungefähr gleicher Messdauer zu sehen, die jedoch einen signifikanten Unterschied in der Messlänge haben.
Eventuelle Messfehler können durch mangelhafte Kopplung der Acrylzylinder untereinander bzw. mit den Sonden entstehen.
Anstatt den kompletten Körper zu durchdringen, kann der Impuls schon an der Grenzfläche zwischen den Zylinder reflektiert werden,
wodurch der Eindruck eines kürzeren Körpers entsteht. 
Auch Sprünge im Material können nicht vollständig ausgeschlossen werden, sie sind aber unwahrscheinlich. \\

Die Messwerte für das Augenmodell sind offensichtlich fehlerhaft. 
Ein menschliches Auge hat einen Durchmesser von $ 2,30 \, \unit{\centi\meter} $ \cite{ap07}.
Der experimentell bestimmte Wert liegt bei $l_\text{Retina,modell} = 2,76 \,\unit{\centi\meter}$, ein Wert, der auf den ersten Blick zwar recht gut erscheint,
alledings nur, wenn man vernachlässigt, dass das Modell dreimal größer als ein echtes menschliches Auge ist.
So liegt der experimentell bestimmte Wert für ein realistisch großes Auge bei
$ l_\text{Retina,echt} = \dfrac{1}{3} l_\text{Retina,modell} = 0,92 \, \unit{\centi\meter}$.
Der experimentelle Wert hat also eine Abweichung von $ 60,00 \, \% $ zum Literaturwert.