\section{Theorie}
\label{sec:theorie}


Schallwellen sind longitudinale Wellen, die in der Form
\begin{equation}
    p(x, t) = p_0 + v_0 Z \cos(\omega t - k x)
    \label{eq:druckwelle}
\end{equation}
als Druckschwankungen im Raum propagieren.

Die akustische Impedanz $Z = c \rho$ beschreibt dabei den
Widerstand, auf den Schallwellen im Medium der Dichte $\rho$
erfahren.
Dabei ist die Schallgeschwindigkeit $c$ materialabhängig, in einer
Flüssigkeit z.B. bestimmen die Kompressibilität $\kappa$ 
und Dichte über
\begin{equation}
    c_\text{Fl} = \sqrt{\frac{1}{\kappa \rho}}
\end{equation}
die Schallgeschwindigkeit. \\

In Festkörpern bestehen Schallwellen dagegen nicht nur aus 
Longitudinal-, sondern auch aus Transversalwellen.
Hier gilt
\begin{equation}
    c_\text{Fk} = \sqrt{\frac{E}{\rho}} \,,
\end{equation}
wobei das Elastizitätsmodul $E$ anstelle der Kompressibilität
$\kappa$ eine Rolle spielt. \\


\subsection{Reflexion}

Trifft eine Schallwelle auf eine Grenzfläche, wird ein Teil
reflektiert, der andere wird transmittiert und teilweise
absorbiert.
Mit der akustischen Impedanz $Z_i$ des Mediums $i$ ergibt sich
der Reflexionskoeffizient aus
\begin{equation}
    R = \left(\frac{Z_1 - Z_2}{Z_1 + Z_2}\right)^2 \,.
    \label{eq:reflexkoeff}
\end{equation}

\subsection{Transmission und Absorption}

Der Anteil der Welle, der nicht reflektiert, sondern 
transmittiert wird, bestimmt sich dann aus

\begin{equation}
    T = 1 - R
    \label{eq:transmisskoeff}
\end{equation}


Über eine Strecke $x$ verliert eine Schallwelle einen Teil ihrer
Energie durch Absorption, mit einer Anfangsintensität $I_0$ gilt
\begin{equation}
    I(x) = I_0 \mathrm{e}^{-\alpha x} \,.
\end{equation}

Dabei ist $\alpha$ der Absorptionskoeffizient des Mediums.

\subsection{Impuls-Echo-Verfahren}

Mit einem piezo-elektrischen Kristall als Ultraschallsender und
-empfänger, der in einem elektrischen Feld zu Schwingungen
angeregt werden kann und Ultraschallwellen abstrahlt,


