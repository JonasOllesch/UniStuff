\section{Diskussion}
\label{sec:Diskussion}
Als erstes wurde Schallgeschwindigkeit in Acrylglas gemessen.
Diese werden mit einem Theoriewert von
$c_{\text{Acryl},\text{t}} = 2730 \,\unit{\frac{\meter}{\second}} $ verglichen.
Das entspricht einer Abweichung von $13 \, \%$ beim Impuls-Echo-Verfahren bzw. einer Abweichung von $88.70 \, \%$ beim Durchschallungsverfahren. 
Allein an der Größe der Abweichung kann gesehen werden, dass die Messwerte eine großen Streuung unterliegen. Nach diesen beiden Messreihen ist das Echo-Impuls-Verfahren deutlich besser geeingnet, um die Schallgeschwindigkeit in einem Medium zu messen.

Mehr Messwerte und eine genauere Messung sind notwendig, um eindeutigere Ergebnisse zu erhalten.
In \autoref{fig:graph1a} und \autoref{fig:graph1b} sind Messwerte bei ungefähr gleicher Messdauer zu sehen, die jedoch einen signifikanten Unterschied in der Messlänge haben.
Messfehler bei zusammengesetzten Zylinder entstehen durch eine unzureichende Kopplung. Der Impuls wird dann schon an der Schnittstelle reflektiert und nicht erst am Ende des zweiten Zylinders. \\

Die Messwerte für das Augenmodell sind offensichtlich fehlerhaft. Ein menschliches Auge hat einen Durchmesser von $ 2,3 \, \unit{cm} $ \cite{ap07}. Der experimentelle Wert liegt bei $ 0.92 \, \unit{cm}$.
Der experimentelle Wert hat eine Abweichung von $ 60 \, \% $ zum Literaturwert.