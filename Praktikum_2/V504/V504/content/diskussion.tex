\section{Diskussion}
\label{sec:Diskussion}

Der in \autoref{subsec:raumladungsgebiet} ermittelte Exponent $m = 1,30 \pm 0,04$ weicht um $13,33 \%$ vom Theoriewert
$m = 1,5$ ab, was in Anbetracht der Empfindlichkeit der Messapparatur ein akzeptabler Fehler ist. \\

Bei der Messung des Sättigungsstroms zu der Heizspannung von $5 \, \unit{\volt}$ sind noch weitere Messwerte notwendig, um eine bessere Aussage über den Sättigungsstrom machen zu können, da auch bei $250 \, \unit{\volt}$ noch kein eindeutiges Plateau zu erkennen ist.
Bei der Betrachtung der Ergebnisse zeigt sich, dass die Temperaturen aus der Untersuchung des Anlaufstromgebiets und der Kathodentemperatur aus Leistungsbilanz deutlich unterschiedliche Werte liefern.
Die Austrittsarbeit von Wolfram wurde auf einen Wert von $\text{e}_0 \Phi = (3,352 \pm 0,045) \,\unit{\eV}$ bestimmen. Bei einem Literaturwert von $4,54 \, \unit{\electronvolt}$ \cite{ap10} ergibt sich eine realative Abweichung von $ 26,17 \, \%$.
Diese große Abweichung kann dadurch erklärt werden, dass die gemessenen Ströme, insbesondere in der sechsten Messreihe sehr gering waren, welche schwierig sind exakt zu messen.
Eine weitere Fehlerquelle ist die endliche Ausdehnung der Anode, die möglicherweise nicht alle emittierte Elektronen aufnehmen konnte. Diese Fehlerquelle kann aber nur durch eine verbesserte Messapparatur verringert werden.