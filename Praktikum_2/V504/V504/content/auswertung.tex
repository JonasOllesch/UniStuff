\section{Auswertung}
\label{sec:auswertung}

Zur besseren Übersichtlichkeit ist die Auswertung im Folgenden in einige Unterabschnitte unterteilt.

\subsection{Kennlinien und Sättigungsstrom}

Die Paare aus Anodenspannung und Anodenstrom der fünf verschiedenen Kennlinien sind in \autoref{tab:kennlini1_1}, \autoref{tab:kennlini1_2}, \autoref{tab:kennlini1_3}, \autoref{tab:kennlini1_4} und \autoref{tab:kennlini1_5} dargestellt.

%\begin{table}[H]
%    \centering
%    \caption{Anodenstrom $I_1$ mit den dazugehörigen Anodenspannungen $U$.}
%    \label{tab:kennlini1_2}
%\begin{minipage}[c]{0.45\textwidth}
%    \begin{tabular}{S S}
%        \toprule
%        & \multicolumn{1}{c}{$U_\text{H} = 3,2 \,\unit{\volt}$} \\
%        \cmidrule(lr){2-2}
%        {$U \mathbin{/} \unit{\volt}$} & {$I_1 \mathbin{/} \unit{\milli\ampere}$} \\
%        \midrule
%        {5}&       {0.005}\\
%        {7}&       {0.010}\\
%        {9}&       {0.013}\\
%        {10}&      {0.015}\\
%        {11}&      {0.017}\\
%        {13}&      {0.020}\\
%        {15}&      {0.024}\\
%        {17}&      {0.028}\\
%        {19}&      {0.031}\\
%        {20}&      {0.034}\\
%        {21}&      {0.036}\\
%        {23}&      {0.040}\\
%        {25}&      {0.043}\\
%        {27}&      {0.046}\\
%        {29}&      {0.049}\\
%        {30}&      {0.051}\\
%        {35}&      {0.057}\\
%        {40}&      {0.061}\\
%        {45}&      {0.065}\\
%        {50}&      {0.067}\\
%        {55}&      {0.069}\\
%        {60}&      {0.070}\\
%        {80}&      {0.070}\\
%        {90}&      {0.072}\\
%        {100}&     {0.074}\\
%        \bottomrule
%    \end{tabular}
%\end{minipage}
%\begin{minipage}[c]{0.45\textwidth}
%    \begin{tabular}{S S}
%        \toprule
%        & \multicolumn{1}{c}{$U_\text{H} = 3,5 \,\unit{\volt}$} \\
%        \cmidrule(lr){2-2}
%        {$U \mathbin{/} \unit{\volt}$} & {$I_2 \mathbin{/} \unit{\milli\ampere}$} \\
%        \midrule
%        {5}&       {0.007}\\
%        {7}&       {0.011}\\
%        {9}&       {0.016}\\
%        {10}&      {0.018}\\
%        {12}&      {0.022}\\
%        {14}&      {0.027}\\
%        {16}&      {0.031}\\
%        {18}&      {0.036}\\
%        {20}&      {0.042}\\
%        {22}&      {0.046}\\
%        {24}&      {0.052}\\
%        {26}&      {0.057}\\
%        {28}&      {0.063}\\
%        {30}&      {0.067}\\
%        {32}&      {0.070}\\
%        {34}&      {0.074}\\
%        {36}&      {0.077}\\
%        {38}&      {0.081}\\
%        {40}&      {0.084}\\
%        {45}&      {0.090}\\
%        {50}&      {0.094}\\
%        {55}&      {0.096}\\
%        {60}&      {0.097}\\
%        {65}&      {0.099}\\
%        {70}&      {0.102}\\
%        {80}&      {0.105}\\
%        {90}&      {0.018}\\
%        {100}&     {0.111}\\
%        {110}&     {0.113}\\
%        {120}&     {0.114}\\
%        {130}&     {0.115}\\
%        {140}&     {0.116}\\
%        {150}&     {0.117}\\
%        \bottomrule
%    \end{tabular}
%\end{minipage}
%\end{table}
%    
%
%
%\begin{table}[H]
%    \centering
%    \caption{Anodenströme $I_i$ mit den dazugehörigen Anodenspannungen $U$ für die dritte bis fünfte Heizspannung.}
%        \label{tab:kennlini3_5}
%    \begin{minipage}{0.45\textwidth}
%        \sisetup{table-format=2.1}
%        \begin{tabular}{S S S}
%            \toprule
%            & \multicolumn{1}{c}{$U_\text{H} = 4 \,\unit{\volt}$} & \multicolumn{1}{c}{$U_\text{H} = 4,2 \,\unit{\volt}$} \\
%            \cmidrule(lr){2-2} \cmidrule(lr){3-3}
%            {$U \mathbin{/} \unit{\volt}$} & {$I_3 \mathbin{/} \unit{\milli\ampere}$} & {$I_4 \mathbin{/} \unit{\milli\ampere}$} \\
%            \midrule
%            {2}&       {0.002}&{0.003}\\
%            {4}&       {0.007}&{0.008}\\
%            {6}&       {0.012}&{0.015}\\
%            {8}&       {0.016}&{0.020}\\
%            {10}&      {0.024}&{0.039}\\
%            {12}&      {0.030}&{0.036}\\
%            {14}&      {0.036}&{0.044}\\
%            {16}&      {0.042}&{0.053}\\
%            {18}&      {0.049}&{0.061}\\
%            {20}&      {0.056}&{0.070}\\
%            {22}&      {0.064}&{0.080}\\
%            {24}&      {0.071}&{0.089}\\
%            {26}&      {0.080}&{0.101}\\
%            {28}&      {0.088}&{0.110}\\
%            {30}&      {0.093}&{0.121}\\
%            {32}&      {0.101}&{0.129}\\
%            {34}&      {0.110}&{0.140}\\
%            {36}&      {0.117}&{0.146}\\
%            {38}&      {0.124}&{0.156}\\
%            {40}&      {0.132}&{0.168}\\
%            {42}&      {0.138}&{0.177}\\
%            {44}&      {0.145}&{0.188}\\
%            {46}&      {0.152}&{0.200}\\
%            {48}&      {0.159}&{0.212}\\
%            {50}&      {0.165}&{0.222}\\
%            {52}&      {0.170}&{0.235}\\
%            {54}&      {0.167}&{0.244}\\
%            {56}&      {0.173}&{0.252}\\
%            {58}&      {0.180}&{0.262}\\
%            {60}&      {0.186}&{0.269}\\
%            {65}&      {0.197}&{0.306}\\
%            {70}&      {0.209}&{0.335}\\
%            {75}&      {0.218}&{0.360}\\
%            {80}&      {0.223}&{0.385}\\
%            {85}&      {0.228}&{0.405}\\
%            {90}&      {0.227}&{0.425}\\
%            {95}&      {0.230}&{0.446}\\
%            {100}&     {0.232}&{0.459}\\
%            {110}&     {0.238}&{0.474}\\
%            {120}&     {0.243}&{0.496}\\
%            {130}&     {0.246}&{0.512}\\
%            {140}&     {0.247}&{0.524}\\
%            {150}&     {0.249}&{0.530}\\
%            {160}&     {0.251}&{0.532}\\
%            {170}&     {0.252}&{0.535}\\
%            {180}&     {0.254}&{0.540}\\
%            {190}&     {0.255}&{0.543}\\
%            {200}&     {0.256}&{0.546}\\
%            \bottomrule
%        \end{tabular}
%    \end{minipage}
%    \begin{minipage}{0.45\textwidth}
%    \sisetup{table-format=2.1}
%    \begin{tabular}{S S}
%        \toprule
%        & \multicolumn{1}{c}{$U_\text{H} = 5,4 \,\unit{\volt}$} \\
%        \cmidrule(lr){2-2}
%        {$U \mathbin{/} \unit{\volt}$} & {$I_5 \mathbin{/} \unit{\ampere}$} \\
%        \midrule
%        {5}&       {0.015}\\
%        {10}&      {0.034}\\
%        {15}&      {0.056}\\
%        {16}&      {0.062}\\
%        {18}&      {0.072}\\
%        {20}&      {0.085}\\
%        {22}&      {0.094}\\
%        {24}&      {0.105}\\
%        {26}&      {0.116}\\
%        {28}&      {0.130}\\
%        {30}&      {0.140}\\
%        {32}&      {0.155}\\
%        {34}&      {0.162}\\
%        {36}&      {0.175}\\
%        {38}&      {0.190}\\
%        {40}&      {0.207}\\
%        {42}&      {0.221}\\
%        {44}&      {0.233}\\
%        {46}&      {0.251}\\
%        {48}&      {0.266}\\
%        {50}&      {0.281}\\
%        {52}&      {0.296}\\
%        {54}&      {0.314}\\
%        {56}&      {0.336}\\
%        {58}&      {0.353}\\
%        {60}&      {0.376}\\
%        {65}&      {0.426}\\
%        {70}&      {0.473}\\
%        {75}&      {0.526}\\
%        {80}&      {0.580}\\
%        {85}&      {0.629}\\
%        {90}&      {0.671}\\
%        {95}&      {0.717}\\
%        {100}&     {0.750}\\
%        {105}&     {0.814}\\
%        {110}&     {0.890}\\
%        {115}&     {0.950}\\
%        {130}&     {1.003}\\
%        {140}&     {1.090}\\
%        {150}&     {1.181}\\
%        {160}&     {1.284}\\
%        {170}&     {1.370}\\
%        {180}&     {1.455}\\
%        {190}&     {1.540}\\
%        {200}&     {1.616}\\
%        {210}&     {1.688}\\
%        {220}&     {1.765}\\
%        {230}&     {1.833}\\
%        {240}&     {1.950}\\
%        {250}&     {1.996}\\
%        \bottomrule
%    \end{tabular}
%    \end{minipage}
%\end{table}
%

\begin{table}[H]
    \centering
    \caption{Anodenstrom $I_1$ mit der dazugehörigen Anodenspannung $U$ für die Heizspannung $U = 3,2 \, \unit{\volt} $.}
    \label{tab:kennlini1_1}
    \begin{tabular}{S S S S S[table-format=3.0] S}
      \toprule
      {$U \mathbin{/} \unit{\volt}$} & {$I_1 \mathbin{/} \unit{\milli\ampere}$} & {$U \mathbin{/} \unit{\volt}$} & {$I_1 \mathbin{/} \unit{\milli\ampere}$} & {$U \mathbin{/} \unit{\volt}$} & {$I_1 \mathbin{/} \unit{\milli\ampere}$}  \\
      \midrule
        {5}&       {0,005} & {20}&      {0,034} & {45}&      {0,065}\\
        {7}&       {0,010} & {21}&      {0,036} & {50}&      {0,067}\\
        {9}&       {0,013} & {23}&      {0,040} & {55}&      {0,069}\\
        {10}&      {0,015} & {25}&      {0,043} & {60}&      {0,070}\\
        {11}&      {0,017} & {27}&      {0,046} & {80}&      {0,070}\\
        {13}&      {0,020} & {29}&      {0,049} & {90}&      {0,072}\\
        {15}&      {0,024} & {30}&      {0,051} & {100}&     {0,074}\\
        {17}&      {0,028} & {35}&      {0,057} & {-}   &    {-}    \\
        {19}&      {0,031} & {40}&      {0,061} & {-}   &    {-}    \\
      \bottomrule
    \end{tabular}
\end{table}

\begin{table}[H]
    \centering
    \caption{Anodenstrom $I_2$ mit der dazugehörigen Anodenspannung $U$ für die Heizspannung $U = 3,5 \, \unit{\volt} $.}
    \label{tab:kennlini1_2}
    \begin{tabular}{S S S S S[table-format=3.1] S}
      \toprule
      {$U \mathbin{/} \unit{\volt}$} & {$I_2 \mathbin{/} \unit{\milli\ampere}$} & {$U \mathbin{/} \unit{\volt}$} & {$I_1 \mathbin{/} \unit{\milli\ampere}$} & {$U \mathbin{/} \unit{\volt}$} & {$I_1 \mathbin{/} \unit{\milli\ampere}$}  \\
      \midrule
            {5}&       {0,007}&        {26}&      {0,057}&        {60}&      {0,097}\\
            {7}&       {0,011}&        {28}&      {0,063}&        {65}&      {0,099}\\
            {9}&       {0,016}&        {30}&      {0,067}&        {70}&      {0,102}\\
            {10}&      {0,018}&        {32}&      {0,070}&        {80}&      {0,105}\\
            {12}&      {0,022}&        {34}&      {0,074}&        {90}&      {0,018}\\
            {14}&      {0,027}&        {36}&      {0,077}&        {100}&     {0,111}\\
            {16}&      {0,031}&        {38}&      {0,081}&        {110}&     {0,113}\\
            {18}&      {0,036}&        {40}&      {0,084}&        {120}&     {0,114}\\
            {20}&      {0,042}&        {45}&      {0,090}&        {130}&     {0,115}\\
            {22}&      {0,046}&        {50}&      {0,094}&        {140}&     {0,116}\\
            {24}&      {0,052}&        {55}&      {0,096}&        {150}&     {0,117}\\

      \bottomrule
    \end{tabular}
\end{table}

\begin{table}[H]
    \centering
    \caption{Anodenstrom $I_3$ mit der dazugehörigen Anodenspannung $U$ für die Heizspannung $U = 4,0 \, \unit{\volt} $.}
    \label{tab:kennlini1_3}
    \begin{tabular}{S S S S S[table-format=3.1] S}
      \toprule
      {$U \mathbin{/} \unit{\volt}$} & {$I_3 \mathbin{/} \unit{\milli\ampere}$} & {$U \mathbin{/} \unit{\volt}$} & {$I_1 \mathbin{/} \unit{\milli\ampere}$} & {$U \mathbin{/} \unit{\volt}$} & {$I_1 \mathbin{/} \unit{\milli\ampere}$}  \\
      \midrule
        {2}  &    { 0,002} & {34}  &    {0,110} & {75 }    & {0,218} \\
        {4}  &    { 0,007} & {36}  &    {0,117} & {80 }    & {0,223} \\
        {6}  &    { 0,012} & {38}  &    {0,124} & {85 }    & {0,228} \\
        {8}  &    { 0,016} & {40}  &    {0,132} & {90 }    & {0,227} \\
        {10}  &    {0,024} & {42}  &    {0,138} & {95 }    & {0,230} \\
        {12}  &    {0,030} & {44}  &    {0,145} & {100}    & {0,232} \\
        {14}  &    {0,036} & {46}  &    {0,152} & {110}    & {0,238} \\
        {16}  &    {0,042} & {48}  &    {0,159} & {120}    & {0,243} \\
        {18}  &    {0,049} & {50}  &    {0,165} & {130}    & {0,246} \\
        {20}  &    {0,056} & {52}  &    {0,170} & {140}    & {0,247} \\
        {22}  &    {0,064} & {54}  &    {0,167} & {150}    & {0,249} \\
        {24}  &    {0,071} & {56}  &    {0,173} & {160}    & {0,251} \\
        {26}  &    {0,080} & {58}  &    {0,180} & {170}    & {0,252} \\
        {28}  &    {0,088} & {60}  &    {0,186} & {180}    & {0,254} \\
        {30}  &    {0,093} & {65}  &    {0,197} & {190}    & {0,255} \\
        {32}  &    {0,101} & {70}  &    {0,209} & {200}    & {0,256} \\
      \bottomrule
    \end{tabular}
\end{table}


\begin{table}[H]
    \centering
    \caption{Anodenstrom $I_4$ mit der dazugehörigen Anodenspannung $U$ für die Heizspannung $U = 4,2 \, \unit{\volt} $.}
    \label{tab:kennlini1_4}
    \begin{tabular}{S S S S S[table-format=3.1] S}
      \toprule
      {$U \mathbin{/} \unit{\volt}$} & {$I_4 \mathbin{/} \unit{\milli\ampere}$} & {$U \mathbin{/} \unit{\volt}$} & {$I_1 \mathbin{/} \unit{\milli\ampere}$} & {$U \mathbin{/} \unit{\volt}$} & {$I_1 \mathbin{/} \unit{\milli\ampere}$}  \\
      \midrule
        {2}  &     {0,003}       & {34}  &    {0,140} & {75 }    & {0,360} \\
        {4}  &     {0,008}       & {36}  &    {0,146} & {80 }    & {0,385} \\
        {6}  &     {0,015}       & {38}  &    {0,156} & {85 }    & {0,405} \\
        {8}  &     {0,020}       & {40}  &    {0,168} & {90 }    & {0,425} \\
        {10}  &    {0,039}       & {42}  &    {0,177} & {95 }    & {0,446} \\
        {12}  &    {0,036}       & {44}  &    {0,188} & {100}    & {0,459} \\
        {14}  &    {0,044}       & {46}  &    {0,200} & {110}    & {0,474} \\
        {16}  &    {0,053}       & {48}  &    {0,212} & {120}    & {0,496} \\
        {18}  &    {0,061}       & {50}  &    {0,222} & {130}    & {0,512} \\
        {20}  &    {0,070}       & {52}  &    {0,235} & {140}    & {0,524} \\
        {22}  &    {0,080}       & {54}  &    {0,244} & {150}    & {0,530} \\
        {24}  &    {0,089}       & {56}  &    {0,252} & {160}    & {0,532} \\
        {26}  &    {0,101}       & {58}  &    {0,262} & {170}    & {0,535} \\
        {28}  &    {0,110}       & {60}  &    {0,269} & {180}    & {0,540} \\
        {30}  &    {0,121}       & {65}  &    {0,306} & {190}    & {0,543} \\
        {32}  &    {0,129}       & {70}  &    {0,335} & {200}    & {0,546} \\
      \bottomrule
    \end{tabular}
\end{table}

\begin{table}[H]
    \centering
    \caption{Anodenstrom $I_5$ mit der dazugehörigen Anodenspannung $U$ für die Heizspannung $U = 5,0 \, \unit{\volt} $.}
    \label{tab:kennlini1_5}
    \begin{tabular}{S S S S S[table-format=3.1] S}
      \toprule
      {$U \mathbin{/} \unit{\volt}$} & {$I_5 \mathbin{/} \unit{\milli\ampere}$} & {$U \mathbin{/} \unit{\volt}$} & {$I_1 \mathbin{/} \unit{\milli\ampere}$} & {$U \mathbin{/} \unit{\volt}$} & {$I_1 \mathbin{/} \unit{\milli\ampere}$}  \\
      \midrule
            {5}&       {0,015}&   {44}&      {0,233}& {105}&     {0,814}\\
            {10}&      {0,034}&   {46}&      {0,251}& {110}&     {0,890}\\
            {15}&      {0,056}&   {48}&      {0,266}& {115}&     {0,950}\\
            {16}&      {0,062}&   {50}&      {0,281}& {130}&     {1,003}\\
            {18}&      {0,072}&   {52}&      {0,296}& {140}&     {1,090}\\
            {20}&      {0,085}&   {54}&      {0,314}& {150}&     {1,181}\\
            {22}&      {0,094}&   {56}&      {0,336}& {160}&     {1,284}\\
            {24}&      {0,105}&   {58}&      {0,353}& {170}&     {1,370}\\
            {26}&      {0,116}&   {60}&      {0,376}& {180}&     {1,455}\\
            {28}&      {0,130}&   {65}&      {0,426}& {190}&     {1,540}\\
            {30}&      {0,140}&   {70}&      {0,473}& {200}&     {1,616}\\
            {32}&      {0,155}&   {75}&      {0,526}& {210}&     {1,688}\\
            {34}&      {0,162}&   {80}&      {0,580}& {220}&     {1,765}\\
            {36}&      {0,175}&   {85}&      {0,629}& {230}&     {1,833}\\
            {38}&      {0,190}&   {90}&      {0,671}& {240}&     {1,950}\\
            {40}&      {0,207}&   {95}&      {0,717}& {250}&     {1,996}\\
            {42}&      {0,221}&   {100}&     {0,750}&  {-} & {-}\\
      \bottomrule
    \end{tabular}
\end{table}


Die dazugehörigen Plots sind in \autoref{fig:kennlini} zu erkennen.

\begin{figure}[H]
    \centering
    \includegraphics{build/kennlinien.pdf}
    \caption{Anodenstrom $I_\text{A}$ in Abhängigkeit der Anodenspannung $U_\text{A}$ bei unterschiedlichen Heizspannungen.}
    \label{fig:kennlini}
\end{figure}

Die unterschiedlichen Sättigungsströme lassen sich relativ genau aus den aufgenommenen Messdaten erkennen.
Sie sind in \autoref{tab:sättigungsstromis} dargestellt.

\begin{table}
    \centering
    \caption{Sättigungsstrom $I_\text{S}$ zu unterschiedlichen Heizspannungen $U_\text{H}$ und -strömen $I_\text{H}$.}
    \label{tab:sättigungsstromis}
    \begin{tabular}{S S[table-format=1.2] S}
        \toprule
        {$U_\text{H} \mathbin{/} \unit{\volt}$} & {$I_\text{H} \mathbin{/} \unit{\ampere}$} & {$I_\text{S} \mathbin{/} \unit{\milli\ampere}$} \\
        \midrule
            3.2 & 1.95& 0.074\\
            3.5 & 2.0 & 0.117\\
            4.0 & 2.1 & 0.256\\
            4.2 & 2.2 & 0.546\\
            5.0 & 2.4 & 1.996\\
        \bottomrule
    \end{tabular}
\end{table}


\subsection{Untersuchung des Raumladungsgebiets}

Anhand der größtmöglichen Heizspannung soll nun das Raumladungsgebiet untersucht werden.
Hier wird also die fünfte Messreihe genauer betrachtet.
Dazu wird eine lineare Regression der Form
\begin{equation*}
    y = m x + b
\end{equation*}
durchgeführt, wobei Strom und Spannung logarithmisch gegeneinander aufgetragen werden. \\

Diese Regression mit den Parametern
\begin{equation*}
    m = 1,30 \pm 0,04
\end{equation*}
und
\begin{equation*}
    b = -13,25 \pm 0,01
\end{equation*}
ist in \autoref{fig:linregkennl5} aufgetragen.

\begin{figure}[H]
    \centering
    \includegraphics{build/linregraumladung.pdf}
    \caption{Lineare Regression an den logarithmisch geplotteten Werten aus Messreihe 5. Die Vertikale gibt den Grenzbereich des Raumladungsgebietes an.}
    \label{fig:linregkennl5}
\end{figure}


\subsubsection{Untersuchung des Anlaufstromgebiets}

Die zum Anlaufstrom aufgenommenen Messdaten sind in \autoref{tab:anlaufstrom} dargestellt.
Dabei wird die korrigierte Spannung $U_\text{korr}$ über
\begin{equation*}
    U_\text{korr} = U_\text{gemessen} - R_i \, I
\end{equation*} 
bestimmt, wobei $I$ der gemessene Strom und $R_i$ der Innenwiderstand von $1 \,\unit{\mega\ohm}$ sind.

\begin{table}
    \centering
    \caption{Gemessene Spannung $U_\text{gemessen}$, gemessener Strom $I$ sowie korrigierte Spannung $U_\text{korr}$.}
    \label{tab:anlaufstrom}
    \begin{tabular}{S[table-format=1.2] S[table-format=2.3] S S[table-format=1.2] S[table-format=2.3] S}
        \toprule
        {$U_\text{gemessen} \mathbin{/} \unit{\volt}$} & {$I \mathbin{/} \unit{\nano\ampere}$} & {$U_\text{korr} \mathbin{/} \unit{\volt}$}&{$U_\text{gemessen} \mathbin{/} \unit{\volt}$} & {$I \mathbin{/} \unit{\nano\ampere}$} & {$U_\text{korr} \mathbin{/} \unit{\volt}$} \\
        \midrule
        0         &       11    &   0,011   & 0,55      &       0,67  &   0,55067 \\     
        0,05      &       7,6   &   0,0576  & 0,6       &       0,54  &   0,60054 \\         
        0,1       &       5,5   &   0,1055  & 0,65      &       0,43  &   0,65043 \\     
        0,15      &       4,4   &   0,1544  & 0,7       &       0,36  &   0,70036 \\      
        0,2       &       3,5   &   0,2035  & 0,75      &       0,28  &   0,75028 \\     
        0,25      &       2,85  &   0,25285 & 0,8       &       0,26  &   0,80026 \\      
        0,3       &       2,2   &   0,3022  & 0,85      &       0,222 &   0,85022 \\     
        0,35      &       1,7   &   0,3517  & 0,9       &       0,19  &   0,90019 \\      
        0,4       &       1,35  &   0,40135 & 0,95      &       0,17  &   0,95017 \\     
        0,45      &       1,1   &   0,4511  & 0,96      &       0,16  &   0,96016 \\     
        0,5       &       0,88  &   0,50088 & {-}       &{-}          &         {-}\\           
        \bottomrule
    \end{tabular}
\end{table}

Nun soll ein Ausdruck für die Kathodentemperatur gefunden werden.
Dazu wird \eqref{eq:stromladexp} nach T umgestellt, sodass sich eine Regressionsgleichung der Form
\begin{equation*}
    \ln(I) = -\dfrac{\text{e}_0}{\text{k}_\text{B} T} V + \ln(c) = m x + b
\end{equation*}
ergibt, dessen grafische Darstellung in \autoref{fig:anlaufstrom} zu erkennen ist.

\begin{figure}
    \centering
    \includegraphics{build/reganlaufstrom.pdf}
    \caption{Logarithmische Darstellung der zum Anlaufstromgebiet aufgenommenen Messdaten samt linearer Regression.}
    \label{fig:anlaufstrom}
\end{figure}

Mit
\begin{equation*}
    m = (-4,373 \pm 0,100) \, \unit{\dfrac{1}{\volt}}
\end{equation*}
und
\begin{equation*}
    b = (-18,572 \pm 0,058) 
\end{equation*}
lässt sich über
\begin{equation*}
    m = -\dfrac{\text{e}_0}{\text{k}_\text{B} T} \Leftrightarrow T = - \dfrac{\text{e}_0}{\text{k}_\text{B} m}
\end{equation*}
die Kathodentemperatur zu
\begin{equation*}
    T = (2650 \pm 60) \,\unit{\kelvin}
\end{equation*}
bestimmen. 


\subsection{Kathodentemperatur aus Leistungsbilanz}

Mit den in \autoref{tab:sättigungsstromis} verwendeten Heizspannungen und -strömen kann nach
\begin{equation*}
    T = \left(\dfrac{U_\text{H} I_\text{H} - N_\text{WL}}{\eta \sigma f}\right)^\frac{1}{4}
\end{equation*}
die Kathodentemperatur bestimmt werden, wobei $N_\text{WL} = 0,9 \,\unit{\watt}$ die Wärmeleitung der Apparatur, $\eta = 0,28$ der Emissionsgrad der Oberfläche,
$\sigma = 5,7 \cdot 10^{-12} \,\unit{\watt \,\centi\meter^{-2} \,\kelvin^{-4}}$ die Boltzmannsche Strahlungskonstante und $f = 0,32 \,\unit{\centi\meter^2}$ die emittierende Kathodenoberfläche darstellen \cite{ap09}. \\

So ergeben sich die in \autoref{tab:tabelle3} dargestellten Werte.

%\begin{table}[H]
%    \centering
%    \caption{Kathodentemperatur $T$ zu unterschiedlichen Heizströmen.}
%    \label{tab:kathtempheizstrom}
%    \begin{tabular}{S S S S S}
%        \toprule
%        {$T_1 \mathbin{/} \unit{\kelvin}$} & {$T_2 \mathbin{/} \unit{\kelvin}$} & {$T_3 \mathbin{/} \unit{\kelvin}$} & {$T_4 \mathbin{/} \unit{\kelvin}$} & {$T_5 \mathbin{/} \unit{\kelvin}$} \\
%        \midrule
%
%        \bottomrule
%    \end{tabular}
%\end{table}

Gemittelt ergibt sich so eine Kathodentemperatur von
\begin{equation*}
    T = (1951,430 \pm 123,461) \,\unit{\kelvin} \,.
\end{equation*}


\subsection{Bestimmung der Austrittsarbeit}

Mithilfe von \eqref{eq:stromdichte} lässt sich nach Umstellen mit $j_\text{S} = \frac{I_\text{S}}{f}$ ein Ausdruck für die Austrittsarbeit $\text{e}_0 \,\Phi$ finden.
Dieser lautet
\begin{equation*}
    \text{e}_0 \,\Phi = -\text{k}_\text{B} T \,\ln \left(\dfrac{I_\text{S} \, \text{h}^3}{4 \, \pi \, \text{e}_0 \, \text{k}^2_\text{B} \, \text{m}_0 \,f \,  T^2}\right) \,.
\end{equation*} \\

Die zu den bereits in \autoref{tab:sättigungsstromis} bestimmten Sättigungsströmen gehörigen Austrittsarbeiten sind in \autoref{tab:tabelle3} dargestellt.

%\begin{table}
%    \centering
%    \caption{Austrittsarbeiten $\text{e}_0 \Phi$ zu unterschiedlichen Sättigungsströmen.}
%    \label{tab:austrittsarbeit}
%    \begin{tabular}{S S S S S}
%        \toprule
%        {$\text{e}_0\Phi_1 \mathbin{/} \unit{\eV}$} & {$\text{e}_0\Phi_2 \mathbin{/} \unit{\eV}$} & {$\text{e}_0\Phi_3 \mathbin{/} \unit{\eV}$} & {$\text{e}_0\Phi_4 \mathbin{/} \unit{\eV}$} & {$\text{e}_0\Phi_5 \mathbin{/} \unit{\eV}$}  \\
%        \midrule
%        %%%%% Hier könnten Ihre Werte stehen
%
%
%
%
%        %%%%%%
%        \bottomrule
%    \end{tabular}
%\end{table}

\begin{table}[H]
    \centering
    \caption{Messwerte der Sättigungsstromstärken, Heizstromstärken und -spannungen, sowie die daraus errechneten Kathodentemperaturen T und Austrittsarbeiten $\text{e}_0\Phi$.}
    \label{tab:tabelle3}
    \begin{tabular}{S[table-format=1.2] S S S S}
        \toprule
        {$I_{H} \mathbin{/} \unit{\ampere}$} & {$U_{H} \mathbin{/} \unit{\volt}$} & {$T \mathbin{/} \unit{\kelvin}$} & {$I_{S} \mathbin{/} \unit{\milli\ampere}$} & {$\text{e}_0\Phi \mathbin{/} \unit{\eV}$} \\
        \midrule
        1,95&3,2&1798,207&0,074&3.291\\
        2   &3,5&1859,031&0,117&3.340\\
        2,1 &4,0&1957,580&0.256&3.402\\
        2,2 &4,2&1983,175&0,546&3.322\\
        2,4 &5,0&2159,161&1,996&3.407\\
        \bottomrule
    \end{tabular}
\end{table}



Gemittelt ergibt sich hier

\begin{equation*}
    \text{e}_0 \Phi = (3,352 \pm 0,045) \,\unit{\eV} \,.
\end{equation*}