\section{Theorie}
\label{sec:Theorie}

Einer elektromagnetischem Welle kann ein Poynting-Vektor zu geordnet werden. Der Betrag wird als 
\begin{equation}
    \left| \vec{S}\right|  = v \epsilon \epsilon_0  \vec{E}²
    \label{eq:betragpoynting}
\end{equation}
 geschrieben werden.

Trifft eine ebene Lichtwelle unter dem Winkel  $\alpha$  aus dem Vakuum auf eine Grenzfläche zu einem anderen Medium, so wird ein Teil der Strahlung gebrochen $\vec{E_r}$ und der andere reflektiert $\vec{E_d}$. Es wird angenommen, dass kein Licht vom Medium absorbiert wird.


\begin{figure}
    \centering
    \includegraphics{Brechung an einer Ebene.pdf}
    \caption{Brechung an einer Ebene.} 
    \label{fig:abb1}
\end{figure}

Wie in der oberen Grafik \ref{fig:abb1} zu sehen ist, ist der Brechungswinkel $ \beta$ kleiner als der Brechungswinkel $\alpha$, da das Licht im Medium eine geringere Geschwindigkeit hat als im Vakuum.
Daraus bedingt sich eine Querschnittsänderung der einlaufenden Strahlung von $F_e$ auf $F_d$. 
Es gilt die Energieerhaltung, die sich als
\begin{equation}
    S_e F_e = S_r F_e +  S_d F_d
    \label{eq:Energieerhaltung}
\end{equation}
oder auch 

\begin{equation}
    S_e \cos(\alpha) = S_r \cos(\alpha) +  S_d \cos(\beta)
    \label{eq:Energieerhaltungwinkel}
\end{equation}
schreiben lässt.
Die Gleichung \eqref{eq:Energieerhaltungwinkel} lässt sich mit hilfe der Gleichung \eqref{eq:betragpoynting} für die Poynting-Vektoren zu 
\begin{equation}
    c \epsilon_0 \vec{E_e}² \cos(\alpha) = c \epsilon_0 \vec{E_r}^2 \cos(\alpha) + v \epsilon \epsilon_0 \vec(E_d)^2 \cos(\beta)
    \label{eq:eins}
\end{equation}
umstellen. Der Brechungsindex ist das Verhältnis zwischen der Lichtgeschwindigkeit im Vakuum und der im Medium 
\begin{equation}
    n = \dfrac{c}{v} ,
    \label{eq:Brechungsindex}
\end{equation}
damit kann anstelle der Gleichung \eqref{eq:eins} auch 
\begin{equation}
    (\vec{E_e}^2  - \vec{E_r}^2)n \cos(\alpha) = \epsilon \vec(E_d)^2 \cos(\beta)
    \label{eq:zwei}
\end{equation}
geschrieben werden.

Wenn die Maxwellsche Relation
\begin{equation*}
    n^2 = \epsilon
    \label{eq:Maxwellschrelation}
\end{equation*}
in die Gleichung \eqref{eq:zwei} eingesetzt wird, ergibt sich 
\begin{equation}
    (\vec{E_e}^2  - \vec{E_r}^2) \cos(\alpha) = n \vec(E_d)^2 \cos(\beta).
    \label{eq:drei}
\end{equation}

Die eingehende Welle, die durch den Vektor $\vec{E_e}$ gegeben ist, besteht aus einem parallel und senkrecht polarisierten Teil.
\begin{equation*}
    \vec{E_e} = \vec{E_\parallel}  + \vec{E_\perp}
    \label{eq:vier}
\end{equation*}
Die unterschiedlich polarisierten Teile der Welle verhalten sich an der Grenzfläche verschieden.
Aus diesem Grund ist es notwendig den parallelen und senkrechten Teil seperat zu betrachten. 

\subject{Senkrecht Polarisation}
\label{senkrechtpolarisation}

Die Welle in der Abbildung \ref{fig:abb3} schwingt nach der Definition senkrecht zur Einfallsebene.
Die Gesetzen der Elektrodynamik machen eine Aussage über die Änderung der elektrischen Feldestärke beim Übergang von einem Medium in ein anderes.
Aus dem Verschinden des Linienintegral längs einer geschlossenen Kurve folgt, dass die Tangentialkomponente der Feldestärke stetig durch die Grenzfläche hindurchgeht.
Aus dieser Stetigkeit gilt
%Irgendwas hier wirft einen Fehler, aber ich finde den leider nicht
%\begin{equation}
%    \vec(E_r_\perp) = - \vec{E_e_\perp} \dfrac{n \cos(\beta) - \cos(\alpha)}{n \cos(\beta) + \cos(\alpha)}.
%    \label{eq:senkrechtpol}
%\end{equation}
%
%Durch die Hilfe des Snelliusschen Brechungsgesetzes 
%
%\begin{equation*}
%    n = \dfrac{\sin(\alpha)}{\sin(\beta)}
%    \label{eq:Snelli}
%\end{equation*}
%
%können die $ \beta $ Terme eliminiert werden und sich die Formel
%
%\begin{equation}
%    \vec(E_r_\perp) = - \vec(E_e_\perp) \dfrac{(\sqrt{n^2-\sin^2(\alpha)}- \cos(\alpha))^2}{n^2 -1} .
%    \label{eq:senkrechtpol2}
%\end{equation}
%
%\begin{figure}
%    \centering
%    \includegraphics{Brechung mit paralleler Strahlung.pdf}
%    \caption{Brechung senkrecht polarisierter Wellen.} 
%    \label{fig:abb3}
%\end{figure}