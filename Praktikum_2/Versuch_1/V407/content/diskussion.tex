\section{Diskussion}
\label{sec:Diskussion}

Während des Experiments sind mehrere Fehlerquellen aufgefallen. Zunächst konnte auch bei ausgeschalteten Laser ein Nullstrom gemessen werden. Dieser entstand wahrscheinlich aufgrund Sonnenlicht, welches trotz abgedunkelten Fenstern auf die Photozelle fiel.
Da der Nullstrom im $\unit{\nano \ampere}$-Bereich gemessen wurde, wurde dieser vernachlässigt.
Weiter war der verwendete Laser nicht stark genug fokussiert, damit er vollständig in die Öffnung der Photozelle fällt. 
So wurde immer nur ein Teil des Lichtes eingefangen und es musste darauf geachtet werden, dass immer der gleiche Teil des Lasers eingefangen wurde.


Der Brechungsindex für Silizium liegt laut Literatur \cite{ap02} für Licht mit einer Wellenlänge von $ 632,8 \, \unit{\nano\meter}$ bei $ \text{n} = 3,88 $.
Wird der Brechungsindex aus der Messreihe vom senkrecht polarisierten Licht $ \text{n} = 3,4 \pm 0,4 $ betrachtet, erhält man eine Abweichung von $ 12,37 \,\%$.


Bei der parallelen Polarisation werden nur die Werte zwischen $5° $ und $ 75°$ gemittelt, da bei größeren Winkeln der Brechungsindex entartet.
Als Mittelwert für die Brechungsindizes dieser Winkel ergibt sich $ \text{n} = 3,71 \pm 0,21 $ mit einer Abweichung von $ 4,38 \,\%$ vom Literaturwert.


Für den Brewsterwinkel ergibt sich ein Brechungsindex von $ \text{n} = 3,606 $ mit einer Abweichung von $ 7,061 \,\%$. Der aus allen drei Messungen gemittelte Wert $\bar{n} = 3,57 \pm 0,16$ hat eine Abweichung von $ 7,99 \,\%$.
