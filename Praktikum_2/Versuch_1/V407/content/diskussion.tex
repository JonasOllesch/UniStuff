\section{Diskussion}
\label{sec:Diskussion}

Während des Experiments sind mehrere Fehlerquellen aufgefallen. Zunächst konnte auch bei ausgeschalteten Laser ein Nullstrom gemessen werden. Dieser entstand wahrscheinlich aufgrund Sonnenlicht, welches trotz abgedunkelten Fenstern auf die Photozelle fiel.
Da der Nullstrom im Nanoampere Bereich gemessen wurde, wurde dieser vernachlässigt.
Weiter war der verwendete Laser nicht stark genug fokussiert, damit er vollständig in die Öffnung der Photozelle fällt. So wurde immer nur ein Teil des Lichtes eingefangen und es musste darauf geachtet werden, dass immer der gleiche Teil des Lasers eingefangen wurde.
Der Brechungsindex für Silizium liegt laut Literatur \cite{ap02} bei $ \text{n} = 3.88 $ für Licht mit einer Wellenlänge von $ 632.8 \, \unit{\nano\meter}$.
Wird der Brechungsindex aus der Messreihe vom senkrecht polarisierten Licht $ \text{n} = 3.4 \pm 0.4 $ betrachtet, erhält man eine Abweichung von $ 12.37 \% \,. $
Mit der parallelen Polarisation wird eine Abweichung von $ 3.09 \% \, $ erziehlt, obwohl dieser Wert nur über das Mitteln alles Winkel erreicht wird, da der Brechungsindex zwischen $ \text{n} = 8.069 $ und  $ \text{n} = 1.215 $ je nach Winkel variiert.
Das sind Werte, die weit außerhalb der Abweichung liegen.
Für den Brewsterwinkel ergibt sich ein Brechungsindex von $ \text{n} = 3.606 $ mit einer Abweichung von $ 7.061 \% \, .$ Der gemittelte Wert hat eine Abweichung von $ 4.639 \% \, .$
