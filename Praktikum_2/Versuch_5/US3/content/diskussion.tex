\section{Diskussion}
\label{sec:diskussion}

Zum Aufbau ist anzumerken, dass das Prisma nicht fest am Schauch fixiert werden konnte. Die Kunststoffplatte, welche die Stabilität des Prisma garantieren sollte, hat nicht in die vorgesehende Halterung gepasst.
Ohne das Festsetzen des Prismas am Schlauch konnte es zu Luftblasen in dem Ultraschallgel kommen und damit zu Fehlern in der Messung.

An den ersten drei Plots \autoref{fig:Graph1a},\autoref{fig:Graph1b} und \autoref{fig:Graph1c} ist eindeutig ein linearer Zusammenhang zwischen der Schrömungsgeschwindigkeit und der Frequenzverschiebung festzustellen.
Die Plots \autoref{fig:Graph2a1} und \autoref{fig:Graph2b1} weisen einen ähnlichen Verlauf auf. 
Eine mehr Messdaten sind notwendig, um eine Ausgeleichsfunktion zu plotten. Aufgrund der wenigen Messdaten an den Peaks ist ein Maximum bei einer Messtiefe von ungefähr $0.0185  \, \unit{\milli\meter}$ anzunehmen.

Die Plots \autoref{fig:Graph2a2} und \autoref{fig:Graph2b2} bestätigen keinen Zusammenhang zwischen Messtiefe und Momentangeschwindigkeit.
Was die Aussage zulässt, dass die Flussgeschwindigkeit nicht von dem Abstand zum Mittelpunkt des Rohres abhängt.