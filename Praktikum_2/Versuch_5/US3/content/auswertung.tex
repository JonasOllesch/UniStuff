\section{Auswertung}
\label{sec:auswertung}

Vor der Auswertung der aufgenommenen Messdaten werden zunächst nach \eqref{eq:dopplerwinkel} die Dopplerwinkel $\alpha_\theta$ bestimmt.
Mit $c_L = 1800 \,\unit{\frac{\meter}{\second}}$ und $c_P = 2700 \,\unit{\frac{\meter}{\second}}$ergeben sich

\begin{align*}
    \alpha_{15°} &= 80,06 \,° \,,\\
    \alpha_{30°} &= 70,53 \,° \,,\\
    \alpha_{45°} &= 61,87 \,° \,. 
\end{align*}

\subsection{Bestimmung der Strömungsgeschwindigkeit}

Unter Umstellung von \eqref{eq:dopplerversch} ergibt sich für die Strömungsgeschwindigkeit $v$ der Ausdruck
\begin{equation}
    v = \frac{\Delta\nu \,c}{2 \,\nu_0 \cos\alpha} \,.
    \label{eq:momgeschwi}
\end{equation}

Die unterschiedlichen Strömungsgeschwindigkeiten für die verschiedenen Pumpleistungen und Dopplerwinkel sind dabei in \autoref{tab:1winkel1}, \autoref{tab:1winkel2} und \autoref{tab:1winkel3} dargestellt und jeweils gegen  
$\frac{\Delta\nu}{\cos\alpha}$ geplottet. \\

Dabei sind $\nu_0 = 2 \,\unit{\mega\hertz}$ sowie $c = c_L = 1800 \,\unit{\frac{\meter}{\second}}$ für alle drei Dopplerwinkel gegeben.
Die Pumpleistung in $\%$ ergibt sich dabei aus
\begin{equation*}
    \frac{P_{\text{max}}}{P} \,,
\end{equation*}
wobei die maximale Umdrehungszahl der Pumpe $8600 \, \text{rpm}$ entspricht. \\

\begin{figure} 
    \begin{minipage}[t]{.5\textwidth}
    \centering
    \begin{table}[H]
        \centering
        \captionsetup{justification=centering}
        \caption{Pumpleistungen, \\Frequenzverschiebungen und \\ Strömungsgeschwindigkeiten bei \\ einem Prismawinkel von $15 \,°$.}
        \label{tab:1winkel1} 
       \begin{tabular}{S S S}
        \toprule 
        {Pumpleistung $\mathbin{/}\%$} & {$\Delta \nu \mathbin{/} \unit{\hertz}$} & {$v \mathbin{/} \unit{\frac{\meter}{\second}}$}  \\
        \midrule 
            75.58   &   354  & 0.16 \\
            81.40   &   419  & 0.20 \\
            87.21   &   365  & 0.17 \\
            93.02   &   556  & 0.26 \\
            100.0   &   494  & 0.23 \\
        \bottomrule
       \end{tabular} 
    \end{table}
    \end{minipage}
    \begin{minipage}[t]{.5\textwidth} 
        \centering
        \includegraphics[scale=0.75]{build/Graph1a.pdf} % Das hier durch \includegraphics{\build\graph1a.pdf} ersetzen, wie immer einfach
        \captionsetup{justification=centering}
        \captionof{figure}{Frequenzverschiebung $\frac{\Delta \nu}{\cos\alpha}$ \\ in Abhängigkeit der \\ Strömungsgeschwindigkeit $v$ \\ ($\alpha = 15 \,°$).} 
        %\begin{figure}
        %    \centering
        %    \captionsetup{justification=centering}
        %    \caption{Frequenzverschiebung $\frac{\Delta \nu}{\cos\alpha}$ \\ in Abhängigkeit der \\ Strömungsgeschwindigkeit $v$ \\ ($\alpha = 15 \,°$).} 
        %    \label{fig:graph1a}
        %\end{figure}
    \end{minipage} 
\end{figure} 


\begin{figure} 
    \begin{minipage}[t]{.5\textwidth}
    \centering
    \begin{table}[H]
        \centering
        \captionsetup{justification=centering}
        \caption{Pumpleistungen, \\Frequenzverschiebungen und \\ Strömungsgeschwindigkeiten bei \\ einem Prismawinkel von $30 \,°$.}
        \label{tab:1winkel2} 
       \begin{tabular}{S S S}
        \toprule 
        {Pumpleistung $\mathbin{/}\%$} & {$\Delta \nu \mathbin{/} \unit{\hertz}$} & {$v \mathbin{/} \unit{\frac{\meter}{\second}}$}  \\
        \midrule 
                75.58   &     520  & 0.27 \\
                81.40   &     596  & 0.31 \\
                87.21   &     670  & 0.35 \\
                93.02   &     564  & 0.29 \\
                100.0   &     868  & 0.45 \\
        \bottomrule
       \end{tabular} 
    \end{table}
    \end{minipage}
    \begin{minipage}[t]{.5\textwidth} 
        \centering
        \vspace*{0pt}\rule{.95\textwidth}{12em} % Das hier durch \includegraphics{\build\graph1a.pdf} ersetzen, wie immer einfach
        \captionsetup{justification=centering}
        \captionof{figure}{Frequenzverschiebung $\frac{\Delta \nu}{\cos\alpha}$ \\ in Abhängigkeit der \\ Strömungsgeschwindigkeit $v$ \\ ($\alpha = 30 \,°$).}  
        %\begin{figure}
        %    \centering
        %    \captionsetup{justification=centering}
        %    \caption{Frequenzverschiebung $\frac{\Delta \nu}{\cos\alpha}$ \\ in Abhängigkeit der \\ Strömungsgeschwindigkeit $v$ \\ ($\alpha = 30 \,°$).}  
        %    \label{fig:graph1b}
        %\end{figure}
    \end{minipage} 
\end{figure} 

\begin{figure}[H] 
    \begin{minipage}[t]{.5\textwidth}
    \centering
    \begin{table}[H]
        \centering
        \captionsetup{justification=centering}
        \caption{Pumpleistungen, \\Frequenzverschiebungen und \\ Strömungsgeschwindigkeiten bei \\ einem Prismawinkel von $45 \,°$.}
        \label{tab:1winkel3} 
       \begin{tabular}{S S S}
        \toprule 
        {Pumpleistung $\mathbin{/}\%$} & {$\Delta \nu \mathbin{/} \unit{\hertz}$} & {$v \mathbin{/} \unit{\frac{\meter}{\second}}$}  \\
        \midrule 
                75.58   &     858   & 0.55 \\
                81.40   &     960   & 0.61 \\
                87.21   &     1016  & 0.65 \\
                93.02   &     1172  & 0.75 \\
                100.0   &     1397  & 0.89 \\
        \bottomrule
       \end{tabular} 
    \end{table}
    \end{minipage}
    \begin{minipage}[t]{.5\textwidth} 
        \centering
        \vspace*{0pt}\rule{.95\textwidth}{12em} % Das hier durch \includegraphics{\build\graph1a.pdf} ersetzen, wie immer einfach
        \captionsetup{justification=centering}
        \captionof{figure}{Frequenzverschiebung $\frac{\Delta \nu}{\cos\alpha}$ \\ in Abhängigkeit der \\ Strömungsgeschwindigkeit $v$ \\ ($\alpha = 45 \,°$).}  
        %\begin{figure}
        %    \centering
        %    \captionsetup{justification=centering}
        %    \caption{Frequenzverschiebung $\frac{\Delta \nu}{\cos\alpha}$ \\ in Abhängigkeit der \\ Strömungsgeschwindigkeit $v$ \\ ($\alpha = 45 \,°$).}  
        %    \label{fig:graph1c}
        %\end{figure}
    \end{minipage} 
\end{figure} 






\subsection{Streuintensität und Momentangeschwindigkeit}

Die benötigten Momentangeschwindigkeiten berechnen sich erneut aus \eqref{eq:momgeschwi}, wieder gelten $\nu_0 = 2 \,\unit{\mega\hertz}$ und $c = c_L = 1800 \,\unit{\frac{\meter}{\second}}$.
Die Messtiefe der Sonde, umgerechnet von $\unit{\micro\second}$ in $\unit{\milli\meter}$, die berechneten Geschwindikeiten $v$ sowie die Streuintensität $I$ für eine Pumpleistung von $70 \,\%$ bzw. $45 \,\%$ finden sich in \autoref{tab:2a} und \autoref{tab:2b}. \\

Zusätzlich sind die Momentangeschwindigkeit und Streuintensitäten in Abhängigkeit der Messtiefe in \autoref{fig:graph2a} und \autoref{fig:graph2a} dargestellt.

\begin{figure}[H] 
    \begin{minipage}[t]{.6\textwidth}
    \centering
    \begin{table}[H]
        \centering
        \captionsetup{justification=centering}
        \caption{Messtiefen, \\ Momentangeschwindigkeiten $v$ \\ und Streuintensitäten $I$ \\ bei einer Pumpleistung von $70 \,\%$.}
        \label{tab:2a} 
       \begin{tabular}{S S S S}
        \toprule 
        {Messtiefe $\mathbin{/} \unit{\micro\second}$} & {Messtiefe $\mathbin{/} \unit{\milli\meter}$} & {$v \mathbin{/} \unit{\frac{\meter}{\second}} $} & {$I \mathbin{/} 1000 \cdot \unit{\dfrac{\volt^2}{\second}}$}  \\
        \midrule 
                6.67   &12.0          &0.0& 14.0   \\
                6.94   &12.5          & 44.6& 53.0  \\
                7.22   &13.0          & 47.9& 45.0  \\
                7.5   &13.5          & 63.7& 61.0  \\
                7.78   &14.0          & 70.0& 74.0  \\
                8.06   &14.5          & 73.2& 98.0  \\
                8.33   &15.0          & 73.2& 124.0 \\
                8.61   &15.5          & 63.7& 133.0 \\
                8.89   &16.0          & 52.5& 134.0 \\
                9.17   &16.5          & 43.0& 157.0 \\
                9.44   &17.0          & 49.3& 159.0 \\
                9.72   &17.5          & 57.3& 332.0 \\
                10.00  &18.0          & 60.5& 685.0 \\
                10.28  &18.5           & 57.3& 423.0 \\
                10.56  &19.0           & 54.1& 121.0 \\
                10.83  &19.5           & 50.9& 75.0  \\
        \bottomrule
       \end{tabular} 
    \end{table}
    \end{minipage}
    \begin{minipage}[t]{.5\textwidth} 
        \centering
        \vspace*{0pt}\rule{.95\textwidth}{12em} % Das hier durch \includegraphics{\build\graph1a.pdf} ersetzen, wie immer einfach
        \captionsetup{justification=centering}
        \captionof{figure}{Momentangeschwindigkeit \\ und Streuintensität \\ in Abhängigkeit der Messtiefe.}  
        %\begin{figure}
        %    \centering
        %    \captionsetup{justification=centering}
        %    \caption{Momentangeschwindigkeit und Streuintensität in Abhängigkeit der Messtiefe.}  
        %    \label{fig:graph1c}
        %\end{figure}
    \end{minipage} 
\end{figure} 


\begin{figure} 
    \begin{minipage}[t]{.6\textwidth}
    \centering
    \begin{table}[H]
        \centering
        \captionsetup{justification=centering}
        \caption{Messtiefen, \\ Momentangeschwindigkeiten $v$ \\ und Streuintensitäten $I$ \\ bei einer Pumpleistung von $45 \,\%$.}
        \label{tab:2b} 
       \begin{tabular}{S S S S}
        \toprule 
        {Messtiefe $\mathbin{/} \unit{\micro\second}$} & {Messtiefe $\mathbin{/} \unit{\milli\meter}$} & {$v \mathbin{/} \unit{\frac{\meter}{\second}} $} & {$I \mathbin{/} 1000 \cdot \unit{\dfrac{\volt^2}{\second}}$}  \\
        \midrule 
                6.67        &   12.0   &  0.0      &29.0    \\
                6.94        &   12.5   &  19.1     &39.0    \\
                7.22        &   13.0   &  19.1     &72.0    \\
                7.5         &   13.5   &  22.3     &94.0    \\
                7.78        &   14.0   &  25.5     &128.0   \\
                8.06        &   14.5   &  25.5     &132.0   \\
                8.33        &   15.0   &  25.5     &145.0   \\
                8.61        &   15.5   &  25.5     &136.0   \\
                8.89        &   16.0   &  22.3     &117.0   \\
                9.17        &   16.5   &  19.1     &111.0   \\
                9.44        &   17.0   &  22.3     &68.0    \\
                9.72        &   17.5   &  22.3     &127.0   \\
                10.00       &   18.0   &  22.3     &443.0   \\
                10.28       &   18.5   &  22.3     &530.0   \\
                10.56       &   19.0   &  22.3     &199.0   \\
                10.83       &   19.5   &  22.3     &54.0    \\
        \bottomrule
       \end{tabular} 
    \end{table}
    \end{minipage}
    \begin{minipage}[t]{.5\textwidth} 
        \centering
        \vspace*{0pt}\rule{.95\textwidth}{12em} % Das hier durch \includegraphics{\build\graph1a.pdf} ersetzen, wie immer einfach
        \captionsetup{justification=centering}
        \captionof{figure}{Momentangeschwindigkeit \\ und Streuintensität \\ in Abhängigkeit der Messtiefe.}  
        %\begin{figure}
        %    \centering
        %    \captionsetup{justification=centering}
        %    \caption{Momentangeschwindigkeit und Streuintensität in Abhängigkeit der Messtiefe.}  
        %    \label{fig:graph1c}
        %\end{figure}
    \end{minipage} 
\end{figure} 



