\section{Auswertung}
\label{sec:auswertung}

Vor der Auswertung der aufgenommenen Messdaten werden zunächst nach \eqref{eq:dopplerwinkel} die Dopplerwinkel $\alpha_\theta$ bestimmt.
Mit $c_L = 1800 \,\unit{\frac{\meter}{\second}}$ und $c_P = 2700 \,\unit{\frac{\meter}{\second}}$ergeben sich

\begin{align*}
    \alpha_{15°} &= 80,06 \,° \,,\\
    \alpha_{30°} &= 70,53 \,° \,,\\
    \alpha_{45°} &= 61,87 \,° \,. 
\end{align*}

\subsection{Bestimmung der Strömungsgeschwindigkeit}

Unter Umstellung von \eqref{eq:dopplerversch} ergibt sich für die Strömungsgeschwindigkeit $v$ der Ausdruck
\begin{equation}
    v = \frac{\Delta\nu \,c}{2 \,\nu_0 \cos\alpha} \,.
    \label{eq:momgeschwi}
\end{equation}

Die unterschiedlichen Strömungsgeschwindigkeiten für die verschiedenen Pumpleistungen und Dopplerwinkel sind dabei in \autoref{tab:1winkel1}, \autoref{tab:1winkel2} und \autoref{tab:1winkel3} dargestellt und jeweils gegen  
$\frac{\Delta\nu}{\cos\alpha}$ geplottet. \\

Dabei sind $\nu_0 = 2 \,\unit{\mega\hertz}$ sowie $c = c_L = 1800 \,\unit{\frac{\meter}{\second}}$ für alle drei Dopplerwinkel gegeben.
Die Pumpleistung in $\%$ ergibt sich dabei aus
\begin{equation*}
    \frac{P_{\text{max}}}{P} \,,
\end{equation*}
wobei die maximale Umdrehungszahl der Pumpe $8600 \, \text{rpm}$ entspricht. \\

\begin{figure}
    \begin{subtable}{0.43\textwidth}
        \centering
       \begin{tabular}{c c c}
        \toprule 
        {Pumpleistung $\mathbin{/}\%$} & {$\Delta \nu \mathbin{/} \unit{\hertz}$} & {$v \mathbin{/} \unit{\frac{\meter}{\second}}$}  \\
        \midrule 
            75.58   &   354  & 0.16 \\
            81.40   &   419  & 0.20 \\
            87.21   &   365  & 0.17 \\
            93.02   &   556  & 0.26 \\
            100.0   &   494  & 0.23 \\
        \bottomrule
       \end{tabular}
       \caption{Pumpleistungen, Frequenzverschiebungen und Strömungsgeschwindigkeiten bei einem Prismawinkel von $15 \,°$.}
       \label{tab:1winkel1} 
       \qquad
    \end{subtable}
    \begin{subfigure}{0.57\textwidth} 
        \centering
        \includegraphics[width=\textwidth]{build/Graph1a.pdf} % Das hier durch \includegraphics{\build\graph1a.pdf} ersetzen, wie immer einfach
        \caption{Frequenzverschiebung $\frac{\Delta \nu}{\cos\alpha}$ in Abhängigkeit der Strömungsgeschwindigkeit $v$ ($\alpha = 15 \,°$).} 
        \label{fig:graph1a}
        \qquad
    \end{subfigure}
    \caption{Tabelle und Plot bei einem Prismawinkel von $15 \,°$.} 
\end{figure} 

\begin{figure}
    \begin{subtable}{0.43\textwidth}
        \centering
       \begin{tabular}{c c c}
        \toprule 
        {Pumpleistung $\mathbin{/}\%$} & {$\Delta \nu \mathbin{/} \unit{\hertz}$} & {$v \mathbin{/} \unit{\frac{\meter}{\second}}$}  \\
        \midrule 
                75.58   &     520  & 0.27 \\
                81.40   &     596  & 0.31 \\
                87.21   &     670  & 0.35 \\
                93.02   &     564  & 0.29 \\
                100.0   &     868  & 0.45 \\
        \bottomrule
       \end{tabular}
       \caption{Pumpleistungen, Frequenzverschiebungen und Strömungsgeschwindigkeiten bei einem Prismawinkel von $30 \,°$.}
        \label{tab:1winkel2}  
    \end{subtable}
    \begin{subfigure}{0.57\textwidth} 
        \centering
        \includegraphics[width=\textwidth]{build/Graph1b.pdf} % Das hier durch \includegraphics{\build\graph1a.pdf} ersetzen, wie immer einfach
        \caption{Frequenzverschiebung $\frac{\Delta \nu}{\cos\alpha}$ in Abhängigkeit der Strömungsgeschwindigkeit $v$ ($\alpha = 30 \,°$).}  
        \label{fig:graph1b}
        \qquad
    \end{subfigure}
    \caption{Tabelle und Plot bei einem Prismawinkel von $30 \,°$.} 
\end{figure}  

\begin{figure}
    \begin{subtable}{0.43\textwidth}
        \centering
       \begin{tabular}{c c c}
        \toprule 
        {Pumpleistung $\mathbin{/}\%$} & {$\Delta \nu \mathbin{/} \unit{\hertz}$} & {$v \mathbin{/} \unit{\frac{\meter}{\second}}$}  \\
        \midrule 
                75.58   &     858   & 0.55 \\
                81.40   &     960   & 0.61 \\
                87.21   &     1016  & 0.65 \\
                93.02   &     1172  & 0.75 \\
                100.0   &     1397  & 0.89 \\
        \bottomrule
       \end{tabular}
       \caption{Pumpleistungen, Frequenzverschiebungen und Strömungsgeschwindigkeiten bei einem Prismawinkel von $45 \,°$.}
        \label{tab:1winkel3}
    \end{subtable}
    \begin{subfigure}{0.57\textwidth} 
        \centering
        \includegraphics[width=\textwidth]{build/Graph1c.pdf}  
        \caption{Frequenzverschiebung $\frac{\Delta \nu}{\cos\alpha}$ in Abhängigkeit der Strömungsgeschwindigkeit $v$ ($\alpha = 45 \,°$).}
        \label{fig:graph1c}
        \qquad
    \end{subfigure}
    \caption{Tabelle und Plot bei einem Prismawinkel von $45 \,°$.} 
\end{figure}    

\subsection{Streuintensität und Momentangeschwindigkeit}

Die benötigten Momentangeschwindigkeiten berechnen sich erneut aus \eqref{eq:momgeschwi}, wieder gelten $\nu_0 = 2 \,\unit{\mega\hertz}$ und $c = c_L = 1800 \,\unit{\frac{\meter}{\second}}$.
Die Messtiefe der Sonde, umgerechnet von $\unit{\micro\second}$ in $\unit{\milli\meter}$, die berechneten Geschwindikeiten $v$ sowie die Streuintensität $I$ für eine Pumpleistung von $70 \,\%$ bzw. $45 \,\%$ finden sich in \autoref{tab:2a} und \autoref{tab:2b}.
Bei der Umrechnung der Messtiefe muss zunächst die Vorlaufzeit des Ultraschalls durch das Prisma brücksichtigt werden, da die Schallgeschwindigkeit im Prisma mit $c_p = 2700 \unit{\frac{\meter}{\second}}$ größer ist als in der Dopplerflüssigkeit.
Die Länge der Vorlaufzeit im Prisma beträt $ l = 30,7 \unit{milli\meter}$

Gemessen wurde in $0,5 \unit{\micro\second}$- Schritten, dabei sind die entsprechenden Messtiefen in $\unit{\milli\meter}$ aufsteigend in den dazugehörigen Tabellen eingetragen. \\

Zusätzlich sind die Momentangeschwindigkeit und Streuintensitäten in Abhängigkeit der Messtiefe in \autoref{fig:graph2a} und \autoref{fig:graph2a} dargestellt.

\begin{figure}[H]
    \begin{subfigure}{0.57\textwidth} 
        \centering
        \includegraphics[width=\textwidth]{build/Graph2a.pdf} 
        \caption{Momentangeschwindigkeit und Streuintensität in Abhängigkeit der Messtiefe.}
        \label{fig:graph2a}
        \qquad
    \end{subfigure}
    \begin{subtable}{0.43\textwidth}
        \centering
       \begin{tabular}{c c c}
        \toprule 
        {Messtiefe $\mathbin{/} \unit{\milli\meter}$} & {$v \mathbin{/} \unit{\frac{\meter}{\second}} $} & {$I \mathbin{/} 1000 \cdot \unit{\dfrac{\volt^2}{\second}}$}  \\
        \midrule 
           31.83     &      0.0     &     14.0    \\
           32.73     &     44.6     &     53.0    \\
           33.63     &     47.9     &     45.0    \\
           34.53     &     63.7     &     61.0    \\
           35.43     &     70.0     &     74.0    \\
           36.33     &     73.2     &     98.0    \\
           37.23     &     73.2     &    124.0    \\
           38.13     &     63.7     &    133.0    \\
           39.03     &     52.5     &    134.0    \\
           39.93     &     43.0     &    157.0    \\
           40.83     &     49.3     &    159.0    \\
           41.73     &     57.3     &    332.0    \\
           42.63     &     60.5     &    685.0    \\
           43.53     &     57.3     &    423.0    \\
           44.43     &     54.1     &    121.0    \\
           45.33     &     50.9     &     75.0    \\
        \bottomrule
       \end{tabular}
       \caption{Messtiefen, Momentangeschwindigkeiten $v$ und Streuintensitäten $I$ bei einer Pumpleistung von $70 \,\%$.}
        \label{tab:2a}  
    \end{subtable}
    \caption{Momentangeschwindigkeit und Streuintensität sowie passender Plot bei einer Pumpleistung von $70 \,\%$} 
\end{figure}    


\begin{figure}[H]
    \begin{subfigure}{0.57\textwidth} 
        \centering
        \includegraphics[width=\textwidth]{build/Graph2b.pdf} 
        \caption{Momentangeschwindigkeit und Streuintensität in Abhängigkeit der Messtiefe.}
        \label{fig:graph2b}
        \qquad
    \end{subfigure}
    \begin{subtable}{0.43\textwidth}
        \centering
       \begin{tabular}{c c c}
        \toprule 
        {Messtiefe $\mathbin{/} \unit{\milli\meter}$} & {$v \mathbin{/} \unit{\frac{\meter}{\second}} $} & {$I \mathbin{/} 1000 \cdot \unit{\dfrac{\volt^2}{\second}}$}  \\
        \midrule 
            31.83      &  0.0      &29.0    \\
            32.73      &  19.1     &39.0    \\
            33.63      &  19.1     &72.0    \\
            34.53      &  22.3     &94.0    \\
            35.43      &  25.5     &128.0   \\
            36.33      &  25.5     &132.0   \\
            37.23      &  25.5     &145.0   \\
            38.13      &  25.5     &136.0   \\
            39.03      &  22.3     &117.0   \\
            39.93      &  19.1     &111.0   \\
            40.83      &  22.3     &68.0    \\
            41.73      &  22.3     &127.0   \\
            42.63     &  22.3     &443.0   \\
            43.53     &  22.3     &530.0   \\
            44.43     &  22.3     &199.0   \\
            45.33     &  22.3     &54.0    \\
        \bottomrule
    \end{tabular} 
    \caption{Messtiefen, Momentangeschwindigkeiten $v$ und Streuintensitäten $I$ bei einer Pumpleistung von $45 \,\%$.}
     \label{tab:2b}  
 \end{subtable}
 \caption{Momentangeschwindigkeit und Streuintensität sowie passender Plot bei einer Pumpleistung von $45 \,\%$} 
\end{figure}     



