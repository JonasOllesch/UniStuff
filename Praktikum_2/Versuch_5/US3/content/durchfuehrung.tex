\section{Durchführung}
\label{sec:Durchführung}

Die Durchführung ist in zwei unterschiedliche
Teile getrennt. \\

Dabei wird zunächst die Frequenzverschiebung in Abhängigkeit
des Dopplerwinkels gemessen, um die Strömungsgeschwindigkeit
zu bestimmen, danach wird bei unterschiedlichen Pumpleistungen 
die Strömungsgeschwindigkeit sowie die Streuintensität gemessen.

\subsection*{Versuchsaufbau}

Der Aufbau besteht aus einem dreigeteilten Schlauch-Pumpen-System, das von einem
mit Glaskugeln versetzten Glyceringemisch durchflossen wird.
Anbei steht eine $2 \, \unit{\mega\hertz}$ Ultraschallsonde sowie einige Prismen, mithilfe
derer diverse Parameter der Flüssigkeit bestimmt werden können.


\subsection*{Frequenzverschiebung in Abhängigkeit vom Dopplerwinkel}

Für fünf verschiedene Flussgeschwindigkeiten, also fünf unterschiedliche
Pumpleistungen wird mithilfe der Ultraschallsonde am mittleren
Schlauch die Frequenzverschiebung $\Delta \nu$ für alle drei verschiedenen
Dopplerwinkel des Prismas bestimmt, um einen Ausdruck für die 
Strömungsgeschwindigkeit zu finden. \\

Dabei sei das 'Sample Volume' der Sonde auf 'Large' zu stellen

\subsection*{Bestimmung des Strömungsprofils}

Das Sample Volume der Sonde wird nun auf 'Small' gestellt.
Mithilfe des 'Depth'-Reglers wird die Messtiefe eingestellt.
Es wird in $0,75 \,\unit{\milli\meter}$-Schritten gemessen, bis der
$2,5 \,\unit{\centi\meter}$ dicke Schlauch durchdrungen ist. \\

Die Messung wird einmal für eine Pumpleistung von $70 \,\%$ und
ein weiteres Mal für eine Leistung von $45 \,\%$ durchgeführt.
