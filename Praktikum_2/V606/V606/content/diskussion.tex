\section{Diskussion}
\label{sec:Diskussion}

Die Güte des Selektivverstärkers wurde auf einen Wert von 20 eingestellt, was zusammen mit Wert \eqref{eq:guete} eine relative Abweichung von $112,77 \, \%$ ergibt.
 
Zu den experiemntellen Werten aus \autoref{tab:tab6} wird die relative Abweichung brechnet, was die \autoref{tab:tab7} ergibt.

\begin{table}[H]
    \centering
    \caption{Relative Abweichung der Suszeptibilität $\chi$ unterschiedlicher Proben zur Theorie.}
    \label{tab:tab7}
    \begin{tabular}{S S S}
      \toprule
       {Stoff} & {$\chi_{\text{R}}$} & {$\chi_{\text{U}}$}  \\
      \midrule
      {$\text{Dy}_2\text{O}_3$}  &{$10,06 \, \%$}    &{$ 1418,65 \, \%$}\\
      {$\text{Gd}_2\text{O}_3$}  &{$89,27 \, \%$}    &{$ 31,97 \, \%$}\\
      {$\text{Nd}_2\text{O}_3$}  &{$79,25 \, \%$}    &{$ 250,50 \, \%$}\\
      \bottomrule
    \end{tabular}
\end{table}

Bei der Messung der Filterkurve war es nicht möglich, die Eingangsspannung genau einzustellen. 
Die eingestellte Frequenz schwankte um bis zu $ 8 \, \unit{\kilo\hertz}$, deswegen war es auch nicht möglich Messwerte für den Bereich zwischen $ 23 \, \unit{\kilo\hertz} $ und $ 30 \, \unit{\kilo\hertz} $  aufzunehmen.
Der maximale Wert konnte aber einigermaßen genau bestimmt werden, da mit dem Frequenzgenerator aus dem zweiten Messabschnitt eine ähnliche Frequenz gemessen wurde. \\

Die Suszeptibilität unterliegt während der Messung Schwankungen, da die Suszeptibilität von der Temperatur abhängig ist. 
Durch Berührungen kann die Temperatur der Probe geändert werden. 
Zumindest kurze Berührungen konnten nicht verhindert werden, da die Proben in die Spule eingeführt werden musste, der Einfluss auf die Messgenauigkeit sollte aber vernachlässigbar gering sein. \\
Weitere Ungenauigkeiten entstehen  dadurch, dass die Brückenspannung bei jedem Abgleich nicht auf null abgesenkt werden konnte. \\

An \autoref{tab:tab7} ist zu erkennen, dass die relative Abweichung aus der Differenz der Widerstände durchschnittlich geringer ist als bei der Differenz der Spannungen, was darauf hinweist, dass diese Methode genauer ist.
Nur bei $\text{Gd}_2\text{O}_3$ ist die Abweichung bei der Spannungsmethode geringer ist als bei der Widerstandsmethode.
