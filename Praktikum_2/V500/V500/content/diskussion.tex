\section{Diskussion}
\label{sec:Diskussion}

In diesem Experiment wurde $ \frac{h}{e}$ auf $(2,645 \pm 0,355) \cdot 10^{-15} \,\si{\joule\second\coulomb^{-1}}$ bestimmt.
Der Theroriewert beträgt $4,136 \cdot 10^{-15} \,\si{\joule\second\coulomb^{-1}}$, was eine relative Abweichung von $36 \, \%$ bedeutet. Die Genauigkeit der Ausgleichsgerade \autoref{fig:gegspanngegfreq} kann dadurch erklärt werden, dass die Spektrallinien in Breit und Höhe nicht exakt fokussiert werden konnten.
Dadurch war die Intensität an der Photonenstrom geringer, was wiederum zu größeren Abweichungen führt. \\

Im gesamten Experiment wurden Ströme im $\unit{\nano\ampere}$-Bereich gemessen. Diese unterliegen Schwankungen durch störende elektrische Felder.
Eine weitere Fehlerquelle ist das Vakuum in der Photozelle, welches mit Sicherheit nicht vollständig ist. Die Verunreinigung resultiert in einem systematisch geringeren elektrischen Strom.
Andere systematische Fehler entstehen durch Verunreinigungen in den Linsen und die der Quecksilberdampflampe. Trotz allen möglichen Fehlerquellen und der Größenordnung des Theroriewerts von $\frac{h}{e}$ kann von einem erfolgreichen Experiment gesprochen werden. \\
Es konnte auch ein Dunkelstrom gemessen werden, welches aber aufgrund seiner geringen Größe nicht weiter geachtet wurde.\\