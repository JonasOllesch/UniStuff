\section{Theorie}
\label{sec:theorie}

Treffen Lichtquanten auf eine Metalloberfläche,
lösen sie, sofern ihre Energie ausreicht, Elektronen aus. \\

Um diesen, auch Photoeffekt genannten Effekt näher zu untersuchen,
kann der in \autoref{fig:abb1} dargestellte Aufbau verwendet werden.

\begin{figure}
    \centering
    \includegraphics{figures/Abb1.pdf}
    \caption{Aufbau zur Untersuchung des Photoeffekts \cite{ap10}.}
    \label{fig:abb1}
\end{figure}

Bei Durchführungen dieses Versuches ergeben sich die folgenden Erkenntnisse:

\begin{enumerate}
    \item Die Zahl der pro Zeit ausgelösten Elektronen ist proportional zur Lichtintensität.
    \item Die Energie der Photoelektronen ist proportional zur Photonenfrequenz, aber unabhängig von der Lichtintensität.
    \item Unterhalb einer Grenzfrequenz tritt der Photoeffekt nicht auf.
\end{enumerate}

Diese Erkenntnisse sind jedoch nicht ausschließlich mit einem Wellenmodell vereinbar.
Nur unter der Annahme, dass die Photonenenergie nicht gleichmäßig über die Wellenfläche verteilt,
sondern in Volumina subatomarer Größe konzentriert ist, lassen sich die Ergebnisse erklären. \\

Nach Einstein sind diese Korpuskeln, oder auch Lichtquanten identisch zu den Planckschen Energiequanten,
sie erfüllen also die folgenden Eigenschaften:

\begin{enumerate}
    \item Monochromatisches Licht, also Licht der Frequenz $\nu$ besteht aus Photonen der Lichtgeschwindigkeit $c$,
            die sich geradlinig mit der Energie $\text{h}\nu$ bewegen.
    \item Der Energieübertrag des Photons auf das Elektron ist momentan und teilt sich in die Austrittsarbeit $A_\text{k}$, 
            also die Energie, die das Elektron zum Austritt aus der Metalloberfläche benötigt, 
            und die kinetische Energie des Elektrons auf. Die Energiebilanz nimmt dann die 
            Form
            \begin{equation}
                \text{h} \nu = E_\text{kin} + A_\text{k}
                \label{eq:energiebilanz}
            \end{equation}
            an.
            Daraus folgt direkt, dass der Photoeffekt nur auftritt, wenn
            \begin{equation*}
                \text{h} \nu > A_\text{k} \,.
            \end{equation*}
    \item Die Lichtintensität ist proportional zur Photonenzahl pro Zeit- und Raumwinkeleinheit.
\end{enumerate}

