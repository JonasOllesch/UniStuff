\section{Auswertung}
\label{sec:auswertung}

Zur besseren Übersichtlichkeit ist die Auswertung im Folgenden in einige Unterabschnitte unterteilt.

\subsection{Kennlinien und Sättigungsstrom}

Die Paare aus Anodenspannung und Anodenstrom der fünf verschiedenen Kennlinien sind in \autoref{tab:kennlini1_2} und \autoref{tab:kennlini3_5} dargestellt.

\begin{table}[H]
    \centering
    \caption{Anodenstrom $I_1$ mit den dazugehörigen Anodenspannungen $U$.}
    \label{tab:kennlini1_2}
\begin{minipage}[c]{0.45\textwidth}
    \begin{tabular}{S S}
        \toprule
        & \multicolumn{1}{c}{$U_\text{H} = 3,2 \,\unit{\volt}$} \\
        \cmidrule(lr){2-2}
        {$U \mathbin{/} \unit{\volt}$} & {$I_1 \mathbin{/} \unit{\ampere}$} \\
        \midrule
        %%%%% Hier könnten Ihre Werte stehen




        %%%%%%
        \bottomrule
    \end{tabular}
\end{minipage}
\begin{minipage}[c]{0.45\textwidth}
    \begin{tabular}{S S}
        \toprule
        & \multicolumn{1}{c}{$U_\text{H} = 3,5 \,\unit{\volt}$} \\
        \cmidrule(lr){2-2}
        {$U \mathbin{/} \unit{\volt}$} & {$I_2 \mathbin{/} \unit{\ampere}$} \\
        \midrule
        %%%%% Hier könnten Ihre Werte stehen




        %%%%%%
        \bottomrule
    \end{tabular}
\end{minipage}
\end{table}
    


\begin{table}[H]
    \centering
    \caption{Anodenströme $I_i$ mit den dazugehörigen Anodenspannungen $U$ für die dritte bis fünfte Heizspannung.}
        \label{tab:kennlini3_5}
    \begin{minipage}{0.45\textwidth}
        \sisetup{table-format=2.1}
        \begin{tabular}{S S S}
            \toprule
            & \multicolumn{1}{c}{$U_\text{H} = 4 \,\unit{\volt}$} & \multicolumn{1}{c}{$U_\text{H} = 4,2 \,\unit{\volt}$} \\
            \cmidrule(lr){2-2} \cmidrule(lr){3-3}
            {$U \mathbin{/} \unit{\volt}$} & {$I_3 \mathbin{/} \unit{\ampere}$} & {$I_4 \mathbin{/} \unit{\ampere}$} \\
            \midrule
            %%%%% Hier könnten Ihre Werte stehen
    
    
    
            %%%%%
            \bottomrule
        \end{tabular}
    \end{minipage}
    \begin{minipage}{0.45\textwidth}
    \sisetup{table-format=2.1}
    \begin{tabular}{S S}
        \toprule
        & \multicolumn{1}{c}{$U_\text{H} = 5,4 \,\unit{\volt}$} \\
        \cmidrule(lr){2-2}
        {$U \mathbin{/} \unit{\volt}$} & {$I_5 \mathbin{/} \unit{\ampere}$} \\
        \midrule
        %%%%% Hier könnten Ihre Werte stehen




        %%%%%%
        \bottomrule
    \end{tabular}
    \end{minipage}
\end{table}

Die dazugehörigen Plots sind in \autoref{fig:kennlini} zu erkennen.

%%%%%%%%%%%%%%%%%%%%%%%%%\begin{figure}[H]
%%%%%%%%%%%%%%%%%%%%%%%%%    \centering
%%%%%%%%%%%%%%%%%%%%%%%%%    \includegraphics{build/kennlinien.pdf}
%%%%%%%%%%%%%%%%%%%%%%%%%    \caption{Anodenstrom $I_\text{A}$ in Abhängigkeit der Anodenspannung $U_\text{A}$ bei unterschiedlichen Heizspannungen.}
%%%%%%%%%%%%%%%%%%%%%%%%%    \label{fig:kennlini}
%%%%%%%%%%%%%%%%%%%%%%%%%\end{figure}

Die unterschiedlichen Sättigungsströme lassen sich relativ genau aus den aufgenommenen Messdaten erkennen.
Sie sind in \autoref{tab:sättigungsstromis} dargestellt.

\begin{table}
    \centering
    \caption{Sättigungsstrom $I_\text{S}$ zu unterschiedlichen \\ Heizspannungen $U_\text{H}$ und -strömen $I_\text{H}$.}
    \label{tab:sättigungsstromis}
    \begin{tabular}{S S S}
        \toprule
        {$U_\text{H} \mathbin{/} \unit{\volt}$} & {$I_\text{H} \mathbin{/} \unit{\ampere}$} & {$I_\text{S} \mathbin{/} \unit{\ampere}$} \\
        \midrule
        %%%%% Hier könnten Ihre Werte stehen




        %%%%%%
        \bottomrule
    \end{tabular}
\end{table}


\subsection{Untersuchung des Raumladungsgebiets}

Anhand der größtmöglichen Heizspannung soll nun das Raumladungsgebiet untersucht werden.
Hier wird also die fünfte Messreihe genauer betrachtet.
Dazu wird eine lineare Regression der Form
\begin{equation*}
    y = m x + b
\end{equation*}
durchgeführt, wobei Strom und Spannung logarithmisch gegeneinander aufgetragen werden. \\

Diese Regression mit den Parametern
\begin{equation*}
    m = (... \pm ...)
\end{equation*}
und
\begin{equation*}
    b = (... \pm ...)
\end{equation*}
ist in \autoref{fig:linregkennl5} aufgetragen.

%%%%%%%%%%%%%%%%%%%%%%%\begin{figure}[H]
%%%%%%%%%%%%%%%%%%%%%%%    \centering
%%%%%%%%%%%%%%%%%%%%%%%    \includegraphics{build/linregraumladung.pdf}
%%%%%%%%%%%%%%%%%%%%%%%    \caption{Lineare Regression an den logarithmisch geplotteten Werten aus Messreihe 5. Die Vertikale gibt den Grenzbereich des Raumladungsgebietes an.}
%%%%%%%%%%%%%%%%%%%%%%%    \label{fig:linregkennl5}
%%%%%%%%%%%%%%%%%%%%%%%\end{figure}


\subsubsection{Untersuchung des Anlaufstromgebiets}

Die zum Anlaufstrom aufgenommenen Messdaten sind in \autoref{tab:anlaufstrom} dargestellt.
Dabei wird die korrigierte Spannung $U_\text{korr}$ über
\begin{equation*}
    U_\text{korr} = U_\text{gemessen} - R_i \, I
\end{equation*} 
bestimmt, wobei $I$ der gemessene Strom und $R_i$ der Innenwiderstand von $1 \,\unit{\mega\ohm}$ sind.

\begin{table}
    \centering
    \caption{Gemessene Spannung $U_\text{gemessen}$, gemessener Strom $I$ sowie korrigierte Spannung $U_\text{korr}$.}
    \label{tab:anlaufstrom}
    \begin{tabular}{S S S}
        \toprule
        {$U_\text{gemessen} \mathbin{/} \unit{\volt}$} & {$I \mathbin{/} \unit{\ampere}$} & {$U_\text{korr} \mathbin{/} \unit{\volt}$} \\
        \midrule
        %%%%% Hier könnten Ihre Werte stehen




        %%%%%%
        \bottomrule
    \end{tabular}
\end{table}

Nun soll ein Ausdruck für die Kathodentemperatur gefunden werden.
Dazu wird \eqref{eq:stromladexp} nach T umgestellt, sodass sich eine Regressionsgleichung der Form
\begin{equation*}
    \ln(I) = -\dfrac{\text{e}_0}{\text{k}_\text{B} T} V + \ln(c) = m x + b
\end{equation*}
ergibt, dessen grafische Darstellung in \autoref{fig:anlaufstrom} zu erkennen ist.

%%%%%%%%%\begin{figure}
%%%%%%%%%    \centering
%%%%%%%%%    \includegraphics{build/reganlaufstrom.pdf}
%%%%%%%%%    \caption{Logarithmische Darstellung der zum Anlaufstromgebiet aufgenommenen Messdaten samt linearer Regression.}
%%%%%%%%%    \label{fig:anlaufstrom}
%%%%%%%%%\end{figure}

Mit
\begin{equation*}
    m = (... \pm ...)
\end{equation*}
und
\begin{equation*}
    b = (... \pm ...)
\end{equation*}
lässt sich über
\begin{equation*}
    m = -\dfrac{\text{e}_0}{\text{k}_\text{B} T} \Leftrightarrow T = - \dfrac{\text{e}_0}{\text{k}_\text{B} m}
\end{equation*}
die Kathodentemperatur zu
\begin{equation*}
    T = (... \pm ...) \,\unit{\kelvin}
\end{equation*}
bestimmen. 


\subsection{Kathodentemperatur aus Leistungsbilanz}

Mit den in \autoref{tab:sättigungsstromis} verwendeten Heizspannungen und -strömen kann nach
\begin{equation*}
    T = \left(\dfrac{U_\text{H} I_\text{H} - N_\text{WL}}{\eta \sigma f}\right)^\frac{1}{4}
\end{equation*}
die Kathodentemperatur bestimmt werden, wobei $N_\text{WL} = 0,9 \,\unit{\watt}$ die Wärmeleitung der Apparatur, $\eta = 0,28$ der Emissionsgrad der Oberfläche,
$\sigma = 5,7 \cdot 10^{-12} \,\unit{\watt \,\centi\meter^{-2} \,\kelvin^{-4}}$ die Boltzmannsche Strahlungskonstante und $f = 0,32 \,\unit{\centi\meter^2}$ die emittierende Kathodenoberfläche darstellen \cite{ap09}. \\

So ergeben sich die in \autoref{tab:kathtempheizstrom} dargestellten Werte.

\begin{table}
    \centering
    \caption{Kathodentemperatur $T$ zu unterschiedlichen Heizströmen.}
    \label{tab:kathtempheizstrom}
    \begin{tabular}{S S S S S}
        \toprule
        {$T_1 \mathbin{/} \unit{\kelvin}$} & {$T_2 \mathbin{/} \unit{\kelvin}$} & {$T_3 \mathbin{/} \unit{\kelvin}$} & {$T_4 \mathbin{/} \unit{\kelvin}$} & {$T_5 \mathbin{/} \unit{\kelvin}$} \\
        \midrule
        %%%%% Hier könnten Ihre Werte stehen




        %%%%%%
        \bottomrule
    \end{tabular}
\end{table}

Gemittelt ergibt sich so eine Kathodentemperatur von
\begin{equation*}
    T = (... \pm ...) \,\unit{\kelvin} \,.
\end{equation*}


\subsection{Bestimmung der Austrittsarbeit}

Mithilfe von \eqref{eq:stromdichte} lässt sich nach Umstellen mit $j_\text{S} = \frac{I_\text{S}}{f}$ ein Ausdruck für die Austrittsarbeit $\text{e}_0 \,\Phi$ finden.
Dieser lautet
\begin{equation*}
    \text{e}_0 \,\Phi = -\text{k}_\text{B} T \,\ln \left(\dfrac{I_\text{S} \, \text{h}^3}{4 \, \pi \, \text{e}_0 \, \text{k}^2_\text{B} \, \text{m}_0 \,f \,  T^2}\right) \,.
\end{equation*} \\

Die zu den bereits in \autoref{tab:sättigungsstromis} bestimmten Sättigungsströmen gehörigen Austrittsarbeiten sind in \autoref{tab:austrittsarbeit} dargestellt.

\begin{table}
    \centering
    \caption{Austrittsarbeiten $\text{e}_0 \Phi$ zu unterschiedlichen Sättigungsströmen.}
    \label{tab:austrittsarbeit}
    \begin{tabular}{S S S S S}
        \toprule
        {$\text{e}_0\Phi_1 \mathbin{/} \unit{\eV}$} & {$\text{e}_0\Phi_2 \mathbin{/} \unit{\eV}$} & {$\text{e}_0\Phi_3 \mathbin{/} \unit{\eV}$} & {$\text{e}_0\Phi_4 \mathbin{/} \unit{\eV}$} & {$\text{e}_0\Phi_5 \mathbin{/} \unit{\eV}$}  \\
        \midrule
        %%%%% Hier könnten Ihre Werte stehen




        %%%%%%
        \bottomrule
    \end{tabular}
\end{table}

Gemittelt ergibt sich hier

\begin{equation*}
    \text{e}_0 \Phi = (... \pm ...) \,\unit{\eV} \,.
\end{equation*}