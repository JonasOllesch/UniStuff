\section{Durchführung}
\label{sec:Durchführung}

\subsection*{Teilchenabhängigkeit von der Spannung}

Vor das Fenster des Zählrohres wird eine $\beta$-Quelle gestellt.
In Abhängigkeit von der Spannung $U$ werden beginnend bei $350 \,\unit{\volt}$ in Inkrementen von $10 \,\unit{\volt}$
über Zeitintervalle von zwei Minuten die einfallenden Teilchen $N$ gemessen.

\subsection*{Nachentladungen}

Um die auftretenden Nachentladungen sichtbar zu machen, wird das Oszilloskop auf ein Fenster von $50 \,\unit{\micro\second}$
gestellt, sodass nur noch ein einziger Impuls sichtbar ist.
Es wird eine niedrige sowie eine höhere Spannung eingestellt und anschließend die auf dem Oszilloskop sichtbaren Ereignisse verglichen.

\subsection*{Messung der Totzeit über die Zwei-Quellen-Methode}

Mithilfe einer weiteren Probe werden bei einer geeigneten Spannung die Teilchenzahlen $N$ für die erste, die zweite, sowie
beider Proben zusammen gemessen und anschließend die summierten Teilchenzahlen verglichen.

\subsection*{Freigesetzte Ladung pro Teilchen}

Zur Bestimmung der freigesetzten Ladung pro Teilchen wird, parallel zur ersten Messreihe, der Strom I gemessen.
