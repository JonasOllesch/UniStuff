\section{Diskussion}
\label{sec:Diskussion}

Bei der Interpretation der Messergebnisse müssen ein paar Fehlerquellen diskutiert werden.

Bei diesem Experiment wurde der Plateau-Bereich des Zählrohres untersucht. Die Messreihe ging dabei von $350 \,\unit{\volt}$ bis $700 \,\unit{\volt}$, dabei kann das obere Ende des Plateaus bei ungefähr $600 \, \unit{\volt}$ identifiziert werden.
Im Gegensatz dazu ist das untere Ende nicht so gut ablesbar. Eine Erweiterung der Messdaten in einem Bereich von unter $350 \, \unit{\volt}$ würde helfen, das untere Ende des Plateaus besser eingrenzen zu können.

Bei einem gebräuchlichen Geiger-Müller-Zählrohr liegt die Totzeit zwischen $10^{-4} \, \unit{\second}$ und $10^{-5} \, \unit{\second}$ \cite{ap01}.
Die genäherte Totzeit von $ 125 \,\unit{\micro\second}$ liegt in der nähe diese Bereichs und ist deswegen auch als realistisch anzusehen.
Die relative Abweichung zur oberen Grenze beträgt $25 \, \% $

Die über die Zwei-Quellen-Methode bestimmte Totzeit von $ \left( 86,5 \pm 4 \right) \,\unit{\micro\second}$ liegt genau im Bereich eines herkömmlichen Geiger-Müller-Zählrohres.
Zwischen den beiden gemessenen Totzeiten liegt ein Unterschied von $ 69,25 \% .$

Der Plot \autoref{fig:messung3g} zeigt eine größere freigesetzte Ladung $\Delta Q $ bei einer größeren Spannung $ \unit{\volt}$. Die Ausgleichsfunktion zeigt einen linearen Zusammenhang zwischen den beiden Größen.
Der Anstieg der freigesetzten Ladung kann durch das stärkere elektrische Feld zwischen Kathode und Anode im Geiger-Müleler-Zählrohr erklärt werden, welches die ionisierten Teilchen stärker beschleunigt werden und dem häufigeren auftreten von Nachentladungen.