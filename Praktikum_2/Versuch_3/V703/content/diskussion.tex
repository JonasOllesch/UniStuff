\section{Diskussion}
\label{sec:Diskussion}

Bei der Interpretation der Messergebnisse müssen ein paar Fehlerquellen diskutiert werden.
Ein vorhandenes defektes Kabel, das zu Beginn des Versuches zu groben Ungenauigkeiten führte,
wurde identifiziert und noch vor dem Aufnehmen von Messwerten gewechselt, kann als Fehlerquelle also
weitesgehend ausgeschlossen werden.
Störsignale, die von radioaktiven Teilchen im Erdmantel, oder aus der Atmosphäre stammen, sowie Strahlung von anderen radioaktiven Präparaten in der Nähe und auch komische Strahlung, 
wurden bei der Auswertung aufgrund ihrer geringen Relevanz nicht beachtet.

Bei diesem Experiment wurde der Plateau-Bereich des Zählrohres untersucht. Die Messreihe ging dabei von $350 \,\unit{\volt}$ bis $700 \,\unit{\volt}$, dabei kann das obere Ende des Plateaus bei ungefähr $600 \, \unit{\volt}$ identifiziert werden.
Im Gegensatz dazu ist das untere Ende nicht so gut ablesbar. Eine Erweiterung der Messdaten in einem Bereich von unter $350 \, \unit{\volt}$ würde helfen, das untere Ende des Plateaus besser eingrenzen zu können.

Bei einem gebräuchlichen Geiger-Müller-Zählrohr liegt die Totzeit zwischen $10^{-4} \, \unit{\second}$ und $10^{-5} \, \unit{\second}$ \cite{ap01}.
Die genäherte Totzeit von $ 125 \,\unit{\micro\second}$ liegt in diesem Bereich und ist deswegen auch als realistisch anzusehen,
die über die Zwei-Quellen-Methode bestimmte Totzeit... %% Hier müsste dann noch mit der RICHTIGEN berechneten Totzeit verglichen werden...


%% Freigesetzte Ladung..., Nachentladungen...