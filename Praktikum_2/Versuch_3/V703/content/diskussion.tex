\section{Diskussion}
\label{sec:Diskussion}

Bei der Interpretation der Messergebnisse müssen ein paar Fehlerquellen diskutiert werden.
Zuerst ist ein defektes Kabel zu nennen, dass vor dem Beginn der ersten Messreihe ausgewechselt werden muss. Aufgrund dieses Defektes wurden entweder deutlich mehr Teilchen als theoretisch möglich gewesen wäre oder kein Teilchen im Geiger-Müller-Zählrohr gemessen.
Nachdem Austausch des Kabels wurden keine weiteren defekten Bauteile festgestellt, aber kleinere defekte Messgeräte können nicht ausgeschlossen werden.
Störsignale, die von radioaktiven Teilchen im Erdmantel, oder aus der Atmosphäre stammen, sowie Strahlung von anderen radioaktiven Präparaten in der Nähe und auch komische Strahlung, werden nicht beachtet.

Bei diesem Experiment wurde der Plateau-Bereich des Zählrohres untersucht. Die Messreihe ging dabei von $350 \,\unit{\volt}$ bis $700 \,\unit{\volt}$ , dabei kann das obere Ende des Plateaus bei ungefähr $600 \, \unit{\volt}$ identifiziert werden.
Im Gegensatz dazu ist das untere Ende nicht so gut ablesbar. Eine Erweiterung der Messdaten in einem Bereich von unter $350 \, \unit{\volt}$ würde helfen, das untere Ende des Plateaus besser eingrenzen zu können.

Bei einem gebräuchlichen Geiger-Müller-Zählrohr liegt die Totzeit zwischen $10^{-4} \, \unit{\second}$ und $10^{-5} \, \unit{\second}$ \cite{ap01}. Die genäherte Totzeit von $ 75 \,\unit{\micro\second}$ liegt in diesem Bereich und ist deswegen auch als realistisch anzusehen.