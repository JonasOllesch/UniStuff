\section{Diskussion}
\label{sec:Diskussion}
\paragraph*{Probleme des Versuchsausbaus}
Es ist zu beachten das es bei dem Versuch zu Schwierigkeiten kam.
Die Messung für $U=\qty{175}{\volt}$ konnte durchgefürt werden. Doch nachdem die Spannung erhöht wurde konnten keine weiteren Tröpfchen beobachtet werden.
Auch nach dem reinigen der Kammer und austauschen der Lampe konnte das Experiment nicht fortgesetzt werden. Die Daten für alle anderen Spannungen stammen deshalb aus einer vorherrigen Durchführung.
Dennoch können die Werte als korrekt angesehen werden und die Rechnung sind nicht beeinflusst.
Das tatsächliche Problem am Versuchausaufbau ist nicht klar geworden aber es kommt bei diesem Versuch oft zu Problemen.
\paragraph*{Messwerte und Genauigkeit}
Die Messwerte unterliegen vorallem der Reaktionsgeschwindigkeit von Menschen.
Eine Person beobachtet das Teilchen und eine andere stoppt die Zeit. Dies führt zu Abweichungen der Daten.
Dadurch lässt sich auch die relative Abweichung der bestimmten Konstanten von ihren Literaturwert erklären.
Der Literaturwert für $e_\text{0,lit}$ ist gegeben mit $\qty{1.602176e-19}{\coulomb}$ \cite[575]{Metzler}.
Das bedeutet eine Abweichung von $\qty{24(32)}{\percent}$. Für die Avogadro-Konstante mit $N_\text{A,lit}=\qty{6.022141e23}{\per\mole}$\cite[575]{Metzler} ergibt sich so eine Abweichung von $\qty{19(21)}{\percent}$.
Auch wenn es zu relativ grossen Abweichungen kommt liegt der Literarturert im Sicherheitsintervall.
Wenn der Versuch über noch mehr Teilchen und Durchfläufe durchgeführt wird kann die Genaugigkeit deutlich verbessert werden.
Zudem gibt es noch weitere Störfaktoren die dazu führen das die Ergebnisse schlechter werden.
Dazu zählt zum Beispiel das Problem, dass die Kammer nicht perfekt von außen abgeschirmt ist und die Tröpfchen so von Druck unterschieden beeinflusst werden.
Es kommt auch zu Abweichungen in der Bestimmung der Viskosität und der Temperatur da diese nicht durch eine Funktion bestimmt werden können sondern aus der Tabelle und dem Graphen abgelesen werden müssen.
Trotz der Schwierigkeiten in der Durchführung ist der Versuch durchaus gut geeignet um die Konstanten $e$ und $N_\text{A}$ zu bestimmen. Vorallem wenn noch mehr Messungen durchgeführt werden.




