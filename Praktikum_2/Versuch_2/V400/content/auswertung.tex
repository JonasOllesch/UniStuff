\section{Auswertung}
\label{sec:auswertung}

\subsection{Das Reflexionsgesetz}

In \autoref{tab:messung1} sind für verschiedene Einfallswinkel $\alpha_1$ die Ausfallswinkel $\alpha_2$ dargestellt.
Die Winkel konnten dabei aufgrund der Ausdehnung der Laserstrahlen sowie der Genauigkeit der Winkelskala auf ungefähr $1 \,°$ mit zufriedenstellender Genauigkeit abgelesen werden.

\begin{table}[H]
    \centering
    \caption{Einfallswinkel $\alpha_1$ und Ausfallswinkel $\alpha_2$.}
    \label{tab:messung1}
    \begin{tabular}{S[table-format=2.0] S[table-format=2.0]}
      \toprule
        {$\alpha_1$ $\mathbin{/} \, \unit{\degree}$} & {$\alpha_2$ $\mathbin{/} \, \unit{\degree}$}\\
      \midrule
         0              &               0 \\
        20              &              22 \\
        30              &              32 \\
        40              &              42 \\
        50              &              52 \\
        60              &              61 \\
        70              &              72 \\
        80              &              81 \\
        85              &              86 \\
    \bottomrule
    \end{tabular}
\end{table}

\newpage
\subsection{Brechungsgesetz}
\label{subsec:brechungsgesetz}

In \autoref{tab:messung2} sind für verschiedene Einfallswinkel $\alpha$ die Brechungswinkel $\beta$ aufgetragen.

\begin{table}[H]
    \centering
    \caption{Einfallswinkel $\alpha$ und Brechungswinkel $\beta$.}
    \label{tab:messung2}
    \begin{tabular}{S[table-format=2.0] S[table-format=2.1]}
      \toprule
        {$\alpha$ $\mathbin{/} \, \unit{\degree}$} & {$\beta$ $\mathbin{/} \, \unit{\degree}$}\\
      \midrule
       0                  &                0   \\
      10                  &                7.5 \\
      20                  &                14  \\
      30                  &                20  \\
      40                  &                26  \\
      50                  &                33.5\\
      60                  &                36  \\
      70                  &                39.5\\
    \bottomrule
    \end{tabular}
\end{table}

Anhand dieser Messdaten kann nun mithilfe des Snelliusschen Brechungsgesetzes \eqref{eq:snellius} der Brechungsindex $n$ der planparallelen Plexiglasplatte bestimmt werden.
Da der Laserstrahl aus der Luft in das optisch dichtere Medium der Platte eindringt, seien $n_1 = 1$ sowie $n_2 = n$. \\

Der Brechungsindex ergibt sich also über

\begin{equation*}
    n = \frac{\sin(\alpha)}{\sin(\beta)} \,.
\end{equation*} \\

Im Mittel ergibt sich daraus ein Brechungsindex von

\begin{equation*}
    n = 1,43 \pm 0,05 \,.
\end{equation*} \\

Gesucht ist außerdem die Lichtgeschwindigkeit $v$ im Medium der planparallelen Platte, die sich ebenfalls über das Brechungsgesetz bestimmen lässt.
Dabei gilt
\begin{equation*}
    \frac{v_1}{v_2} = \frac{c}{v} = \frac{\sin(\alpha)}{\sin(\beta)} \,.
\end{equation*} \\

Im Mittel ergibt sich damit

\begin{equation*}
    v = (2,10 \pm 0,08) \cdot 10^8 \,\unit{\frac{\meter}{\second}} \,.
\end{equation*} 


\subsection{Strahlversatz}

Mithilfe der in \autoref{subsec:brechungsgesetz} aufgenommen Messdaten wird nun der Strahlversatz $s$ über 

\begin{equation}
    s = d \,\frac{\sin(\alpha - \beta)}{\cos(\beta)}
    \label{eq:strahlversatz}
\end{equation}
bestimmt.

Zusätzlich dazu wird der Strahlversatz außerdem über das Brechungsgesetz berechnet.
Unter Verwendung des in \autoref{subsec:brechungsgesetz} ermittelten Brechungsindexes $n = 1,43$ wird der Brechungswinkel $\beta$ berechnet und anschließend in \eqref{eq:strahlversatz} eingesetzt.

Die experimentell sowie rechnerisch ermittelten Strahlversätze sind in \autoref{tab:messung3} dargestellt.

\begin{table}[H]
    \centering
    \caption{Einfallswinkel $\alpha$, experimentell ermittelte Strahlversätze $s_{\text{e}}$ sowie rechnerische Strahlversätze $s_{\text{t}}$.}
    \label{tab:messung3}
    \begin{tabular}{S[table-format=2.0] S[table-format=1.2] S[table-format=1.2]}
      \toprule
        {$\alpha \mathbin{/} \, \unit{\degree}$} & {$ s_{\text{e}} \mathbin{/} \, \unit{\centi\meter}$} & {$ s_{\text{t}} \mathbin{/} \, \unit{\centi\meter}$} \\
      \midrule
       0          &       0.00     &   0.00  \\
      10          &       0.26     &   0.31  \\
      20          &       0.63     &   0.65  \\
      30          &       1.08     &   1.03  \\
      40          &       1.57     &   1.51  \\
      50          &       1.99     &   2.10  \\
      60          &       2.94     &   2.84  \\
      70          &       3.85     &   3.75  \\
    \bottomrule
    \end{tabular}
\end{table}

\subsection{Prisma}
\label{subsec:Prisma}

\begin{table}[H]
    \centering
    \caption{Einfallswinkel, Austrittswinkel und Ablenkung von einem grünen Lasern.}
    \label{tab:messung4a}
    \begin{tabular}{S S S}
      \toprule
        {$\alpha_{1} \mathbin{/} \, \unit{\degree}$} & {$\alpha_{2} \mathbin{/} \, \unit{\degree}$}  & {$\delta_{2} \mathbin{/} \, \unit{\degree}$} \\
      \midrule
          28  &   88  & 58.43\\
          30  &   78  & 50.26\\
          35  &   65  & 42.70\\
          40  &   57  & 39.94\\
          45  &   52  & 39.50\\
          50  &   46  & 38.93\\
    \bottomrule
    \end{tabular}
\end{table}

\begin{table}[H]
  \centering
  \caption{Einfallswinkel, Austrittswinkel und Ablenkung von einem roten Lasern.}
  \label{tab:messung4b}
  \begin{tabular}{S S S}
    \toprule
      {$\alpha_{r} \mathbin{/} \, \unit{\degree}$} & {$\alpha_{r} \mathbin{/} \, \unit{\degree}$} & {$\delta_{r} \mathbin{/} \, \unit{\degree}$}\\
    \midrule
        28  & 83  &  53.762\\
        30  & 77  &  49.443\\
        35  & 64  &  42.045\\
        40  & 56  &  39.365\\
        45  & 51  &  38.967\\
        50  & 45  &  38.438\\
  \bottomrule
  \end{tabular}
\end{table}

\autoref{tab:messung4a} und \autoref{tab:messung4b} zeigen die Einfallswinkel und Ausfallswinkel des roten und grünen Lasers.
Die Ablenkung $\delta$ wird dabei über
\begin{equation}
  \delta = \left( \alpha_{1} + \alpha_{2} \right)- \left( \beta_{1} + \beta_{2} \right) \, .
  \label{eq:ablenkung}
\end{equation}
berechnet.
Für den Brechungsindex von Kronglas wird der Wert $ n = 1.555$ \cite{ap02} verwendet.

\subsection{Beugung am Gitter}
\label{subsec:Gitter}
Mit Hilfe der Formel \eqref{eq:beugmaxgitter} kann aus den Messdaten die Wellenlänge des Laserlichts berechnet werden, welches auf das Gitter fällt. 
Die Werte befinden sich in \autoref{tab:messung5a_grun}, \autoref{tab:messung5a_rot}, \autoref{tab:messung5b_grun}, \autoref{tab:messung5b_rot}, \autoref{tab:messung5c_grun} und \autoref{tab:messung5c_rot}.
\begin{table}[H]
  \centering
  \caption{Beugung eines grünen Lasers an einem Beugungsgitter mit 600 Linien $\mathbin{/} \unit{\milli\meter}$.}
  \label{tab:messung5a_grun}
  \begin{tabular}{S S S}
    \toprule
      {$  \text{k} $} & {$\varphi \mathbin{/} \unit{\degree} $}  & {$ \lambda \mathbin{/} \unit{\nano\meter}$} \\
    \midrule
        1   & -19 & 542\\
        -1  & 19  & 542\\
  \bottomrule
  \end{tabular}
\end{table}

\begin{table}[H]
  \centering
  \caption{Beugung eines roten Lasers an einem Beugungsgitter mit 600 Linien $\mathbin{/} \unit{\milli\meter}$.}
  \label{tab:messung5a_rot}
  \begin{tabular}{S S S}
    \toprule
      {$  \text{k} $} & {$\varphi \mathbin{/} \unit{\degree} $}  & {$ \lambda \mathbin{/} \unit{\nano\meter}$} \\
    \midrule
        1   & -23 & 651\\
        -1  & 23  & 651\\
  \bottomrule
  \end{tabular}
\end{table}

Es folgen die Wellenlängen für ein 300 Linien $\mathbin{/} \unit{\milli\meter}$ Gitter daraus ergibt sich $ \lambda = 524 \pm 3.4 \, \unit{\nano\meter}$ für den grünen Laser und $ \lambda = 633 \pm 6 \, \unit{\nano\meter}$ für den roten Laser.
\begin{table}[H]
  \centering
  \caption{Beugung eines grünen Lasers an einem Beugungsgitter mit 300 Linien $\mathbin{/} \unit{\milli\meter}$.}
  \label{tab:messung5b_grun}
  \begin{tabular}{S S S}
    \toprule
      {$  \text{k} $} & {$\varphi \mathbin{/} \unit{\degree} $}  & {$ \lambda \mathbin{/} \unit{\nano\meter}$} \\
    \midrule
        -3  & -28   & 521 \\
        -2  & -18.5 & 528 \\
        -1  & -9    & 521 \\
        0   &  0    &  0  \\
        1   & 9     & 521 \\ 
        2   & 18.5  & 528 \\
        3   & 28    & 521 \\  
  \bottomrule
  \end{tabular}
\end{table}

\begin{table}[H]
  \centering
  \caption{Beugung eines roten Lasers an einem Beugungsgitter mit 300 Linien $\mathbin{/} \unit{\milli\meter}$.}
  \label{tab:messung5b_rot}
  \begin{tabular}{S S S}
    \toprule
      {$  \text{k} $} & {$\varphi \mathbin{/} \unit{\degree} $}  & {$ \lambda \mathbin{/} \unit{\nano\meter}$} \\
    \midrule
        -3  & -35 & 637 \\
        -2  & -22 & 624 \\
        -1  &  -11  & 636 \\
        0   &  0   & 0   \\
        1   & 11   & 636 \\ 
        2   & 22  & 624 \\
        3   & 35  & 637 \\  
  \bottomrule
  \end{tabular}
\end{table}


Die Mittelwerte für die Wellenlänge für das 100 Linien $\mathbin{/} \unit{\milli\meter}$ sind $ \lambda = 536 \pm 18 \, \unit{\nano\meter}$ für den grünen Laser und $ \lambda = 653 \pm 19 \, \unit{\nano\meter}$ für den roten Laser.
\begin{table}[H]
  \centering
  \caption{Beugung eines grünen Lasers an einem Beugungsgitter mit  100 Linien $\mathbin{/} \unit{\milli\meter}$.}
  \label{tab:messung5c_grun}
  \begin{tabular}{S S S S S S}
    \toprule
      {$  \text{k} $} & {$\varphi \mathbin{/} \unit{\degree} $}  & {$ \lambda \mathbin{/} \unit{\nano\meter}$} & {$  \text{k} $} & {$\varphi \mathbin{/} \unit{\degree} $}  & {$ \lambda \mathbin{/} \unit{\nano\meter}$} \\
    \midrule
    -10& -37      &601   &   0 &   0       &0  \\
    -9 & -29.5    &547   &  1  & 3        &523 \\
    -8 & -25      &528   &  2  & 6        &522 \\
    -7 & -22      &535   &  3  & 9        &521 \\
    -6 & -18.5    &528   &  4  & 12.5     &541 \\
    -5 & -15.5    &534   &  5  & 15.5     &534 \\
    -4 & -12.5    &541   &  6  & 19       &542 \\
    -3 & -9       &521   &  7  & 22.5     &546 \\
    -2 & -6       &522   &  8  & 26       &547 \\
    -1 & -3       &523   &  9  & 28       &521 \\   
  \bottomrule
  \end{tabular}
\end{table}


\begin{table}[H]
  \centering
  \caption{Beugung eines roten Lasers an einem Beugungsgitter mit 100 Linien $\mathbin{/} \unit{\milli\meter}$.}
  \label{tab:messung5c_rot}
  \begin{tabular}{S S S S S S}
    \toprule
      {$  \text{k} $} & {$\varphi \mathbin{/} \unit{\degree} $}  & {$ \lambda \mathbin{/} \unit{\nano\meter}$} & {$  \text{k} $} & {$\varphi \mathbin{/} \unit{\degree} $}  & {$ \lambda \mathbin{/} \unit{\nano\meter}$} \\
    \midrule
        -8  &      -31    & 643     &0 &       0     & 0   \\
        -7  &      -26.5  & 637     &1 &      4      & 697 \\ 
        -6  &      -22.5  & 637     &2 &      7.5    & 652 \\
        -5  &      -19    & 651     &3 &      11     & 636 \\
        -4  &      -15    & 647     &4 &      15     & 647 \\
        -3  &      -11    &636      &5 &      19     & 651 \\
        -2  &      -7.5   & 652     &6 &      23     &651  \\
        -1  &      -4     &697      &\dots&   \dots  & \dots   \\  
  \bottomrule
  \end{tabular}
\end{table}