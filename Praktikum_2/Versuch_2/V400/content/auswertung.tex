\section{Auswertung}
\label{sec:auswertung}

\subsection{Das Reflexionsgesetz}

In \autoref{tab:messung1} sind für verschiedene Einfallswinkel $\alpha_1$ die Ausfallswinkel $\alpha_2$ dargestellt.
Die Winkel konnten dabei aufgrund der Ausdehnung der Laserstrahlen sowie der Genauigkeit der Winkelskala auf ungefähr $1 \,°$ mit zufriedenstellender Genauigkeit abgelesen werden.

\begin{table}[H]
    \centering
    \caption{Einfallswinkel $\alpha_1$ und Ausfallswinkel $\alpha_2$.}
    \label{tab:messung1}
    \begin{tabular}{S[table-format=2.0] S[table-format=2.0]}
      \toprule
        {$\alpha_1$ $\mathbin{/} \, \unit{\degree}$} & {$\alpha_2$ $\mathbin{/} \, \unit{\degree}$}\\
      \midrule
         0              &               0 \\
        20              &              22 \\
        30              &              32 \\
        40              &              42 \\
        50              &              52 \\
        60              &              61 \\
        70              &              72 \\
        80              &              81 \\
        85              &              86 \\
    \bottomrule
    \end{tabular}
\end{table}


\subsection{Brechungsgesetz}
\label{subsec:brechungsgesetz}

In \autoref{tab:messung2} sind für verschiedene Einfallswinkel $\alpha$ die Brechungswinkel $\beta$ aufgetragen.

\begin{table}[H]
    \centering
    \caption{Einfallswinkel $\alpha$ und Brechungswinkel $\beta$.}
    \label{tab:messung2}
    \begin{tabular}{S[table-format=2.0] S[table-format=2.1]}
      \toprule
        {$\alpha$ $\mathbin{/} \, \unit{\degree}$} & {$\beta$ $\mathbin{/} \, \unit{\degree}$}\\
      \midrule
       0                  &                0   \\
      10                  &                7.5 \\
      20                  &                14  \\
      30                  &                20  \\
      40                  &                26  \\
      50                  &                33.5\\
      60                  &                36  \\
      70                  &                39.5\\
    \bottomrule
    \end{tabular}
\end{table}

Anhand dieser Messdaten kann nun mithilfe des Snelliusschen Brechungsgesetzes \eqref{eq:snellius} der Brechungsindex $n$ der planparallelen Plexiglasplatte bestimmt werden.
Da der Laserstrahl aus der Luft in das optisch dichtere Medium der Platte eindringt, seien $n_1 = 1$ sowie $n_2 = n$. \\

Der Brechungsindex ergibt sich also über

\begin{equation*}
    n = \frac{\sin(\alpha)}{\sin(\beta)} \,.
\end{equation*} \\

Im Mittel ergibt sich daraus ein Brechungsindex von

\begin{equation*}
    n = 1,43 \pm 0,05 \,.
\end{equation*} \\

Gesucht ist außerdem die Lichtgeschwindigkeit $v$ im Medium der planparallelen Platte, die sich ebenfalls über das Brechungsgesetz bestimmen lässt.
Dabei gilt
\begin{equation*}
    \frac{v_1}{v_2} = \frac{c}{v} = \frac{\sin(\alpha)}{\sin(\beta)} \,.
\end{equation*} \\

Im Mittel ergibt sich damit

\begin{equation*}
    v = (2,10 \pm 0,08) \cdot 10^8 \,\unit{\frac{\meter}{\second}} \,.
\end{equation*} 


\subsection{Strahlversatz}

Mithilfe der in \autoref{subsec:brechungsgesetz} aufgenommen Messdaten wird nun der Strahlversatz $s$ über 

\begin{equation}
    s = d \,\frac{\sin(\alpha - \beta)}{\cos(\beta)}
    \label{eq:strahlversatz}
\end{equation}
bestimmt.

Zusätzlich dazu wird der Strahlversatz außerdem über das Brechungsgesetz berechnet.
Unter Verwendung des in \autoref{subsec:brechungsgesetz} ermittelten Brechungsindexes $n = 1,43$ wird der Brechungswinkel $\beta$ berechnet und anschließend in \eqref{eq:strahlversatz} eingesetzt.

Die experimentell sowie rechnerisch ermittelten Strahlversätze sind in \autoref{tab:messung3} dargestellt.

\begin{table}[H]
    \centering
    \caption{Einfallswinkel $\alpha$, experimentell ermittelte Strahlversätze $s_{\text{e}}$ sowie rechnerische Strahlversätze $s_{\text{t}}$.}
    \label{tab:messung3}
    \begin{tabular}{S[table-format=2.0] S[table-format=1.2] S[table-format=1.2]}
      \toprule
        {$\alpha \mathbin{/} \, \unit{\degree}$} & {$ s_{\text{e}} \mathbin{/} \, \unit{\centi\meter}$} & {$ s_{\text{t}} \mathbin{/} \, \unit{\centi\meter}$} \\
      \midrule
       0          &       0.00     &   0.00  \\
      10          &       0.26     &   0.31  \\
      20          &       0.63     &   0.65  \\
      30          &       1.08     &   1.03  \\
      40          &       1.57     &   1.51  \\
      50          &       1.99     &   2.10  \\
      60          &       2.94     &   2.84  \\
      70          &       3.85     &   3.75  \\
    \bottomrule
    \end{tabular}
\end{table}



