\thispagestyle{plain}

\section*{Kurzfassung}
In dieser Arbeit betrachten wir die Kopplung zwischen Neutrinos als Majoranateilchen, also Teilchen, die mit ihren Antiteilchen übereinstimmen, und Majoronen, den Teilchen, die,
ähnlich wie das Higgs-Boson für alle anderen Teilchen, Neutrinos ihre Massen verleihen.
Über den hypothetischen neutrinolosen Doppelbetazerfall und Beobachtungen der Spektren von Supernovae lassen die die zulässigen Kopplungparameterbereiche limitieren.
Es gelingt uns, mithilfe beobachteter Daten der Supernova SN1987A, die Parameter $|g_{i j}|$ der Kopplungsmatrix $g$ auf $3 \, \cdot \, 10^{-7} < |g_{i j}| < 2 \, \cdot \, 10^{-5}$ einzuschränken.
Gemeinsam mit der Neutrinomassengrenze von $m_1 < \SI{0.8}{\eV}$ aus dem KATRIN-Experiment und der Einschränkung $g_1 < 10^{-4}$ aus einer Betrachtung der Neutrinospektren lässt sich der in \autoref{fig:finalplot}
dargestellte Parameterbereich ausschließen. 
Nach abschließendem Vergleich mit aus dem neutrinolosen Doppelbetazerfall erhaltenen Obergrenzen auf die Elektron-Elektron-Kopplung mit $\num{0.4} \, \cdot \, 10^{-5} < g_{ee} < \num{0.9} \, \cdot \, 10^{-5}$ können wir den
zulässigen Bereich noch weiter einschränken.

\section*{Abstract}
\begin{foreignlanguage}{english}
In this thesis, we discuss the coupling between neutrinos as majorana particles, thus particles that coincide with their anti particles, and majorons, the particles that, like the Higgs boson does for all other
particles, give the neutrios their mass.
The valid coupling parameter regions are impacted by the hypothetical neutrinoless double beta decay and observed supernova spectra.
We succeed in limiting the coupling parameters $|g_{i j}|$ of the coupling matrix $g$ to a range outside $3 \, \cdot \, 10^{-7} < |g_{i j}| < 2 \, \cdot \, 10^{-5}$.
Together with the neutrino mass limit of $m_1 < \SI{0.8}{\eV}$ obtained from the KATRIN experiment and the restriction $g_1 < 10^{-4}$ from neutrino spectra, we can gain the exclusion region represented in
\autoref{fig:finalplot}. \\
Finally, by comparing our results to the upper limits on $g_{ee}$ obtained by Hau Zhang in \cite{hauhau} with $\num{0.4} \, \cdot \, 10^{-5} < g_{ee} < \num{0.9} \, \cdot \, 10^{-5}$ we further expand our exclusion region.

\end{foreignlanguage}
