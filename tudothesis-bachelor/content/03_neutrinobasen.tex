\chapter{Neutrinos und Neutrinomischung in verschiedenen Basen}
\label{chap:neutrinobasen}

\section{Flavourbasis} %\cite[Kap. 2.1]{oberauer}
\label{subsec:flavourbasis}
Voraussetzung für die Neutrinooszillationen ist, dass Masseneigenzustände $\nu_i$ als Lösungen der Diracgleichung mit den Masseneigenwerten $m_i \neq 0 (i = 1, 2, 3)$ existieren.
Diese Masseneigenzustände bestimmen die Propagation des Neutrinos im Vakuum und müssen nicht zwangsläufig identisch zu den Flavoureigenzuständen sein.
Hier betrachten wir die Flavoureigenzustände $\nu_\alpha (\alpha = e, \mu, \tau)$ als Superposition der Masseneigenzustände mit
\begin{equation}
    \nu_\alpha = \sum_i U_{\alpha i} \nu^{(h_i)}_i \,,
    \label{eq:flavourbasis}
\end{equation}
wobei $U_{\alpha i}$ die unitäre Mischungsmatrix zwischen Massenbasis und Flavourbasis darstellt \cite[Kap. 2.1]{oberauer}.
Das Superskript $(h_i) = \pm 1$ beschreibt dabei die Helizität des Eigenzustands.


\section{Supernovae}

Im Inneren eines Sterns, der massiver als zehn unserer Sonnenmassen ist und sich dem Ende seines Lebens zuneigt, sind die Elemente nach absteigender Masse geschichtet.
Der Kern des Sterns, bestehend aus Eisen, Elektronen, Positronen, Photonen und vereinzelten Nukleonen, ist umgeben von Schichten aus Silizium, Sauerstoff, Kohlenstoff, Helium und Wasserstoff.
Je mehr Eisen der Stern fusioniert, desto weniger Material bleibt ihm zur weiteren Kernfusion übrig.
Zu diesem Zeitpunkt des Lebens des Sterns wird das Gleichgewicht zum nach innen wirkenden Gravitationsdruck durch den Strahlungsdruck der Elektronen nach außen aufrechterhalten. 
Auf Dauer ist dieses Gleichgewicht nicht stabil.

Die Elektronen und Protonen im Kern des Stern kombinieren im Prozess des Elektroneneinfangs zu einem Neutron und einem Elektronneutrino $e^- + p \rightarrow n + \nu_e$ \cite{supernovaepaper}.
So sinkt der Elektronendruck immer weiter, bis der Kern instabil wird und in sich zusammenstürzt, er kollabiert.
Die Kollapsgeschwindigkeit übersteigt dabei die Schallgeschwindigkeit im stellaren Medium, es erreicht also keine Information des zusammenbrechenden inneren Kerns den nachstürzenden äußeren Kern.
Mit einer anfänglichen Dichte von $10^9$ bis $10^{10} \, \si{\gram \per \centi\cubic\meter}$ erreicht er schnell Dichten, die in Atomkernen typisch sind.
In Zuge dessen ändert sich die Zusammensetzung im Kern.
Nun besteht er nur noch aus Protonen und Neutronen, die als Fermionen nach dem Pauliverbot niemals im exakt gleichen Zustand existieren dürfen.
Zusätzlich zu diesem Verbot kommt hinzu, dass Protonen sich auch aufgrund ihrer gleichen Ladung abstoßen.

Je dichter der Kern beim Kollaps also wird, desto drastischer erhöht sich der Druck nach außen, bis der Innendruck den Außendruck übersteigt und der Kern im \textit{Rebound} eine Schockwelle erzeugt, die nach außen durch den Stern propagiert.
Dabei verliert die Schockwelle durch das Aufbrechen des einfallenden Sternmaterials Energie.
Zusammen mit der Energie, die an Elektronneutrinos, die beim Elektroneneinfang mit den im Kern freiwerdenden Protonen entstehen, verloren geht, wird die Schockwelle im Kern verzögert, bevor sie vollends nach außen propagiert.
Der Kern des sterbenden Sterns mit seinem vergleichsweise kühlen Inneren und heißen, Neutrinos aller Flavours abstrahlenden Mantel wird nun als Protoneutronenstern bezeichnet.
Die im (auch Neutrinosphäre genannten) Mantel freiwerdenden Neutrinos entstehen in so großer Zahl, dass eine Leistung von mehr als $10^{45} \,\si{\watt}$ \cite{supernovaepaper} frei wird, die die verzögerte Schockwelle wiederbelebt.
So explodiert der Stern in einer spektakulären Explosion, die wir als Supernova bezeichnen.

\section{Materiebasis} %\cite{päspaper}
\label{subsec:materiebasis}

Dabei wird ein großer Teil der Supernovaenergie durch Neutrinos transportiert.
Ist es diesen nun möglich, bei der Propagation durch das Supernovamedium über den Zerfall $\tilde{\nu}^{(h_i)}_i (p_i) \rightarrow \tilde{\nu}^{(h_j)}_j (p_j) + J(q)$
Energie in Form von Majoronen $J$ mit Impuls $q$ abzustrahlen, kann sich, je nach Kopplungsstärke zwischen Neutrino und Majoron, das Energiespektrum der Supernova verändern.
Diese Zerfälle wirken sich auf das Oszillations- und Propagationsverhalten der Neutrinos aus und können mithilfe der Materiebasis beschrieben werden.

Drücken wir dabei das linkshändige vierkomponentige Feld $\nu_L$ in chiraler Darstellung der Gammamatrizen über ein zweikomponentiges Feld $\phi$ mit $\nu^T_L = (\phi, 0)^T$ aus, 
näheres dazu findet sich in \cite{komponentendinger}, lässt sich der Lagrangian in der Materiebasis als
\begin{equation}
    \mathcal{L}_\text{tot} = \mathcal{L}_0 + \mathcal{L}_\text{med} + \mathcal{L}_\text{int}
    \label{eq:materielagrange}
\end{equation}
mit
\begin{align*}
    \mathcal{L}_0          &=   i \sum_i \left[\phi^\dagger_i \left(\partial_t - \vec{\sigma} \times \nabla \right) \phi_i - \frac{m_i}{2} \left(\phi^T_i \sigma_2 \phi - \phi^\dagger_i \sigma_2 \phi^*_i\right) \right] \,,\\
    \mathcal{L}_\text{med} &= - \sum_{i j} \phi^\dagger_i V_{i j} \phi_j  \,,\\
    \mathcal{L}_\text{int} &= - J \sum_{i j} g^M_{i j} \left( \phi^T_i \sigma_2 \phi_j + \phi^\dagger_i \sigma_2 \phi^*_j \right)
\end{align*}
schreiben \cite{päspaper}.
Dabei beschreibt der freie Lagrangian $\mathcal{L}_0$ die Propagation im Vakuum, $\mathcal{L}_\text{med}$ die in der Potentialmatrix $V$ berücksichtigten Effekte von Materie und $\mathcal{L}_\text{int}$ Wechselwirkungen zwischen Neutrinos und Majoronen.
Die Kopplungsstärke zwischen Majoronen und Neutrinos der Massenbasis ist dabei in $g^M_{i j}$ zusammengefasst, $\vec{\sigma} = (\sigma_1, \sigma_2, \sigma_3)$ bezeichnet den Vektor der in \eqref{eq:paulimatrizen} zu findenden Paulimatrizen
und $m_i$ den Masseneigenwert des jeweiligen Neutrinos in der Massenbasis.
Wir berücksichtigen hier nur die einfachste Art der Majoronmodelle, also nehmen wir eine Proportionalität zwischen Kopplungs- und Massenmatrix $g^M_{i j} \propto m_{i j}$ an \cite{päspaper}.

Für gewöhnlich würden wir hier, um die Eigenzustände des bereits erwähnten Zerfalls $\tilde{\nu}^{(h_i)}_i (p_i) \rightarrow \tilde{\nu}^{(h_j)}_j (p_j) + J(q)$ zu bestimmen, 
die aus $\mathcal{L}_\text{tot}$ resultierenden Feldgleichungen 
\begin{equation}
    i \left(\partial_t - \vec{\sigma} \times \nabla \right) \phi_i + i m_i \sigma_2 \phi^*_i - \sum^3_j V_{i j} \phi_j = 0 \,,
\end{equation}
wie in \cite{komponentendinger} gezeigt, mithilfe von Planarwellenspinoren definiter Helizität lösen. 
Allerdings stimmen die so erhaltenden Lösungen mit den aus der Diagonalisierung der Mikheyev-Smirnov-Wolfenstein-Gleichung folgenden Lösungen überein.
Sie beschreibt die Zeitentwicklung schwacher Eigenzustände und die Mischung zwischen diesen und den Energieeeigenzuständen durch
\begin{equation}
    i \partial_t \nu^{(h)}_i = \left(H^\text{rel}_{i j} + U_{i \alpha} V_{\alpha \beta} U^\dagger_{\beta j}\right) \nu^{(h)}_j \,.
    \label{eq:MSW-gleichung}
\end{equation}
Aufgrund der geringen Massen und hohen Energien der Neutrinos, bei denen auch Energie- und Materieeigenzustände übereinstimmen, 
betrachten wir den hochrelativistischen Limes des Hamiltonians $H^\text{rel}_{i j} \approx \left(p + \frac{m^2_i}{2 p}\right) \delta_{ij}$ mit dem Kroneckerdelta
\begin{equation}
    \delta_{ij} = \begin{cases}
                    1, \quad i=j \\
                    0, \quad \text{sonst} \,.
                  \end{cases}
                  \label{eq:kronecker}
\end{equation}
Dabei ist $V$ die diagonale Potentialmatrix der schwachen Basis, also
\begin{equation}
    V = \left( \begin{array}{c c c}
        V_C + V_N   &   0     &     0   \\ 
        0           &   V_N   &     0   \\ 
        0           &   0     &     V_N  \\
        \end{array}\right) \,,
\end{equation}
wobei $V_C = \sqrt{2} h G_F n_B (Y_\text{e}) + Y_{\nu_\text{e}}$ das von geladenen Strömen erzeugte Potential und 
$V_N = \sqrt{2} h G_F n_B \left(-\frac{1}{2} Y_N + Y_{\nu_\text{e}}\right)$ das durch ungeladene Ströme erzeugte Potential beschreiben.
Mit der Baryonendichte $n_B$ des betrachteten Mediums gilt für ein beliebiges Teilchen $i$ mit korrespondierendem Antiteilchen $\bar{i}$ $Y_i = \frac{n_i - n_{\bar{i}}}{n_B}$ \cite{päspaper}.
Geladene Ströme bezeichnen dabei schwache Wechselwirkungen, bei denen ein $W^+$- oder $W^-$- Boson ausgetauscht wird und ungeladene Ströme nur unter Austausch von neutralen $Z$-Bosonen stattfindende schwache Wechselwirkungen.

Aus der Diagonalisierung der rechten Seite der MSW-Gleichung \cite{komponentendinger, PhysRevD.37.1935} folgen die Materieeigenzustände
\begin{equation}
    \tilde{\nu}^{(h_i)}_i = \sum_i \tilde{U}_{i j} \nu^{(h_j)}_j \,.
    \label{eq:materiebasis}
\end{equation} 
Ähnlich wie die Flavoureigenzustände lassen sie sich als Superposition der Masseneigenzustände beschreiben.
Sie unterscheiden sich lediglich in der Mischungsmatrix $\tilde{U}$.

Hier können wir die in \autoref{subsec:flavourbasis} auftretende Mischungsmatrix $U$ als
\begin{equation}
    U = U_{2 3} \, U_{1 3} \, U_{1 2} \, U_0
    \label{eq:kopplungflavour}
\end{equation}
parametrisieren.
Bei den Matrizen $U_{i j}$ handelt es sich dabei um Rotationsmatrizen, die in der $i j$-Ebene eine Rotation um den Winkel $\theta_{i j}$ durchführen.
Die in \autoref{sec:neutrinomasse} behandelte mögliche CP-Verletzung findet sich als Phase in der Matrix $U_0$ mit $U_0 = \text{diag}(1, \mathrm{e}^{-2 i \delta_1}, \mathrm{e}^{-2 i \delta_2})$ wieder, wobei
$\delta_1$ und $\delta_2$ komplexe, beliebige Majoranaphasen darstellen\cite{neutrinorotmat}.

Wie im Chooz-Experiment festgestellt wurde, ist der Chooz-Winkel $\theta_{1 3}$ nicht, wie lange Zeit angenommen, null \cite{theta13}.
Nach aktuellen Daten der Particle Data Group gilt $\sin^2 \left(\theta_{1 3} \right) = \num{2.20 +- 0.07} \cdot 10^{-2}$ \cite{neutrinospdg}.

Wir können $\theta_{1 2}$ als solaren Mischungswinkel $\theta_\odot$ und $\theta_{2 3}$ als atmosphärischen Mischungswinkel $\theta_\text{atm}$ identifizieren.
Da zusätzlich, vor allem für leichte Neutrinos in der Nähe der Neutrinosphäre der Supernova, $|V_{\alpha \alpha}| \gg \frac{m^2_i}{2 p}$ für alle Elemente der Potentialmatrix gilt, unterscheiden sich die Materieeigenzustände nur um eine
beliebige Drehung $\theta'$ in der $\nu_\mu$ - $\nu_\tau$ Ebene von den Eigenzuständen der schwachen Basis.
Wir wählen $\theta' = -\theta_{2 3}$, um möglichst einfache Ausdrücke der Materieeigenzustände zu erhalten, wie sie in \autoref{tab:materieeigis} dargestellt sind.
\begin{table}[H]
    \centering
    \begin{tabular}{S S S}
      \toprule
    {Materieeigenzustände $\tilde{\nu}^{(h_i)}_i$} & {Schwache Eigenzustände $\nu_{\alpha'}$} & {Potential $V$} \\
      \midrule
       {$\tilde{\nu}^+_1$} & {$\bar{\nu}_e$}                                                        &  {$- (V_C + V_N)$} \\
       {$\tilde{\nu}^+_2$} & {$\bar{\nu}_{\mu'}  = c_{2 3} \bar{\nu}_\mu - s_{2 3} \bar{\nu}_\tau$} &  {$- V_N$} \\
       {$\tilde{\nu}^+_3$} & {$\bar{\nu}_{\tau'} = s_{2 3} \bar{\nu}_\mu + c_{2 3} \bar{\nu}_\tau$} &  {$- V_N$} \\
       {$\tilde{\nu}^-_1$} & {$\nu_{\mu'}        = c_{2 3} \nu_\mu       - s_{2 3} \nu_\tau$}       &  {$V_N$} \\
       {$\tilde{\nu}^-_2$} & {$\nu_{\tau'}       = s_{2 3} \nu_\mu       + c_{2 3} \nu_\tau$}       &  {$V_N$} \\
       {$\tilde{\nu}^-_3$} & {$\nu_e$}                                                              &  {$V_C + V_N$} \\
    \bottomrule
    \end{tabular}
    \caption{Materieeigenzustände $\tilde{\nu}^\pm_i$ positiver und negativer Helizität im Limes $|V_{\alpha \alpha}| \gg \frac{m^2_i}{2 p}$ als Rotation der schwachen Eigenzustände. Die Eigenzustände sind dabei so
            angeordnet, dass das Potential in der Tabelle nach unten hin ansteigt. Es gilt $c_{2 3} = \cos(\theta_{2 3})$ und $s_{2 3} = \sin(\theta_{2 3})$.}
    \label{tab:materieeigis}
\end{table}

Genauso lassen sich auch die Kopplungsmatrixelemente $\tilde{g}_{i j}$ der Materiebasis als Rotation der Elemente der schwachen Basis identifizieren.
In den hier betrachteten Modellen ist die Kopplungsmatrix der Massenbasis diagonal, es gilt
\begin{equation}
    g^W = \left( \begin{array}{c c c}
        g_1         &   0     &     0   \\ 
        0           &   g_2   &     0   \\ 
        0           &   0     &     g_3  \\
    \end{array}\right) \,.
    \label{eq:schwachekopplung}
\end{equation}
Damit ist
\begin{equation}
    g_{\alpha' \beta'} \equiv \ \tilde{g}_{i j} = U(-\theta_{2 3}) \, g^W_{\alpha \beta} \, U^T(-\theta_{2 3}) = U_{1 3} U_{1 2} \, U^*_0 \, g^M_{i j} \, U^\dagger_0 \, U_{1 2} U_{1 3} \,,
\end{equation}
wenn wir zusätzlich benutzen, dass die schwache Kopplungsmatrix und die der Massenbasis über $g^W = U \, g^M \, U^T$ verknüpft sind.
Die Matrixelemente $|g_{ij}|$ bezeichnen wir wie folgt:
\begin{align}
    \tilde{g} &= \left( \begin{array}{c c c}
        g_{ee}                  &   g_{e \mu'}          &     g_{e \tau'}       \\ 
        g_{e \mu'}              &   g_{\mu' \mu'}       &     g_{\tau' \mu'}    \\ 
        g_{e \tau'}             &   g_{\tau' \mu'}      &     g_{\tau \tau}     \\
    \end{array}\right) \label{eq:materiekoppmat} \,.
\end{align}

Die Kopplungsparameter $g_2$ und $g_3$ lassen sich dabei in Abhängigkeit von $g_1$ und $m_1$, also der Masse des ersten Masseneigenzustands, ausdrücken als
\begin{align}
    g_2 = g_1 \sqrt{1 + \frac{\Delta m^2_\odot}{m^2_1}}\,, && g_3 = g_1 \sqrt{1 + \frac{\Delta m^2_\odot + \Delta m^2_\text{atm}}{m^2_1}} \,,
    \label{eq:g2g3}
\end{align}
wobei wir die Definitionen $\Delta m^2_{1 2} = m^2_2 - m^2_1 = \Delta m^2_\odot$ und $\Delta m^2_{23} = m^2_3 - m^2_2 = \Delta m^2_\text{atm}$ benutzen.
So ist ein Zusammenhang der Kopplungsmatrizen aller drei Basen hergestellt, den wir verwenden, um die Einschränkungen auf die Kopplungsstärken deutlich zu machen.


%% Unnötige Kopplungsmatrix
%    &= \left( \begin{array}{c c c}
%        c^2_{1 2} c^2_{1 3} g_1 + s^2_{12} c^2_{13} \mathrm{e}^{-2 i \delta_1} g_2 + s^2_{13} \mathrm{e}^{-2 i \delta_2}  g_3               &   -s_{12}c_{12}c_{13} g_1 + c_{12} s_{12} c_{13} \mathrm{e}^{-2 i \delta_1} g_2       &   -c^2_{12} s_{13} c_{13} g_1 + s^2_{12} c^2_{13} \mathrm{e}^{-2 i \delta_1} g_2 + s_{13} c_{13} \mathrm{e}^{-2 i \delta_2} g_3  \\ 
%        -s_{12}c_{12}c_{13} g_1 + c_{12} s_{12} c_{13} \mathrm{e}^{-2 i \delta_1} g_2                                                       &   s^2_{12} g_1 + c^2_{12} \mathrm{e}^{-2 i \delta_1} g_2                              &   s_{12} c_{12} s_{13} g_1 + s_{12}c_{12}s_{13} \mathrm{e}^{-2 i \delta_1} g_2   \\ 
%        -c^2_{12} s_{13} c_{13} g_1 + s^2_{12} c^2_{13} \mathrm{e}^{-2 i \delta_1} g_2 + s_{13} c_{13} \mathrm{e}^{-2 i \delta_2} g_3       &   s_{12} c_{12} s_{13} g_1 + s_{12}c_{12}s_{13} \mathrm{e}^{-2 i \delta_1} g_2        &   c^2_{21} s^2_{13} g_1 + s^2_{12}s^2_{13} \mathrm{e}^{-2 i \delta_1} g_2 + c^2_{13} \mathrm{e}^{-2 i \delta_2} g_3  \\
%    \end{array}\right)




