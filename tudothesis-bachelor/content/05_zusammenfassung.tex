\chapter{Zusammenfassung und Ausblick}

Wie wir sehen, schränken die diskutierten Argumente die mögliche Wahl der Kopplungsparameter drastisch ein.
So können wir die Bereiche $m_1 > \SI{0.8}{\eV}$ dank des KATRIN-Experiments, $g_1 > 10^{-4}$ aus dem Argument der Neutrinospektren sowie $g_{i j}$ $3 \cdot 10^{-7} < |g_{i j}| < 2 \cdot 10^{-5}$
mithilfe beobachteter Luminositäten vollständig ausschließen und damit den Bereich zulässiger Wertepaare
$(g_1,m_1)$ deutlich einschränken.

Zu den erzielten Ergebnissen sei gesagt, dass sie nur die einfachsten Majoronmodelle berücksichtigen.
Hier haben wir die Neutrinomasse über die spontane Brechung der Leptonenzahlsymmetrie eingeführt.
Dadurch werden eine Vielzahl von Modellen berücksichtigt, es ist aber auch möglich, die Neutrinomasse und Majoronen auf andere Art und Weise einzuführen.
Auch die Wahl von $\theta_{1 3} = 0$ beeinflusst die resultierenden Ausschlussregionen, auch wenn der tatsächliche Wert sich nicht sehr stark von null unterscheidet.
Darüber hinaus ist die Argumentation über Supernovae nur eine Möglichkeit, Einschränkungen auf die Kopplungsparameter zu erhalten.
Über den neutrinolosen Doppel-$\beta$-Zerfall können wir ebenfalls zumindest eine Obergrenze auf $g_{ee}$ erhalten.

Wie in \cite{hauhau} gezeigt, liegt diese Obergrenze zwischen
\begin{equation}
    \num{0.4} \cdot 10^{-5} < g_{ee} < \num{0.9} \cdot 10^{-5} \,.
    \label{eq:g_eedoppelbeta}
\end{equation} 
Gemeinsam unseren Einschränkungen ergibt sich so \autoref{fig:mitdobeta}.
%\begin{figure}[H]
%    \centering
%    \includegraphics[width=.8\textwidth]{build/meinsanddobeta.pdf}
%    \caption{Grafische Darstellung der hier erhaltenen Ausschlussregion zuzüglich der aus dem Doppelbetazerfall erhaltenen Obergrenzen auf $g_{ee}$. In grün sind die positiven und in lindgrün die negativen
%            unteren Obergrenzen, analog in hellen und dunklen Blautönen das obere Limit der Obergrenzen dargestellt. Gestrichelte Linien kennzeichnen dabei Zweige mit der Phase $\delta = \frac{\pi}{2}$, durchgezogene Linien die mit $\delta = 0$.}
%    \label{fig:mitdobeta}
%\end{figure}

Abschließend lässt sich sagen, dass es uns durch das Auftreten weiterer Supernovae innerhalb unserer Galaxie möglich sein wird, die beobachteten Daten besser mit unseren Simulationen abzugleichen.
Gemeinsam mit komplexeren Modellen lässt sich die Ausschlussregion so immer weiter modifizieren, bis es uns eines Tages gelingt, die Existenz von Majoronen zweifellos zu beweisen oder die
Ausschlussregion so weit einzuschränken, dass das Neutrino als Majoranateilchen nahezu ausgeschlossen werden kann.



