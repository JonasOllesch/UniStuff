\chapter{Zusammenfassung und Ausblick}

Wie wir sehen, schränken die diskutierten Argumente die mögliche Wahl der Kopplungsparameter drastisch ein.
So können wir die Bereiche $m_1 > \SI{0.8}{\eV}$ dank des KATRIN-Experiments, $g_1 > 10^{-4}$ aus dem Argument der Neutrinospektren sowie $|g_{i j}| > \frac{\num{0.83} \cdot 10^{-8}}{m_J} \,\si{\mega\eV}$
mithilfe beobachteter Luminositäten vollständig ausschließen und damit den Bereich zulässiger Wertepaare
$(g_1,m_1)$ deutlich einschränken.

Zu den erzielten Ergebnissen sei gesagt, dass sie nur die einfachsten Majoronmodelle berücksichtigen.
Hier haben wir die Neutrinomasse über die spontane Brechung der Leptonenzahlsymmetrie eingeführt.
Dieser Ansatz umfasst eine Vielzahl von Modellen, es ist aber auch möglich, die Neutrinomasse und Majoronen auf andere Art und Weise einzuführen.
Im Vergleich zu \cite{päspaper} berücksichtigen wir hier zwar den Einfluss des Chooz-Winkels $\theta_{1 3}$, die Auswirkungen auf die Kopplungsparameter $|g_{i j}|$ sind aber sehr gering, da $\theta_{13}$ nur geringfügig verschieden von null ist.
Damit sind alle Beiträge der Form $\sin^2 \theta_{13}$ nahezu vernachlässigbar klein gegen Beiträge der Form $\cos^2\theta_{13}$.
Setzen wir $\theta_{13}$ vollständig auf null, ergibt sich logischerweise die in \cite{päspaper} beschriebene Form der Kopplungsmatrix $g$.
Darüber hinaus ist die Argumentation über Supernovae nur eine Möglichkeit, Einschränkungen auf die Kopplungsparameter zu erhalten.
Über den neutrinolosen Doppel-$\beta$-Zerfall können wir ebenfalls eine Obergrenze auf $g_{ee}$ erhalten.
Wie in \cite{hauhau} gezeigt, liegt diese Obergrenze zwischen
\begin{equation}
    g_{ee} < \left(\num{0.4} - \num{0.9}\right) \, \cdot \, 10^{-5} \,,
    \label{eq:g_eedoppelbeta}
\end{equation} 
also, je nach Wahl der Majonenmasse, mindestens eine Größenordnung über der von uns angenommenen Obergrenze.
Gemeinsam unseren Einschränkungen ergibt sich so \autoref{fig:mitdobeta}.
\begin{figure}[H]
    \centering
    \includegraphics[width=.8\textwidth]{build/finalmitdobeta.pdf}
    \caption{Grafische Darstellung der hier erhaltenen Ausschlussregion zuzüglich der aus dem Doppelbetazerfall erhaltenen Obergrenzen auf $g_{ee}$. In blau ist die aus der unteren Obergrenze resultierend, 
    analog grün die aus dem obere Limit der Obergrenze resultierende Ausschlussregion dargestellt. Die Auswahl der für die Ausschlussregion relevanten Kurven erfolgt dabei analog zu dem in \autoref{sec:kopplungen}
    geschilderten Vorgehen für $g_{ee}$.}
    \label{fig:mitdobeta}
\end{figure}
Die Grenze des Doppelbetazerfalls gibt eine allgemeinere Grenze auf die Kopplung $g_{ee}$, schränkt die von uns gefundene Ausschlussregion also nicht weiter ein.

Durch das Auftreten weiterer Supernovae innerhalb unserer Galaxie wird es uns möglich sein, die beobachteten Daten besser mit unseren Simulationen abzugleichen.
Gemeinsam mit komplexeren Modellen lässt sich die Ausschlussregion so immer weiter modifizieren, bis es uns eines Tages gelingt, die Existenz von Majoronen zweifellos zu beweisen oder die
Ausschlussregion so weit einzuschränken, dass das Neutrino als Majoranateilchen nahezu ausgeschlossen werden kann.
Ein guter Kandidat für eine neue Supernova ist der Stern Betelgeuse, dessen auffällige Luminosität vermuten lässt, dass er in absehbarer Zeit in einer Supernova explodieren könnte, die sich auch ohne Teleskop beobachten lassen wird.




