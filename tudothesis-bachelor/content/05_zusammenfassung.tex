\chapter{Zusammenfassung und Ausblick}

Wie wir sehen, schränken die diskutierten Argumente die mögliche Wahl der Kopplungsparameter drastisch ein.
So können wir den Bereich $m_1 > \SI{0.8}{\eV}$ und $g_1 > 10^{-4}$ vollständig ausschließen.
Auch unterhalb dieser Grenzen ist die Wahl von $m_1$ und $g_1$ deutlich eingeschränkt.

Zu den erzielten Ergebnissen sei gesagt, dass sie nur die einfachsten Majoronmodelle berücksichtigen.
Hier haben wir die Neutrinomasse über die spontane Brechung der Leptonenzahlsymmetrie eingeführt.
Dadurch werden eine Vielzahl von Modellen berücksichtigt, es ist aber auch möglich, die Neutrinomasse und Majoronen auf andere Art und Weise einzuführen.
Auch die Wahl von $\theta_{1 3} = 0$ beeinflusst die resultierenden Ausschlussregionen, auch wenn der tatsächliche Wert sich nicht sehr stark von null unterscheidet.
Darüber hinaus ist die Argumentation über Supernovae nur eine Möglichkeit, Einschränkungen auf die Kopplungsparameter zu erhalten.
Über den neutrinolosen Doppel-$\beta$-Zerfall können wir ebenfalls Grenzen erhalten, die in ... dargestellt sind. %%%%% Mit Haus Code zusammenpacken
Näheres zum neutrinolosen Doppel-$\beta$-Zerfall findet sich in ..., mit dessen Zusammenarbeit dieser Teil der Arbeit entstanden ist. %%% Haus BA zitieren

Durch das Auftreten weiterer Supernovae innerhalb unserer Galaxie wird es uns möglich sein, die beobachteten Daten besser mit unseren Simulationen abzugleichen und damit in Zukunft die Ausschlussregionen weiter zu modifizieren.



