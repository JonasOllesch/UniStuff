\chapter{Einleitung}
\label{chap:einleitung}

Neutrinos sind nach dem Standardmodell der Teilchenphysik neutrale, masselose Leptonen, die nur über die schwache Wechselwirkung an andere Elementarteilchen koppeln.
In zwei unabhängigen Experimenten konnten von Arthur McDonald Takaaki Kajita allerdings bewiesen werden, dass der lange vorhergesagte Prozess der Neutrinooszillationen tatsächlich existiert.
Diese Neutrinooszillation, also das zeitlich und räumlich periodische Wechseln des Neutrinoflavours, setzt allerdings eine endliche Ruhemasse voraus.
Damit können Neutrinos nicht, wie vom Standardmodell angenommen, masselos sein.

Das Standardmodell muss also um neue Physik ergänzt werden, um die Neutrinomasse zu berücksichtigen.
Eine Möglichkeit, die endliche, aber, verglichen mit den anderen Leptonen verschwindend geringe Masse zu erklären, ist, Neutrinos als Majoranateilchen einzuführen.
Majoranateilchen beschreiben Teilchen, die mit ihren Antiteilchen übereinstimmen.
Im Falle der Neutrinos müsste so die jedem Lepton zugeordnete Leptonzahl, die bei allen bisher beobachteten Prozess eine Erhaltungsgröße darstellt, spontan gebrochen sein.
Diese spontane Symmetriebrechung führen wir später durch das Majoronenfeld ein, das linkshändige Neutrinos mit rechtshändigen Antineutrinos und umgekehrt verknüpft.
Es besitzt die Majoronen, sogenannte Goldstone-Bosonen, als Botenteilchen. \\
Um diese Majoronen zu finden, werden konkret zwei Prozesse näher betrachtet.
Der neutrinolose Doppelbetazerfall, bei dem zwei Neutronen in einem Atomkern gleichzeitig unter möglicher Emission eines Majorons in zwei Protonen und Elektronen zerfallen und Supernovaexplosionen, bei denen
Majoronen die Kühlung und die Neutrinospektren beeinflussen könnten. \\
Hier legen wir den Fokus auf den Prozess der Supernovae.
Anhand verschiedener Basen werden wir uns die Propagation von Neutrinos in dichten Medien deutlich machen und konkret beleuchten, 
inwiefern Eigenzustände der Massenbasis und der Materiebasis miteinander verknüpft sind.
Durch Betrachtung verschiedener Argumente und aufgenommener Daten der Supernova SN1987A wird es uns so gelingen, Grenzen auf die Neutrinomasse und Kopplungen zwischen Majoronen und Neutrinos zu erhalten.
Haben wir diese Grenzen erhalten, stellen wir die Ergebnisse analog zu \cite{päspaper} grafisch dar, um ein anschauliches Bild für die entstehenden Ausschlussregionen zu erhalten und ziehen abschließend einen Vergleich
zu den von Hau Zhang in \cite{hauhau} erzielten Ergebnissen des Doppelbetazerfalls.


