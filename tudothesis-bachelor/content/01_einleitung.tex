\chapter{Einleitung}

Neutrinos sind nach dem Standardmodell der Teilchenphysik neutrale, masselose Leptonen, die nur über die schwache Wechselwirkung an andere Elementarteilchen koppeln.
In zwei unabhängigen Experimenten konnten von Arthur McDonald Takaaki Kajita allerdings bewiesen werden, dass Neutrinos oszillieren.
Diese Neutrinooszillation, also das zeitlich und räumlich periodische Wechseln des Neutrinoflavours, setzt eine endliche Ruhemasse voraus.
Damit können Neutrinos nicht, wie vom Standardmodell angenommen, masselos sein. \\

Das Standardmodell muss also um neue Physik ergänzt werden, um die Neutrinomasse zu berücksichtigen.
Eine Theorie, die endliche, aber geringe Masse zu erklären, liegt in den Majoronen, hypothetischen Nambu-Goldstone-Bosonen, die zur spontanen Brechung der Leptonenzahlsymmetrie eingeführt werden.
Sie verknüpfen linkshändige Neutrinos mit rechtshändigen Antineutrinos, das Neutrino wäre also identisch zu seinem Antiteilchen. \\
Um diese Majoronen zu finden, werden konkret zwei Prozesse näher betrachtet.
Der Doppelbetazerfall, bei dem zwei Neutronen in einem Atomkern gleichzeitig unter möglicher Emission eines Majorons in zwei Protonen und Elektronen zerfallen und Supernovaexplosionen, bei denen
Majoronen die Kühlung und die Neutrinospektren beeinflussen könnten. \\

Der Fokus dieser Arbeit soll auf den Supernovaexplosionen liegen, konkret in der Aktualisierung der in \cite{päspaper} erzielten Ergebnisse.


