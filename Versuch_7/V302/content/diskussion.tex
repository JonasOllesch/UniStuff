\section{Diskussion}
\label{sec:Diskussion}
Vergleicht man die Abweichung zwischen der Induktivitätsmessbrücke und der Maxwellbrücke, so fällt auf, dass ....  %%%%%%%%%%%% Hier die Werte vergleichen!!!
Diese Abweichung lässt sich durch die Annahmen, dass die Spule $L_2$ bei der Induktivitätsmessbrücke keinen Innenwiderstand besitzt, erklären.
Zwar wurde auch angenommen, dass der Kondensator $C_4$ der Maxwellbrücke ebenfalls verlustfrei ist, allerdings arbeiten Kondensatoren prinzipiell verlustärmer. \\

Zum Vergleich der Theoriekurve und den Messdaten der Wien-Robinson-Brücke lässt sich sagen, dass sie, bis auf das Datentupel $(ν, U_{Br}(ν)) = (800 \,\unit{\hertz}, 720 \,\unit{\milli\volt})$ recht gut zueinander passen. \\

Der Klirrfaktor von $...$ ist ebenfalls sehr gering, die Genauigkeit um $ν_0$ ist mit einem Minimum von $1 \,\unit{\milli\volt}$ sehr hoch. 