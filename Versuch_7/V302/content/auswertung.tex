\section{Auswertung}
\label{sec:Auswertung}

\subsection{Wheatstonesche Brücke}

Die zur Bestimmung von $R_x$ notwendigen Messdaten sind in \autoref{tab:wheatstone} dargestellt.
Dabei ist $R_4$ durch $R_4 = 1000 - R_3$ gegeben.

\begin{table}[H]
  \centering
  \caption{Messungen der bekannten Widerstände $R_2$, $R_3$ und $R_4$.}
  \label{tab:wheatstone}
  \begin{tabular}{S S S}
    \toprule
    & {Messung 1} & {Messung 2} \\
    \midrule
    {Für Wert 12} \\
    {$R_2 \mathbin{/} \unit{\ohm}$} & 1000 & 664 \\
    {$R_3 \mathbin{/} \unit{\ohm}$} &  282 & 371 \\
    {$R_4 \mathbin{/} \unit{\ohm}$} &  718 & 629 \\
    \\
    {Für Wert 14} \\ 
    {$R_2 \mathbin{/} \unit{\ohm}$} & 1000 & 664\\
    {$R_3 \mathbin{/} \unit{\ohm}$} &  475 & 576\\
    {$R_4 \mathbin{/} \unit{\ohm}$} &  525 & 424\\
    \bottomrule
  \end{tabular}
\end{table}

Nach \eqref{fig:wheatstonebrücke} lässt sich nun $R_x$ berechnen. 
Der relative Fehler für $R_2$ beträgt dabei $0.2 \,\%$, für das Verhältnis $\frac{R_3}{R_4}$ beträgt er $0.5 \,\%$
Für Wert 12 ergibt sich dann für die beiden Messungen gemittelt
\begin{equation*}
  R_x = ... \,\unit{\ohm}
\end{equation*} und für Wert 14

\begin{equation*}
  R_x = ... \,\unit{\ohm} \,.
\end{equation*}


\subsection{Kapazitätsmessbrücke}

Die zur Berechnung von $R_x$ und $C_x$ nötigen Messdaten sind in \autoref{tab:kapamessbrü}
dargestellt. Erneut ist $R_4 = 1000 - R_3$, der relative Fehler von $C_2$ beträgt $0.2 \,\%$, für $R_2$
beträgt er $3 \,\%$.

\begin{table}[H]
  \centering
  \caption{Messungen der bekannten Widerstände $R_2$, $R_3$ und $R_4$ und der Kapazität $C_2$.}
  \label{tab:kapamessbrü}
  \begin{tabular}{S S S}
    \toprule
    & {Messung 1} & {Messung 2} \\
    \midrule
    {Für Wert 9} \\
    {$C_2 \mathbin{/} 10^{-9} \,\unit{\farad}$} &  750 & 597\\
                {$R_2 \mathbin{/} \unit{\ohm}$} &  281 & 347 \\
                {$R_3 \mathbin{/} \unit{\ohm}$} &  632 & 582 \\
                {$R_4 \mathbin{/} \unit{\ohm}$} &  368 & 418 \\
  \end{tabular}
\end{table}

Mit \eqref{eq:resxkapbrü} und \eqref{eq:kapxkapbrü} ergeben sich, erneut gemittelt,

\begin{equation*}
  R_x = ... \,\unit{\ohm}
\end{equation*}
und
\begin{equation*}
  C_x = ... \,\unit{\farad} \,.
\end{equation*}


\subsection{Induktivitätsmessbrücke}

In \autoref{tab:indumessbrü} sind die zur Berechnung von $R_x$ und $L_x$ notwendigen Daten dargestellt.
Die relative Abweichung von $L_2$ beträgt dabei $0.2 \,\%$, die von $R_2$ und $\frac{R_3}{R_4}$ verändern
sich nicht.

\begin{table}[H]
  \centering
  \caption{Messungen der bekannten Widerstände $R_2$, $R_3$ und $R_4$ und der \\ Induktivität $L_2$.}
  \label{tab:indumessbrü}
  \begin{tabular}{S S[table-format=3.1] S[table-format=3.1]}
    \toprule
    & {Messung 1} & {Messung 2} \\
    \midrule
    {Für Wert 16} \\
    {$L_2 \mathbin{/} 10^{-3} \,\unit{\henry}$} &   14,6  &    20,1   \\
                {$R_2 \mathbin{/} \unit{\ohm}$} &   49    &    63     \\
                {$R_3 \mathbin{/} \unit{\ohm}$} &  903    &   871     \\
                {$R_4 \mathbin{/} \unit{\ohm}$} &   97    &   129     \\
  \end{tabular}
\end{table}

Mit \eqref{eq:resxindubrü} und \eqref{eq:induxindubrü} ergeben sich dann

\begin{equation*}
  R_x = ... \,\unit{\ohm}
\end{equation*} und

\begin{equation*}
  L_x = ... \,\unit{\henry} \,.
\end{equation*}


\subsection{Wien-Robinson-Brücke}



