\section{Auswertung}
\label{sec:Auswertung}

\subsection{Wheatstonesche Brücke}

Die zur Bestimmung von $R_x$ notwendigen Messdaten sind in \autoref{tab:wheatstone} dargestellt.
Dabei ist $R_4$ durch $R_4 = 1000 - R_3$ gegeben.

\begin{table}[H]
  \centering
  \caption{Messungen der bekannten Widerstände $R_2$, $R_3$ und $R_4$.}
  \label{tab:wheatstone}
  \begin{tabular}{S S S}
    \toprule
    & {Messung 1} & {Messung 2} \\
    \midrule
    {Für Wert 12} \\
    {$R_2 \mathbin{/} \unit{\ohm}$} & 1000 & 664 \\
    {$R_3 \mathbin{/} \unit{\ohm}$} &  282 & 371 \\
    {$R_4 \mathbin{/} \unit{\ohm}$} &  718 & 629 \\
    \\
    {Für Wert 14} \\ 
    {$R_2 \mathbin{/} \unit{\ohm}$} & 1000 & 664\\
    {$R_3 \mathbin{/} \unit{\ohm}$} &  475 & 576\\
    {$R_4 \mathbin{/} \unit{\ohm}$} &  525 & 424\\
    \bottomrule
  \end{tabular}
\end{table}

Nach \eqref{fig:wheatstonebrücke} lässt sich nun $R_x$ berechnen. 
Der relative Fehler für $R_2$ beträgt dabei $0,2 \,\%$, für das Verhältnis $\frac{R_3}{R_4}$ beträgt er $0,5 \,\%$
Für Wert 12 ergibt sich dann für die beiden Messungen gemittelt
\begin{equation*}
  R_x =392,2 \pm 1,5 \,\unit{\ohm}
\end{equation*} und für Wert 14

\begin{equation*}
  R_x = 903,4 \pm 3,4 \,\unit{\ohm} \,.
\end{equation*}


\subsection{Kapazitätsmessbrücke}

Die zur Berechnung von $R_x$ und $C_x$ nötigen Messdaten sind in \autoref{tab:kapamessbrü}
dargestellt. Erneut ist $R_4 = 1000 - R_3$, der relative Fehler von $C_2$ beträgt $0,2 \,\%$, für $R_2$
beträgt er $3 \,\%$.

\begin{table}[H]
  \centering
  \caption{Messungen der bekannten Widerstände $R_2$, $R_3$ und $R_4$ und der Kapazität $C_2$.}
  \label{tab:kapamessbrü}
  \begin{tabular}{S S S}
    \toprule
    & {Messung 1} & {Messung 2} \\
    \midrule
    {Für Wert 9} \\
    {$C_2 \mathbin{/} 10^{-9} \,\unit{\farad}$} &  750 & 597\\
                {$R_2 \mathbin{/} \unit{\ohm}$} &  281 & 347 \\
                {$R_3 \mathbin{/} \unit{\ohm}$} &  632 & 582 \\
                {$R_4 \mathbin{/} \unit{\ohm}$} &  368 & 418 \\
  \end{tabular}
\end{table}

Mit \eqref{eq:resxkapbrü} und \eqref{eq:kapxkapbrü} ergeben sich, erneut gemittelt,

\begin{equation*}
  R_x = (483 \pm 7) \,\unit{\ohm}
\end{equation*}
und
\begin{equation*}
  C_x = (4,33 \pm 0,07)10^{-7} \,\unit{\farad} \,.
\end{equation*}


\subsection{Induktivitätsmessbrücke}

In \autoref{tab:indumessbrü} sind die zur Berechnung von $R_x$ und $L_x$ notwendigen Daten dargestellt.
Die relative Abweichung von $L_2$ beträgt dabei $0,2 \,\%$, die von $R_2$ und $\frac{R_3}{R_4}$ verändern
sich nicht.

\begin{table}[H]
  \centering
  \caption{Messungen der bekannten Widerstände $R_2$, $R_3$ und $R_4$ und der \\ Induktivität $L_2$.}
  \label{tab:indumessbrü}
  \begin{tabular}{S S[table-format=3.1] S[table-format=3.1]}
    \toprule
    & {Messung 1} & {Messung 2} \\
    \midrule
    {Für Wert 16} \\
    {$L_2 \mathbin{/} 10^{-3} \,\unit{\henry}$} &   14,6  &    20,1   \\
                {$R_2 \mathbin{/} \unit{\ohm}$} &   49    &    63     \\
                {$R_3 \mathbin{/} \unit{\ohm}$} &  903    &   871     \\
                {$R_4 \mathbin{/} \unit{\ohm}$} &   97    &   129     \\
  \end{tabular}
\end{table}

Mit \eqref{eq:resxindubrü} und \eqref{eq:induxindubrü} ergeben sich dann gemittelt

\begin{equation*}
  R_x = (440,8 \pm 1,7) \,\unit{\ohm}
\end{equation*} und

\begin{equation*}
  L_x = (1,358 \pm 0,007) \,\unit{\henry} \,.
\end{equation*}

\newpage

\subsection{Wien-Robinson-Brücke}

Die Frequenzabhängigkeit der Brückenschaltung wird nach Frequenz $ν$ und Brückenspannung $U_{Br}$ in
\autoref{tab:wienrobinson} aufgetragen.

\begin{table}[H]
  \centering
  \caption{Messungen der Frequenz $ν$ und Brückenspannung $U_{Br}$.}
  \label{tab:wienrobinson}
  \begin{tabular}{S[table-format=5.0] S[table-format=3.1]}
    \toprule
    {Frequenz $ν$ in $\unit{\hertz}$} & {Brückenspannung $U_{Br} \mathbin{/} \unit{\milli\volt}$} \\
    \midrule
    20    & 620   \\
    40    & 560   \\
    80    & 310   \\
    150   &  68   \\
    151   &  28   \\
    152   &  26   \\
    153   &  22   \\
    154   &  18   \\
    155   &  16   \\
    156   &  15   \\
    157   &   8   \\
    158   &   7   \\
    159   &   6   \\
    160   &   3,6 \\
    161   &   1   \\
    162   &   3   \\
    163   &   6   \\
    164   &   8   \\
    165   &  12   \\
    166   &  14   \\
    167   &  17   \\
    168   &  19   \\
    169   &  22   \\
    170   &  25   \\
    180   &  50   \\
    190   &  72   \\
    200   & 100   \\
    400   & 360   \\
    800   & 720   \\
    1600  & 600   \\
    3200  & 610   \\
    6400  & 610   \\
    12800 & 605   \\
    25600 & 600   \\
    30000 & 580   \\
  \end{tabular}
\end{table}

Als theoretische Kreisfrequenz $ν_0 = \frac{ω_0}{2π} = \frac{R C}{2π}$ ergibt sich hier mit 
$R = 1000 \,\unit{\ohm}$ und $C = 993 \cdot 10^{-9} \,\unit{\farad}$

\begin{equation*}
  ν_0 = 160,3 \,\unit{\hertz} \,.
\end{equation*}

Damit lässt sich der in \autoref{fig:vergleichsplot} dargestellte Vergleich von Theorie und Messung darstellen.
Die Theoriekurve wird dabei nach \eqref{eq:omegadingens} berechnet.

\begin{figure}
  \centering
  \includegraphics{build/Theorie.pdf}
  \caption{Vergleich der Messdaten und der Theorie.}
  \label{fig:vergleichsplot}
\end{figure}

Die Speisespannung ist in \autoref{tab:speisespann} aufgetragen.

\begin{table}[H]
  \centering
  \caption{Messungen der Frequenz $ν$ und Speisespannung $U_{S}$.}
  \label{tab:speisespann}
  \begin{tabular}{S[table-format=5.0] S[table-format=3.1]}
    \toprule
    {Frequenz $ν$ in $\unit{\hertz}$} & {Speisespannung $U_{S} \mathbin{/} \unit{\milli\volt}$} \\
    \midrule
       20 & 1900 \\
       40 & 1900 \\
       80 & 2000 \\
      160 & 2000 \\
      320 & 1950 \\
      640 & 1900 \\
     1280 & 1900 \\
     2560 & 1900 \\
     5120 & 1950 \\
    10240 & 1950 \\
    20480 & 1800 \\
    30000 & 1900 \\
  \end{tabular}
\end{table}


Zur abschließenden Berechnung des Klirrfaktors wird angenommen, dass alle $U_n$ mit $n > 2$ nicht zur Summe
beitragen.

$U_2$ berechnet sich dann über die effektiven Brückenspannung $U_{Br,eff} = \frac{U_{Br}}{2 \sqrt{2}}$
an der Stelle $ν_0$ und $Ω = 2$. Es ergibt sich

\begin{equation*}
  U_2 = \frac{3,535*10^(-4) \,\unit{\volt}}{\sqrt{\frac{(2^2 - 1)^2}{9|(1-2^2)^2+9 \cdot 2^2|}}} = (2,371)*10^{-4} \,\unit{\volt}
\end{equation*} und damit

\begin{equation*}
  k = \frac{U_2}{U_1} = 1,234*10^{-3} \,.
\end{equation*}


