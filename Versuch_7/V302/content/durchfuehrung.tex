\section{Durchführung}
\label{sec:Durchführung}
Am Anfang wird die Speisespannung und die Speisefrequenz eingestellt und für die gesamte Messung beibehalten.
\subsection{Wheatstonebrücke}

Es wird die Schaltung der \autoref{fig:wheatstonebrücke} aufgebaut.
Der bekannt Widerstand $R_2$ wird festgelegt. Danach wird das Potentiometer so variiert, dass die Brückenspannung minimal wird.
Die Werte des Potentiometers und der bekannten Bauteile werden aufgeschrieben.
Der Vorgang wird mit einem anderen $R_2$ für den gleichen $R_x$ wiederholt.
Die gleiche Messung wird darauf für einen anderen unbekannten Widerstand durchgeführt.

\subsection{Induktivitätsmessbrücke}
Es wird die Schaltung der \autoref{fig:indumessbrü} aufgebaut. Die beiden Potentiometer werden so eingestellt, dass die Brückenspannung erneut minimal wird. Dazu werden die Potentiometer im Wechsel zu einander justiert.
Ist die minimale Spannung erreicht, werden die eingestellten Widerstände gemessen und die Werte der bekannten Bauteile notiert.
Die Messung wird mit einer anderen Spule $L_2$ wiederholt.

\subsection{Maxwellbrücke}
Es wird die Schaltung der 
%\autref{fig:maxwellbrü}
aufgebaut, dabei werden die selben unbekannten Bauteile verwendet wie bei der Induktivitätsmessbrücke.
Die Brückenspannung wird erneut durch die Potentiometer minimiert, sobalt dies erreicht werden Werte von $L_2$, $L_3$ und $L_4$ gemessen.
