\section{Durchführung}
\label{sec:Durchführung}

Bei allen Versuchen werden die Brückenschaltungen mit einer Frequenz von $1000 \, \, \unit{\hertz}$ betrieben.
Die Brückenspannungen werden auf einem digitalen Oszilloskop abgelesen.

\subsection{Wheatstone Brücke}

Die Schaltung wird \autoref{fig:wheatstonebrücke} entsprechend aufgebaut, das über das Potentiometer
realisierte Widerstandsverhältnis $\frac{R_3}{R_4}$ wird solange variiert, bis die Brückenspannung 
näherungsweise verschwindet.
Werte für $R_2$, $R_3$ und $R_4$ werden notiert, die Messung wird erneut für einen verschiedenen 
Widerstand $R_2$ gemessen und anschließend erneut mit zwei verschiedenen $R_2$ für einen weiteren 
unbekannten Widerstand $R_x$ wiederholt.


\subsection{Kapazitätsmessbrücke}

Wie in \autoref{fig:kapamessbrü} dargestellt wird die Kapazitätsmessbrücke aufgebaut.
Die beiden Potentiometer (für $R_2$ bzw. $\frac{R_3}{R_4}$) werden erneut solange abwechselnd variiert, 
bis die Brückenspannung verschwindet, die Werte für $C_2$, $R_2$, $R_3$ und $R_4$ werden notiert, die Messung
wird für eine weitere unbekannte Kapazität wiederholt.


\subsection{Induktivitätsmessbrücke}
\label{subsec:indumessdurch}

Analog zur Kapazitätsmessbrücke wird durch abwechselnde Variation der Potentiometer die Brückenspannung
der wie in \autoref{fig:indumessbrü} aufgebauten Schaltung minimiert, $L_2$, $R_2$, $R_3$ und $R_4$ 
werden notiert.
Die Messung für eine weitere Induktivität $L_2$ durchgeführt.


\subsection{Maxwellbrücke}

Gemäß \autoref{fig:maxwellbrü} wird mit der in \autoref{subsec:indumessdurch} verwendeten 
Induktivität $L_x$ aufgebaut. 
Erneut werden die nun einzeln als Potentiometer aufgebauten Widerstände $R_3$ und $R_4$ wechselnd variiert,
die Werte für $C_2$, $R_2$, $R_3$ und $R_4$ werden notiert.


\subsection{Wien-Robinson-Brücke}

Entsprechend \autoref{fig:wienrobinson} wird eine Wien-Robinson-Brücke aufgebaut.
Es wird ein Frequenzbereich von $20 \, \unit{\hertz}$ bis $30000 \, \unit{\hertz}$ durchlaufen und die
Brückenspannung notiert.
Dabei wird die Frequenz für jede Messung verdoppelt, bis sich ein Minimum feststellen lässt.
Im Frequenzbereich dieses Minimums wird nun genauer, in $1 \, \unit{\hertz}$-Schritten gemessen.
Sobald der Bereich des Minimums sichtbar verlassen ist, wird die Frequenz erneut verdoppelt, bis 
$30000 \, \unit{\hertz}$ erreicht sind.
Nach Abschluss der Messreihe wird für denselben Frequenzbereich die Speisespannung $U_S$ notiert.

%% Eigentlich fertig

