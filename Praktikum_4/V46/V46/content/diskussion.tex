\section{Diskussion}
\label{sec:Diskussion}

%\subsection{Bestimmung der magnetischen Flussdichte}

Vor der Messung der Faraday-Rotation, für die 3. Probe änderte sich die angegebene Stromstärke des Elektromagneten von $\SI{10}{\ampere}$ auf $\SI{9.8}{\ampere}$.
Aus diesem Grund wird die maximale gemessene Feldstärke von $\SI{405}{\milli\tesla}$ für die ersten Proben mit einem Faktor von $\SI{1.02}{}$ auf $\SI{413.1}{\milli\tesla}$ korrigiert.
Der Verlauf des magnetischen Feldstärke ist symmetrisch um die Position der Proben, was den Erwartungen entspricht.
Auch ist das Magnetfeld im Mittelpunkt nahezu homogen, sodass auch die längste Probe, mit einer Dicke von $\SI{5.11}{\milli\meter}$, von einem nahezu homogenen Magnetfeld durchdrungen wird.\\

Der Quotient aus effektiver und reeller Masse beträgt  $\frac{m_{\text{eff}}}{m_\text{e}} \approx \SI{0.063}{}$ \cite{GrossMarx+2018}.
Die gemessenen Verhältnisse betragen $\frac{m^1_{\text{eff}}}{m_\text{e}} = \SI{0.349 \pm 0.012}{}$ und $\frac{m^2_{\text{eff}}}{m_\text{e}} = \SI{0.10087 \pm 0.00070}{}$,
was relative Abweichungen von $\Delta^1_\text{\%} = \SI{453 \pm 19}{\%}$ und $\Delta^2_\text{\%} = \SI{60.1 \pm 1.1}{\%}$ bedeutet.
Das bedeutet, dass die gemessenen Werte signifikant von dem Literaturwerten abweichen.\\



Während der Justage konnte die Apparatur nicht so eingestellt werden, dass die Minima genau einen Abstand von $\SI{90}{\degree}$ voneinander haben.
Das deutet darauf hin, dass mit einer besseren Justage ein genaueres Ergebnis erzielt werden kann.
Zusätzlich hat starkes Rauschen das Ablesen der Messwerte erschwert.  
Es ergeben sich größere Ungenauigkeiten beim Ablesen der Winkel für die resultierenden Minima.
\autoref{fig:Winkel_nom} zeigt hier eine große Streuung besonders für die Messwerte der 2. Probe.
