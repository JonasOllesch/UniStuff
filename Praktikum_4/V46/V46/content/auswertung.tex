\section{Auswertung}
\label{sec:auswertung}

Die Fehlerfortpflanzung wird mithilfe des Paketes \textsc{uncertainties} \cite{uncertainties} durchgeführt.
Für die lineare Regression wird \textsc{scipy.curve\_fit} \cite{scipy} verwendet.
Grafiken werden mit \textsc{matplotlib} \cite{matplotlib} erstellt.

\subsection{Stärke des Magnetfeldes}

Zunächst wird die Stärke des Magnetfeldes untersucht, welches durch den verwendeten Elektromagneten erzeugt wird.
Diese Messung wurde zwar zuletzt durchgeführt, jedoch ist die maximale magnetische Flussdichte für die weitere Auswertung entscheidend.
Die gemessenen Flussdichten sind in \autoref{fig:BFeld} dargestellt.
Der maximale Wert beläuft sich auf $B_{\text{max}} = \SI{405}{\milli\tesla}$. 
Vor dieser Messung wurde am Gleichstromnetzteil eine Stromstärke $\SI{10}{\ampere}$ eingestellt, jedoch fiel die Stromstärke für diese Messung auf $\SI{9.8}{\ampere}$
und der vorherige Wert konnte nicht erneut erreicht werden.
Aus diesem Grund wird die Stärke des Magnetfeldes im Folgenden um einem Faktor von $1,02$ korrigiert.

\begin{figure}[H]
    \centering
    \includegraphics{build/BFeld.pdf}
    \caption{Gemessene Stärke des verwendeten Magnetfeldes in Abhängigkeit der Sondenposition.}
    \label{fig:BFeld}
\end{figure}

\subsection{Faraday-Effekt}

Die entscheidenen Eigenschaften der untersuchten Proben sind in \autoref{tab:Probeneigenschaften} dargestellt.

\begin{table}[H]
    \centering
    \caption{Dicke und Dotierungsstärke der verwendeten Proben.}
    \label{tab:Probeneigenschaften}
    \begin{tabular}{c c c c}
    \toprule
      {-} & {Probe 1} & {Probe 2} & {Probe 3} \\
    \midrule
          {Dotierung $N \mathbin{/} \unit{\centi\meter^{-3}}$}  &  $2.8 \cdot 10^{18}$ & $1.2 \cdot 10^{18}$ &   {-}    \\    
          {Dicke     $L \mathbin{/} \unit{\milli\meter^{-3}}$}  &  $1.296$             & $1.36$              &   $ 5.11$\\    
    \bottomrule
    \end{tabular}
\end{table}

Die gemessenen Rotationswinkel für alle drei Proben sind in \autoref{tab:Winkeldaten} dargestellt.
Diese Winkel werden mit $\alpha_{\text{rad}} = \alpha_{\text{deg}} \cdot \frac{\pi}{\SI{180}{\degree}}$ vom Gradmaß in das dimensionslose Bogenmaß umgerechnet.
Die umgerechneten Winkel sind in \autoref{tab:WinkelinBogenmaß} abgebildet.
Zunächst wird der normierte Drehwinkel nach  
\begin{equation*}
    \theta_{\text{frei}} = \frac{1}{2 L}\left(\theta_1 - \theta_2\right) 
\end{equation*}
berechnet, wobei $L$ die Länge der Probe ist und $\theta_1$ und $\theta_2$ die Drehwinkel repräsentieren.
Diese Grafik ist in \autoref{fig:Winkel_nom} dargestellt. 

Nun wird der normierte Winkel der Faraday-Rotation der undotierten Probe von den normierten Winkel der dotierten Proben abgezogen, um  das Signal der quasi freien Elektronen zu erhalten.
Die Differenz in zwischen der undotierten und der dotierten Probe ist in \autoref{fig:Winkel_nom_diff} aufgetragen.\\

Jetzt wird die x-Achse um den Faktor $\lambda$ skaliert, damit einer lineare Regression nach der Form 

\begin{equation*}
    \theta_{\text{frei}} = \alpha \cdot \lambda^2 \cdot \frac{N B}{n} + \beta \cdot \frac{N B}{n}\, .
\end{equation*}
durchgeführt werden kann.

Die freien Parametern werden zu %Zeilenumbruch
\begin{equation*}
    \alpha_1 = \SI{-36.45 \pm 0.48}\cdot 10^{-12} \unit{\frac{1}{\meter³\tesla}} \,, \, \beta_1 = \SI{426.6 \pm 2.1}\cdot 10^{-24} \, \unit{\frac{1}{\meter\tesla}} \\ 
\end{equation*}
und
\begin{equation*}
    \alpha_2 = \SI{-213.6 \pm 1.1}\cdot 10^{-12} \unit{\frac{1}{\meter³\tesla}} \,, \,  \beta_2 = \SI{1764.6  \pm 4.9}\cdot 10^{-24} \, \unit{\frac{1}{\meter\tesla}}\, .  \\
\end{equation*}

Diese Gleichung wird nun mit \eqref{eq:theta_frei} verglichen.
Aus dem Parameter $\alpha$ kann nun die effektive Masse freier Ladungsträger nach der Gleichung
\begin{equation*}
    \sqrt{\frac{8 \pi² \epsilon_0 c³}{e³}} \cdot \frac{1}{\sqrt{\alpha}} = m^*
\end{equation*} 
berechnet werden.

Die berechnete Masse beläuft sich somit auf 
\begin{equation*}
    m^*_1 = \SI{773.8 \pm 5.1}{\cdot 10^{-34}}\,\si{\kilo\gram}\, \\
\end{equation*}
und
\begin{equation*}
    m^*_2 = \SI{3196.5 \pm 8.4}{\cdot 10^{-35}}\,\si{\kilo\gram} \,.   \\
\end{equation*}

\begin{figure}[H]
    \centering
    \includegraphics{build/winkelnom.pdf}
    \caption{Normierter Rotationswinkel gegen die Wellenlänge $\lambda$ zusammen mit einer linearen Regression der untersuchten Wellen.}
    \label{fig:Winkel_nom}
\end{figure}


\begin{figure}[H]
    \centering
    \includegraphics{build/diffwinkelnom.pdf}
    \caption{Differenz zwischen dem Faraday-Winkel der undotierten und dotierten Probe gegen eine lineare x-Achse.}
    \label{fig:Winkel_nom_diff}
\end{figure}


\begin{figure}[H]
    \centering
    \includegraphics{build/diffwinkelnom2.pdf}
    \caption{Differenz zwischen dem Faraday-Winkel der undotierten und dotierten Probe gegen eine skalierte x-Achse.}
    \label{fig:Winkel_nom_diff2}
\end{figure}


\begin{table}[H]
    \centering
    \caption{Messwerte für die gemessenen Rotationswinkel für beide Polrichtungen des Magnetfeldes für drei verschiedene Proben Galliumarsenid im Gradmaß.}
    \label{tab:Winkeldaten}
    \begin{tabular}{S[table-format=1.3] | c c |c c | c c}
    \toprule
      {$\lambda \mathbin{/} \unit{\micro\meter}$ } & {$\theta^{1_\text{deg}}_1$} & {$\theta^{1_\text{deg}}_2$} & {$\theta^{2_\text{deg}}_1$} & {$\theta^{2_\text{deg}}_2$} & {$\theta^{3_\text{deg}}_1$} & {$\theta^{3_\text{deg}}_2$} \\
    \midrule
    {$1.06$}    & {$61°00'$}&   {$83°54'$}&   {$78°32'$}&   {$83°45'$}&   {$69°30'$}&  {$32°58'$} \\
    {$1.24$}    & {$86°45'$}&   {$84°52'$}&   {$19°06'$}&   {$62°38'$}&   {$69°06'$}&  {$29°54'$} \\
    {$1.45$}    & {$71°20'$}&   {$85°03'$}&   {$77°09'$}&   {$40°05'$}&   {$44°55'$}&  {$23°52'$} \\
    {$1.72$}    & {$70°30'$}&   {$83°16'$}&   {$60°30'$}&   {$80°06'$}&   {$27°28'$}&  {$31°36'$} \\
    {$1.96$}    & {$67°28'$}&   {$81°50'$}&   {$69°40'$}&   {$39°11'$}&   {$29°11'$}&  {$27°08'$} \\
    {$2.156$}   & {$64°45'$}&   {$79°00'$}&   {$72°13'$}&   {$58°08'$}&   {$19°21'$}&  {$22°50'$} \\
    {$2.34$}    & {$44°10'$}&   {$53°38'$}&   {$40°54'$}&   {$20°46'$}&   {$20°55'$}&  {$24°11'$} \\
    {$2.510$}   & {$20°00'$}&   {$35°15'$}&   {$16°40'$}&   {$25°15'$}&   {$19°36'$}&  {$21°53'$} \\
    {$2.65$}    & {$61°20'$}&   {$76°01'$}&   {$57°44'$}&   {$64°45'$}&   {$11°30'$}&  {$35°03'$} \\
    \bottomrule
    \end{tabular}
\end{table}


\begin{table}[H]
    \centering
    \caption{Werte für die Rotationswinkel für beide Polrichtungen des Magnetfeldes für drei verschiedene Proben Galliumarsenid im Bogenmaß.}
    \label{tab:WinkelinBogenmaß}
    \begin{tabular}{S[table-format=1.3] | c c |c c | c c}
    \toprule
    {$\lambda \mathbin{/} \unit{\micro\meter}$ } & {$\theta^{1_\text{rad}}_1$} & {$\theta^{1_\text{rad}}_2$} & {$\theta^{2_\text{rad}}_1$} & {$\theta^{2_\text{rad}}_2$} & {$\theta^{3_\text{rad}}_1$} & {$\theta^{3_\text{rad}}_2$} \\
    \midrule
    {$1.06$}    & {1.0647}&   {1.4643}&   {1.3707}&   {1.4617}&   {1.2130}&  {0.5754} \\
    {$1.24$}    & {1.5141}&   {1.4812}&   {0.3334}&   {1.0932}&   {1.2060}&  {0.5219} \\
    {$1.45$}    & {1.2450}&   {1.4844}&   {1.3465}&   {0.6996}&   {0.7839}&  {0.4166} \\
    {$1.72$}    & {1.2305}&   {1.4533}&   {1.0559}&   {1.3980}&   {0.4794}&  {0.5515} \\
    {$1.96$}    & {1.1775}&   {1.4283}&   {1.2159}&   {0.6839}&   {0.5093}&  {0.4736} \\
    {$2.156$}   & {1.1301}&   {1.3788}&   {1.2604}&   {1.0146}&   {0.3377}&  {0.3985} \\
    {$2.34$}    & {0.7709}&   {0.9361}&   {0.7138}&   {0.3624}&   {0.3651}&  {0.4221} \\
    {$2.510$}   & {0.3491}&   {0.6152}&   {0.2909}&   {0.4407}&   {0.3421}&  {0.3819} \\
    {$2.65$}    & {1.0705}&   {1.3267}&   {1.0076}&   {1.1301}&   {0.2007}&  {0.6117} \\
    \bottomrule
    \end{tabular}
\end{table}

