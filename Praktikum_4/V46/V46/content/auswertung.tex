\section{Auswertung}
\label{sec:auswertung}

Die Fehlerfortpflanzung wird mithilfe des Paketes \textsc{uncertainties} \cite{uncertainties} durchgeführt.
Für die lineare Regression wird \textsc{scipy.curve\_fit} \cite{scipy} verwendet.
Grafiken werden mit \textsc{matplotlib} \cite{matplotlib} erstellt,.

\subsection{Stärke des Magnetfeldes}

Zunächst wird die Stärke des Magnetfeldes untersucht, welches durch den verwendeten Elektromagneten erzeugt wird.
Diese Messung wurde zwar zuletzt durchgeführt, jedoch ist maximale magnetische Flussdichte für die weitere Auswertung entscheidend.
Die gemessenen Flussdichten sind in \autoref{fig:BFeld}.
Der maximale Werte beläuft sich auf $B_{max} = \SI{405}{\milli\tesla}$. 
Vor dieser Messung wurde am Gleichstromnetzteil eine Stromstärke $\SI{10}{\ampere}$ eingestellt, jedoch fiel die Stromstärke für diese Messung auf $\SI{9.8}{\ampere}$
und der vorherige Wert konnte nicht erneut erreicht werden.
Aus diesem Grund wird die Stärke des Magnetfeldes im folgenden um einem Faktor von $1.02$ korrigiert.

\begin{figure}[H]
    \centering
    \includegraphics{build/BFeld.pdf}
    \caption{Gemessene Stärke des verwendeten Magnetfeldes in Abhängigkeit der Sondenposition.}
    \label{fig:BFeld}
\end{figure}

\subsection{Faraday-Effekt}

Die entscheidenen Eigenschaften der untersuchten Proben sind in \autoref{tab:Probeneigenschaften} dargestellt.

\begin{table}[H]
    \centering
    \caption{Dicke und Dotierungsstärke der verwendeten Proben.}
    \label{tab:Probeneigenschaften}
    \begin{tabular}{c c c c}
    \toprule
      {-} & {Probe 1} & {Probe 2} & {Probe 3} \\
    \midrule
          {Dotierung $N \mathbin{/} \unit{\centi\meter^{-3}}$}  &  $2.8 \cdot 10^{18}$ & $1.2 \cdot 10^{18}$ &   {-}    \\    
          {Dicke     $L \mathbin{/} \unit{\milli\meter^{-3}}$}  &  $1.296$             & $1.36$              &   $ 5.11$\\    
    \bottomrule
    \end{tabular}
\end{table}

Die gemessenen Rotationswinkel für alle drei Proben sind in \autoref{tab:Winkeldaten} dargestellt.
Zunächst wird der normierte Drehwinkel nach  $\theta_{\text{frei}} = \frac{1}{2 L}\left(\theta_1 - \theta_2\right)$ 
und gegen das Quadrat der untersuchte Wellenlänge aufgetragen.
Dieser Grafik ist in \autoref{fig:Winkel_nom} dargestellt. 
Die lineare hat die Form

\begin{equation*}
    \theta_{frei} = \alpha \cdot \lambda^2 \cdot \frac{N B}{n} + \beta \cdot \frac{N B}{n}\, .
\end{equation*}

Die freien Parametern werden zu 
\begin{align*}
    \alpha_1 = \SI{8.31 \pm 0.48}\cdot 10^{-13} \unit{\frac{1}{\meter²\tesla}} & \beta_1 = \SI{216.1 \pm 2.1}\cdot 10^{-24} \unit{\frac{1}{\tesla}} \, \text{und}\\ 
    \alpha_2 = \SI{-34.9 \pm 1.1}\cdot 10^{-12} \unit{\frac{1}{\meter²\tesla}} & \beta_2 = \SI{166.6  \pm 4.9}\cdot 10^{-24} \unit{\frac{1}{\tesla}} \.  \\
\end{align*}

Diese Gleichung wird nun mit \eqref{eq:theta_frei} verglichen.
Aus dem Parameter $\alpha$ kann nun die effektiv Masse freier Ladungsträger nach der Gleichung
\begin{equation*}
    \sqrt{\frac{8 \pi² \epsilon_0 c³}{e³}} = m^*
\end{equation*} 
berechnet werden.

Die berechnete Masse beläuft sich somit auf 
\begin{equation*}
    m^*_1 = \SI{2.873 \pm 0.084}{\kilo\gram} \cdot 10^{-31} \, \text{und} \\
    m^*_2 = \SI{1.403 \pm 0.023}{\kilo\gram} \cdot 10^{-31} \,.   \\
\end{equation*}

\begin{figure}[H]
    \centering
    \includegraphics{build/winkelnom.pdf}
    \caption{Normierter Rotationswinkel gegen das Quadrat zusammen mit einer linearen Regression der untersuchten Wellen..}
    \label{fig:Winkel_nom}
\end{figure}


\begin{table}[H]
    \centering
    \caption{Messwerte für die gemessenen Rotationswinkel für beide Polrichtungen des Magnetfeldes für drei verschiedene Proben Galliumarsenid.}
    \label{tab:Winkeldaten}
    \begin{tabular}{S[table-format=1.3] | c c |c c | c c}
    \toprule
      {$\lambda \mathbin{/} \unit{\micro\meter}$ } & {$\theta^1_1$} & {$\theta^1_2$} & {$\theta^2_1$} & {$\theta^2_2$} & {$\theta^3_1$} & {$\theta^3_2$} \\
    \midrule
    {$1.06$}    & {$61°00'$}&   {$82°54'$}&   {$78°32'$}&   {$83°45'$}&   {$69°30'$}&  {$32°59'$} \\
    {$1.24$}    & {$86°45'$}&   {$84°52'$}&   {$19°06'$}&   {$62°38'$}&   {$69°06'$}&  {$29°54'$} \\
    {$1.45$}    & {$71°20'$}&   {$85°03'$}&   {$77°09'$}&   {$40°05'$}&   {$44°55'$}&  {$23°52'$} \\
    {$1.72$}    & {$70°30'$}&   {$83°16'$}&   {$60°30'$}&   {$80°06'$}&   {$27°28'$}&  {$31°36'$} \\
    {$1.96$}    & {$67°28'$}&   {$81°50'$}&   {$69°40'$}&   {$39°11'$}&   {$29°11'$}&  {$29°08'$} \\
    {$2.156$}   & {$64°45'$}&   {$79°00'$}&   {$72°13'$}&   {$58°08'$}&   {$19°21'$}&  {$22°50'$} \\
    {$2.34$}    & {$44°10'$}&   {$53°38'$}&   {$16°40'$}&   {$20°46'$}&   {$20°55'$}&  {$24°11'$} \\
    {$2.510$}   & {$20°00'$}&   {$35°15'$}&   {$16°40'$}&   {$25°15'$}&   {$19°36'$}&  {$21°53'$} \\
    {$2.65$}    & {$61°20'$}&   {$76°01'$}&   {$57°44'$}&   {$64°45'$}&   {$11°30'$}&  {$35°03'$} \\
    \bottomrule
    \end{tabular}
\end{table}
