\section{Auswertung}
\label{sec:auswertung}

\subsection{Bestimmung des Hintergrund}

Zunächst wird die Hintergrundrate des Versuchsaufbaus bestimmt. Dafür wird eine Messung durchgeführt, bei der sich keine Probe in der Apertur befindet.
Die Länge dieser Messung beträgt $\approx \SI{24.15}{\hour}$, hat insgesamt $294337$ Signale registriert und ist in \autoref{fig:Hintergrund} abgebildet. 
Es ist zu erkennen, dass bei sehr hohen Kanälen der Hintergrund dauerhaft ansteigt. Da dies unabhängig von der verwendeten Probe ist werden im folgenden alle Kanäle über $8000$ auf null gesetzt.
Der gemessene Hintergrund wird bei den untersuchten Proben auf die jeweiligen Messzeiten umgerechnet und von den gemessenen Signalen abgezogen.  

\begin{figure}[H]
    \centering
    \includegraphics[width=\textwidth]{build/Hintergrund.pdf}
    \caption{Hintergrund gemessen über einen längeren Zeitraum.}
    \label{fig:Hintergrund}
\end{figure}


\subsection{Kalibrierung der Energie}

Den Kanälen wird eine Energie zugeordnet, indem ein Strahler in die Apertur eingesetzt wird, der bei bekannten Energien strahlt. 
Danach wird aus den Abständen zwischen den Emissionslinien die Energie der einzelnen Kanälen bestimmt.  
Zur Kalibrierung wird $\text{Eu}-152$ verwendet.
Ein Histogram der Messung ist in \autoref{fig:Europium} abgebildet.

\begin{figure}[H]
    \centering
    \includegraphics[width=\textwidth]{build/Hintergrund.pdf}
    \caption{Histogram der Europiumsignalen.}
    \label{fig:Europium}
\end{figure}