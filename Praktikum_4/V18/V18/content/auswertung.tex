\section{Auswertung}
\label{sec:auswertung}

\subsection{Bestimmung des Hintergrundes}

Zunächst wird die Hintergrundrate des Versuchsaufbaus bestimmt. Dafür wird eine Messung durchgeführt, bei der sich keine Probe in der Apparatur befindet.
Die Länge dieser Messung beträgt $\approx \SI{24.15}{\hour}$, hat insgesamt $294337$ Signale registriert und ist in \autoref{fig:Hintergrund} abgebildet. 
Es ist zu erkennen, dass bei sehr hohen Kanälen der Hintergrund dauerhaft ansteigt. Da dies unabhängig von der verwendeten Probe ist werden im Folgenden alle Kanäle über $8000$ auf null gesetzt.
Der gemessene Hintergrund wird bei den untersuchten Proben auf die jeweiligen Messzeiten umgerechnet und von den gemessenen Signalen abgezogen.  

\begin{figure}[H]
    \centering
    \includegraphics[width=\textwidth]{build/Hintergrund.pdf}
    \caption{Hintergrund gemessen über einen längeren Zeitraum.}
    \label{fig:Hintergrund}
\end{figure}


\subsection{Kalibrierung der Energie}

Den Kanälen wird eine Energie zugeordnet, indem ein Strahler in die Apparatur eingesetzt wird, der bei bekannten Energien strahlt. 
Danach wird aus den Abständen zwischen den Emissionslinien die Energie der einzelnen Kanälen bestimmt.  
Zur Kalibrierung wird Eu-152 verwendet.
Ein Histogramm der Messung ist in \autoref{fig:Europium} abgebildet.
Dafür werden die Peaks im Spektrum über \texttt{scipy.signal.find\_peaks} \cite{scipy} bestimmt.
Der lokale Hintergrund wird über eine lineare Regression bestimmt, die durch nächsten zehn Kanäle, die jeweils links und rechts vom Peak liegen, bestimmt.
Danach wird über alle Kanäle, die den Peak bilden, summiert.

Um die Energie des Spektrums zu bestimmen werden die gemessenen Emissionslinien den aus der Literatur \cite{LNHB} bekannten Linien zugeordnet.
Die den Peaks zugeordneten Kanäle sind in \autoref{tab:EmissionsAlignment} eingetragen.
Das Maximum mit der Kanalnummer 594 hat mit Abstand den größten Linieninhalt und deswegen wird diesem die Emissionsenergie $\SI{121.7817 \pm 0.0003}{}$ mit der größten Emissionswahrscheinlichkeit zugeordnet.
Daraus folgt, dass dem Peak mit der Kanalnummer 201 keine Spektrallinie zugeordnet werden kann, da Europium keine Emissionen mit weniger Energie besitzt.

Um die Energie der anderen Kanäle zu bestimmen, wird eine lineare Regression $\text{Energie} = \alpha \cdot \text{Kanal} + \beta$ zwischen den Kanälen und der dazugehörigen Energie durchgeführt.
Diese Regression ist in \autoref{fig:EnergieKanal} zu sehen und die Parameter werden als

\begin{equation*}
    \alpha = \SI{0.207306 \pm 0.000049}{\frac{\kilo\eV}{\text{Kanal}}} \quad  \text{und} 
\end{equation*}
\begin{equation*}
    \beta = \SI{-1.17 \pm 0.18}{\kilo\eV}
\end{equation*}
bestimmt.


\begin{figure}[H]
    \centering
    \includegraphics[width=\textwidth]{build/Europium.pdf}
    \caption{Histogramm der Europiumsignalen.}
    \label{fig:Europium}
\end{figure}



\begin{table}
    \centering
    \caption{Kanalnummer, Energie, Emissionswahrscheinlichkeit $W$, Spektrallinieninhalt $Z$ und der Detektoreffizienz $Q$ von Europium.}
    \label{tab:EmissionsAlignment}
    \begin{tabular}{c S[table-format=4.4(2), separate-uncertainty] S[table-format=2.3(2)] S[table-format=4.0(3)] S[table-format=2.2(4)]}
        \toprule
        Kanalnummer & {$E \mathbin{/} \si{\kilo\eV}$} & {$W \mathbin{/} \%$} & {Linieninhalt} & {$Q \mathbin{/} \%$ }\\
        \midrule           
        594  & 121.7817(3) & 28.41(13) & 7736(351)  & 48.67(2239) \\ 
        1187 & 244.6974(8) & 7.55(4)   & 1376(151)  & 32.59(1764) \\ 
        1666 & 344.2785(12)& 26.59(12) & 3134(238)  & 21.07(965) \\ 
        1780 & 367.7891(20)& 0.862(5)  & 153 (63)   & 31.84(18513) \\ 
        1987 & 411.1165(12)& 2.238(10) & 206 (63)   & 16.46(7371) \\ 
        2146 & 443.965(3)  & 2.80(2)   & 294 (81)   & 18.83(1442) \\ 
        3763 & 778.9045(24)& 12.97(6)  & 709 (159)  & 9.77(503) \\ 
        4190 & 867.380(3)  & 4.243(23) & 217 (85)   & 9.18(4990) \\ 
        4656 & 964.079(18) & 14.50(6)  & 626 (149)  & 7.72(369) \\ 
        5247 & 1085.837(10)& 10.13(6)  & 419 (122)  & 7.41(489) \\ 
        5368 & 1112.076(30)& 13.41(6)  & 468 (130)  & 6.25(330) \\ 
        6797 & 1408.013(30)& 20.85(8)  & 611 (160)  & 5.25(244) \\ 
        \bottomrule
    \end{tabular}
\end{table}


\begin{figure}[H]
    \centering
    \includegraphics[width=\textwidth]{build/EnergieKanäle.pdf}
    \caption{Die Energieabhängigkeit der Messkanäle.}
    \label{fig:EnergieKanal}
\end{figure}


\subsection{Bestimmung der Detektoreffizienz}
\label{subsec:DetektorEffiSec}

Die Effizienz des Germaniumsdetektors kann über die Gleichung 

\begin{equation}
    Q = \frac{4 \pi \cdot Z}{A \cdot W \cdot t \cdot \Omega}\,
    \label{eq:DetektorEffi}
\end{equation}
bestimmt werden, wobei $Z$ der Spektrallinieninhalt, $A$ die Aktivität, $W$ die Emissionswahrscheinlichkeit, $t$ die Messzeit und $\Omega$ der Raumwinkel.
Der Raumwinkel kann mit der Gleichung

\begin{equation*}
    \Omega = 2 \pi\left(1 - \frac{a}{\sqrt{a² + r²}}\right)   \,,
\end{equation*}

wobei $a$ der Abstand zwischen Quelle und Detektor ist und $r$ der Radius des Detektors.
Der Radius des Detektors beträgt $\SI{22.5}{\milli\meter}$. 
Der Abstand $a$ setzt sich aus dem Abstand zwischen der Quelle und der Aluminiumhülle und dem Abstand zwischen Aluminiumhülle und Detektor, der mit $\SI{15}{\milli\meter}$ gegeben ist, zusammen.
Daraus ergibt sich ein Gesamtabstand von $a = \SI{86}{\milli\meter} = \SI{71}{\milli\meter} + \SI{15}{\milli\meter}$ und ein Raumwinkel von 

\begin{equation*}
    \Omega = 0.2046
\end{equation*}


Die Aktivität der Quelle ist mit $\SI{4130 \pm 60}{\becquerel}$ am 01.10.2000 gegeben.
Diese Aktivität folgt dem Zerfallsgesetz, was zusammen mit einer Halbwertszeit von $T_{1/2} = \SI{13.522 \pm 0.016}{\year}$, eine Aktivität von $A = \SI{1211 \pm 17}{\becquerel}$ am Tage des Experimentes bedeutet.
Die berechnete Detektoreffizienz kann in \autoref{tab:EmissionsAlignment} abgelesen werden.


Die Effizienz für die anderen Kanäle kann über eine Exponentialfunktion der Form $Q(E) = \alpha \cdot \left({\frac{E}{\unit{\kilo\eV}}}\right)^{\beta}$ bestimmt werden.
Die freien Parameter werden als

\begin{align*}
    \alpha &= \SI{24 \pm 10}{\frac{1}{\kilo\eV}} \quad  \text{und} \\
    \beta  &= \num{-0.799 \pm 0.078}\\
\end{align*}
%\begin{equation*}
%
%\end{equation*}

bestimmt.
Der zugehörige Graph kann in \autoref{fig:EffKanal} betrachtet werden.


\begin{figure}[H]
    \centering
    \includegraphics[width=\textwidth]{build/EffizienzKanal.pdf}
    \caption{Berechnete Detektoreffizienz für den betrachten Energiebereich.}
    \label{fig:EffKanal}
\end{figure}

\subsection{Untersuchung eines monochromatischen Gammastrahlers}

Der untersuchte $\gamma$-Strahler ist $Cs$-137.
Das gemessene Spektrum ohne Hintergrund ist in \autoref{fig:CeasiumSpektrum} dargestellt.\\
Zunächst wird der Photopeak untersucht.
Die Spitze des Peaks liegt im Kanal $3196$, was einer Energie von $\SI{661.383 \pm 0.093}{\kilo\eV}$ entspricht.
Insgesamt werden $\num{9139 \pm 476}$ Signale dem Photopeak zugeordnet.
Der Photopeak ist näher in \autoref{fig:CaesiumPhotopeak} abgebildet. 
Dieser Photopeak wird mit einer Gaußverteilung der Form

\begin{equation*}
    f(E) = \frac{\alpha}{\sqrt{2 \pi \sigma²}} \cdot e^{-\frac{(E-\mu)²}{2 \sigma²}}
\end{equation*}
genähert, wobei die Parameter als 

\begin{align*}
    \alpha  &= \SI{1878.3 \pm 1.0}{\frac{1}{\kilo\eV}} \,, \\
    \mu     &= \SI{661.23702 \pm 0.00062}{\kilo\eV}   \quad  \text{und}             \\
    \sigma  &= \SI{0.97609 \pm 0.00062}{\kilo\eV}\\
\end{align*}
bestimmt.

Die volle Breite bei halber Höhe wird als

\begin{equation*}
    E_{\text{FWHM}} =  \SI{2.30}{\kilo\eV}
\end{equation*}

und die volle Breite bei zehntel Höhe als

\begin{equation*}
    E_{\text{FWZM}} =  \SI{4.19}{\kilo\eV}
\end{equation*}

bestimmt. 

Das bedeutet in Verhältnis von 
\begin{equation*}
    \frac{E_{\text{FWZM}}}{E_{\text{FWHM}}} = 1.82\,.
\end{equation*}

Im Weiteren wird das Comptonkontinuum näher untersucht.
Aus dem bereits experimentell bestimmten Comptonpeak kann num mit \autoref{eq:Comptonpeak} und \autoref{eq:Rückstreupeak} %Formel in der Theorie einfügen 
 die Comptonkante und der Rückstreupeak berechnet werden.
Beide Werte sind zusammen mit den theoretischen Werten in \autoref{tab:CaesiumEnergien} angegeben.

\begin{table}
    \centering
    \caption{Theoretisch und experimentell bestimmte Energien des Photopeak, der Comptonkante und des Rückstreupeaks von Cs-137. Die theoretischen Werte werden mithilfe von \cite{LNHB} bestimmt.}
    \label{tab:CaesiumEnergien}
    \begin{tabular}{l S[table-format=3.2(4), separate-uncertainty] S[table-format=3.4(4), separate-uncertainty]}
        \toprule
        {-} & {Experiment} & {Theorie}\\
        \midrule 
        Photopeak       & $\SI{661.23 \pm 0.23 }{\kilo\eV}$ & $\SI{661.6553 \pm 0.030}{\kilo\eV}$  \\ 
        Comptonkante    & $\SI{476.95 \pm 0.39 }{\kilo\eV}$ & $\SI{477.332 \pm 0.028}{\kilo\eV}$    \\ 
        Rückstreupeaks  & $\SI{184.290 \pm 0.033 }{\kilo\eV}$ & $\SI{184.3228 \pm 0.0023}{\kilo\eV}$  \\ 
        \bottomrule
    \end{tabular}
\end{table}

Das Comptonkontinuum kann über \autoref{eq:Comptonkontinuum} %Auch in der Theorie einfügen.
genähert werden. 
Da die Comptonkante verschmiert ist, wird die Regression zwischen dem Wendepunkt der Comptonkante und der Energie vor Beginn des Rückstreupeaks durchgeführt und dann zu kleineren Energien verlängert.
Der dazugehörigen Graph ist in \autoref{fig:Comptonplot} abgebildet. 
Bei einer Energie des Comptonpeak von $\SI{661.6553}{\kilo\eV}$ wurde der Vorfaktor als $\num{3.755 \pm 0.090}$ bestimmt.
Über diese Ausgleichsfunktion kann auch der Linieninhalt des Comptonkontinuum bestimmt werden.
Dafür wird über die Funktion von $E = 0$ bis zu $E = \SI{527.46}{\kilo\eV}$ integriert.
Aus dieser Integration folgt, dass im Comptonkontinuum insgesamt $\num{4369}$ Signale enthalten sind.\\ 

Wie in \autoref{eq:Absorptionswahrscheinlichkeit} kann die Wahrscheinlichkeit $P$ mit der ein $\gamma$-Quant absorbiert wird mit der Gleichung $P(l) = 1- \exp{(-\mu \cdot l)}$ beschrieben,
dabei ist $l$ die Länge des Detektors und beträgt $l = \SI{3.9}{\centi\meter}$ \cite{v18}.
$\mu$ ist der Absorptionskoeffizient, welcher aus \autoref{fig:Absorptionskoeffizient} abgelesen werden kann. 
Die Absorptionskoeffizient für den Photopeak und die Comptonkante betragen

\begin{equation*}
   \mu_\text{Photo} = \SI{0.007}{\frac{1}{\centi\meter}} \quad \text{und}
\end{equation*}
\begin{equation*}
    \mu_\text{Compton} = \SI{0.37}{\frac{1}{\centi\meter}}\,. 
\end{equation*}

Mit \autoref{eq:Absorptionswahrscheinlichkeit} ergeben sich Absorptionswahrscheinlichkeiten von

\begin{equation*}
    P_\text{Photo} = \num{0.027} \quad \text{und}
 \end{equation*}
\begin{equation*}
     P_\text{Compton} = \num{0.764}\,. 
\end{equation*}


\begin{figure}[H]
    \centering
    \includegraphics[width=\textwidth]{build/Caesium.pdf}
    \caption{Spektrum eines $Cs$-137 Strahlers.}
    \label{fig:CeasiumSpektrum}
\end{figure}

\begin{figure}[H]
    \centering
    \includegraphics[width=\textwidth]{build/Caesium_Photopeak.pdf}
    \caption{Photopeak der Cs-137 Quelle.}
    \label{fig:CaesiumPhotopeak}
\end{figure}


\begin{figure}[H]
    \centering
    \includegraphics[width=\textwidth]{build/CaesiumE.pdf}
    \caption{Graph des Comptonkontinuum bei Cs-137.}
    \label{fig:Comptonplot}
\end{figure}

\subsection{Bestimmung der Aktivität von Co-60}

Das Energiespektrum einer Co-60 Probe ist in \autoref{fig:Co60} abgebildet.
Zur Bestimmung der Quellenaktivität wird die \autoref{eq:DetektorEffi} zur Aktivität $A$ umgestellt.
Daraus ergibt sich
\begin{equation*}
    A = \frac{4\pi \cdot Z}{Q \cdot W \cdot t \cdot \Omega}\,.
\end{equation*}
Der Raumwinkel beträgt wieder $\Omega = 0.2046$. 
und die Messzeit beläuft sich auf $\SI{4324}{\second}$.
Der Linieninhalt wird analog wie in \autoref{subsec:DetektorEffiSec} berechnet und ist zusammen mit der Emissionswahrscheinlichkeit aus \cite{LNHB} in \autoref{tab:Co60} gegeben.
Für die Aktivität ergibt sich ein Mittelwert von $A_\text{Co60} = \SI{124 \pm 22}{\becquerel}$

\begin{table}
    \centering
    \caption{Kanalnummer, Energie, Emissionswahrscheinlichkeit $W$, Spektrallinieninhalt $Z$ und die Aktivität von Co-60.}
    \label{tab:Co60}
    \begin{tabular}{c S[table-format=4.3(1), separate-uncertainty] S[table-format=3.4(1)] S[table-format=3.0(3)] S[table-format=2.0(4)]}
        \toprule
        Kanalnummer & {$E \mathbin{/} \si{\kilo\eV}$} & {$W \mathbin{/} \%$} & {Linieninhalt} & {$A \mathbin{/} \unit{\becquerel}$ }\\
        \midrule           
        5662 &  1332.492(4) 	 & 99.9826(6)   & 734(152)  & 124(30) \\ 
        6430 &  1173.228(3) 	 & 99.85(3)     & 656(153)  & 123(33) \\ 
        \bottomrule
    \end{tabular}
\end{table}



\begin{figure}[H]
    \centering
    \includegraphics[width=\textwidth]{build/Cobalt60.pdf}
    \caption{Energiespektrum einer Co-60 Probe.}
    \label{fig:Co60}
\end{figure}

\subsection{Untersuchung der Zerfallskette von Uranophan}

In dieser Messung wird ein Gestein als Quelle verwendet, welches das Mineral Uranophan $\ce{Ca[UO2|SiO3OH]2 5H2O}$ enthält.
Das aufgenommene Spektrum ist in \autoref{fig:Uranophan} zu sehen.
Alle Peaks, die identifiziert werden konnten, sind in \autoref{tab:Uranophan} aufgelistet.
Daneben sind die Isotope eingetragen, zu denen diese Linie zugeordnet wird, dabei werden ausschließlich Isotope aus der Zerfallskette von U-238 betrachtet.  
Es werden auch Peaks im Spektrum gefunden, die keinen Isotopen zugeordnet werden können.
Zusätzlich besitzt Th-234 zwei Emissionslinien $E_1 = \SI{92.38 \pm 0.01}{\kilo\eV}$ und $E_2 = \SI{92.80 \pm 0.02}{\kilo\eV}$, die bei der Analyse des Spektrums nicht voneinander getrennt werden können.
Aus diesem Grund werden beide Emissionswahrscheinlichkeit zusammengerechnet.
%Außerdem treten in der Untersuchten Zerfallsreihe zwei Emissionslinien auf die nahe beieinander liegen und zu unterschiedlichen Isotopen gehören.
%Dazu gehören $ \SI{186.211 \pm 0.013}{\kilo\eV}$ von Ra-226 und $\SI{186.15 \pm 0.02}{\kilo\eV}$ von Pa-234.
%Der Linieninhalt der $\SI{186.03 \pm 0.15}{\kilo\eV}$ wird nach dem Verhältnis der Emissionswahrscheinlichkeiten aufgeteilt.
Die Mittelwerte der Aktivitäten sind in \autoref{tab:Aktivitäten} dargestellt.


\begin{table}
    \centering
    \caption{Experimentelle und theoretische Energien, Emissionswahrscheinlichkeiten, Isotop, Linieninhalt und Aktivität Emissionslinien aus dem Uranophanspektrum.}
    \label{tab:Uranophan}
    \begin{tabular}{S[table-format=4.3(4), separate-uncertainty] S[table-format=4.3(1), separate-uncertainty] S[table-format=3.4(1)] c S[table-format=5.0(3)] S[table-format=5.0(4)]}
        \toprule
        {$E_\text{Exp} \mathbin{/} \si{\kilo\eV}$} & {$E_\text{Theo} \mathbin{/} \si{\kilo\eV}$} & {$W \mathbin{/} \%$} & {Isotop} & {Linieninhalt} & {$A \mathbin{/} \unit{\becquerel}$ }\\
        \midrule           
        77.40(0.17)     &   {-}          & {-}          & {-}         &  22922(822)  & {-}           \\
        92.74(0.17)     &   92.80 (2)    & 4.33 (29)    & {Th-234}    &  6939(284)   & 5637(670)   \\
        186.03(0.15)    &   186.211 (13) & 1.78 (19)    & {Ra-226}    &  12247(404)  & 42215(5278)   \\
        242.00(0.14)    &   241.997 (3)  & 7.268 (22)   & {Pb-214}    &  11466(390)  & 11944(759)    \\
        295.07(0.13)    &   295.224 (2)  & 18.414 (36)  & {Pb-214}    &  23016(591)  & 11087(691)    \\
        351.876(0.120)  &    351.932 (2) &  35.60 (7)   & {Pb-214}    &  38556(866)  & 11058(738)    \\
        608.728(0.094)  &    609.312 (7) & 45.49(19)    & {Bi-214}    &  23906(719)  & 8315(811)     \\
        664.700(0.092)  &   665.453 (22) & 1.530 (7)    & {Bi-214}    &  629(104)    & 6989(1345)    \\
        701.808(0.092)  &         {-}    &   {-}        &   {-}       &  176(53)     & {-}           \\
        741.196(0.093)  &        {-}     &   {-}        &   {-}       &  241(73)     & {-}           \\
        767.109(0.094)  &   768.356 (10) & 4.892 (16)   &  {Bi-214}   &  2117(234)   & 8238(1275)    \\
        784.938(0.095)  &    785.96 (9)  &  1.064 (13)  &   {Pb-214}  &  440(86)     & 8030(1812)    \\
        805.66(0.10)    &   805.80 (5)   &   2.5 (3)    &   {Pa-234}  &  471(98)     & 3735(992)     \\
        933.16(0.11)    &    934.061 (12)&   3.10 (1)   &   {Bi-214}  &  997(154)    & 7161(1408)    \\
        999.29(0.12)    &     {-}        &    {-}       &      {-}    &  247(71)     & {-}           \\
        1119.32(0.14)   & 1 120.287 (10) &  14.91 (3)   &   {Bi-214}  &  4201(366)   & 7255(1162)    \\
        1154.56(0.14)   & 1 155.19 (2)   &   1.635 (7)  &   {Bi-214}  &  402(101)    & 6497(1866)    \\
        1180.89(0.15)   &      {-}       &    {-}       &       {-}   &  59(27)      & {-}           \\
        1237.28(0.16)   &1 238.111 (12)  &  5.831 (14)  &   {Bi-214}  &  1426(210)   & 6822(1397)    \\
        1278.74(0.17)   &1 280.96 (2)    & 1.435 (6)    &   {Bi-214}  &  266(77)     & 5325(1746)    \\
        1334.71(0.18)   &    {-}         &       {-}    &      {-}    &  50(20)      & {-}           \\
        1376.38(0.19)   & 1 377.669 (12) &3.968 (11)    &   {Bi-214}  &  958(184)    & 7337(1790)    \\
        1385.29(0.19)   &      {-}       &      {-}     &      {-}    &  62(22)      & {-}           \\
        1406.65(0.19)   & 1 407.98 (4)   & 2.389 (8)    &  {Bi-214}   &  384(110)    & 4973(1619)    \\
        1509.06(0.22)   &1 509.228 (15)  & 2.128 (10)   &  {Bi-214}   &  479(138)    & 7360(2417)    \\
        1536.42(0.22)   &      {-}       &      {-}     &    {-}      &  30(19)      & {-}           \\
        1582.24(0.23)   &      {-}       &    {-}       &      {-}    &  66(30)      & {-}           \\
        \bottomrule
    \end{tabular}
\end{table}


\begin{table}
    \centering
    \caption{Bestimmte Isotope und ihre Aktivitäten.}
    \label{tab:Aktivitäten}
    \begin{tabular}{c S[table-format=5.0(4)]}
        \toprule
        {Isotop} & {$A \mathbin{/} \unit{\becquerel}$}\\
        \midrule           
                {Th-234} & 5637(670)\\
                {Ra-226} & 42214(5279)\\
                {Pb-214} & 10530(1486)\\
                {Bi-214} & 7025(994)\\
                {Pa-234} & 3735(992)\\
                \bottomrule
    \end{tabular}
\end{table}

\begin{figure}[H]
    \centering
    \includegraphics[width=\textwidth]{build/Uranophan.pdf}
    \caption{Energiespektrum einer Uranophanquelle.}
    \label{fig:Uranophan}
\end{figure}




