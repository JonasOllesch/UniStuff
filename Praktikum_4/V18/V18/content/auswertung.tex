\section{Auswertung}
\label{sec:auswertung}

\subsection{Bestimmung des Hintergrund}

Zunächst wird die Hintergrundrate des Versuchsaufbaus bestimmt. Dafür wird eine Messung durchgeführt, bei der sich keine Probe in der Apertur befindet.
Die Länge dieser Messung beträgt $\approx \SI{24.15}{\hour}$, hat insgesamt $294337$ Signale registriert und ist in \autoref{fig:Hintergrund} abgebildet. 
Es ist zu erkennen, dass bei sehr hohen Kanälen der Hintergrund dauerhaft ansteigt. Da dies unabhängig von der verwendeten Probe ist werden im folgenden alle Kanäle über $8000$ auf null gesetzt.
Der gemessene Hintergrund wird bei den untersuchten Proben auf die jeweiligen Messzeiten umgerechnet und von den gemessenen Signalen abgezogen.  

\begin{figure}[H]
    \centering
    \includegraphics[width=\textwidth]{build/Hintergrund.pdf}
    \caption{Hintergrund gemessen über einen längeren Zeitraum.}
    \label{fig:Hintergrund}
\end{figure}


\subsection{Kalibrierung der Energie}

Den Kanälen wird eine Energie zugeordnet, indem ein Strahler in die Apertur eingesetzt wird, der bei bekannten Energien strahlt. 
Danach wird aus den Abständen zwischen den Emissionslinien die Energie der einzelnen Kanälen bestimmt.  
Zur Kalibrierung wird Eu-152 verwendet.
Ein Histogram der Messung ist in \autoref{fig:Europium} abgebildet.
Dafür werden die Peaks im Spektrum über \texttt{scipy.signal.find\_peaks} \cite{scipy} bestimmt.
Der lokale Hintergrund wird über eine lineare Regression bestimmt, die durch nächsten zehn Kanäle, die jeweils links und rechts vom Peak liegen, bestimmt.
Danach wird über alle Kanäle die den Peak bilden summiert.

Um die Energie des Spektrums zu bestimmen werden die gemessenen Emissionslinien den aus der Literatur \cite{LNHB} bekannten Linien zugeordnet.
Die den Peaks zugeordneten Kanäle können in \autoref{tab:EmissionsAlignment}.
Das Maximum mit der Kanalnummer 594 hat mit Abstand den größten Linieninhalt und deswegen wird diesem die Emissionsenergie $\SI{121.7817 \pm 0.0003}{}$ mit der größten Emissionswahrscheinlichkeit zugeordnet.
Daraus folgt, dass dem Peak mit der Kanalnummer 201 keine Spektrallinie zugeordnet werden kann, da Europium keine Emissionen mit weniger Energie besitzt.

Um die Energie der anderen Kanäle zu bestimmen wird eine lineare Regression $\text{Energie} = \alpha \cdot \text{Kanal} + \beta$ zwischen den Kanälen und der dazugehörigen Energie durchgeführt.
Diese Regression ist in \autoref{fig:EnergieKanal} zu sehen und die Parameter werden als

\begin{equation*}
    \alpha = \SI{0.207306 \pm 0.000049}{\frac{\kilo\eV}{\text{Kanal}}} \quad  \text{und} 
\end{equation*}
\begin{equation*}
    \beta = \SI{-1.17 \pm 0.18}{\kilo\eV}
\end{equation*}
bestimmt.


\begin{figure}[H]
    \centering
    \includegraphics[width=\textwidth]{build/Europium.pdf}
    \caption{Histogram der Europiumsignalen.}
    \label{fig:Europium}
\end{figure}



\begin{table}
    \centering
    \caption{Kanalnummer, Energie, Emissionswahrscheinlichkeit $W$, Spektrallinieninhalt $Z$ und der Detektoreffizienz $Q$ von Europium.}
    \label{tab:EmissionsAlignment}
    \begin{tabular}{c S[table-format=4.4(2), separate-uncertainty] S[table-format=2.3(2)] S[table-format=4.0(3)] S[table-format=2.2(4)]}
        \toprule
        Kanalnummer & {$E \mathbin{/} \si{\kilo\eV}$} & {$W \mathbin{/} \%$} & {Linieninhalt} & {$Q \mathbin{/} \%$ }\\
        \midrule           
        594  & 121.7817(3) & 28.41(13) & 7736(351)  & 48.67(2239) \\ 
        1187 & 244.6974(8) & 7.55(4)   & 1376(151)  & 32.59(1764) \\ 
        1666 & 344.2785(12)& 26.59(12) & 3134(238)  & 21.07(965) \\ 
        1780 & 367.7891(20)& 0.862(5)  & 153 (63)   & 31.84(18513) \\ 
        1987 & 411.1165(12)& 2.238(10) & 206 (63)   & 16.46(7371) \\ 
        2146 & 443.965(3)  & 2.80(2)   & 294 (81)   & 18.83(1442) \\ 
        3763 & 778.9045(24)& 12.97(6)  & 709 (159)  & 9.77(503) \\ 
        4190 & 867.380(3)  & 4.243(23) & 217 (85)   & 9.18(4990) \\ 
        4656 & 964.079(18) & 14.50(6)  & 626 (149)  & 7.72(369) \\ 
        5247 & 1085.837(10)& 10.13(6)  & 419 (122)  & 7.41(489) \\ 
        5368 & 1112.076(30)& 13.41(6)  & 468 (130)  & 6.25(330) \\ 
        6797 & 1408.013(30)& 20.85(8)  & 611 (160)  & 5.25(244) \\ 
        \bottomrule
    \end{tabular}
\end{table}


\begin{figure}[H]
    \centering
    \includegraphics[width=\textwidth]{build/EnergieKanäle.pdf}
    \caption{Die Energieabhängigkeit der Messkanäle.}
    \label{fig:EnergieKanal}
\end{figure}


\subsection{Bestimmung der Detektoreffizienz}

Die Effizienz des Germaniumsdetektors kann über die Gleichung 

\begin{equation*}
    Q = \frac{Z}{A \cdot W \cdot t \cdot \Omega}\,,
\end{equation*}
wobei $Z$ der Spektrallinieninhalt, $A$ die Aktivität, $W$ die Emissionswahrscheinlichkeit, $t$ die Messzeit und $\Omega$ der Raumwinkel.
Der Raumwinkel kann mit der Gleichung

\begin{equation*}
    \Omega = 2 \pi\left(1 - \frac{a}{\sqrt{a² + r²}}\right)   \,,
\end{equation*}

wobei $a$ der Abstand zwischen Quelle und Detektor ist und $r$ der Radius des Detektors.
Der Radius des Detektors beträgt $\SI{22.5}{\milli\meter}$. 
Der Abstand $a$ setzt sich aus dem Abstand zwischen der Quelle und der Aluminiumhülle und dem Abstand zwischen Aluminiumhülle und Detektor, der mit $\SI{15}{\milli\meter}$ gegeben ist, zusammen.
Daraus ergibt sich ein Gesamtabstand von $a = \SI{86}{\milli\meter} = \SI{71}{\milli\meter} + \SI{15}{\milli\meter}$ und ein Raumwinkel von 

\begin{equation*}
    \Omega = 0.2046
\end{equation*}


Die Aktivität der Quelle ist mit $\SI{4130 \pm 60}{\becquerel}$ am 01.10.2000 gegeben.
Diese Aktivität folgt dem Zerfallsgesetz, was zusammen mit einer Halbwertszeit von $T_{1/2} = \SI{13.522 \pm 0.016}{\year}$, eine Aktivität von $A = \SI{1211 \pm 17}{\frac{1}{\second}}$ am Tage des Experimentes bedeutet.
Die berechnete Detektoreffizienz kann in \autoref{tab:EmissionsAlignment} abgelesen werden.


Die Effizienz für die anderen Kanäle kann über eine Exponentialfunktion der Form $Q(E) = \alpha \cdot {\frac{E}{\unit{\kilo\eV}}}^{\beta}$ bestimmt werden.
Die freien Parameter wurden zu 

\begin{align*}
    \alpha &= \SI{24 \pm 10}{\frac{1}{\kilo\eV}} \quad  \text{und} \\
    \beta  &= \num{-0.799 \pm 0.078}\\
\end{align*}
%\begin{equation*}
%
%\end{equation*}

bestimmt.
Der zugehörige Graph kann in \autoref{fig:EffKanal} betrachtet werden.


\begin{figure}[H]
    \centering
    \includegraphics[width=\textwidth]{build/EffizienzKanal.pdf}
    \caption{Berechnete Detektoreffizienz für den betrachten Energiebereich.}
    \label{fig:EffKanal}
\end{figure}

\subsection{Untersuchung eines monochromatischen Gammastrahlers}

Der untersuchte $\gamma$-Strahler ist $Cs$-137.
Das gemessene Spektrum ohne Hintergrund ist in \autoref{fig:CeasiumSpektrum} dargestellt.\\
Zunächst wird der Photopeak untersucht.
Die Spitze des Peaks liegt im Kanal $3196$, was einer Energie von $\SI{661.383 \pm 0.093}{\kilo\eV}$ entspricht.
Der Photopeak ist näher in \autoref{fig:CaesiumPhotopeak} abgebildet.
Dieser Photopeak wird mit einer Gaußverteilung der Form

\begin{equation*}
    f(E) = \frac{\alpha}{\sqrt{2 \pi \sigma²}} \cdot e^{-\frac{(E-\mu)²}{2 \sigma²}}
\end{equation*}
genähert, wobei die Parameter als 

\begin{align*}
    \alpha  &= \SI{1878.3 \pm 1.0}{\frac{1}{\kilo\eV}} \,, \\
    \mu     &= \SI{661.23702 \pm 0.00062}{\kilo\eV}   \quad  \text{und}             \\
    \sigma  &= \SI{0.97609 \pm 0.00062}{\kilo\eV}\\
\end{align*}
bestimmt.

Die volle Breite bei halber Höhe wird als

\begin{equation*}
    E_{\text{FWHM}} =  \SI{2.30}{\kilo\eV}
\end{equation*}

und die volle Breite bei zehntel Höhe als

\begin{equation*}
    E_{\text{FWZM}} =  \SI{4.19}{\kilo\eV}
\end{equation*}

bestimmt.

Das bedeutet in Verhältnis von 
\begin{equation*}
    \frac{E_{\text{FWZM}}}{E_{\text{FWHM}}} = 1.82\,.
\end{equation*}

Im weiteren wird das Comptonkontinuum näher untersucht.
Aus dem breits experimentell bestimmten Comptonpeak kann num mit \autoref{eq:Comptonpeak} und \autoref{eq:Rückstreupeak}  die Comptenkante und der Rückstreupeak berechnet werden.
Bei Werte sind zusammen mit den theoretischen Werten in Tabelle \autoref{tab:CaesiumEnergien} angegeben.

\begin{table}
    \centering
    \caption{Theoretisch und experimentell bestimmte Energien des Photopeak, der Comptonkante und des Rückstreupeaks von Cs-137. Die theoretischen Werte wurde mithilfe von \cite{LNHB} bestimmt.}
    \label{tab:CaesiumEnergien}
    \begin{tabular}{l S[table-format=3.2(4), separate-uncertainty] S[table-format=3.4(4), separate-uncertainty]}
        \toprule
        {-} & {Experiment} & {Theorie}\\
        \midrule 
        Photopeak       & $\SI{661.23 \pm 0.23 }{\kilo\eV}$ & $\SI{661.6553 \pm 0.030}{\kilo\eV}$  \\ 
        Comptonkante    & $\SI{476.95 \pm 0.39 }{\kilo\eV}$ & $\SI{477.332 \pm 0.028}{\kilo\eV}$    \\ 
        Rückstreupeaks  & $\SI{184.290 \pm 0.033 }{\kilo\eV}$ & $\SI{184.3228 \pm 0.0023}{\kilo\eV}$  \\ 
        \bottomrule
    \end{tabular}
\end{table}





\begin{figure}[H]
    \centering
    \includegraphics[width=\textwidth]{build/Caesium.pdf}
    \caption{Spektrum eines $Cs$-137 Strahlers.}
    \label{fig:CeasiumSpektrum}
\end{figure}

\begin{figure}[H]
    \centering
    \includegraphics[width=\textwidth]{build/Caesium_Photopeak.pdf}
    \caption{Photopeak der Cs-137 Quelle.}
    \label{fig:CaesiumPhotopeak}
\end{figure}
