\section{Versuchsaufbau und Durchführung}
\label{sec:Durchführung}

Der Versuch ist \autoref{fig:aufbau} entsprechend aufgebaut.

\begin{figure}[H]
    \centering
    \includegraphics{figures/setup.pdf}
    \caption{Foto des verwendeten Versuchsaufbaus einschließlich einer Grafik, die den konkreten Aufbau des Germaniumdetektors zeigt \cite{v18}.}
    \label{fig:aufbau}
\end{figure}

Der hier verwendete Detektor, bestehend aus einer n-dotierten Lithiumschicht an einem Germaniumkristall ist von einer schützenden Aluminiumhülle umhüllt.
Um diese Hülle zu durchdringen, benötigen die Photonen mindestens eine Energie von etwa $\SI{50}{\kilo\eV}$.
Der Detektor selbst befindet sich in einer großen schwarzen Box unterhalb der Befestigung für die Quelle, aber über dem Dewar,%keine Ahnung aus was die große schwarze Kiste besteht. In der Anleitung habe ich nichts gefunden, aber es könnte Aluminium sein.  
das zum Kühlen des Detektors verwendet wird.
Auf der linken Seite der Box, nicht mehr in \autoref{fig:aufbau} zu erkennen, befindet sich der Computer, über den das Programm zur Aufnahme
der Messdaten läuft.
Auf der anderen Seite befindet sich die Elektronik, über die einerseits eine Biasspannung von $\SI{5}{\kilo\volt}$ angelegt wird,
aber auch der Zählprozess abläuft.
Je nach Amplitude, also Energie des eingehenden Signals werden die Impulse mithilfe eines Multi Channel Analysers (MCA) in unterschiedliche
Kanäle sortiert. \\

Für die Messung selbst wird mit einem Metallstab der Abstand zwischen Quelle und Detektor vermessen.
Als erstes Element wird $^{152} Eu$ vermessen, um den Detektor zu kalibrieren.
Anschließend wird $^{137} Cs$ als monochromatischer und $^{60} Co$ als nicht-monochromatischer Gammastrahler vermessen.
Es folgt die Messung eines Steins, welcher das Mineral ''Uranophan'' enthält.  %Die Anführungszeichen haben die irgendwie zerschossen, die die waren sowieso unwissenschaftlich xD
All diese Messung laufen über einen Zeitraum von jeweils $45$ Minuten.
Abschließend wird die letzte Quelle aus dem Detektor entfernt und eine deutlich längere Hintergrundmessung über etwa einen Tag gestartet.
