\section{Diskussion}
\label{sec:Diskussion}


\subsection{Energiekalibrierung}

Die lineare Regression ergab den Zusammenhang $E(\text{Kanal}) = \SI{0.207306 \pm 0.000049}{\frac{\kilo\eV}{\text{Kanal}}} \cdot \text{Kanal} + \SI{-1.17 \pm 0.18}{\kilo\eV}$.
Diese Funktion hat einen negativen y-Achsenabschnitt, der nicht physikalisch ist, da der Detektor im negativen Bereich der Funktion Energie an die Umgebung abgeben würde.
Jedoch ist der Detektor nur für $\gamma$-Quanten mit einer Energie von über $\SI{50}{\kilo\eV}$ sensitiv \cite{v18}.
Aus diesem Grund wird der negative Bereich vernachlässigt.


\subsection{Detektoreffizienz}

Die Funktion für die Detektoreffizienz divergiert für kleine Energien.
Bei Energie, die kleiner als $\approx \SI{53.12}{\kilo\eV}$ sind ist die Detektoreffizienz größer als $1$, 
da solche kleinen Energien in diesem Versuch nicht beachtet werden, sollte dieser Effekt die Messung nicht beeinflussen.


\subsection{Monochromatischen Gammastrahlers}

Der Peak der monochromatischen Gammastrahlers Cs-137 wird mit einer Gaußverteilung genähert.
Bei einer Gaußverteilung hat die volle Breite bei halber Höhe ein festes Verhältnis zur vollen Breite bei zehntel Höhe von 1.823.
Für den Caesiumpeak wurde ein Verhältnis von $\frac{E_{\text{FWZM}}}{E_{\text{FWHM}}} = 1.82\,.$ berechnet.
Beide Werte stimmen angemessen überein. Somit wird bestätigt, dass es sich um eine Gaußverteilung handelt.


Es wird die experimentell bestimmten und theoretische berechneten Eigenschaften des Caesiumspektrum verglichen.
Die Werte sind in \autoref{tab:CaesiumEnergienVergleich} aufgetragen. Daraus ist zu erkennen, dass die experimentell bestimmten Werte sehr gut mit den Theoriewerten übereinstimmen.

\begin{table}
    \centering
    \caption{Theoretisch und experimentell bestimmte Energien des Photopeak sowie ihre relative Abweichungen, der Comptonkante und des Rückstreupeaks von Cs-137. Die theoretischen Werte werden mithilfe von \cite{LNHB} bestimmt.}
    \label{tab:CaesiumEnergienVergleich}
    \begin{tabular}{l S[table-format=3.2(4), separate-uncertainty] S[table-format=3.4(4), separate-uncertainty]  S[table-format=1.3(4)]}
        \toprule
        {-} & {Experiment} & {Theorie} & {Relative Abweichung $ \mathbin{/} \%$}\\
        \midrule 
        Photopeak       & $\SI{661.23 \pm 0.23 }{\kilo\eV}$ & $\SI{661.6553 \pm 0.030}{\kilo\eV}$      & 0.064(0.035) \\ 
        Comptonkante    & $\SI{476.95 \pm 0.39 }{\kilo\eV}$ & $\SI{477.332 \pm 0.028}{\kilo\eV}$       & 0.08(0.08) \\ 
        Rückstreupeaks  & $\SI{184.290 \pm 0.033 }{\kilo\eV}$ & $\SI{184.3228 \pm 0.0023}{\kilo\eV}$   & 0.018(0.018) \\ 
        \bottomrule
    \end{tabular}
\end{table}

Aus dem Verhältnis der Absorptionswahrscheinlichkeiten $P_\text{Photo} = \num{0.027}$ und $P_\text{Compton} = \num{0.764}$ 
ist ein Verhältnis zwischen Comptonkontinuum und Photopeak von $\frac{P_\text{Compton}}{P_\text{Photo}} \approx 28.30$ zu erwarten.
Es wurde jedoch ein Verhältnis von $\frac{P^{\text{Exp}}_\text{Compton}}{P^{\text{Exp}}_\text{Photo}} \approx 0.48$ bestimmt.
Für diese Diskrepanz kann keine Erklärung gefunden werden.


\subsection{Zerfallskette von Uranophan}

Im Spektrum der  Zerfallskette von Uranophan werden viele Maxima gefunden, die nicht eindeutig einem Isotop zugeordnet werden können,
jedoch ist deren Linieninhalt deutlich kleiner als bei den anderen Peaks, die zugeordnet werden können.
Das lässt den Schluss zu, dass es sich um Hintergrund handelt, der nicht herausgerechnet werden konnte.
Diese Erklärung ist nicht direkt für das Maximum bei $\SI{77.40 \pm 0.17}{\kilo\eV}$, da der Linieninhalt mit $\num{22 922 \pm 822}$ signifikant ist.

Weiter handelt es sich beim der Zerfallskette von U-238 im Vergleich zu den anderen Zerfallsketten um eine lange Kette.
Es könnten nicht alle Isotope der Zerfallskette identifiziert werden, da sie beim Zerfall keine $\gamma$-Emissionen erzeugen.
Das ist zum Beispiel bei U-238 selbst der Fall \cite{LNHB}. 
Anders als bei den anderen Proben handelt es sich bei dem Uranophan um ein natürliches Gestein.
In diesem Gestein könnten auch andere radioaktive Elemente enthalten sein, die nicht identifiziert werden können. 
