\section{Theory}
\label{sec:theorie}

To reasonably discuss the experiment's execution as well as later evaluate the measurements, some theoretical basics are needed.

\subsection{X-rays}

The term X-ray describes electromagnetic radiation with an energy of around $\SI{1}{\kilo\eV}$ to $\SI{100}{\kilo\eV}$ \cite{gamma}.
As shown in \autoref{fig:xraytube}, one possibility of producing such radiation is through the use of an X-ray tube.

\begin{figure}[H]
    \centering
    \includegraphics{figures/x-ray_tube.pdf}
    \caption{Schematic view of an X-ray tube with cathode C and anode A \cite{xraytube}.}
    \label{fig:xraytube}
\end{figure}

A wire is heated by a voltage $U_\text{heat}$ until it becomes incandescent.
Thus, it emits electrons which are then accelerated towards the anode through a high voltage between cathode C and anode A.
When they hit the anode material, they lose energy through interaction, namely through bremsstrahlung and ionisation.
Bremsstrahlung describes radiation that is emitted when electrons interact with the Coulomb field of an atomic nucleus and lose energy in the form of photons.
Ionisation happens if the incoming electron hits another electron inside the anode and transfers enough energy for the anode electron to leave its atom.
The free space inside the atomic shell is then refilled by electrons of higher energy levels which in turn emit their energy in the form of photons.
These photons posses exactly the energy between the levels, so
\begin{equation*}
    E_\gamma = \Delta E = E' - E
\end{equation*}
for energy level $E'$ on the higher and energy $E$ on the lower level.
This means that the characteristic spectrum and thus the photon energy is dependent on the anode material since different materials posses different binding energies.
Approximately,
\begin{equation*}
    E_n \propto \frac{Z^2}{n^2}
\end{equation*}
holds for the energy of level $n$ with atomic number $Z$.
While bremsstrahlung is emitted as a continuous spectrum, the X-rays created by ionisation posses a characteristic and discrete wavelength depending
on the energy level of the ionised electron as seen in \autoref{fig:röntgenspektrum}.

\begin{figure}[H]
    \centering
    \includegraphics{figures/röntgenspektrum.pdf}
    \caption{Schematic view of the X-ray spectrum of a material showing the continuous spectrum of the bremsstrahlung as well as the characteristic peaks for ionisation \cite{röntgenspektrum}.}
    \label{fig:röntgenspektrum}
\end{figure}

It can also be seen that the bremsspectrum only begins at a certain wavelength $\lambda_\text{min}$.
This wavelength can be derived from
\begin{equation*}
    E = h f = \frac{h c}{\lambda}
\end{equation*}
to be
\begin{equation*}
    \lambda_\text{min} = \frac{h c}{e U_\text{A}}
\end{equation*}
with energy $E$, frequency $f$, the Planck constant $h$, light speed $c$, elementary charge $e$ and acceleration voltage $U_\text{A}$. \\

X-rays behave differently when passing through and being reflected by different materials.

\subsection{Single-interface Refraction}

When hitting a single flat surface, the refractive index for X-rays can be described by
\begin{equation}
    n = 1 - \delta + i\beta \,.
    \label{eq:refractiveindex}
\end{equation}
The terms
\begin{equation*}
    \delta(r) = \frac{\lambda^2}{2 \pi} r_\text{e} \rho(r) \sum_{j=1}^N \frac{f_j^0 + f_j'}{Z}
\end{equation*} and
\begin{equation*}
    \beta(r) =  \frac{\lambda^2}{2 \pi} r_\text{e} \rho(r) \sum_{j=1}^N \frac{f_j''}{Z} = \frac{\lambda}{4 \pi} \mu(r)
\end{equation*} describe a dispersion an absorption term with $\delta > 0$ and $\beta \propto \mu(r)$ where $\mu(r)$ is the linear
absorption coefficient, where $f_j = f_j^0 + f_j'(E) + if_j''(E)$ describes forced oscillation strengths with dispersion and absorption corrections $f_j'$ and $f_j''$.
The classical electron radius is denoted by $r_\text{e}$ and the electron density is given by $\rho(r)$ with electron number $Z$ \cite{tolan}. \\
The process of refraction and reflection is shown in \autoref{fig:singleinterface}.

\begin{figure}[H]
    \centering
    \includegraphics[width=.9\textwidth]{figures/scattering.png}
    \caption{Refraction and reflection of a plane electromagnetic wave on a flat surface with grazing angle $\alpha_\text{i}$, 
    reflection angle $\alpha_\text{f} = \alpha_\text{i}$, transmission angle $\alpha_\text{t}$ and incoming, transmitted and reflected wave vectors $k_\text{i}$, $k_\text{t}$ and $k_\text{f}$ \cite{tolan}.}
    \label{fig:singleinterface}
\end{figure}

With $\delta \sim 10^{-6}$ and $\beta \sim 10^{-7}$ it can be seen that $|n| < 1$, although it may still be close to $1$.
This means that, for angles below a critical angle $\alpha_\text{c}$, total reflection is possible.
Assuming the exit angle $\alpha_\text{t}$ of the refracted beam to be zero, the critical angle is given by
\begin{equation}
    \alpha_\text{c} \approx \sqrt{2\delta} = \lambda \sqrt{\frac{r_\text{e} \rho}{\pi}} \,,
    \label{eq:critangle}
\end{equation} 
where $r_\text{e}$ is the classical electron radius, $\rho$ the electron density and $\lambda$ the wavelength of the X-ray photon \cite{tolan}. \\

In order to now describe the amplitude of transmitted and reflected X-rays, we use the Fresnel equations.
Normally, for different polarisations of photons, different equations apply.
But here, since the refractive indices $n_1$ and $n_2$ of the vacuum and medium are almost identical (with corrections at $\mathcal{O}(10^{-6})$),
the different equations are identical. \\
For the reflected amplitude,
\begin{equation}
    r = \frac{n_1 \cos \alpha_1 - n_2 \cos\alpha_2}{n_1 \cos \alpha_1 + n_2 \cos\alpha_2}
    \label{eq:reflectedamplitude}
\end{equation}
holds while the transmitted amplitude is described by
\begin{equation}
    t = \frac{2 n_1}{n_1 \cos \alpha_1 + n_2 \cos\alpha_2} \,.
    \label{eq:transmittedamplitude}
\end{equation}
The angles $\alpha_1$ and $\alpha_2$ describe the angles of the reflected and namely transmitted light. \\

These formulae allow us to calculate the so-called Fresnel reflectivity $R_\text{F} = |r|^2$, which becomes
\begin{equation}
    R_\text{F} \simeq \left(\frac{\alpha_\text{c}}{2\alpha_i}\right)^4
    \label{eq:fresnelreflectivity}
\end{equation}
for incident angles $\alpha_i > 3 \alpha_\text{c}$ \cite{tolan}.
For angles smaller than the critical angle, the Fresnel reflectivity is of $\mathcal{O}(1)$, then quickly drops at $\frac{\alpha_\text{i}}{\alpha_\text{c}} = 1$
and converges against zero for $\frac{\alpha_\text{i}}{\alpha_\text{c}} > 1$, comparable to the Fermi-Dirac-distribution for electrons.


\subsection{Multi-interface Refraction}

Until now, we have only looked at refraction of X-rays on a single interface surface.
If the X-ray, or any electromagnetic radiation, instead hits a layered system, as seen in \autoref{fig:multilayeredsystem}, the radiation is refracted and transmitted at every single layer.

\begin{figure}[H]
    \centering
    \includegraphics[width=.5\textwidth]{figures/multi_layer.png}
    \caption{Refraction and reflection of a plane electromagnetic wave inside a $N+1$-layered system \cite{tolan}.}
    \label{fig:multilayeredsystem}
\end{figure}
This also means that the transmitted rays from one layer can interfere with the reflected radiation from a deeper layer, 
causing oscillations in the systems reflectivity.
These so-called \textit{Kiessig oscillations} are shown in \autoref{fig:kiessigoscillations} and in turn can be used to calculate the layer thickness.
With the z-component of the wave vector transfer $q = \vec{k}_\text{f} - \vec{k}_\text{i}$, $q_\text{z} = 2 k \sin\alpha_i$, the layer thickness is
\begin{equation}
    d = \frac{2\pi}{\Delta q_\text{z}} \approx \frac{\lambda}{2 \Delta \alpha_i} \,.
    \label{eq:layerthickness}
\end{equation} 

\begin{figure}[H]
    \centering
    \includegraphics[width=.5\textwidth]{figures/kiessig_oszillation.png}
    \caption{Graphic representation of Kiessig oscillations inside a multi-layered system with critical angles shown for polystyrene (PS) film and Si \cite{tolan}.}
    \label{fig:kiessigoscillations}
\end{figure}

To calculate the Kiessig oscillations, the lowest layer is taken as a layer with infinite thickness without reflection from below, meaning $R_{N+1} = X_{N+1} = 0$.
The rest of the layers is then recursively calculated from the bottom up, giving the recursive formula
\begin{equation}
    X_j = \frac{R_j}{T_j} = \exp(-2i k_{z,j} z_j)\frac{r_{j,j+1} + X_{j+1} \exp(2i k_{z,j+1} z_j)}{1 + r_{j,j+1} X_{j+1} \exp(2i k_{z,j+1} z_j)} \,,
    \label{eq:parrattalgo}
\end{equation}
where
\begin{equation*}
    r_{j,j+1} = \frac{k_{z,j} - k_{z,j+1}}{k_{z,j} + k_{z,j+1}}
\end{equation*}
is the Fresnel coefficient of the $j$-th interface \cite{tolan}.
This approach is called \textit{Parratt algorithm}.

\subsection{Rough Surfaces}

Real surfaces are seldom even.
So we take an ensemble of even surfaces to approximate the rough surface as seen in \autoref{fig:approxrough} and modify the Fresnel coefficient so that
\begin{equation}
    \tilde{r}_{j,j+1} = r_{j,j+1} \exp(-2 k_{z,j} k_{z,j+1}\sigma_j^2) \,,
    \label{eq:modifiedfresnelfcoefficient}
\end{equation}
where 
\begin{equation*}
    \sigma^2 = \int(z-\mu_j) P_j(z) \text{d}z = \int(z-z_j) P_j(z) \text{d}z
\end{equation*} 
is the root-mean-square roughness for a probability distribution $P_j$, here a Gaussian with mean $\mu_j = z_j$.

\begin{figure}
    \centering
    \includegraphics[width=.5\textwidth]{figures/parrott_rough.png}
    \caption{Schematic representation of a rough surface approximated by even surfaces at height $z + z_j$ with probability distribution $P_j$ \cite{tolan}.}
    \label{fig:approxrough}
\end{figure}

It should be noted that this modification works only for systems where the roughness $\sigma_j$ is much smaller than the layer thickness $d_j$.
For system with greater roughness, the smaller even surfaces must be treated as different layers with their own refractive indices etc. \cite{tolan}.

\subsection{Geometry Factor}

It must also be noted that the beam used in this experiment still has a width $b_\text{i}$.
Especially for small angles, it will cover more are than the sample with width $D$ (shown in \autoref{fig:geometryfactor}), 
which has to be corrected by
\begin{equation}
    G = \begin{cases}
        \frac{D \sin\alpha_\text{i}}{d_0}&, \quad \alpha_ \text{i} < \alpha_\text{g} \\
        1&,\quad \alpha_\text{i} > \alpha_\text{g}
    \end{cases}
    \label{eq:GeometryFactor}
\end{equation}
with total beam width $d_0$ and effective beam area $D \sin\alpha_\text{i}$ \cite{v44}.

\begin{figure}[H]
    \centering
    \includegraphics{figures/beam_width.pdf}
    \caption{Schematic view at the beam hitting the sample at a low angle $\alpha_\text{i}$ where the beam area is larger than that of the sample \cite{v44}.}
    \label{fig:geometryfactor}
\end{figure}




