\section{Discussion}
\label{sec:Diskussion}


The measurement geometry angle is $\alpha_\text{g} = \SI{0.520 \pm 0.020}{\degree}$ while the theoretical value is $\alpha_\text{Theo}  = \SI{0.573 \pm 0.057}{\degree}$.
This results in a relative deviation of $\Delta_\alpha = \SI{9.25}{\%}$.\\
The Parratt algorithm delivers $\delta_{\text{Si}}   =  \num{6.93e-6}$ and $\delta_\text{Poly}  =  \num{9.70e-7}$.
The literature values are $\delta_{\text{Si, lit}} = \num{7.6e-6}$ and $\delta_{\text{Poly, lit}} = \num{3.5e-6}$ \cite{v44}.
The corresponding relative deviations are $\Delta_{\delta, \text{SI}} = \num{8.82}{\%}$ and $\Delta_{\delta, \text{Poly}} = \num{72.29}{\%}$.\\
Critical angle $\alpha_{\text{Poly}, \text{c}} = 0.080 \, \unit{\degree}$ and $\alpha_{\text{Si}, \text{c}} = 0.213\, \unit{\degree}$ together with the literature values 
$\alpha_{\text{Si}, \text{theo}} = 0.174 \, \unit{\degree}$ and $\alpha_{\text{Poly}, \text{theo}} = 0.153 \, \unit{\degree}$.
This yields deviations of $\Delta_{\alpha_{\text{Si}, \text{c}}} = 22.41\,\%$ and $\Delta_{\alpha_{\text{Poly}, \text{c}}} = 47.71\,\%$ .


The layer thickness is calculate in two ways giving the values $d_1 = \SI{8.8 \pm 0.7e-08}{\meter}$ and $d_2 = \SI{8.49e-8}{\meter}$.
The deviation of the layer thicknesses is $3.65\, \%$.\\

The sample might have a scratch, as can be seen in \autoref{fig:Omega2Theta1}. Because the reflectivity has an abnormality at around $\alpha \approx \SI{0.6}{\%}$.
Most of the calculated values have a high relative deviation from their literature value. This is probably a consequence of a poor fit.
Although several methods were tried, no better parameters could be determined. The best agreement is in the layer thickness with $3.65\, \%$,
which suggests that the measurement it at least consistent with itself.
