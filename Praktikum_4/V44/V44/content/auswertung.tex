\section{Evaluation}
\label{sec:auswertung}

\sisetup{round-mode = places, round-precision = 2}%
The following regression are performed with the method \texttt{curve\_fit} from the python \cite{py}-package \textit{scipy} \cite{scipy}.
Propagation of the uncertainties are done with the package \textit{uncertainties} \cite{uncertainties}.
Plots are created with \textit{matplotlib} \cite{matplotlib}.

\subsection{Alignment}

\subsubsection{Detector scan}

To adjust the primary beam a detector scan is performed.
The recorded data and the Gaussian-fit in depicted in \autoref{fig:Detectorscan}.
The resulting reflectivity should represent a Gaussian distribution of the shape

\begin{equation*}
    R\left(\alpha\right) = \frac{a}{\sqrt{2\pi\sigma²}} \cdot e^{-\frac{\left(x -\mu\right)^2}{2\sigma^2}} + b \,.
\end{equation*}

The parameter $a$, $b$, $\mu$ and $\sigma$ are determined as 
\begin{align*}
    a &=  \SI{118153.296 \pm 0.070}{\degree}\,, \\
    b & = \num{11647.81 \pm 0.16}\,, \\
    \mu    &= \SI{ -0.1087117 \pm   0.0000023}{\degree}\quad \text{and} \\
    \sigma &= \SI{  3.706477 \pm  0.0000024}{\degree}\,. \\
\end{align*}

The $\text{FWHM}$ and the maximum of the reflectivity $I_\text{max}$ are established as 

\begin{equation*}
    \text{FWHM} = \SI{0.087}{\degree} \quad \text{and}
\end{equation*}
\begin{equation*}
    I_{\text{max}} = \num{1283375.91 \pm 0.68}\,.
\end{equation*}


\begin{figure}[H]
    \centering
    \includegraphics[]{build/Detectorscan.pdf}
    \caption{The intensity of the detector scan in dependence on the incidence angle $\alpha \mathbin{/} \unit{\degree}$.}
    \label{fig:Detectorscan}
\end{figure}

\subsubsection{Z-Scan}

During the Z-Scan the z-position of the sample is changed. 
The sample is moved from below the into the primary beam until it is completely block reducing the reflectivity to nearly zero.
This measurement is shown in \autoref{fig:Z-Scan}.
The width of the beam is 
\begin{equation*}
    d_0 = \SI{0.20 \pm 0.02}{\milli\meter}.
    \label{eq:BeamWidth}
\end{equation*}

The uncertainty of the width is the distance between two measurement point.

\begin{figure}[H]
    \centering
    \includegraphics[]{build/Z1Scann.pdf}
    \caption{The measurement intensity for different positions of the probe. In addition the width of the beam is included.} 
    \label{fig:Z-Scan}
\end{figure}



\subsubsection{Rockingscan}

A Rockingscan is used to determine the geometry angle $\alpha_\text{g}$. 
For this the \autoref{fig:Rockingscan} is evaluated, where $\alpha_\text{g}$ is half the base of the triangle.
In this measurement $\alpha_\text{g}$ is measurement as 
\begin{equation*}
    \alpha_\text{g} = \SI{0.520 \pm 0.020}{\degree} = \num{0.00908 \pm 0.00035}\,.
\end{equation*}

Together with the width of the sample $D = \SI{20}{\milli\meter}$ \cite{v44}, the width of the beam from \autoref{eq:BeamWidth} and \autoref{eq:GeometryFactor} the theoretical value is calculated to be 

\begin{equation*}
    \alpha_\text{Theo}  = \SI{0.573 \pm 0.057}{\degree} = \num{0.0100 \pm 0.0010}\,.
\end{equation*}



\begin{figure}[H]
    \centering
    \includegraphics[]{build/Rockingcurve.pdf}
    \caption{Intensity of the Rockingscan with $2\alpha = 0$ and the corresponding geometry angle.} 
    \label{fig:Rockingscan}
\end{figure}

\subsection{Omega/2Theta}

The results of the Omega/2Theta scan can be seen in \autoref{fig:Omega2Theta1}. 
The setup records data for $ \SI{5}{\second}$.
To gain the true reflectivity the diffuse reflectivity is subtracted from the compact scan and modified by

\begin{equation*}
    R(I) =   \frac{I}{5 \cdot I_\text{max}}\,.
\end{equation*}

The factor 5 is included because the maximum intensity is only measured for $\SI{1}{\second}$.



\begin{figure}[H]
    \centering
    \includegraphics{build/Omega2Theta.pdf}
    \caption{Intensity of the two Omega/Theta scans, where the incidence angle of the diffuse scan is shifted by $\SI{0.2}{\degree}$.} 
    \label{fig:Omega2Theta1}
\end{figure}
