\section{Execution}
\label{sec:execution}

\subsection{Alignment}
\label{subsec:alignment}

Before starting the measurements, the Sagnac Interferometer first has to be thoroughly aligned.
The mirrors and other elements of the aperture are named as seen in \autoref{fig:sagnac}.
First, mirror $M_2$ is adjusted so that the beam coming from $M_1$ hits it right in the middle.
Adjustment plates are used to ensure the beam passes through the center of both mirror $M_a$ as well as $M_b$, using the two screws on $M_2$ to correct misalignment.
Then, using the same adjustment plates, the split beams are centered on $M_b$.
If the beams pass right through the center of the adjustment plates on their way away from $M_b$ as well, they should now overlap on the screen behind the interferometer.
Fine adjustments on $M_b$ are necessary to overlap the two beams. \\
For further alignment, as the two overlapping beams are polarised perpendicular to each other, a polarisation filter that is rotated by $45°$ is needed.
As soon as it is installed, an interference pattern can be observed on the screen, as seen in \autoref{fig:fringes}.
This interference pattern should be without any fringes, requiring further alignment until both beams are perfectly parallel to each other.

\begin{figure}[H]
    \centering
    \includegraphics{figures/V64_komisch.pdf}
    \caption{Resulting interference pattern on the screen behind the interferometer, including fringes caused by slight misalignment of the beams \cite{v64}.}
    \label{fig:fringes}
\end{figure}


\subsection{Measurement}
\label{subsec:measurement}

\subsubsection{Contrast}
\label{subsec:DurchführungKontras}
Using $M_2$, one of the beams is offset horizontally, creating two beams that still converge into one outside the interferometer.
A glass holder is introduced between the PBSC and mirror $M_4$ and the screen is removed so that the beam passes into the photo diodes placed behind.
By rotating the screw on the glass holder, it is possible to vary the measured voltage on the oscilloscope and voltmeter.
Now, for different polarisations between $0°$ and $180°$, adjusted via the polarisation filter in front of the PBSC, the interference minimum and maximum is measured,
taking three measurements for each maximum and minimum. \\

\subsubsection{Refractive index of glass}
\label{subsec:DruchführungGlass}
Now, the number of interference maxima is to be measured.
For that, the glass holder with a thickness of $T = \SI{1}{\milli\meter}$ is rotated for angles between $-8°$ and $2°$.
The number of maxima is displayed on a controller below the oscilloscope.
This measurement is repeated ten times, paying close attention to a constant appropriate rotation speed, as even small variations can cause oscillations that impact the measurement quality. \\

\subsubsection{Refractive index of air}
\label{subsec:DurchführungLuft}
Finally, the gas cell with a length of $\SI{100.0 \pm 0.1}{\milli\meter}$ is installed in one of the two beams.
It is evacuated using the vacuum pump until it reaches a pressure of around $\SI{5}{\milli\bar}$.
Then, air is slowly added back in.
The number of interference maxima, as seen on the oscilloscope is then documented as a function of pressure in steps of $\SI{50}{\milli\bar}$.
This measurement is repeated three times.

