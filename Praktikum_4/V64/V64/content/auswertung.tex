\section{Evaluation}
\label{sec:auswertung}


The error propagation is performed using  \textsc{uncertainties} \cite{uncertainties}. The linear interpolation is performed using \textsc{scipy.curve\_fit} \cite{scipy}.
Figures are compiled using \textsc{matplotlib} \cite{matplotlib}. 
%\input{./build/my_table.tex}

\subsection{Contrast}

With the first measurement, the contrast of the setup is determined as described in \autoref{subsec:DurchführungKontras}.
For this the contrast is calculated with \autoref{eq:contrast} for three different measurement series. 
The resulting contrasts and the theoretical course are plotted in \autoref{fig:contrast}.

\begin{figure}[H]
    \centering
    \includegraphics[width=\textwidth]{build/Kontrast.pdf}
    \caption{Comparison between the measurement contrasts and the expected course.}
    \label{fig:contrast}
\end{figure}

The maximas of each series is measured at $K¹_{\text{max}} = \SI{140}{\degree},K²_{\text{max}} = \SI{130}{\degree}$ and $K³_{\text{max}} = \SI{50}{\degree}$.
The highest average contrast is $\overline{K} =  \SI{0.910 \pm 0.014}{}$ at a angle of $\theta = \SI{130}{\degree}$.
A table of all measured currents and the resulting contrast can be seen in \autoref{tab:contrast}.
\input{./build/contrast.tex}


\subsection{Refractive Index of Glass}

The setup is modified according to \autoref{subsec:DruchführungGlass}. 
The refractive index of the inserted glass plate can be calculated from the examine angle and the zero passes at the diode according to \eqref{eq:refractiveindexglass}.
Vacuum wavelength is given as $\lambda_\text{vac} = \SI{632.990}{\nano\meter}$ \cite{v64} and the thickness of the glass plate as $T = \SI{1}{\milli\meter}$.
The measured zero passes and the calculated refractive index of the plate can be seen in \autoref{tab:glas}. The average value is $n = \SI{1.506 \pm 0.047}{}$.
\input{./build/glas.tex}


\subsection{Refractive Index of Air}

For this setup is changed to match \autoref{subsec:DurchführungLuft}.
To determent the refractive index of air at a temperature of $T = \SI{15}{\degree}$ and $p = \SI{1}{\bar}$ with the Lorentz-Lorenz relation, 
the polarizability of the air is calculated first. 
The refractive index is calculated via \eqref{eq:refractiveindexairExp}.
The length of the pressure chamber $L = \SI{100 \pm 0.1}{\milli\meter}$.
A plot between the calculated refractive index and the pressure and the linear approximation of the Lorentz-Lorenz relation can be seen in \autoref{fig:n_ns_p}.

\begin{figure}[H]
    \centering
    \includegraphics[width=\textwidth]{build/Brechungsindex.pdf}
    \caption{The refractive index is plotted against the pressure. In addition, the linear regression of the approximation of the Lorentz-Lorenz relation is shown.}
    \label{fig:n_ns_p}
\end{figure}

The calculated polarizabilities and the resulting refractive indices are shown in \autoref{tab:Air}.

\begin{table}
    \centering
    \caption{Polarizability and the refractive indices calculated with the Lorentz-Lorenz relation at $T = \SI{20}{\degree}$ and $p = \SI{1}{\bar}$ for four measurement series.}
    \label{tab:Air}
    \begin{tabular}{c c}
        \toprule
        Polarizability $ \mathbin{/} \unit{\frac{\ampere²\second⁴}{\kilo\gram}}$ & Refractive index \\
        \midrule 
        ${\left( 254.5   \pm  3.7 \right) \cdot 10^{-37}}$ & {$1.000360  \pm  0.000052$} \\
        ${\left( 254.8   \pm  5.3 \right) \cdot 10^{-37}}$ & {$1.000360  \pm  0.000076$} \\
        ${\left( 239.5   \pm  1.8 \right) \cdot 10^{-37}}$ & {$1.000339  \pm  0.000025$} \\
        ${\left( 206.2   \pm  2.9 \right) \cdot 10^{-37}}$ & {$1.000291  \pm  0.000041$} \\        
        \bottomrule
    \end{tabular}
\end{table}

Therefore the mean index is $\overline{n} = \SI{1.000337 \pm 0.000028}{}\,.$

