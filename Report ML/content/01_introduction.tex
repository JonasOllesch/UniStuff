\chapter{Introduction and Motivation}
\label{ch:intro}

Over the last few years, the quality of AI generated content has improved unfathomably.
Be it the appearance of generative models like \textit{Chat-GPT}, an almost human chatbot seemingly able to answer every question,
or methods such as \textit{Stable Diffusion} which can be used to generate almost convincingly real images, 
AI has by now taken a hold in almost every field. 
The average consumer thus gains access to an entirely new level of convenience. \\

Still, this creates perhaps as many dangers as it provides merits.
One of those dangers can be seen even while just writing this report - as GitHub's \textit{Copilot} is always eager to provide a "helpful" suggestion,
luring the author into just stopping thinking and let the AI do the work. \\

Here, however, we will deal with a different kind of problem: how can it be ensured that AI generated images are correctly identified?
As of now, it is still possible, at least for the more-informed, to differentiate between generated and real images, but considering the rapid progress over the last years,
it is unlikely to remain that way for much longer. \\ %maybe reference the generated videos from earlier this year here

Apart from the risks it poses concerning the spreading of misinformation in the form of, e.g., deepfakes, this also creates issues regarding copyright and intellectual property.
Since generative AI models need to be trained on a large dataset of real art in order to create images on their own, it is likely that at least a portion of the training images
were taken without the artists' consent. \\

To mitigate these problems, it is useful to develop methods of recognizing AI-generated content.
The obvious, and somewhat ironic, approach is to train a neural network on a dataset of real and generated art
since it is unrealistic to apply more conventional, non-machine learning methods to such a large amount of data. \\

So, in this report, we will answer a couple of questions: Can a convolutional neural network (CNN) reliably differentiate between real and AI generated art compared to a random forest?
And more specifically, how well can it differentiate between real and AI generated art of the same epoch? \\






