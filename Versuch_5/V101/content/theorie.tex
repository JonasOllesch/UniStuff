\section{Theorie}
\label{sec:Theorie}
Rotiert eine punktförmige Masse $m$ um eine Drehachse, ist ihr Trägheitsmoment $I$ im Abstand $r$ von jener Drehachse gegeben durch $I = m r^2$.
Dieses Gesetz lässt sich nun auch auf ausgedehnte Massen übertragen. Durch Aufteilung des ausgedehnten Körper in kleinere Teillmassen $m_i$ im Abstand $r_i$ von der Drehachse ergibt sich
das Gesamtträgheitsmoment zu

\begin{equation}
    I = \sum_i^N r^2_i m_i\,.
    \label{eq:trägmomsum}
\end{equation}

Werden die Teilmassen nun infinitisimal, lässt sich \eqref{eq:trägmomsum} schreiben als
\begin{equation}
    I = \int r^2 \mathrm{d}m\,.
    \label{eq:trägmomint}
\end{equation}

Im hier durchgeführten Versuch werden insbesondere die in \autoref{fig:trägmomSKZ} dargestellten Trägheitsmomente benötigt.

\begin{figure}
    \centering
    \includegraphics{TrägheitsmomenteKugelusw.pdf}
    \caption{Trägheitsmoment eines langen Stabes, einer Kugel und eines Zylinders\cite{ap05}.}
    \label{fig:trägmomSKZ}
\end{figure}
Für eine dünnen Stab ist das Trägheitsmoment gegeben durch 

\begin{equation}
    I_{St} = \dfrac{m}{12} l^2 \,.
    \label{eq:trägdünstab}
\end{equation}

Für eine Kugel gilt
\begin{equation}
    I_{K} = \dfrac{2}{5} m R^2 \,.
    \label{eq:trägKugel}
\end{equation}

Für das Trägheitsmoment eines Zylinders, der sich parallel zur Symmetrieachse dreht, gilt
\begin{equation}
    I_{Z} = \dfrac{m}{2} R^2 \,,
    \label{eq:trägZylinderparall}
\end{equation}
wenn der Zylinders sich senkrecht zur Symmetrieachse dreht, ist das Tr'gheitsmoment durch 
\begin{equation}
    I_{zh} = m \left(\dfrac{R^2}{4} + \dfrac{h^2}{12} \right) \,,
    \label{eq:trägZylindersenkrecht}
\end{equation}
gegeben. \\

Wird die Drehachse nun so gewählt, dass sie nicht länger durch den Körperschwerpunkt verläuft, lässt sich das neue Trägheitsmoment bezüglich dieser Achse über den Satz von Steiner berechnen.
Für eine parallel um den Abstand $a$ verschobene Drehachse gilt

\begin{equation}
    I = I_S + ma^2\,,
    \label{eq:steiner}
\end{equation}
wobei $m$ die Gesamtmasse des Körpers und $I_S$ das Trägheitsmoment bei Drehungen um die Schwerpunktachse darstellt. \\

Nun sei ein schwingungsfähiges System zu betrachten. Greift nun eine Kraft $\vec{F}$ im Abstand $\vec{r}$ von der Drehachse an, wirkt der Drehung um den Winkel $\varphi$, aufgrund der eingesetzten Feder,
ein rücktreibendes Drehmoment $\vec{M}$ entgegen. 
Dieses Drehmoment ist gegeben durch
\begin{equation*}
    \vec{M} = \vec{F} \times \vec{r}
\end{equation*}
bzw.
\begin{equation}
    M = |\vec{M}| = |\vec{F}||\vec{r}| \sin\varphi = F r \sin\varphi\,,
    \label{eq:drehmombetrag}
\end{equation}
sofern nur der Betrag betrachtet wird.
An Gleichung \eqref{eq:drehmombetrag} lässt sich bereits erkennen, dass der Betrag des Drehmomentes für senkrecht angreifende Kräfte, also $\varphi=\frac{π}{2}$ maximal wird. \\

Wird das System 'losgelassen', stellt sich eine harmonische Schwingung ein, die Schwingungsdauer ist dabei mit der Winkelrichtgröße $D$ und dem Trägheitsmoment $I$ gegeben durch
\begin{equation}
    T = 2 π \sqrt{\frac{I}{D}}\,,
    \label{eq:periodendauer}
\end{equation}
wobei sich die Winkelrichtgröße über
\begin{equation}
    M = D \varphi
    \label{eq:winkelrichtgröße}
\end{equation}
definiert lässt. 
Umstellen von \eqref{eq:periodendauer} nach $I$ liefert einen Ausdruck für das Trägheitsmoment, der Zusammenhang zwischen Schwingungsdauer, Drehmoment und Trägheitsmoment lautet
\begin{equation}
    I = \frac{T^2}{4 π^2} D \,.
    \label{eq:trägüschwi}
\end{equation}\\

Durch Messung von $\varphi$ als Funktion der Kraft $F$, \textit{statische Methode}, oder, bei bekanntem Trägheitsmoment $I$, durch Messung der Schwingungsdauer $T$, \textit{dynamische Methode}, lässt sich
$D$ bestimmen. Eine gleichzeitige Bestimmung von $I$ und $D$ wird durch eine Kombination beider Methoden ermöglicht. \\

Bei der statischen Methode wird $\varphi$ als Funktion der Kraft $F$ gemessen. In der dynamischen Methode wird bei einem bekanntem Trägheitsmoment $I$ die Schwingungsdauer $T$ gemessen, damit lässt sich die Winkelrichtgröße $D$ berechnen.
Eine gleichzeitige Bestimmung von $I$ und $D$ wird durch eine Kombination beider Methoden ermöglicht.\\

Mithilfe von \eqref{eq:drehmombetrag} lässt sich \eqref{eq:winkelrichtgröße} zu
\begin{equation}
  D = \frac{F r}{\varphi}
  \label{eq:WinkelrichtgröFr}
\end{equation}
umschreiben. 