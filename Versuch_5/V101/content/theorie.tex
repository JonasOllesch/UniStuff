\section{Theorie}
\label{sec:Theorie}
Rotiert eine punktförmige Masse $m$ um eine Drehachse, ist ihr Trägheitsmoment $I$ im Abstand $r$ von jener Drehachse gegeben durch $I = m r^2$.
Dieses Gesetz lässt sich nun auch auf ausgedehnte Massen übertragen. Durch Aufteilung des ausgedehnten Körper in kleinere Teillmassen $m_i$ im Abstand $r_i$ von der Drehachse ergibt sich
das Gesamtträgheitsmoment zu

\begin{equation}
    I = \sum_i^N r^2_i m_i\,.
    \label{eq:trägmomsum}
\end{equation}

Werden die Teilmassen nun infinitisimal, lässt sich \eqref{eq:trägmomsum} schreiben als
\begin{equation}
    I = \int r^2 \mathrm{d}m\,.
    \label{eq:trägmomint}
\end{equation}

Im hier durchgeführten Versuch werden insbesondere die in \autoref{fig:trägmomSKZ} dargestellten Trägheitsmomente benötigt.

\begin{figure}
    \centering
    \includegraphics{TrägheitsmomenteKugelusw.pdf}
    \caption{Trägheitsmoment eines langen Stabes, einer Kugel und eines Zylinders\cite{ap05}.}
    \label{fig:trägmomSKZ}
\end{figure}

Wird die Drehachse nun so gewählt, dass sie nicht länger durch den Körperschwerpunkt verläuft, lässt sich das neue Trägheitsmoment bezüglich dieser Achse über den Satz von Steiner berechnen.
Für eine parallel um den Abstand $a$ verschobene Drehachse gilt

\begin{equation}
    I = I_S + ma^2\,,
    \label{eq:steiner}
\end{equation}
wobei $m$ die Gesamtmasse des Körpers und $I_S$ das Trägheitsmoment bei Drehungen um die Schwerpunktachse darstellt. \\

Nun sei ein schwingungsfähiges System zu betrachten. Greift nun eine Kraft $\vec{F}$ im Abstand $\vec{r}$ von der Drehachse an, wirkt der Drehung um den Winkel $\varphi$, aufgrund der eingesetzten Feder,
ein rücktreibendes Drehmoment $\vec{M}$ entgegen. 
Dieses Drehmoment ist gegeben durch
\begin{equation*}
    \vec{M} = \vec{F} \times \vec{r}
\end{equation*}
bzw.
\begin{equation}
    M = |\vec{M}| = |\vec{F}||\vec{r}| \sin\varphi = F r \sin\varphi\,,
    \label{eq:drehmombetrag}
\end{equation}
sofern nur der Betrag betrachtet wird.
An \eqref{eq:drehmombetrag} lässt sich bereits erkennen, dass der Betrag des Drehmomentes für senkrecht angreifende Kräfte, also $\varphi=\frac{π}{2}$ maximal wird. \\

Wird das System 'losgelassen', stellt sich eine harmonische Schwingung ein, die Schwingungsdauer ist dabei mit der Winkelrichtgröße $D$ und dem Trägheitsmoment $I$ gegeben durch
\begin{equation}
    T = 2 π \sqrt{\frac{I}{D}}\,,
\end{equation}
wobei sich die Winkelrichtgröße über
\begin{equation}
    M = D \varphi
\end{equation}
definiert lässt. \\

Durch Messung von $\varphi$ als Funktion der Kraft $F$ \textit{(statische Methode)} oder, bei bekanntem Trägheitsmoment $I$, durch Messung der Schwingungsdauer $T$ \textit{(dynamische Methode)} lässt sich
$D$ bestimmen. Eine gleichzeitige Bestimmung von $I$ und $D$ wird durch eine Kombination beider Methoden ermöglicht. 



