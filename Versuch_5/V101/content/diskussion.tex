\section{Diskussion}
\label{sec:Diskussion}
%Was zu den Abweichungen, wenn wir die genau haben

Die wohl größte Fehlerquelle bei der Bestimmung der Trägheitsmomente der Puppe liegt darin, dass die Gliedmaßen nur schwer perfekt auszurichten sind. 
Einerseits konnte aufgrund der Gelenke nicht sichergestellt werden, dass Hände und Füße der Puppe voll ausgestreckt waren, was die Annäherung als Zylinder ungenauer macht, 
in der Näherung wurden Hände und Füße gar nicht berücksichtigt.
Andererseits war vor allem der Torso, aber auch alle anderen Körperteile nicht vollständig symmetrisch, was das Schwingungsverhalten beeinflusste und die Näherung weiter verschlechterte.

Für alle drei Körper, also auch Kugel und Zylinder spielen auch Reibung und Luftwiderstand eine Rolle, zusammen mit der hohen Ungenauigkeit der Schwingungsdauermessung 
lassen sich die Abweichungen von Theorie und Messung erklären.