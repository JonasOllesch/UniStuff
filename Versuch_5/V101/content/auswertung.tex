\section{Auswertung}
\label{sec:Auswertung}

\subsection{Bestimmung der Winkelrichtgröße}
\label{subsec:a}
Mithilfe von \eqref{eq:drehmombetrag} lässt sich \eqref{eq:winkelrichtgröße} zu
\begin{equation}
  D = \frac{F r}{\varphi}
  \label{eq:WinkelrichtgröFr}
\end{equation}
umschreiben. 

Die bei der ersten Messung aufgenommenen Daten sind in \autoref{tab:Messung_a} dargestellt.
Dabei ist $\varphi$ die Auslenkung des Systems und $F$ die am im Abstand von $r = 0,2 \,\unit{\meter}$ Kraftmesser gemessene rücktreibende Kraft. 
\begin{table}[H] % Hier noch eine dritte Spalte mit den einzelnen Winkelrichtgrößen wär nicht schlecht...
  \centering
  \sisetup{table-format=3.0}
  \begin{tabular}{S S[table-format=1.3]}
      \toprule
      {$\varphi\mathbin{/}\unit{°}$} & {$F \mathbin{/} \unit{\newton}$}\\
      \midrule
           30  & 0,042 \\
           40  & 0,062 \\
           50  & 0,088 \\
           60  & 0,106 \\  
           70  & 0,128 \\
           80  & 0,140 \\
           90  & 0,168 \\
           100 & 0,190 \\
           110 & 0,200 \\
           120 & 0,250 \\
      \bottomrule
  \end{tabular}
  \caption{Rücktreibende Kraft zu verschiedenen Auslenkungen.}
  \label{tab:Messung_a}
\end{table}

%Mittelwert und Standardabweichung hinschreiben
Aus den gemessenen Werten wurde der Mittelwert und die Standardabweichung berechnet.
Für die Winkelrichtgröße ergibt sich dann  
\begin{equation*}
  D = (0,0203 \pm 0,0020) \,\unit{\newton\meter} \,.  
\end{equation*}

\subsection{Bestimmung des Eigenträgheitsmoments}
\label{subsec:b}
% Was ist mit den Unsicherheiten auf den Messwerten, zumindest auf dem Gewicht? Hatten wir die? Ja wir hatten da eine, aber Katja hat in ihrem Protokoll auch keine angegeben xD
Um das Eigenträgheitsmoment zu bestimmen, wurden zwei Zylinder mit einer Masse von je $222,7 \,\unit{g}$, einem Radius von $16 \,\unit{mm}$ und einer Höhe von $27,1 \,\unit{mm}$ senkrecht zur Drehachse befestigt.
Der Abstand $a$ wurde dabei stets bis zum vorderen Ende des Zylinders gemessen und $\dfrac{h}{2}$ addiert, in \autoref{tab:Messung_b} 
sind also die Abstände bis zum Zylinderzentrum mit den dazugehörigen Schwingungsdauern dargestellt.

\begin{table}[H] % Hier statt 3T lieber 1T auftragen?? Ich meine, Katja hatte das schon beim ersten Protokoll erwähnt, dass 3T unnötig ist...
  %dann gibts halt nur T xD
  \centering
  \begin{tabular}{S[table-format=2.3] S[table-format=2.2]}
      \toprule
      {$a \mathbin{/}\unit{cm}$} & {$T \mathbin{/} \unit{\second}$}\\
      \midrule
           20,355 & 6,166 \\
           21,355 & 6,542 \\
           22,355 & 6,576 \\
           23,355 & 6,856 \\  
           24,355 & 7,109 \\
           25,355 & 7,329 \\
           26,355 & 7,646 \\
           27,355 & 7,793 \\
           28,355 & 8,113 \\
           29,355 & 8,406 \\
      \bottomrule
  \end{tabular}
  \caption{Schwingungsdauern $T$ bei verschiedenen Abständen $a$.}
  \label{tab:Messung_b}
\end{table}
Für das Trägheitsmoment wird die Form $ I = T_D + 2I_z + 2m a^2$ angenommen. 
Durch einsetzen in \eqref{eq:periodendauer} ergibt sich \eqref{eq:periodendauerquadrat}, dabei ist $I_z$ das Trägheitsmoment eines Zylinders, $m$ die Masse und $a$ der Abstand zur Rotationsachse.
\begin{equation}
  T^2 = 4 \pi^2 \frac{I_D}{D} + 8 \pi^2 \frac{m (\frac{R^2}{4} + \frac{h^2}{12})}{D}+ 8 \pi^2 \frac{m}{D} a^2 .
  \label{eq:periodendauerquadrat}
\end{equation}
In \autoref{fig:T2ga2} wird nun das Quadrat der Schwingungsdauer $T^2$ gegen das Abstandquadrat $a^2$ aufgetragen.
Die dazu durchgeführte lineare Regression nimmt die Form $T^2 = A  a^2 + B $ mit den Parametern
\begin{equation*}
  A = (708 \pm 20) \dfrac{1}{s^2m^2}
\end{equation*}
\begin{equation*}
 B = (8,8 \pm 1,3) \dfrac{1}{s^2} 
\end{equation*}
an.

Mithilfe von \eqref{eq:trägüschwi} lässt sich das Eigenträgheitsmoment zu
\begin{equation*}
  I_D = (0,0045 \pm 0,0008) \,\unit{\newton\meter} \,.
\end{equation*}
bestimmen.

Die Unsicherheit berechnet sich dabei aus der Gaußschen Fehlerfortpflanzung mit
\begin{equation}
  Δf(x_1,...,x_N) = \sqrt{\sum_{i=1}^N \left(\frac{\partial f}{\partial x_i}Δx_i\right)^2}
  \label{eq:gaußfehler}
\end{equation}
berechnen.

\begin{figure}[H]
  \centering
  \includegraphics{build/T2ga2.pdf}
  \caption{Lineare Regression der quadrierten Messwerte.}
  \label{fig:T2ga2}
\end{figure}

\subsection{Trägheitsmoment der Kugel}
\label{subsec:c}

Ähnlich zu \autoref{subsec:b} wird nun die Schwingungsdauer einer Kugel mit Durchmesser $d = 12,75 \, \unit{\centi\meter}$ und einer Masse von $m = 811,3 \,\unit{\gram}$ bestimmt und in \autoref{tab:Messung_c} dargestellt.
Die Messung wird mit einer Auslenkung von $\varphi = \dfrac{π}{2} = 90 \,\unit{\degree}$ zehn Mal wiederholt, die Schwingungsdauern werden gemittelt und es wird die Abweichung bestimmt.

\begin{table}[H] % Auch hier lieber noch 1T :D ✓ 
  \centering
  \begin{tabular}{S[table-format=2.0] S[table-format=1.2]}
      \toprule
      {Messung} & {$T \mathbin{/} \unit{\second}$}\\
      \midrule
          1  & 1,626 \\
          2  & 1,586 \\
          3  & 1,686 \\
          4  & 1,576 \\  
          5  & 1,626 \\
          6  & 1,559 \\
          7  & 1,663 \\
          8  & 1,753 \\
          9  & 1,646 \\
          10 & 1,626 \\
      \bottomrule
  \end{tabular}
  \caption{Schwingungsdauern $T$ der Kugel.}
  \label{tab:Messung_c}
\end{table}
Es ergibt sich die Periodendauer
\begin{equation*}
  T= (1,64 \pm 0,05) \, \unit{\second} \,.
\end{equation*}

Aus \eqref{eq:trägüschwi} ergibt sich das Trägheitsmoment der Kugel zu
\begin{equation*}
  I_{K,m} = (0,00138 \pm 0,00017)\, \unit{\newton\meter}  \, .
\end{equation*} \\

Wie in \autoref{fig:trägmomSKZ} zu sehen, ergibt sich der Theoriewert des Trägheitsmoments einer Kugel aus
\begin{equation}
  I_{K} = \frac{2}{5} m R^2
  \label{trägheitsmomK}
\end{equation}
zu
\begin{equation*}
  I_{K,t} = 0,00131 \, \unit{\newton\meter}. % die theorischen Werten haben keine Unsicherheit aber wenn die noch eine bekommen sollen, dann kann man da bestimmt noch was machen xD
\end{equation*}

Die relative Abweichung des gemessenen und theoretischen Wertes beträgt dann
\begin{equation*}
  \left|\frac{I_{K,m}}{I_{K,t}} * 100 - 100 \right| = 5,344 \% \,.
\end{equation*}



\subsection{Trägheitsmoment des Zylinders}
\label{subsec:d}

Analog zur Bestimmung des Trägheitsmoments der Kugel wird nun für den Zylinder vorgegangen. Mit einer Masse von $367,7 \,\unit{\gram}$, einer Höhe von $h = 9,74 \,\unit{\centi\meter}$
und einem Durchmesser von $d = 9,67 \,\unit{\centi\meter}$ ergeben sich für zehn Messungen die in \autoref{tab:Messung_d} dargestellten Schwingungsdauern.

\begin{table}[H]
  \centering
  \begin{tabular}{S[table-format=2.0] S[table-format=1.2]}
      \toprule
      {Messung} & {$T \mathbin{/} \unit{\second}$}\\
      \midrule
          1  & 0,90 \\
          2  & 0,77 \\
          3  & 0,86 \\
          4  & 0,82 \\  
          5  & 0,75 \\
          6  & 0,77 \\
          7  & 0,88 \\
          8  & 0,77 \\
          9  & 0,79 \\
          10 & 0,81 \\
      \bottomrule
  \end{tabular}
  \caption{Schwingungsdauern $T$ des Zylinders.}
  \label{tab:Messung_d}
\end{table}

Gemittelt ergibt sich hier die Schwingungsdauer
\begin{equation*}
  T = (0,82 \pm 0,05) \, \unit{\second} \,.
\end{equation*}

Das experimentelle Trägheitsmoment lässt sich erneut aus \eqref{eq:trägüschwi} zu
\begin{equation*}
  I_{K,m} = (0,00034 \pm 0,00004) \, \unit{\newton\meter}
\end{equation*} 
bestimmen. \\

Mit dem Trägheitsmoment
\begin{equation}
  I_Z = \frac{1}{2} m R^2
\end{equation}
eines aufrecht zur Drehachse stehenden Zylinders ergibt sich für den Theoriewert
\begin{equation*}
  I_{Z,t} = 0,00043 \,  \unit{\newton\meter}.
\end{equation*}

Die Abweichung beträgt dann
\begin{equation*}
  \left|\frac{I_{Z,m}}{I_{Z,t}} * 100 - 100 \right| = 20,930 \% \,.
\end{equation*}

\newpage

\subsection{Trägheitsmoment einer Holzpuppe}
\label{subsec:e}

Abschließend soll das Trägheitsmoment einer Holzpuppe in zwei unterschiedlichen Positionen bestimmt werden. Dazu wurden zunächst die einzelnen Körperteile abgemessen.
Für die Längen $l$ ergaben sich die in \autoref{tab:Messung_e1} aufgetragenen Werte, die je zehnfach gemessenen Durchmesser $d$ der einzelnen Körperteile finden sich in \autoref{tab:Messung_e2}.
Dabei sei eine Unsicherheit von $1 \, \unit{\milli\meter}$ anzunehmen.

\begin{table}[H]
  \centering
  \begin{tabular}{S[table-format=2.0] S[table-format=2.2]}
      \toprule
      {Körperteil} & {$l \mathbin{/} \unit{\meter}$}\\
      \midrule
        {Kopf}  & {$0,04310 \pm 0,00010$} \\
        {Arme}  & {$0,12920 \pm 0,00010$} \\
        {Torso} & {$0,08740 \pm 0,00010$} \\
        {Beine} & {$0,14630 \pm 0,00010$} \\
      \bottomrule
  \end{tabular}
  \caption{Längen der einzelnen Puppenkörperteile.}
  \label{tab:Messung_e1}
\end{table}

\begin{table}[H]
  \centering
  \sisetup{table-format=1.2}
  \begin{tabular}{S[table-format=2.0] S S S S}
      \toprule
      {Messung} & {$d_{Kopf} \mathbin{/} \unit{\centi\meter}$} & {$d_{Arme} \mathbin{/} \unit{\centi\meter}$} & {$d_{Torso} \mathbin{/} \unit{\centi\meter}$} & {$d_{Beine} \mathbin{/} \unit{\centi\meter}$} \\
      \midrule
        1  & 3,02 & 1,24 & 3,22 & 1,25 \\
        2  & 2,96 & 1,57 & 3,55 & 1,47 \\
        3  & 2,53 & 1,26 & 3,17 & 1,14 \\
        4  & 1,58 & 1,40 & 2,84 & 1,86 \\  
        5  & 2,92 & 1,10 & 2,59 & 1,31 \\
        6  & 2,48 & 0,96 & 2,74 & 1,17 \\
        7  & 2,02 & 1,32 & 2,50 & 1,03 \\
        8  & 2,87 & 1,30 & 3,27 & 0,99 \\
        9  & 2,03 & 0,85 & 3,35 & 1,01 \\
        10 & 1,59 & 1,22 & 3,46 & 1,22 \\
      \bottomrule
  \end{tabular}
  \caption{Durchmesser der einzelnen Puppernkörperteile.}
  \label{tab:Messung_e2}
\end{table}

Nach Mittelung der gemessenen Durchmesser ergeben sich die in \autoref{tab:Messung_e3} dargestellten Radien.

\begin{table}[H]
  \centering
  \begin{tabular}{S[table-format=2.0] S[table-format=2.2]}
      \toprule
      {Körperteil} & {$\bar{d} \mathbin{/} \unit{\meter}$}\\
      \midrule
        {Kopf}  & {$0,0120 \pm 0,0050$} \\
        {Arme}  & {$0,0061 \pm 0,0020$} \\
        {Torso} & {$0,0153 \pm 0,0035$} \\
        {Beine} & {$0,0062 \pm 0,0025$} \\
      \bottomrule
  \end{tabular}
  \caption{Gemittelte Radien der einzelnen Puppenkörperteile.}
  \label{tab:Messung_e3}
\end{table}

\subsubsection{Bestimmung des Trägheitsmoments in Position 1}
\label{subsubsec:pos1}

Zunächst soll das Trägheitsmoment der Holzpuppe mit seitlich ausgestreckten Armen bestimmt werden.
Erneut werden dazu zehnfach die Schwingungsdauern gemessen, diesmal jedoch sowohl für eine Auslenkung von $\varphi = 90 \unit{\degree}$ als auch $\varphi = 120 \unit{\degree}$,
die Messwerte sind in \autoref{tab:Messung_f} aufgetragen.

\begin{table}[H]
  \centering
  \begin{tabular}{S[table-format=2.0] S[table-format=1.3] S}
      \toprule
      {Messung} & {$T \mathbin{/} \unit{\second}$ bei $\varphi = 90 \unit{\degree}$} & {$T \mathbin{/} \unit{\second}$ bei $\varphi = 120 \unit{\degree}$}\\
      \midrule
      1 & 0.663 & 0.660 \\
      2 & 0.533 & 0.663 \\
      3 & 0.663 & 0.663 \\
      4 & 0.623 & 0.663 \\
      5 & 0.553 & 0.643 \\
      6 & 0.597 & 0.753 \\
      7 & 0.600 & 0.643 \\
      8 & 0.597 & 0.667 \\
      9 & 0.597 & 0.643 \\
      10& 0.710 & 0.597 \\
      \bottomrule
  \end{tabular}
  \caption{Schwingungsdauern bei Auslenkungen von $\varphi = 90 \unit{\degree}$ und $\varphi = 120 \unit{\degree}$ in Position 1.}
  \label{tab:Messung_f}
\end{table}

Wie schon bei der Trägheitsmomentsbestimmung der anderen Körper kann nun erneut \eqref{eq:trägüschwi} genutzt werden, um das Trägheitsmoment zu berechnen.
Aus den gemittelten Schwingungsdauern ergeben sich dann
\begin{equation*}
  I_{P_1,1} = (0,00019 \pm 0,00004) \, \unit{\newton\meter} 
\end{equation*} 
für $\varphi = 90 \, \unit{{\degree}}$ und

\begin{equation*}
  I_{P_1,2} = (0,000224 \pm 0,000034) \, \unit{\newton\meter} 
\end{equation*}
für $\varphi = 120 \unit{\degree}$, erneut gemittelt also

\begin{equation*}
  I_{P_1,m} = (0,00022 \pm 0,00003) \, \unit{\newton\meter} \,.
\end{equation*}

Für die Berechnung des theoretischen Trägheitsmoments werden die einzelnen Körperteile als Zylinder approximiert. Dabei sei angenommen, dass die Drehachse genau durch den Mittelpunkt von Kopf und Torso verläuft, es gilt also
\begin{equation*}
  I_{Kopf,t} = \frac{1}{2} \, m_{Kopf} \, R^2_{Kopf}
\end{equation*}
und
\begin{equation*}
  I_{Torso,t} = \frac{1}{2} \, m_{Torso} \, R^2_{Torso} \,.
\end{equation*}

Dabei sind die Beine vertikal um je $R_{Beine}$ und die Arme horizontal um $\frac{h_{Arme}}{2} + R_{Torso}$ zur Drehachse verschoben. 
Über den Satz von Steiner lassen sich jetzt die dazugehörigen Trägheitsmomente berechnen.
Es ergeben sich mithilfe der Zylinderträgheitsmomente aus \autoref{fig:trägmomSKZ}
\begin{equation*}
  I_{Arme,t} = I_{Arme,S} + m a^2 = m \left(\frac{R^2_{Arme}}{4} + \frac{h^2_{Arme}}{12} \right) + m \left(R_{Torso} + \frac{h_{Arme}}{2} \right)^2
\end{equation*}
und
\begin{equation*}
  I_{Beine,t} = I_{Beine,S} + m a^2 = \frac{1}{2} \, m_{Beine} \, R^2_{Beine} + m \, R^2_{Beine}\,.
\end{equation*}

Das Gesamtträgheitsmoment setzt sich dann aus der Summe der Einzelträgheitsmomente zu
\begin{equation*}
  I_{P_1,t} = (0,00028 \pm 0,00015) \, \unit{\newton\meter}
\end{equation*}
zusammen.

\newpage

\subsubsection{Bestimmung des Trägheitsmoments in Position 2}
\label{subsubsec:pos2}

Analog wird nun für die zweite Puppenposition verfahren. 
Nun sind die Arm horizontal nach hinten und die Beine horizontal nach vorne ausgerichtet. 
In \autoref{tab:Messung_g} sind erneut die Schwingungsdauern dargestellt.

\begin{table}[H]
  \centering
  \begin{tabular}{S[table-format=2.0] S[table-format=1.3] S}
      \toprule
      {Messung} & {$T \mathbin{/} \unit{\second}$ bei $\varphi = 90 \unit{\degree}$} & {$T \mathbin{/} \unit{\second}$ bei $\varphi = 120 \unit{\degree}$}\\
      \midrule
      1  & 0,903 & 0,833 \\
      2  & 0,947 & 0,990 \\
      3  & 0,993 & 0,907 \\
      4  & 0,947 & 0,880 \\
      5  & 0,903 & 0,863 \\
      6  & 1,013 & 0,950 \\
      7  & 1,013 & 1,013 \\
      8  & 0,883 & 0,970 \\
      9  & 0,947 & 0,993 \\
      10 & 0,883 & 0,970 \\
      \bottomrule
  \end{tabular}
  \caption{Schwingungsdauern bei Auslenkungen von $\varphi = 90 \, \unit{\degree}$ und $\varphi = 120 \, \unit{\degree}$ \\ in Position 2.}
  \label{tab:Messung_g}
\end{table}

Aus den gemittelten Schwingungsdauern lassen sich wie schon in \autoref{subsubsec:pos1} die gemessenen Trägheitsmomente bestimmen und zu
\begin{equation*}
  I_{P_2,m} = 0,00045 \pm 0,00006 \, \unit{\newton\meter}
\end{equation*}
mitteln.

Zur Berechnung des Theoriewertes wird analog vorgegangen, es ändert sich lediglich das Trägheitsmoment der Beine, die nicht länger vertikal, sondern horizontal ausgerichtet sind.
Mit 
\begin{equation*}
  I_{Beine,t} = I_{Beine,S} + m a^2 = m \left(\frac{R^2_{Beine}}{4} + \frac{h^2_{Beine}}{12} \right) + m \left(\frac{h_{Beine}}{2} \right)^2
\end{equation*}
ergibt sich aus der Summe der Einzelträgheitsmomente
\begin{equation*}
  I_{P_2,t} = (0,00066 \pm 0,00023) \, \unit{\newton\meter} \,.
\end{equation*} \\

\subsection{Abweichungen von Theorie und Messung}

Die Abweichungen von Theorie und Messung belaufen sich auf

\begin{equation*}
  100 * \left(\frac{I_{P_1,m}}{I_{P_1,t}} - 1\right) = 21,429 \%
\end{equation*}
und
\begin{equation*}
  100 * \left(\frac{I_{P_2,m}}{I_{P_2,t}} - 1\right) = 31,818 \% \,.
\end{equation*}

Zum Vergleich der experimentellen und theoretischen Werte werden im Folgenden insbesondere die Verhältnisse der Trägheitsmomente $\frac{I_{P_1}}{I_{P_2}}$ der einzelnen Positionen betrachtet.
Es ergibt sich mit
\begin{equation*}
  d_{m}= \frac{I_{P_1,m}}{I_{P_2,m}} = 0,489
\end{equation*}
und
\begin{equation*}
  d_{t} = \frac{I_{P_1,t}}{I_{P_2,t}} = 0,424 
\end{equation*}
die relative Abweichung von
\begin{equation*}
  d_{ges} = 100*(\frac{d_m}{d_t}-1) = 15,252 \,\% \,.
\end{equation*}