\section{Auswertung}
\label{sec:Auswertung}

\subsection{Bestimmung der Winkelrichtgröße}
\label{subsec:a}
Die Gleichung \eqref{eq:winkelrichtgröße} lässt sich mit Hilfe der Gleichung \eqref{eq:drehmombetrag} zu 
\begin{equation}
  D = \frac{F r}{\varphi} 
\end{equation}

umschreiben. 

Bei der ersten Messung wurden die folgenden Daten \autoref{tab:Messung_a} aufgenommen.
Der Kraftmesser wurde bei alle Messungen bei einem Abstand von $r = 0.2 \unit{meter}$ angesetzt.
\begin{table}[H]
  \centering
  \sisetup{table-format=2.0}
  \begin{tabular}{S S }
      \toprule
      {$\varphi\mathbin{/}\unit{Grad}$} & {$F \mathbin{/} \unit{\newton}$}\\
      \midrule
           30   &         0.042   \\
           40   &         0.062   \\
           50   &         0.088   \\
           60   &         0.106   \\  
           70   &         0.128   \\
           80   &         0.140   \\
           90   &         0.168   \\
           100  &         0.190   \\
           110  &         0.200   \\
           120  &         0.250   \\
      \bottomrule
  \end{tabular}
  \caption{Rücktreibende Kraft zu verschiedenen Auslenkungen.}
  \label{tab:Messung_a}
\end{table}

Aus den gemessenen Werten in \autoref{tab:Messung_a} wurde der Mittelwert und die Standartabweichung berechnet.