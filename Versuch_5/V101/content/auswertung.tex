\section{Auswertung}
\label{sec:Auswertung}

\subsection{Bestimmung der Winkelrichtgröße}
\label{subsec:a}
Mithilfe von \eqref{eq:drehmombetrag} lässt sich \eqref{eq:winkelrichtgröße} zu
\begin{equation}
  D = \frac{F r}{\varphi}
  \label{eq:WinkelrichtgröFr}
\end{equation}
umschreiben. 

Die bei der ersten Messung aufgenommenen Daten sind in \autoref{tab:Messung_a} dargestellt.
Dabei ist $\varphi$ die Auslenkung des Systems und $F$ die am im Abstand von $r = 0,2 \,\unit{\meter}$ Kraftmesser gemessene rücktreibende Kraft. 
\begin{table}[H] % Hier noch eine dritte Spalte mit den einzelnen Winkelrichtgrößen wär nicht schlecht...
  \centering
  \sisetup{table-format=3.0}
  \begin{tabular}{S S[table-format=1.3]}
      \toprule
      {$\varphi\mathbin{/}\unit{°}$} & {$F \mathbin{/} \unit{\newton}$}\\
      \midrule
           30  & 0.042 \\
           40  & 0.062 \\
           50  & 0.088 \\
           60  & 0.106 \\  
           70  & 0.128 \\
           80  & 0.140 \\
           90  & 0.168 \\
           100 & 0.190 \\
           110 & 0.200 \\
           120 & 0.250 \\
      \bottomrule
  \end{tabular}
  \caption{Rücktreibende Kraft zu verschiedenen Auslenkungen.}
  \label{tab:Messung_a}
\end{table}

%Mittelwert und Standardabweichung hinschreiben
Aus den gemessenen Werten wurde der Mittelwert und die Standardabweichung berechnet.
Für die Winkelrichtgröße ergibt sich dann  
\begin{equation*}
  D = 0.0203 \pm 0.0020 \,\unit{\newton\meter} \,.  
\end{equation*}

\subsection{Bestimmung des Eigenträgheitsmoments}
\label{subsec:b}
% Was ist mit den Unsicherheiten auf den Messwerten, zumindest auf dem Gewicht? Hatten wir die?
Um das Eigenträgheitsmoment zu bestimmen, wurden zwei Zylinder mit einer Masse von je $222,7 \,\unit{g}$, einem Radius von $16 \,\unit{mm}$ und einer Höhe von $27,1 \,\unit{mm}$ senkrecht zur Drehachse befestigt.
Der Abstand $a$ wurde dabei stets bis zum vorderen Ende des Zylinders gemessen und $\dfrac{h}{2}$ addiert, in \autoref{tab:Messung_b} 
sind also die Abstände bis zum Zylinderzentrum mit den dazugehörigen Schwingungsdauern dargestellt.

\begin{table}[H] % Hier statt 3T lieber 1T auftragen?? Ich meine, Katja hatte das schon beim ersten Protokoll erwähnt, dass 3T unnötig ist...
  \centering
  \begin{tabular}{S[table-format=2.3] S[table-format=2.2]}
      \toprule
      {$a \mathbin{/}\unit{cm}$} & {$3T \mathbin{/} \unit{\second}$}\\
      \midrule
           20.355 & 18.50 \\
           21.355 & 19.62 \\
           22.355 & 19.73 \\
           23.355 & 20.57 \\  
           24.355 & 21.33 \\
           25.355 & 21.99 \\
           26.355 & 22.94 \\
           27.355 & 23.38 \\
           28.355 & 24.33 \\
           29.355 & 25.22 \\
      \bottomrule
  \end{tabular}
  \caption{Schwingungsdauern $T$ bei verschiedenen Abständen $a$.}
  \label{tab:Messung_b}
\end{table}
Für das Trägheitsmoment wird die Form $ I = T_D + 2I_z + 2m a^2$ angenommen. 
Durch einsetzen in \eqref{eq:periodendauer} ergibt sich \eqref{eq:periodendauerqadrat}, dabei ist $I_z$ das Trägheitsmoment eines Zylinders, $m$ die Masse und $a$ der Abstand zur Rotationsachse.
\begin{equation}
  T^2 = 4 \pi^2 \frac{I_D}{D} + 8 \pi^2 \frac{m (\frac{R^2}{4} + \frac{h^2}{12})}{D}+ 8 \pi^2 \frac{m}{D} a^2 .
  \label{eq:periodendauerqadrat}
\end{equation}
In \autoref{fig:T2ga2} wird nun das Quadrat der Schwingungsdauer $T^2$ gegen das Abstandquadrat $a^2$ aufgetragen.
Die dazu durchgeführte lineare Regression nimmt die Form $T^2 = A  a^2 + B $ mit den Parametern
\begin{equation*}
  A = (708 \pm 20) \dfrac{1}{s^2m^2}
\end{equation*}
\begin{equation*}
 B = (8.8 \pm 1.3) \dfrac{1}{s^2} 
\end{equation*}
an.

Mithilfe von \eqref{eq:periodendauer} lässt sich das Eigenträgheitsmoment zu
\begin{equation*}
  I_D = ... \pm ... \,.
\end{equation*}
bestimmen.

% Hast du den Theoriewert für das Eigenträgheitsmoment irgendwo? :D

\begin{figure}[H]
  \centering
  \includegraphics{build/T2ga2.pdf}
  \caption{Lineare Regression der quadrierten Messwerte.}
  \label{fig:T2ga2}
\end{figure}

\subsection{Trägheitsmoment der Kugel}
\label{subsec:c}

Ähnlich zu \autoref{subsec:b} wird nun die Schwingungsdauer einer Kugel mit Durchmesser $d = 12,75 \, \unit{\centi\meter}$ und einer Masse von $m = 811,3 \,\unit{\gram}$ bestimmt und in \autoref{tab:Messung_c} dargestellt.
Die Messung wird mit einer Auslenkung von $\varphi = \dfrac{π}{2} = 90 \,\unit{\degree}$ zehn Mal wiederholt, die Schwingungsdauern werden gemittelt und es wird die Abweichung bestimmt.

\begin{table}[H] % Auch hier lieber noch 1T :D
  \centering
  \begin{tabular}{S[table-format=2.0] S[table-format=1.2]}
      \toprule
      {Messung} & {$3T \mathbin{/} \unit{\second}$}\\
      \midrule
          1  & 4.88 \\
          2  & 4.76 \\
          3  & 5.06 \\
          4  & 4.73 \\  
          5  & 4.88 \\
          6  & 4.68 \\
          7  & 4.99 \\
          8  & 5.26 \\
          9  & 4.94 \\
          10 & 4.88 \\
      \bottomrule
  \end{tabular}
  \caption{Schwingungsdauern $T$ der Kugel.}
  \label{tab:Messung_c}
\end{table}
Es ergibt sich die Periodendauer
\begin{equation*}
  T= ... \pm ... \,.
\end{equation*}

Aus \eqref{eq:periodendauer} ergibt sich das Trägheitsmoment der Kugel zu
\begin{equation*}
  I_{K,m} = ... \pm ... \,.
\end{equation*} \\

Wie in \autoref{fig:trägmomSKZ} zu sehen, ergibt sich der Theoriewert des Trägheitsmoments einer Kugel aus
\begin{equation}
  I_{K} = \frac{2}{5} m R^2
  \label{trägheitsmomK}
\end{equation}
zu
\begin{equation*}
  I_{K,t} = ... \,.
\end{equation*}

Die relative Abweichung des gemessenen und theoretischen Wertes beträgt dann
\begin{equation*}
  \left|\frac{I_{K,m}}{I_{K,t}} * 100 - 100 \right| = (... \pm ...) \% \,.
\end{equation*}



\subsection{Trägheitsmoment des Zylinders}
\label{subsec:d}

Analog zur Bestimmung des Trägheitsmoments der Kugel wird nun für den Zylinder vorgegangen. Mit einer Masse von $367,7 \,\unit{\gram}$, einer Höhe von $h = 9,74 \,\unit{\centi\meter}$
und einem Durchmesser von $d = 9,67 \,\unit{\centi\meter}$ ergeben sich für zehn Messungen die in \autoref{tab:Messung_d} dargestellten Schwingungsdauern.

\begin{table}[H] % Auch hier lieber noch 1T :D
  \centering
  \begin{tabular}{S[table-format=2.0] S[table-format=1.2]}
      \toprule
      {Messung} & {$3T \mathbin{/} \unit{\second}$}\\
      \midrule
          1  & 2.72 \\
          2  & 2.32 \\
          3  & 2.58 \\
          4  & 2.46 \\  
          5  & 2.26 \\
          6  & 2.32 \\
          7  & 2.64 \\
          8  & 2.32 \\
          9  & 2.39 \\
          10 & 2.45 \\
      \bottomrule
  \end{tabular}
  \caption{Schwingungsdauern $T$ des Zylinders.}
  \label{tab:Messung_d}
\end{table}

Gemittelt ergibt sich hier die Schwingungsdauer
\begin{equation*}
  T = ... \pm ... \,.
\end{equation*}

Das experimentelle Trägheitsmoment lässt sich erneut aus \eqref{eq:periodendauer} zu
\begin{equation*}
  I_{K,m} = ... \pm ...
\end{equation*} 
bestimmen. \\

Mit dem Trägheitsmoment
\begin{equation}
  I_Z = \frac{1}{2} m R^2
\end{equation}
eines aufrecht zur Drehachse stehenden Zylinders ergibt sich für den Theoriewert
\begin{equation*}
  I_{Z,t} = ... \,.
\end{equation*}

Die Abweichung beträgt dann
\begin{equation*}
  \left|\frac{I_{Z,m}}{I_{Z,t}} * 100 - 100 \right| = (... \pm ...) \% \,.
\end{equation*}

\subsection{Trägheitsmoment einer Holzpuppe}
\label{subsec:e}

Abschließend soll das Trägheitsmoment einer Holzpuppe in zwei unterschiedlichen Positionen bestimmt werden. Dazu wurden zunächst die einzelnen Körperteile abgemessen.
Für die Längen $l$ ergaben sich die in \autoref{tab:Messung_e1} aufgetragenen Werte, die je zehnfach gemessenen Durchmesser $d$ der einzelnen Körperteile finden sich in \autoref{tab:Messung_e2}.

\begin{table}[H]
  \centering
  \begin{tabular}{S[table-format=2.0] S[table-format=2.2]}
      \toprule
      {Körperteil} & {$l \mathbin{/} \unit{\centi\meter}$}\\
      \midrule
        {Kopf}  & 4.31  \\
        {Arme}  & 12.92 \\
        {Torso} & 8.74 \\
        {Beine} & 14.63 \\
      \bottomrule
  \end{tabular}
  \caption{Längen der einzelnen Puppenkörperteile.}
  \label{tab:Messung_e1}
\end{table}

\begin{table}[H] % Auch hier lieber noch 1T :D
  \centering
  \sisetup{table-format=1.2}
  \begin{tabular}{S[table-format=2.0] S S S S}
      \toprule
      {Messung} & {$d_{Kopf} \mathbin{/} \unit{\centi\meter}$} & {$d_{Arme} \mathbin{/} \unit{\centi\meter}$} & {$d_{Torso} \mathbin{/} \unit{\centi\meter}$} & {$d_{Beine} \mathbin{/} \unit{\centi\meter}$} \\
      \midrule
        1  & 3.02 & 1.24 & 3.22 & 1.25 \\
        2  & 2.96 & 1.57 & 3.55 & 1.47 \\
        3  & 2.53 & 1.26 & 3.17 & 1.14 \\
        4  & 1.58 & 1.40 & 2.84 & 1.86 \\  
        5  & 2.92 & 1.10 & 2.59 & 1.31 \\
        6  & 2.48 & 0.96 & 2.74 & 1.17 \\
        7  & 2.02 & 1.32 & 2.50 & 1.03 \\
        8  & 2.87 & 1.30 & 3.27 & 0.99 \\
        9  & 2.03 & 0.85 & 3.35 & 1.01 \\
        10 & 1.59 & 1.22 & 3.46 & 1.22 \\
      \bottomrule
  \end{tabular}
  \caption{Durchmesser der einzelnen Puppernkörperteile.}
  \label{tab:Messung_e2}
\end{table}

Nach Mittelung der gemessenen Durchmesser ergeben sich die in \autoref{tab:Messung_e3} dargestellten Werte.

\begin{table}[H]
  \centering
  \begin{tabular}{S[table-format=2.0] S[table-format=2.2]}
      \toprule
      {Körperteil} & {$\bar{d} \mathbin{/} \unit{\centi\meter}$}\\
      \midrule
        {Kopf}  & {$... \pm ...$} \\
        {Arme}  & {$... \pm ...$} \\
        {Torso} & {$... \pm ...$}\\
        {Beine} & {$... \pm ...$} \\
      \bottomrule
  \end{tabular}
  \caption{Gemittelte Durchmesser der einzelnen Puppenkörperteile.}
  \label{tab:Messung_e3}
\end{table}
