\section{Auswertung}
\label{sec:Auswertung}

\subsection{Bestimmung der Winkelrichtgröße}
\label{subsec:a}
Die Gleichung \eqref{eq:winkelrichtgröße} lässt sich mit Hilfe der Gleichung \eqref{eq:drehmombetrag} zu 
\begin{equation}
  D = \frac{F r}{\varphi} 
\end{equation}

umschreiben. 

Bei der ersten Messung wurden die folgenden Daten \autoref{tab:Messung_a} aufgenommen.
Der Kraftmesser wurde bei alle Messungen bei einem Abstand von $r = 0.2  \unit{m}$ angesetzt.
\begin{table}[H]
  \centering
  \sisetup{table-format=2.0}
  \begin{tabular}{S S }
      \toprule
      {$\varphi\mathbin{/}\unit{°}$} & {$F \mathbin{/} \unit{\newton}$}\\
      \midrule
           30   &         0.042   \\
           40   &         0.062   \\
           50   &         0.088   \\
           60   &         0.106   \\  
           70   &         0.128   \\
           80   &         0.140   \\
           90   &         0.168   \\
           100  &         0.190   \\
           110  &         0.200   \\
           120  &         0.250   \\
      \bottomrule
  \end{tabular}
  \caption{Rücktreibende Kraft zu verschiedenen Auslenkungen.}
  \label{tab:Messung_a}
\end{table}

Aus den gemessenen Werten in \autoref{tab:Messung_a} wurde der Mittelwert und die Standartabweichung berechnet.
Für die Windrichtgröße ergit sich $ 0.0203 \pm 0.0020 \unit{Nm} $.

\subsection{Bestimmung des Eigenträgheitsmoments $I_D $ }
\label{subsec:b}
Um das Eigenträgheitsmoment zu bestimmen würden zwei Zylinder mit einer Masse von je $222.7 \unit{g}$ senkrecht zur Drehachse befestigt.
Weiter haben die Zylinder einen Radius von $16\unit{mm}$ und eine Höhe von $ 27.1 \unit{mm}$.
Für diese Messung ergaben sich die Werte der \autoref{tab:Messung_b}.

\begin{table}[H]
  \centering
  \sisetup{table-format=2.0}
  \begin{tabular}{S S }
      \toprule
      {$a \mathbin{/}\unit{cm}$} & {$3T \mathbin{/} \unit{\second}$}\\
      \midrule
           20.355  &         18.50   \\
           21.355  &         19.62   \\
           22.355  &         19.73   \\
           23.355  &         20.57   \\  
           24.355  &         21.33   \\
           25.355  &         21.99   \\
           26.355  &         22.94   \\
           27.355  &         23.38   \\
           28.355  &         24.33   \\
           29.355  &         25.22   \\
      \bottomrule
  \end{tabular}
  \caption{Rücktreibende Kraft zu verschiedenen Auslenkungen.}
  \label{tab:Messung_b}
\end{table}
Für das Trägheitsmoment wird die Form $ I = T_D + 2I_z + 2m a^2$ angenommen. Diese Form wird die Gleichung \eqref{eq:periodendauer} eingesetzten und es ergibt sich die Gleichung \eqref{eq:periodendauerqadrat}, dabei $I_z$ das Trägheitsmoment eines Zylinder $ m$ die Masse und $a$ der Abstand zu der Rotationsachse.
\begin{equation}
  T^2 = 4\pi^2 \dfrac{I_D}{D} + 8\pi^2\dfrac{m(\dfrac{R^2}{4} + \dfrac{h^2}{12})}{D}+ 8 \pi^2 \dfrac{m}{D}a^2 .
  \label{eq:periodendauerqadrat}
\end{equation}
In \autoref{fig:T2ga2} wird $T^2$ gegen $ a^2$ aufgetragen.

\begin{figure}[H]
  \centering
  \includegraphics{build/T2ga2.pdf}
  \caption{Lineare Regression der qadrierten Messewerte.}
  \label{fig:T2ga2}
\end{figure}
Die Regressiongerade hat die Form $T^2 = A  a^2 + B $ mit den Paramtern
\begin{equation*}
  A = (708 \pm 20) \dfrac{1}{s^2m^2}
\end{equation*}
\begin{equation*}
 B = (8.8 \pm 1.3) \dfrac{1}{s^2} 
\end{equation*}
